%\def\option{}
\providecommand{\option}{handout}
\documentclass[aspectratio=169,notheorems,9pt,\option]{beamer}



\makeatletter
\def\readrest#1\enddate{}
\def\setdate@#1-#2-#3\relax{\def\lecturedate{#1.#2.#3.}\readrest}
\def\setdate{\expandafter\setdate@\jobname\relax-\relax-\relax\enddate}

\makeatother

\setdate


\usepackage{marvosym}
%\usepackage[magyar]{babel}
\usepackage{polyglossia}
\setmainlanguage{magyar}

% \usetheme{Bruno}

\usecolortheme{Bruno}
% \usefonttheme{Bruno}
% \useinnertheme{Bruno}
\useoutertheme{Bruno}

\usepackage{pvdefs}
\newtheorem{df}{Definíció}
\newtheorem{theorem}{Tétel}
\newtheorem{lemma}{Lemma}
\newtheorem{proposition}{Állítás}
\newtheorem{corollary}{Következmény}

\defcx{XAPIDLCSGTHEN}
\deftx{STQbCf\lambda\Omega\sigma}
\defmf{X}

\def\diag{\operatorname{diag}}
\def\Id{\mathbf{I}}
\let\PEfont\mathbb
\def\npto{\buildrel p\over \nrightarrow}
\def\ndto{\buildrel d\over \nrightarrow}
\def\dto{\buildrel d\over \rightarrow}
\def\dto{\stackrel{d}{\to}}
\let\nto\nrightarrow

\def\Leb{\operatorname{Leb}}

\newrobustcmd{\continue}{%
  \onslide<\value{beamerpauses}->\relax
}


\beamerdefaultoverlayspecification{<+->}

\mode<presentation>

\RequirePackage[math-style=TeX, bold-style=upright]{unicode-math}
\setmathfont{Latin Modern Math}[Scale=1]
% \setsansfont{Kurier Cond}
\setsansfont[Ligatures=TeX, ItalicFont={Fira Sans Italic}, BoldFont={Fira Sans SemiBold}, Scale=0.9]{Fira Sans Book}
%\setsansfont{Ubuntu}

%\usepackage{mathspec}
%\setmathsfont{lm}
%(Digits,Latin,Greek)
%[Numbers={Lining,Proportional}]{lm}
%\setmathsfont(Digits,Greek)
%[Uppercase=Plain,Lowercase=Regular,Scale=MatchLowercase]
%{GFS Porson}


% Title page
\setbeamerfont{title}{
    size=\LARGE,
    shape=\bfseries
}

\setbeamerfont{subtitle}{
    size=\large,
    shape=\normalfont
}

\setbeamerfont{author}{
    size=\normalsize,
    shape=\normalfont
}

\setbeamerfont{details}{
    size=\footnotesize,
    shape=\normalfont
}

% Slide title
\setbeamerfont{frametitle}{size=\LARGE}
\setbeamerfont{framesubtitle}{
    size=\normalsize,
    %shape=\normalfont\scshape
}

% Blocks
\setbeamerfont{block title}{
    size=\normalfont,
    shape=\strut
}

\setbeamerfont{blockdef}{
    size=\large,
    shape=\bfseries
}

% Description
\setbeamerfont{description item}{shape=\bfseries}


% Footer information line with title, author and slide number
\setbeamertemplate{footline}{%
    \leavevmode%
    \hbox{%
        \usebeamerfont{footline}%
        \begin{beamercolorbox}[
            wd=\textwidth,
            ht=3ex,
            dp=1.25ex
        ]{footline}%
            \hspace{1cm}%
            \insertshorttitle%
            \hspace{2em}%
            \insertdate%
            \hfill%
            \insertframenumber{} / \inserttotalframenumber%
            \hspace{1cm}
        \end{beamercolorbox}%
    }
    \vskip0pt%
}

%% Bruno Beamer theme — Version 1.0
%% A lightweight Beamer theme inspired from the Metropolis theme
%%
%% Written in 2017-2019 by:
%% — Rémi Cérès <remiceres@msn.com>
%% — Mattéo Delabre <bonjour@matteodelabre.me>
%%
%% This work is released under the CC0 1.0 Universal license. See the
%% accompanying LICENSE file for details. To the extent possible under
%% law, Rémi Cérès and Mattéo Delabre have waived all copyright and
%% related or neighboring rights to the Bruno Beamer theme. This work
%% is published from France.

% \mode<presentation>

\setbeamertemplate{sections/subsections in toc}[square]
\setbeamertemplate{itemize item}[square]
\setbeamertemplate{itemize subitem}[circle]

% Allow multi-slide frames
\setbeamertemplate{frametitle continuation}{}

% Horizontal separator
\setbeamertemplate{separator}{%
    \usebeamercolor{separator}%
    \textcolor{fg}{\rule{.7\textwidth}{.5pt}}%
}

% Separate authors with newlines
\renewcommand{\beamer@andtitle}{\\}

% Title page. If you want to add a background image on the title slide,
% use the \background command to set the path to that image. Otherwise,
% no image will be included
\RequirePackage{tikz}
\usetikzlibrary{fadings}
\newcommand{\background}[1]%
    {\newcommand{\bruno@background}{#1}}


%% \newcommand\insertcaption{}
\newcommand{\backgroundcaption}[2][white]{\def\bruno@caption{{#1}{#2}}}

\newcommand{\insertcaption}{\ifdefined\bruno@caption{\expandafter\textcolor\bruno@caption}\fi}

\tikzfading[
    name=title page picture fading,
    left color=transparent!0,
    right color=transparent!100,
]

\setbeamertemplate{title page}{%
    \begin{minipage}{.7\textwidth}
        \raggedright
        \usebeamerfont{title}
        \inserttitle

        \usebeamertemplate{separator}

        \usebeamerfont{author}
        \vspace{2em}
        \insertauthor

        \vspace{2em}
        \usebeamerfont{details}
        \insertinstitute

        \insertdate
    \end{minipage}

    % Include faded image on the right if defined
    \ifdefined\bruno@background
    \begin{tikzpicture}[remember picture, overlay]
        % Crop image to a trapezium on the right
        \clip (current page.south east)
           -- (current page.north east)
           -- ++(-5.1, 0)
           -- ++(-3, -\paperheight)
           -- cycle;

        % Include background image
        \node[
            anchor=south east,
            inner sep=0,
            outer sep=0
        ] at (current page.south east) {
            \includegraphics[height=\paperheight]
                {\bruno@background}
        };

        \node[anchor=south east,
            inner sep=0,
            outer sep=15pt,
        ] at (current page.south east){\insertcaption};

        % Add a slight shadow
        \fill[
            black, path fading=title page picture fading,
            fading angle=-15
        ]
            (current page.south east)
            rectangle
            ++(-10, \dimexpr\paperheight+1cm);

        
    \end{tikzpicture}
    \fi
}

\mode<all>
\AtBeginDocument{%
  \let\phi\varphi
  \let\theta\vartheta
  \let\setminus\smallsetminus
}


\title{Statisztika előadás}
\date{\lecturedate}
\background{images/young_Ronald_Fisher.jpg}
\backgroundcaption{Ronald Fisher (1913)}


\begin{document}

\maketitle

\begin{frame}{Teljesség}
    \begin{df}
      $\cP=\set*{\P[\theta]}{\theta\in\Theta}$ eloszláscsalád, 
      $S$ statisztika. $S$ \textbf{teljes statisztika} (a $\cP$ eloszláscsaládra nézve), ha 
      \begin{displaymath}
        \E[\theta](h(S))=0,\quad\text{minden $\theta\in\Theta$-ra}\implies \P[\theta]{h=0}=1,\quad\text{minden $\theta$-ra}
      \end{displaymath} 
      Azaz $S$ teljes, ha az $S$ függvényei között csak a lényegében nulla változóra teljesül, hogy  
      az eloszlás család minden tagja mellett létezik és nulla a várható értéke.
  
      $S$ \textbf{korlátosan teljes}, ha az $S$ \textbf{korlátos} függvényei között csak a 
      lényegében nulla változóra teljesül, hogy  
      az eloszlás család minden tagja mellett  nulla a várható értéke.
    \end{df}
    \begin{theorem}
      Ha $S$ korlátosan teljes és elégséges a $\cP$ eloszláscsaládra, akkor minimális elégséges. 
    \end{theorem}
    \begin{itemize}
      \item $A\in\sigma(S)$, $\cS$ tetszőleges elégséges $\sigma$-algebra, $T_A=\P[\theta]{X\in A\given \cS}$ közös változat.
      \item $h(S)=\E[\theta]{\I[A]-T_A\given S}=\I[A]-\E{T_A\given S}$ korlátos statisztika, 
      $\E[\theta]{h(S)}=\P[\theta]{A}-\P[\theta]{A}=0$.
      \item Teljesség miatt $h(S)=0$, azaz $\I[A]=\E{T_A\given S}$ $\cP$ majdnem mindenütt. $0\leq T_A\leq 1$ miatt
      \begin{displaymath}
        \E[\theta]{(\I[A]-T_A)^2\given S}=\E{\I[A]-2\I[A]T_A+T_A^2\given S}
        = \E{T_A^2\given S}-\I[A]\leq\E{T_A\given S}-\I[A]=0.
      \end{displaymath}
      \item $\P[\theta]{\I[A]=T_A}=1$ minden $\theta$-ra, 
      azaz $\P[\theta]{A\circ \event*{T_A=1}}=0$, így $A\in\cS^*$ és $\sigma(S)\subset\cS^*$.
    \end{itemize}  
  \end{frame}
  
  
  \begin{frame}{Példa. Nem minden minimális elégséges statisztika korlátosan teljes}
    \begin{itemize}
      \item $U(\theta,\theta+1)$, $\theta\in\real$ eloszlásból származó, $n\geq2$ elemű minta.
      \item $S=(X_1^*,X_n^*)$ minimális elégséges statisztika.
      \item $X_n^*-X_1^*$ eloszlása nem függ $\theta$-tól és $\E[\theta]{X_n^*-X_1^*}=\frac{n-1}{n+1}$.
      \item $h(S)=(X_n^*-X_1^*)\wedge1-\frac{n-1}{n+1}$ olyan korlátos függvénye $S$-nek, amire
      \begin{displaymath}
        \E[\theta]{h(S)}=0,\quad\forall \theta\in\real.
      \end{displaymath}
    \end{itemize}
  \end{frame}
  
  \begin{frame}{További példa teljes statisztikára}
    \begin{itemize}
      \item $N(\theta,1)$ eloszlásból származó $n$ elemű mintánk van.
      \item $S=\sum X_i\sim N(n\theta,n)$ elégséges statisztika.
      \item 
      \begin{displaymath}
        0=\E[\theta]{h(S)}=\int h(x)\frac{1}{\sqrt{2\pi n}} e^{-\frac{(x-n\theta)^2}{2n}}d x 
        =C(\theta)\int h(x)e^{-\frac{x^2}{2n}}e^{\theta x}d x\quad\forall\theta\in\real  
      \end{displaymath} 
    \end{itemize}
    \begin{theorem}
      $h:\real\to\real$ mérhető, $a<b$, 
      \begin{displaymath}
        \int h(x)e^{\theta x} d x,\quad\text{létezik és nulla minden $\theta\in(a,b)$-re}.
      \end{displaymath} 
      Ekkor $h=0$ Lebesgue majdnem mindenütt.
    \end{theorem}
    \begin{itemize}
      \item A tétel alapján, $h(x)e^{-\frac{x^2}{2n}}=0$ mm. és így $\P[\theta]{h(S)=0}=1$ minden $\theta$-ra
      \item $S$ teljes és elégséges, ezért minimális elégséges.
    \end{itemize}
  \end{frame}
  
  \begin{frame}{Momentum generáló függvény}  
    $X$ $p$-dimenziós vektor változó, $M_X(t)=\E{e^{t\cdot X}}$, $t\in\real^p$ az $X$ momentum generáló függvénye. 
    \begin{proposition}
      Ha $X$ momentum generáló függvénye véges az origó egy
      környezetében, akkor meghatározza $X$ eloszlását.
    \end{proposition}
    \continue
    Bizonyítás vázlat.
    \begin{itemize}   
    \item $X$ skalár változó. $h (z)=h (x+iy)=\E{e^{(x+iy) X}}$  a képzetes tengely körüli
      sávban definiált és véges. Itt deriválható (komplex értelemben),
      ezért a valós tengelyen felvett értékek (a momentum generáló függvény)
      meghatározzák a karakterisztikus függvényt, az pedig az eloszlást.
   
    \item $X$ vektor változó. $\alpha$ rögzített vektor, $Y=\alpha\cdot
      X$ skalár változó. $\E{e^{tY}}=\E{e^{(t\alpha)X}}$, azaz $Y$ eloszlását
      $X$ momentum generáló függvénye meghatározza.
    \item $X$ eloszlását az $\alpha\cdot X$ alakú változók eloszlása
      meghatározza, ugyanis a karakterisztikus függvényekre
      $\phi_{X} (\alpha)=\phi_{\alpha\cdot X} (1)$.
    \item Összefoglalva, a momentum generáló függvény meghatározza az
      egy dimenziós vetületek eloszlását, azok pedig $X$ eloszlását.
    \end{itemize}
    
  \end{frame}
  
  \begin{frame}%%{Elégséges feltétel teljességre}
    \begin{theorem}
      Legyen $\mu$ mérték $\B (\real^p)$-n, $h:\real^p\to\real$ mérhető és 
      \begin{displaymath}\textstyle
        A=\set*{\alpha\in\real^p}{\int e^{\alpha \cdot x}h (x)\mu
          (dx)=0}%\quad\text{nem üres nyílt}
      \end{displaymath}
      Ha $A$-nak létezik belső pontja, akkor $h$ $\mu$ majdnem mindenütt nulla.
    \end{theorem}
    \begin{itemize}
    \item Legyen $\alpha_0\in\interior A$, és
      \begin{displaymath}
        \nu_{\pm} (H)=\int e^{\alpha_0x}\abs*{h}_\pm (x) \mu (d x),\quad H\in\B(\real^p).
      \end{displaymath}
      $\nu_{+}$, $\nu_{-}$ véges Borel mértékek,  hiszen $\alpha_0\in A$.
    \item  Ha $\nu_{+} (\real^p)=\nu_{-} (\real^p) = 0$, akkor $h$ $\mu$ majdnem mindenütt
      nulla.
    \item Ha $\nu_{+}(\real^p)=\nu_{-} (\real^p)>0$, akkor konstanssal
      való szorzás után feltehető, hogy $\nu_{+},\nu_{-}$ valószínűségi
      mértékek. Azt akarjuk megmutatni, hogy ez az eset nem fordulhat
      elő. 
      
    \item A két mérték momentum generáló függvénye az origó egy
      környezetében azonos, hiszen
      \begin{displaymath}
        \int e^{t x}\nu_{+} (d x)-\int e^{t x}\nu_{-} (d x)=\int
        e^{(t+\alpha_0)x}h (x)\mu (d x)=0
            \quad\text{ha $\alpha_0+t\in A$.}
      \end{displaymath}
    \item Mivel a momentum generáló függvény az eloszlást meghatározza,
      $\nu_{+}=\nu_{-}$ és %$\nu_{+} (h>0)=\nu_{-} (h>0)=0$
      \begin{displaymath}
        \nu_{-} (h>0)=0=\nu_{+}(h>0)=
        \int \I{h (x) >0} e^{\alpha_0 x}\abs*{h}_+ (x)\mu (d x)=
        \nu_{+} (\real^p)=1,
        \quad\text{\alert{\Large\Lightning}}
      \end{displaymath}
      % Ez ellentmondás, abból indultunk ki, hogy feltehető, hogy $\nu_{+} (\real^p)=1$.
    \end{itemize}
  \end{frame}
  
  \begin{frame}{Exponenciális család}
    \begin{df}
      $\cP$ exponenciális eloszlás család, ha dominált és alkalmas paraméterezéssel 
      $\cP=\set*{\P[\theta]}{\theta\in\Theta}$, $\Theta\subset\real^p$ 
      %$\set*{\P[\theta]}{\theta\in\Theta}$ dominált
      %eloszláscsalád, $\Theta\subset\real^p$ nyílt. Ha 
      valamint alkalmas domináló mértékkel, $T:\mfX\to\real^p$ statisztikával 
      a sűrűségfüggvények alakja 
      \begin{displaymath}
        f_{\theta} (x) =\exp{\theta\cdot T(x)-b (\theta)}.\tag{*}
      \end{displaymath}
      %akkor \textbf{exponenciális családról} beszélünk.
    \end{df}
    \continue
    Tulajdonságok:
    \begin{itemize}
    \item $T$ elégséges statisztika (Neymann faktorizációs tétel)
    \item 
    Ha $\Theta\subset\real^p$-ben van $p+1$ általános helyzetű pont, akkor $T$ minimális elégséges.
  
    $\theta_i\in\Theta$, $i=0,\dots,p$ és $\theta_i-\theta_0$ lineáris burka $\real^p$.  
    
    Ekkor, ha $\frac{f_{\theta_i}(x)}{f_{\theta_i}(y)}$ nem függ $\theta$-tól, 
    akkor $(\theta_i-\theta_0)(T(x)-T(y))=0$, 
    $i=1,\dots,p$ és $T(x)=T(y)$, ami a minimális elégségesség feltétele.
  
    Exponenciális család mindig megadható úgy is, hogy $T$ minimális elégséges legyen. HF.
    % lineáris burka $\real^p$, 
    % azaz ha $\Theta\subset\real^p$ tartalmazza $\real^p$ egy (lineáris) bázisát. 
    % Altérre áttérve ez elérhető.
    
    % $T$ mindig választható minimális elégségesnek, elég 
    % $P T$-t minimális elégséges, ahol $P$ merőleges 
    % vetítés  $\Theta$ lineáris burkára.
  
    \item Ha $\Theta$ belseje nem üres, akkor $T$ teljes is.
    
    \end{itemize}
  \end{frame}
  
  \begin{frame}{Exponenciális család}
    \begin{df}
      $\cP$ exponenciális eloszlás család, ha dominált és alkalmas 
      $\cP=\set*{\P[\theta]}{\theta\in\Theta}$, $\Theta\subset\real^p$ paraméterezéssel 
      %$\set*{\P[\theta]}{\theta\in\Theta}$ dominált
      %eloszláscsalád, $\Theta\subset\real^p$ nyílt. Ha 
      és alkalmas domináló mértékkel, $T:\mfX\to\real^p$ statisztikával 
      a sűrűségfüggvények alakja 
      \begin{displaymath}
        f_{\theta} (x) =\exp{\theta\cdot T(x)-b (\theta)}.\tag{*}
      \end{displaymath}
      %akkor \textbf{exponenciális családról} beszélünk.
    \end{df}
    \continue
    Ha $\Theta$ belseje nem üres, akkor $T$ teljes is.
    \begin{itemize}  
    \item $\lambda$ a domináló mérték, amivel a sűrűség $(*)$ alakú. $\tlambda(H)=\lambda(T\in H)$, $H\in\B(\real^p)$.
  
    \item Ha minden $\theta\in\Theta$-ra, $\E[\theta](h(T))$ létezik és nulla, akkor
    \begin{displaymath}
      0=\E[\theta]{h(T)}=\int_{\mfX} h\circ T \exp{\theta T-b(\theta)}d\lambda 
      =\int_{\real^p} h(t) \exp{\theta t-b(\theta)}\tlambda(d t)
      \implies \int_{\real^p} e^{\theta t} h(t)\tlambda(d t). 
      %,\quad \forall \theta\in\Theta
    \end{displaymath}
  
    \item $h d\tlambda$ az azonosan nulla mérték. Tetszőleges $H\in\B(\real^p)$-re
    \begin{displaymath}
      0=\int_H h d\tlambda=\int_{\mfX} \I{T\in H} h\circ T d\lambda,\quad
      \forall  H\in\B(\real^p)
      \quad\implies\quad \lambda(h\circ T\neq0)=0. 
    \end{displaymath}
    $\P[\theta]{h(T)\neq 0}=0$, azaz $\P[\theta]{h(T)=0}=1$.
    \end{itemize}
  \end{frame}
  
  
  \begin{frame}{Példák exponenciális családra}
    \begin{itemize}
      \item $n$ elemű minta $\exp(\theta)$, $\theta>0$ eloszlásból, $\mfX=(0,\infty)^n$, $\lambda$ a Lebesgue mérték.
      \begin{displaymath}
        f_\theta(x) = \theta^n\exp{-\theta\sum x_i},\quad T(x)=-\sum x_i,\quad b(\theta)=-n\ln\theta.
      \end{displaymath}
      \item $n$ elemű indikátor minta, $p\in(0,1)$ paraméterrel, $\mfX=\smallset*{0,1}^n$, 
       $\lambda$ a számláló mérték
      \begin{align*}
        f_\theta(x)&=p^{\sum x_i}(1-p)^{n-\sum x_i}=\zfrac*{p}{1-p}^{\sum x_i} (1-p)^n\\
        &=\exp{\theta T(x) - b(\theta)},\quad \theta=\ln\frac{p}{1-p}, \quad p =\frac{1}{1+e^{-\theta}},\quad 
        T(x)=\sum x_i,\quad 
        b(\theta)= n\ln(1+e^{\theta})
      \end{align*}
      \item $N(\mu,\sigma^2)$ eloszlásból származó $n$ elemű minta. $\mfX=\real^n$ $\mu\in\real$, $\sigma^2>0$. 
      Sűrűségfüggvény a Lebesgue mértékre.
      \begin{align*}
        f_{\mu,\sigma^2}(x)&=\frac{1}{(2\pi)^{n/2}\sigma^n}\exp{-\frac1{2\sigma^2}\sum(x_i-\mu)^2}\\
        &=\exp{-\frac{1}{2\sigma^2}\sum x_i^2+\frac{\mu}{\sigma^2}\sum x_i -\tilde b(\mu,\sigma^2)},
        \quad \theta=\zjel*{\tfrac1{2\sigma^2},\tfrac\mu{\sigma^2}}, \quad T(x)=\zjel*{-\sum x_i^2,\sum x_i}.
      \end{align*}
    \end{itemize}
  \end{frame}
  
  \begin{frame}{$T$ nem mindig teljes}
    \begin{itemize}
      \item $N(\mu,\sigma^2)$ eloszlásból származó minta $\mu=\sigma^2$  ismeretlen.
      
      Átparaméterezés után $\theta(\mu,\sigma^2)=(\tfrac1{2\sigma^2},\tfrac\mu{\sigma^2})$,
      $\Theta=\set*{(\tau,1)}{\tau>0}$. $T(x)=(-\sum x_i^2,\sum x_i)$ minimális elégséges.
  
      \item $n=2$, 
      \begin{align*}
        \E{T_1}&= \E{-(X_1^2+X_2^2)}=-2(\sigma^2+\mu^2),\\
        \E{T_2}&= \E{(X_1+X_2)} =2\mu,\quad
        \E{T^2_2}=\E{(X_1+X_2)^2} = 2\sigma^2+(2\mu)^2
      \end{align*}
      \item 
      \begin{displaymath}
        \E{2T_1+T_2^2 + T_2} = -4(\sigma^2+\mu^2)+2\sigma^2+4\mu^2+2\mu=2\mu-2\sigma^2=0  
      \end{displaymath}
      $\E[\theta]{h(T)}=0$, minden $\theta=(\tau,1)$-re, ha 
      $h(t_1,t_2)=2t_1+t_2^2+ t_2$.   
      $\E[\theta]{h(T)}=0$ miden $\theta$-ra, de $\P[\theta]{h(T)=0}=0$.
      \begin{displaymath}
        h(T(X))=-(X_1-X_2)^2+2(X_1+X_2)
        %\P[\theta]{-X_1^2+X_1=c}=\P[\theta]{X_1\in\smallset*{x_1,x_2}}=0, 
        %\quad x_{1,2}=\frac{1\pm\sqrt{1-4c}}{2} 
      \end{displaymath}
      Itt $(X_1-X_2,X_1+X_2)$ független nem elfajult normális változók, 
      így $\P[\theta]{h(T)=0}=0$ minden $\theta\in\Theta$-ra. 
      % \begin{displaymath}
      %   \P[\theta]{h(T)=0}=\P[\theta]{-X_1^2+X_1=\sum_{k=2}^n -X_k^2+X_k}=0.
      % \end{displaymath}
  
    \end{itemize}  
  \end{frame}
  
  \end{document}
  \begin{frame}{Nem minden eloszláscsalád exponenciális}
    \begin{itemize}
      \item Ha $\cP$ exponenciális eloszláscsalád, akkor $\P\in\cP$ mértékek sűrűségek mindenhol pozitívak, 
      azaz ekvivalensek.
  
      $n$ elemű minta $U(\theta,\theta+1)$, $\theta\in\real$ eloszlásból.
      A kapott eloszláscsalád nem exponenciális.
      \item Cauchy eloszlás eltolás paraméteres családja: $X-\mu\sim \text{Cauchy}$, $\mu\in\real$.
      
      $X$ eloszlásainak a családja alkothat-e exponenciális családot?
  
      Indirekt feltevéssel %akkor $n$-elemű minta eloszlásainak a családja ismeretlen
      \begin{displaymath}
        \ln f_\mu(x)=\sum_i \ln f_\mu(x_i)=\theta(\mu)T_n(x)-n b(\theta(\mu))+\ln h(x), \quad T_n(x)=\sum_i T(x_i) 
      \end{displaymath}
      ahol $\theta,T:\real\to\real^p$. $\theta(0)=0$ feltehető, mert $\theta(0)T_n(x)$ beolvasztható $\ln h(x)$-be 
      és így a domináló mértékbe.
  
      Feltehető, hogy $\Theta=\set{\theta(\mu)}{\mu\in\real}$ %értékkészletének 
      lineáris burka $\real^p$. Ekkor létezik $\mu_k$ $k=1,\dots,p$, 
      hogy $\theta_k=\theta(\mu_k)$, $k=0,\dots,p$ lineárisan függetlenek. Ekkor alkalmas $c_k$ együtthatókkal.
      \begin{displaymath}
        T_n(x)=\sum_{k=1}^p c_k (\ln f_{\mu_k}-\ln f_0)(x).
      \end{displaymath}
      $n>p+1$ esetén $T_n(x)=T_n(y)$ akkor is előfordul, ha $x^*\neq y^*$, ami ellentmond annak, hogy a 
      rendezett minta minimális elégséges statisztika.
  
    \end{itemize}
    
  \end{frame}
  
  \end{document}
  
  \begin{frame}{Momentum generáló függvény, deriválhatóság}
    \begin{proposition}
      Ha $M (\alpha)=\E{e^{\alpha\cdot X}}<\infty$, az origó egy környezetében,
      akkor ott  $M$ deriválható.
    \end{proposition}
    \begin{itemize}
      
    \item  Létezik $\eps>0$, hogy $\E{e^{h\norm{X}}}<\infty$, ha
      $h<\eps$.
  
      $\real^p$ a normák ekvivalensek, elég
      $\norm{x}_{1}=\frac1p \sum \abs*{x_i}$-re számolni. Mivel $\exp$
      konvex $e^{\norm{x}_1}\leq\frac1p \sum e^{\abs*{x_i}}$ és
      %$\norm{h}_1=\sum h_i=1$ és $h_i\geq0$, akkor
      \begin{displaymath}
        e^{h\abs*{X_i}}\leq e^{hX_i}+e^{-hX_i},\quad
        \E{e^{h\norm{X}_1}}\leq \frac1p \sum_i(M (h e_i)+M (-h e_i))<\infty
      \end{displaymath}
      ahol $e_1,\dots,e_p$ a természetes bázis $\real^p$-ben.
    \item ha $\norm{\alpha}$ kicsi, akkor
      $\abs**{\alpha\cdot X}<\eps\norm{X}$ és a dominált konvergencia tétel miatt
      \begin{displaymath}
        M (\alpha)=\E{\sum_k \frac{(\alpha\cdot X)^n}{n!}}
        =\sum_n \frac{\E{(\alpha\cdot X)^n}}{n!}\quad\implies\quad 
        \partial_{i_1,\dots,i_k}M (0) =\E{\prod_{j}X_{i_j}}
      \end{displaymath}
    \end{itemize}
  
  \end{frame}
  
  \end{document}
  