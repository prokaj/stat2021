%\def\option{}
\providecommand{\option}{handout}
\documentclass[aspectratio=169,notheorems,9pt,\option]{beamer}



\makeatletter
\def\readrest#1\enddate{}
\def\setdate@#1-#2-#3\relax{\def\lecturedate{#1.#2.#3.}\readrest}
\def\setdate{\expandafter\setdate@\jobname\relax-\relax-\relax\enddate}

\makeatother

\setdate


\usepackage{marvosym}
%\usepackage[magyar]{babel}
\usepackage{polyglossia}
\setmainlanguage{magyar}

% \usetheme{Bruno}

\usecolortheme{Bruno}
% \usefonttheme{Bruno}
% \useinnertheme{Bruno}
\useoutertheme{Bruno}

\usepackage{pvdefs}
\newtheorem{df}{Definíció}
\newtheorem{theorem}{Tétel}
\newtheorem{lemma}{Lemma}
\newtheorem{proposition}{Állítás}
\newtheorem{corollary}{Következmény}

\defcx{XAPIDLCSGTHEN}
\deftx{STQbCf\lambda\Omega\sigma}
\defmf{X}

\def\diag{\operatorname{diag}}
\def\Id{\mathbf{I}}
\let\PEfont\mathbb
\def\npto{\buildrel p\over \nrightarrow}
\def\ndto{\buildrel d\over \nrightarrow}
\def\dto{\buildrel d\over \rightarrow}
\def\dto{\stackrel{d}{\to}}
\let\nto\nrightarrow

\def\Leb{\operatorname{Leb}}

\newrobustcmd{\continue}{%
  \onslide<\value{beamerpauses}->\relax
}


\beamerdefaultoverlayspecification{<+->}

\mode<presentation>

\RequirePackage[math-style=TeX, bold-style=upright]{unicode-math}
\setmathfont{Latin Modern Math}[Scale=1]
% \setsansfont{Kurier Cond}
\setsansfont[Ligatures=TeX, ItalicFont={Fira Sans Italic}, BoldFont={Fira Sans SemiBold}, Scale=0.9]{Fira Sans Book}
%\setsansfont{Ubuntu}

%\usepackage{mathspec}
%\setmathsfont{lm}
%(Digits,Latin,Greek)
%[Numbers={Lining,Proportional}]{lm}
%\setmathsfont(Digits,Greek)
%[Uppercase=Plain,Lowercase=Regular,Scale=MatchLowercase]
%{GFS Porson}


% Title page
\setbeamerfont{title}{
    size=\LARGE,
    shape=\bfseries
}

\setbeamerfont{subtitle}{
    size=\large,
    shape=\normalfont
}

\setbeamerfont{author}{
    size=\normalsize,
    shape=\normalfont
}

\setbeamerfont{details}{
    size=\footnotesize,
    shape=\normalfont
}

% Slide title
\setbeamerfont{frametitle}{size=\LARGE}
\setbeamerfont{framesubtitle}{
    size=\normalsize,
    %shape=\normalfont\scshape
}

% Blocks
\setbeamerfont{block title}{
    size=\normalfont,
    shape=\strut
}

\setbeamerfont{blockdef}{
    size=\large,
    shape=\bfseries
}

% Description
\setbeamerfont{description item}{shape=\bfseries}


% Footer information line with title, author and slide number
\setbeamertemplate{footline}{%
    \leavevmode%
    \hbox{%
        \usebeamerfont{footline}%
        \begin{beamercolorbox}[
            wd=\textwidth,
            ht=3ex,
            dp=1.25ex
        ]{footline}%
            \hspace{1cm}%
            \insertshorttitle%
            \hspace{2em}%
            \insertdate%
            \hfill%
            \insertframenumber{} / \inserttotalframenumber%
            \hspace{1cm}
        \end{beamercolorbox}%
    }
    \vskip0pt%
}

%% Bruno Beamer theme — Version 1.0
%% A lightweight Beamer theme inspired from the Metropolis theme
%%
%% Written in 2017-2019 by:
%% — Rémi Cérès <remiceres@msn.com>
%% — Mattéo Delabre <bonjour@matteodelabre.me>
%%
%% This work is released under the CC0 1.0 Universal license. See the
%% accompanying LICENSE file for details. To the extent possible under
%% law, Rémi Cérès and Mattéo Delabre have waived all copyright and
%% related or neighboring rights to the Bruno Beamer theme. This work
%% is published from France.

% \mode<presentation>

\setbeamertemplate{sections/subsections in toc}[square]
\setbeamertemplate{itemize item}[square]
\setbeamertemplate{itemize subitem}[circle]

% Allow multi-slide frames
\setbeamertemplate{frametitle continuation}{}

% Horizontal separator
\setbeamertemplate{separator}{%
    \usebeamercolor{separator}%
    \textcolor{fg}{\rule{.7\textwidth}{.5pt}}%
}

% Separate authors with newlines
\renewcommand{\beamer@andtitle}{\\}

% Title page. If you want to add a background image on the title slide,
% use the \background command to set the path to that image. Otherwise,
% no image will be included
\RequirePackage{tikz}
\usetikzlibrary{fadings}
\newcommand{\background}[1]%
    {\newcommand{\bruno@background}{#1}}


%% \newcommand\insertcaption{}
\newcommand{\backgroundcaption}[2][white]{\def\bruno@caption{{#1}{#2}}}

\newcommand{\insertcaption}{\ifdefined\bruno@caption{\expandafter\textcolor\bruno@caption}\fi}

\tikzfading[
    name=title page picture fading,
    left color=transparent!0,
    right color=transparent!100,
]

\setbeamertemplate{title page}{%
    \begin{minipage}{.7\textwidth}
        \raggedright
        \usebeamerfont{title}
        \inserttitle

        \usebeamertemplate{separator}

        \usebeamerfont{author}
        \vspace{2em}
        \insertauthor

        \vspace{2em}
        \usebeamerfont{details}
        \insertinstitute

        \insertdate
    \end{minipage}

    % Include faded image on the right if defined
    \ifdefined\bruno@background
    \begin{tikzpicture}[remember picture, overlay]
        % Crop image to a trapezium on the right
        \clip (current page.south east)
           -- (current page.north east)
           -- ++(-5.1, 0)
           -- ++(-3, -\paperheight)
           -- cycle;

        % Include background image
        \node[
            anchor=south east,
            inner sep=0,
            outer sep=0
        ] at (current page.south east) {
            \includegraphics[height=\paperheight]
                {\bruno@background}
        };

        \node[anchor=south east,
            inner sep=0,
            outer sep=15pt,
        ] at (current page.south east){\insertcaption};

        % Add a slight shadow
        \fill[
            black, path fading=title page picture fading,
            fading angle=-15
        ]
            (current page.south east)
            rectangle
            ++(-10, \dimexpr\paperheight+1cm);

        
    \end{tikzpicture}
    \fi
}

\mode<all>
\AtBeginDocument{%
  \let\phi\varphi
  \let\theta\vartheta
  \let\setminus\smallsetminus
}


\title{Statisztika előadás}
\date{\lecturedate}
\background{images/young_Ronald_Fisher.jpg}
\backgroundcaption{Ronald Fisher (1913)}


% \ifdefvoid{\option}{
%   % presentation
% }{
%   % handout
%   \usepackage{pgfpages}
%   \pgfpagesuselayout{4 on 1}[a4paper, landscape, border shrink=5mm]
% }
\begin{document}

% \ifdefvoid{\option}{
  \frame{\titlepage}

  \begin{frame}{Tudnivalók}
    \begin{itemize}[<*>]
      \item Tematika, irodalomjegyzék az előadás Canvas oldalán.
      \item Számonkérés: szóbeli vizsga, 
      a vizsgaidőszak elején lesz egy írásbeli vizsga, amin legfeljebb közepest lehet szerezni.
      \item A slide-ok az előadás Canvas oldalán elérhetőek lesznek az óra előtt.
    \end{itemize}
  \end{frame} 
% }{}

\begin{frame}{Ismétlés valószínűségszámításból}

  \begin{itemize}
    \item Sűrűségfüggvény transzformációs formula:
    \begin{displaymath}
      f_{\phi(X)}(t)=f_X(\phi^{-1}(t))\abs*{\det J\phi^{-1}}(t)
    \end{displaymath}
    \item Konvolúciós formula: $X$, $Y$ függetlenek, létezik $f_X$, ekkor létezik $f_{X+Y}$ és 
    \begin{displaymath}
      f_{X+Y}(t)=\E{f_X(t-Y)}=\int f_X(t-y) Q_Y(dy)=\int f_X(t-y)f_Y(y)d y,\quad 
      \text{ha $\exists f_Y$}. % is van sűrűségfüggvénye.}
    \end{displaymath}
    \item Nagy számok törvénye: $X_1,X_2,\dots$ iid, $\E{X_1}$ létezik, ekkor
    \begin{displaymath}
      \frac1n\sum_{i=1}^n X_i\to \E{X_1},\quad \text{egy valószínűséggel.}
    \end{displaymath}
    \item Centrális határeloszlás tétel: $X_1,X_2,\dots$ iid, 
    $\E{X_1}=m\in\real$,  $\D^2{X_1}=\sigma^2\in(0,\infty)$, 
    ekkor 
    \begin{displaymath}
      \frac1{\sqrt{n}}\sum_{i=1}^n \frac{X_i-m}{\sigma}\dto N(0,1)
    \end{displaymath}
    vagy 
    \begin{displaymath}
      \frac1{\sqrt{n}}\sum_{i=1}^n (X_i-m)\dto N(0,\sigma^2),\quad \text{$\sigma^2=0$ esetén is!}
    \end{displaymath}
  \end{itemize}
\end{frame}

\begin{frame}{Statisztikai feladatok}

  ,,\textit{Következtetések levonása a megfigyelésekből.
  Összefüggések feltárása az adatokban.}''
  
  \begin{itemize}
    \item Pontbecslések
    
    pl. Közvélemény-kutatás, a válaszok alapján szeretnénk egy termék kedveltségét megállapítani.
    
    Általánosabban zajos megfigyeléseink (méréseink) vannak egy ismeretlen mennyiségre, 
    amit szeretnénk meghatározni.

    \item Hipotézisvizsgálat
    
    pl. Gyógyszerkutatás, ígéretesnek tűnő hatóanyagok vizsgálunk. A kísérlet eredményeiből 
    szeretnénk arra következtetni, vajon hatásos-e a gyógyszerjelölt.

    \item Konfidencia halmaz.
    
    pl. Közvélemény-kutatásnál nem csak egy adott termék népszerűsége érdekel minket, hanem egy olyan
    lehetőleg minél kisebb intervallum, ami a terméket ténylegesen kedvelők arányát nagy valószínűséggel tartalmazza.

    \item ,,Összefüggések keresése'', klasszifikáció, regresszió.
    
    Minden megfigyelés egy $(x_i,y_i)$ pár, ahol $y_i$ az $f(x_i)$ zajos megfigyelése. 
    $f$ ismeretlen, ezt szeretnénk meghatározni.

    pl. Közvélemény kutatásnál, azon kívül, hogy a terméket kedveli-e a megkérdezett, feljegyzünk egyéb adatokat is:
    életkor, jövedelmi helyzet, családi állapot, stb.  és az érdekel minket, hogy ezek 
    hogyan befolyásolják az adott termék kedveltségét.
    
  \end{itemize}
\end{frame}

\begin{frame}{Fogalmak}
  \begin{itemize}
    \item Minta: $X_1,X_2,\dots,X_n$ független azonos eloszlású megfigyelések.
    
    Általában nem kellene, hogy függetlenek és azonos eloszlásúak legyenek, 
    de csak ezt az esetet nézzük ezen az órán. 

    $n$ a mintaelemszám.
    \item Mintatér: minta lehetséges értékei. Ha skalár megfigyeléseink vannak, akkor $\real^n$ része.
    Jelölés $\mfX$. Ez játssza a lehetséges kimenetelek halmazát. 
    \item Statisztikai mező: $(\Omega,\cA,\cP)$ 
    
    Általában $\Omega=\mfX$.  
    
    $\cP=\set*{\P[\theta]}{\theta\in \Theta}$ valószínűségi mértékek családja $\cA$-n. 
    $\Theta$ neve paramétertér.
 
    Az a feltételezésünk, hogy a megfigyelések $\P^*\in\cP$-ből származnak.  

    \item Statisztika: mérhető függvény a mintatéren. Ha $X=(X_1,\dots,X_n)$ a minta, akkor 
    $S$-et és $S(X)$-et azonosítjuk. Ez egyébként is teljesül, ha $\Omega=\mfX$ és $X_i(x)=x_i$.
  \end{itemize}

\end{frame}

\begin{frame}{Példa}
  Közvélemény-kutatás.
  \begin{itemize}
    \item Minden megfigyelés vagy 0, vagy 1. Nulla ha nem kedveli a terméket, és 1 ha igen.
    \item Mintatér $\mfX=\smallset*{0,1}^n$. $n$ a megfigyelések száma.
    \item Ha minden megkérdezettet, egymástól függetlenül és találomra választjuk a teljes populációból, 
    akkor $X_1,\dots,X_n$ független $\theta$ paraméterű indikátorok, ahol $\theta\in[0,1]$ az ismeretlen paraméter.
    \item Paramétertér $\Theta=[0,1]$
    \item $\P[\theta]{X=x}=\prod_i \P[\theta]{X_i=x_i}=\prod_i \theta^{x_i}(1-\theta)^{1-x_i}$.
    \item $S(x)=\sum_i x_i$ a mintaösszeg, egy példa statisztikára. $S(X)=\sum_i X_i$. 
  \end{itemize}
\end{frame}



% \begin{frame}{Bayes szabály a valószínűségszámítás I előadásból}
%   \begin{proposition}
%     $(A_i)_{i\in I}$ teljes eseményrendszer. $B$ tetszőleges esemény, ekkor
%     \begin{displaymath}
%       \P{A_i|B}=\frac{\P{B|A_i}\P{A_i}}{\sum_j \P{B|A_j}\P{A_j}}
%     \end{displaymath}
%   \end{proposition}
%   \begin{itemize}
%     \item $\P[0]=\P$, $d\P[1]=\frac{\I[B]}{\P{B}}d\P$, azaz a Radon-Nikodym derivált $Z=\frac{\I[B]}\{\P{B}}$.
%     \item $Y=\sum_j j\I{A_j}$, ekkor $\F=\sigma(Y)$ választással $\E[0]{X|\F}=\sum_j \E[0]{X|A_j}\I[A_j]$  
%     \item $\P{A_i|B}=\E[1]{\I{}}$ 

%   \end{itemize}    
% \end{frame}
\begin{frame}{Dominált mértékcsalád}
  \begin{df}
    $\cP$ \textbf{dominált mértékcsalád} az $(\Omega,\cA)$ mérhető téren, 
    ha létezik olyan $\lambda$ $\sigma$-véges 
    mérték amire nézve mindegyik $\P\in\cP$ abszolút folytonos. 
    $\lambda$ neve \textbf{domináló mérték}.
  \end{df}
  \continue
  Emlékeztető.
  \begin{itemize}
    \item $\P\ll\lambda$ azt jelenti, hogy a $\lambda$ nullmértékű halmazok 
    $\P$ nullmértékűek is.
    \item $\P\ll\lambda$ szükséges ahhoz, hogy létezzen $Z$ ($\geq0$, mérhető), 
    amivel $\P{A}=\int_A Zd\lambda$ teljesül.
    
    A Radon-Nikodym tétel szerint elégséges is.
    \item Jelölés: $Z=\frac{d\P}{d\lambda}$ a $\P$ Radon-Nikodym deriváltja 
    $\lambda$-ra nézve.
    \item Mértékcsere integrálban:
    \begin{displaymath}
        \P\ll\lambda,\quad\implies\quad \int h d\P=\int h\frac{d\P}{d\lambda}d\lambda
    \end{displaymath}
    \item Lánc szabály. Ha $\P[1]\ll\P[0]\ll\lambda$, akkor
    \begin{displaymath}
      \frac{d\P[1]}{d\lambda}=\frac{d\P[1]}{d\P[0]}\frac{d\P[0]}{d\lambda}
    \end{displaymath} 
  \end{itemize}
\end{frame}

\begin{frame}{Dominált mértékcsalád, példák}
  \begin{df}<*>
    $\cP$ \textbf{dominált mértékcsalád} az $(\Omega,\cA)$ mérhető téren, 
    ha létezik olyan $\lambda$ $\sigma$-véges 
    mérték amire nézve mindegyik $\P\in\cP$ abszolút folytonos. 
    $\lambda$ neve \textbf{domináló mérték}.
  \end{df}
 
  \begin{itemize}
    \item Ha a mintaelemek közös eloszlása abszolút folytonos a Lebesgue mértékre nézve, akkor az $n$ elemű minta 
    együttes eloszlása abszolút folytonos az $n$ dimenziós Lebesgue mértékre nézve.

    \item Példa. $X_1,\dots,X_n$ normális minta ismeretlen $m$ várható értékkel és $\sigma^2>0$ szórásnégyzettel. 
    
    Ekkor $\mfX=\real^n$, $\Theta=\real\times(0,\infty)$, $\theta=(m,\sigma)$, 
    a domináló mérték $\lambda$ az $n$ dimenziós Lebesgue mérték, az eloszláscsalád tagjait 
    a sűrűségfüggvényükkel lehet megadni:
    \begin{displaymath}
      f_{\theta}(x)=\frac{1}{\sigma^n (2\pi)^{n/2}}\exp{-\frac1{2\sigma^2}\sum_i(x_i-m)^2}
    \end{displaymath}
  \end{itemize}
\end{frame}

\begin{frame}{Dominált mértékcsalád, példák}
  \begin{df}<*>
    $\cP$ \textbf{dominált mértékcsalád} az $(\Omega,\cA)$ mérhető téren, ha létezik olyan $\sigma$-véges 
    mérték amire nézve mindegyik $\P\in\cP$ abszolút folytonos.
  \end{df}
 
  \begin{itemize}
    \item Ha mintaelemek közös eloszlása diszkrét, akkor a mintatér megszámlálható és a mintaelemek együttes eloszlását a 
    $\P[\theta]{X=x}$ valószínűségekkel lehet leírni. Ez nem más mint a számlálómértékre vonatkozó Radon-Nikodym derivált. 

    \item Példa: Közvélemény-kutatás, azaz $n$ elemű indikátor minta. 
    
    Mintatér: $\mfX=\smallset*{0,1}^n$, paramétertér: $\Theta=[0,1]$, minta együttes sűrűségfüggvénye a számláló mértékre nézve:
    \begin{displaymath}
      f_\theta(x)=\P[\theta]{X=x}=\prod_i \theta^{x_i}(1-\theta)^{(1-x_i)},\quad\theta\in\Theta=[0,1]
    \end{displaymath}
  \end{itemize}

\end{frame}

\begin{frame}{Dominált mértékcsalád, példák}
  \begin{df}<*>
    $\cP$ \textbf{dominált mértékcsalád} az $(\Omega,\cA)$ mérhető téren, ha létezik olyan $\sigma$-véges 
    mérték amire nézve mindegyik $\P\in\cP$ abszolút folytonos.
  \end{df}
  Nem minden mértékcsalád dominált és nem minden dominált családra igaz, 
  hogy vagy a számlálómérték, vagy a Lebesgue mérték dominálja.
  
  \begin{itemize}
    \item Exponenciális eloszlású élettartamokat figyelünk meg ismeretlen $c$ értéknél levágva. Azaz 
    $\Theta=\set*{(\lambda,c)}{\lambda,c>0}$, $X_i$ eloszlása $\P[\theta]$ mellett azonos 
    $Z\wedge c$  eloszlásával, ahol $Z\sim\exp(\lambda)$.
    
    HF. Ez az eloszláscsalád nem dominált. 

    \item Előző példa, de $c=1$ ismert. Ekkor $\mfX=[0,1]^n$ és ha 
    $\mu=\Leb+\delta_1$, akkor $\lambda=\mu^{\otimes n}$ domináló mérték.
    A minta sűrűségfüggvénye:
    \begin{displaymath}
      f_{\theta}(x) = 
      \prod_i \zjel*{\theta e^{-\theta x}\I{0<x_i<1}+e^{-\theta}\I{x_i=1}},\quad\theta>0
    \end{displaymath}
    ugyanis 
    \begin{displaymath}
      \P[\theta]{X_i\in H}=
      \P[\theta]{Z\wedge 1\in H}=
      \P[\theta]{Z\in H\cap(0,1)}+\P[\theta]{Z\geq 1}\I{1\in H}
    \end{displaymath}
  \end{itemize}
\end{frame}

\begin{frame}{Elégséges statisztika, elégséges $\sigma$--algebra}

  $(\Omega,\cA,\cP=\set*{\P[\theta]}{\theta\in\Theta})$ statisztikai mező. $\Omega=\mfX$.  
  \begin{df} 
    $S:\mfX\to\real^d$ \textbf{elégséges statisztika}, ha minden $A\in\B(\mfX)$-re $\P[\theta]{X\in A|S}$-nek 
    létezik $\theta$-tól független, közös változata.

    $\cS\subset\cA$ \textbf{elégséges $\sigma$-algebra}, ha minden $A\in\B(\mfX)$-re $\P[\theta]{X\in A|\cS}$-nek 
    létezik $\theta$-tól független, közös változata.
  \end{df} 

  \continue
  Motiváció:
  \begin{itemize}
    \item Cél: A minta alapján $\P^*$ meghatározása, vagy közelítése.

    \item Mikor nem hordoz plusz információt a generáló eloszlásról 
    a minta az $S$ statisztikához képest?

    Ha a minta feltételes eloszlása az eloszláscsalád minden tagjára ugyanaz, 
    azaz $S$ elégséges.

    \item Ha $S$ elégséges, akkor elég $S(X)$-et megjegyezni a mintából.
  \end{itemize}
\end{frame}

\begin{frame}{Ismétlés: feltételes várható érték, feltételes valószínűség}
  \begin{df}
    $X\in L^1$ valószínűségi változó, $\F$ $\sigma$-algebra. $\E{X|\F}$ olyan $\F$ mérhető, integrálható változó, ami teljesíti 
    a parciális átlagolási tulajdonságot:
    \begin{displaymath}
      \E{X\I[A]}=\E{\E{X|\F}\I[A]},\quad\text{minden $A\in\F$-re}.
    \end{displaymath}
  \end{df}
  \begin{itemize}
    \item Ha $\F=\sigma(Y)$, akkor $\E{X|\F}=\E{X|Y}=g(Y)$ alakú, ahol $g$ Borel mérhető.
    \item Ez valójában egy Radon-Nikodym derivált, $\mu(A)=\E{X\I[A]}$ véges előjeles mérték $\F$-n, ami $\P|_{\F}$-re 
    nézve abszolút folytonos. $\E{X|\F}=\frac{d\mu}{d\P|_{\F}}$.
    \item Szemléletes jelentés $g(y)=\lim_{\eps\to0}\E{X|\abs*{Y-y}<\eps}$ ($Y$ eloszlása szerint majdnem minden $y$-ra).
    \item Ha $Y$ diszkét, $\P{Y=y}>0$, akkor $g(y)=\E{X|Y=y}$. Ezt a jelölést szokás használni, a $g$ függvényre, akkor is ha $Y$ nem diszkrét.
    \item Ha $X\in L^2$, akkor $\E{X|\F}\in L^2$ és $X-\E{X|\F}$ merőleges $L^2(\F)$-re.
    $\E{X|\F}$ nem más, mint $X$ vetülete $L^2(\F)$-re.
    \item $\P{A|\F}$ egy másik jelölés $\E{\I[A]|\F}$-re.
  \end{itemize}
\end{frame}

\begin{frame}{Feltételes eloszlás reguláris változata}
  \begin{df}
    $X$ valószínűségi változó, $\F$ $\sigma$-algebra. $Q:\B(\real)\times\Omega\to[0,1]$ az $X$ feltételes 
    eloszlásának reguláris változata, ha 
    \begin{itemize}[<*>]
      \item $\omega\mapsto Q(A,\omega)$, a $\P{X\in A|\F}$ egy változata.
      \item $A\mapsto Q(A,\omega)$ valószínűségi mérték $\B(\real)$-en, minden $\omega$-ra.  
    \end{itemize}
  \end{df}
  \continue
  Ha $X$ vektorváltozó, vagy  általánosabban $(S,\cS)$ értékű, akkor $\B(\real)$ $\cS$-sel helyettesítendő. 
  \begin{proposition}
    Ha $X$ feltételes eloszlásának létezik reguláris változata $Q$, $h(X)\in L^1$, akkor $\E{h(X)|\F}=\int h(x) Q(dx,.)$
  \end{proposition}
  
  \begin{theorem}[bizonyítás nélkül]
    Ha $X$ teljes szeparábilis metrikus térbe képez (pl. $\real$, $\real^n$), akkor tetszőleges feltétel 
    mellett létezik a feltételes eloszlás reguláris változata.
  \end{theorem}
  
\end{frame}

\begin{frame}{A rendezett minta mindig elégséges}
  \begin{df}
    $X_1,\dots,X_n$ minta, 
    $X_1^*=\min_i X_i$ a legkisebb megfigyelés,  
    $X_2^*$ a második legkisebb megfigyelés, \ldots, $X_n^*=\max_i X_i$ 
    a legnagyobb megfigyelés.
    $X^*=(X_1^*,\dots,X_n^*)$ neve \textbf{rendezett minta}. 
    %\continue
    Formulával:
    \begin{displaymath}
      X_k^*=\max\set*{\min\nolimits_{i\in I} X_i}{I\subset\smallset{1,\dots,n},\, \abs*{I}=n-(k-1)} 
    \end{displaymath}
  \end{df}
  \begin{proposition}
    $(X,X^*)\eqinlaw(\Pi X^*,X^*)$, ahol $\Pi$ az $\smallset*{1,\dots,n}$ indexhalmaz 
    $X$-től független véletlen permutációja ($\P{\Pi=\pi}=1/n!$ minden $\pi\in S_n$-re) és 
    \begin{displaymath}
      \pi X^*=(X^*_{\pi^{-1}(1)},\dots,X^*_{\pi^{-1}(n)}).
    \end{displaymath}
  \end{proposition}
  \continue
  Az állítás alapján
  \begin{displaymath}
    \P{X\in H|X^*}=\P{\Pi X^*\in H|X^*}=\P{\Pi x\in H}|_{x=X^*}
    =\frac1{n!}\left.\sum\nolimits_{\pi\in S_n} \I{\pi x\in H}\right|_{x=X^*}
  \end{displaymath}
  Általánosabban: $\E{T(X)|X^*}=\E{T(\Pi X^*)|X^*}=\frac1{n!}\sum_{\pi} T(\pi X^*)$.
\end{frame}

\begin{frame}{A rendezett minta mindig elégséges}
  $X=(X_1,\dots,X_n)$ minta, $X^*=(X_1^*\leq X_2^*\leq\cdots\leq X_n^*)$ a rendezett minta

  \begin{proposition}
    $(X,X^*)\eqinlaw(\Pi X^*,X^*)$, ahol $\Pi$ az $\smallset*{1,\dots,n}$ indexhalmaz 
    $X$-től független véletlen permutációja és 
    \begin{displaymath}
      \pi X^*=(X^*_{\pi^{-1}(1)},\dots,X^*_{\pi^{-1}(n)}).
    \end{displaymath}
  \end{proposition}
  
  \begin{itemize}
    \item $\Pi$ az $X$-től független véletlen permutáció, ekkor $X\eqinlaw \Pi X$
    \begin{displaymath}
      \P{\Pi X\in H}=\E{\P{\Pi X\in H|\Pi}}=\E{\P{\pi X\in H}|_{\pi=\Pi}}
    \end{displaymath} 
    de $\pi X\eqinlaw X$ minden rögzített permutációra és így $\P{\Pi X\in H|\Pi}=\P{X\in H}$.
    \item $(X,X^*)\eqinlaw (\Pi X,(\Pi X)^*)=(\Pi X,X^*)$.
    \item $X=\rho X^*$ alkalmas $\sigma(X)$ mérhető $\rho$ permutációval és 
    $\Pi X=(\Pi\rho) X^*$.
    \item $\Pi\rho$, $X$-től független véletlen permutáció, mert 
    \begin{displaymath}
      \P{\Pi\rho=\pi|X}=\P{\Pi=\pi\rho^{-1}|X}=\frac1{n!},
      \quad\text{minden $\pi\in S_n$-re}. 
    \end{displaymath}
    \item $(X,X^*)\eqinlaw(\Pi X,X^*)=(\Pi\rho X^*,X^*)\eqinlaw(\Pi X^*,X^*)$.
  \end{itemize}
\end{frame}


% \begin{frame}{Bayes szabály}
%   \begin{proposition}
%     Ha $\P[1]\ll \P[0]$, azaz $d\P[1]=Zd\P[0]$ valamely $Z$-vel, akkor 
%     \begin{displaymath}
%       \E[1]{X|\F}=\frac{\E[0]{XZ|\F}}{\E[0]{Z|\F}}\I{\E[0]{Z|\F}>0}
%     \end{displaymath}
%   \end{proposition}
%   \begin{itemize}
%     \item Jobb oldal $\F$ mérhető.
%     \item Kell:  jobboldal teljesíti a parciális átlagolási tulajdonságot.
%     \item  ${d\P[1]}|_\F=\E[0]{Z|\F}{d\P[0]}|_{\F}$
%     \item $A\in \F$
%     \begin{displaymath}
%       \E[1]{\frac{\E[0]{XZ|\F}}{\E[0]{Z|\F}}\I[A]}=
%       \E[0]{\E[0]{XZ|\F}\I[A]}=\E[0]{XZ\I[A]}=\E[1]{X\I[A]}.
%     \end{displaymath}
%   \end{itemize}
% \end{frame}

\begin{frame}{Feltétel elégségességre}
  \begin{proposition}
    $\cP$ dominált mértékcsalád, a $\lambda$ domináló mérték valószínűségi mérték.
    %$f_\theta=\frac{d\P[\theta]}{d\lambda}$, $\P[\theta]\in\cP$.
  
    Ha $\frac{d\P}{d\lambda}$ $\sigma(S)$ mérhető (pontosabban létezik $\sigma(S)$ mérhető változata) minden $\P\in\cP$-re,
    %$f_\theta$ $\sigma(S)$--mérhető, 
    akkor $S$ elégséges statisztika.
  \end{proposition}
  \continue 
  Megfordítás is igaz, bizonyítás később. 
  \pause
  Tetszőleges $\P\in\cP$-re $\frac{d\P}{d\lambda}=f(S)$ és
  \begin{align*}
    \P{X\in A,S\in H}&=\E[\lambda]{\I{X\in A}\I{S\in H}f(S)}\\
    &=\E[\lambda]{\E[\lambda]{\I{X\in A}|S}\I{S\in H}f(S)}
    =\E[\P]{\E[\lambda]{\I{X\in A}|S}\I{S\in H}}
  \end{align*}
  Azaz $\E[\lambda]{\I{X\in A}|S}$ a $\P{X\in A|S}$, $\P\in\cP$  feltételes valószínűségek közös változata.
  %\begin{itemize}
  % \item Feltehető, hogy a domináló mérték valószínűségi mérték, így $d\P[\theta]=f_{\theta}d\lambda$ és $\lambda$ valószínűségi mérték.
  %  \item 
  % Bayes szabály: 
  %     \begin{displaymath}
  %       \P[\theta]{X\in A|S} = \frac{\E[\lambda]{\I{X\in A}f_\theta(X)|S}} {\E[\lambda]{f_\theta(X)|S}}
  %       =\frac{f_\theta(X)\E[\lambda]{\I{X\in A}|S}} {f_\theta(X)}
  %       =\E[\lambda]{\I{X\in A}|S}
  %     \end{displaymath}
  %   azaz $\P[\theta]{X\in A|S}$-nek van $\theta$-tól független közös változata. 
  %   %\item \textbf{Gond:} $h(X)$ nem feltétlenül integrálható $\lambda$ szerint. 
    
    %A pontos érveléshez a domináló mértéket ügyesen kell választani (pl. $h\equiv 1$).
  %\end{itemize}
  \pause
    Valójában a Bayes szabályt használtuk a feltételes várható érték kiszámítására:
    \begin{displaymath}
      \E[\P]{Y|\F}=\frac{\E[\lambda]{Y\frac{d\P}{d\lambda}|\F}}{\E[\lambda]{\frac{dP}{d\lambda}|\F}}
    \end{displaymath}
  \pause

  Ha a $\lambda$ domináló mérték véges, akkor $\tlambda=\frac{1}{\lambda(\Omega)}\lambda$ valószínűségi mérték és az 
  %\begin{displaymath}
    $\frac{d\P}{d\tlambda}=\frac{d\P}{d\lambda}\cdot\frac{d\lambda}{\tlambda}=\lambda(\Omega)\frac{d\P}{d\lambda}$
  %\end{displaymath}
  pontosan akkor $S$ mérhető, ha $\frac{d\P}{d\lambda}$ az.  
\end{frame}

\begin{frame}{Példa, Indikátor minta}
  $X_1,\dots,X_n$, $n$ elemű indikátor minta ismeretlen $\theta\in[0,1]$ paraméterrel.
   \begin{itemize}
     \item $\mfX=\smallset*{0,1}^n$. Domináló mérték $\lambda$: a számláló mérték.
     \item 
      \begin{displaymath}
        f_\theta(x) = \P[\theta]{X=x} 
        = \prod_i \P[\theta]{X_i=x_i}
        =\prod_{i} \theta^{x_i}(1-\theta)^{1-x_i}
        =\theta^{\sum x_i}(1-\theta)^{n-\sum x_i}
     \end{displaymath}
     \item $\lambda$ véges mérték
      \begin{displaymath}
        f_\theta(x)=g_\theta(S(x)),\quad S(x)=\sum x_i,\quad g_\theta(s)=\theta^s(1-\theta)^{n-s},
      \end{displaymath}
      $S=\sum X_i$, azaz \textbf{mintaösszeg}, elégséges statisztika. 
   \end{itemize}
   \continue
   Hogyan adható meg, a $\P[\theta]{X\in A|S}$ feltételes eloszlás?
   \pause
   \begin{itemize}
     \item $S$ diszkrét változó. $S$ lehetséges értékei $0,1,\dots,n$, $S$ binomiális $n$ renddel és $\theta$ paraméterrel.
     \item 
      \begin{displaymath}
        \P[\theta]{X=x|S=s}=
        \frac{\P[\theta]{X=x,\,S=s}}{\P[\theta]{S=s}}=
        \begin{cases}
          0& \sum x_i\neq s\\
          \frac{\theta^s(1-\theta)^{n-s}}{\binom ns \theta^s(1-\theta)^{n-s}}=\frac1{\binom ns} & \sum x_i=s 
        \end{cases}
      \end{displaymath}
      \continue
      $X$ $S$-re vonatkozó feltételes eloszlása: 
        
      $n$ $S$ elemű részhalmazai közül választunk egyet találomra, legyen ez $I$,
      $j\in I$-re $Y_j=1$ egyébként nulla. 
      
      \pause
      $X$ feltételes eloszlása $S$-re nézve azonos $Y$ eloszlásával, azaz minden $\theta$-ra ugyanaz.
   \end{itemize}
\end{frame}

\begin{frame}{Példa,  $N(\theta,1)$ minta}
  $X_1,\dots,X_n$, $n$ elemű minta ismeretlen $\theta\in\real$ várható értékű és egységnyi szórású normális eloszlásból.
   \begin{itemize}
     \item $\mfX=\real^n$, domináló mérték $\lambda$ lehet a Lebesgue mérték, de lehet a $X$ eloszlása $\theta=0$ mellett.
     \item A sűrűségfüggvény a Lebesgue mértékre nézve
      \begin{displaymath}
        f_\theta(x) %= %\P[\theta]{X=x} 
        = \prod_i \frac{1}{\sqrt{2\pi}} e^{-\frac12(x_i-\theta)^2}
        =f_0(x) \prod_{i} e^{x_i\theta-\frac12\theta^2}
        =f_0(x) \exp{\theta\sum_i x_i-\frac {n}2\theta^2}
     \end{displaymath}
     \item 
      \begin{displaymath}
        \frac{d\P[\theta]}{d\P[0]}(x)=\frac{f_\theta(x)}{f_{0}(x)}=\exp{\theta S(x)-\frac {n}2\theta^2},\quad S(x)=\sum_i x_i
      \end{displaymath}
      $S=\sum X_i$, azaz \textbf{mintaösszeg}, elégséges statisztika. 
   \end{itemize}
   \continue
   Hogyan írható le a $\P[\theta]{X\in A|S}$ feltételes eloszlás?
   \pause
   \begin{itemize} 
      \item $\bar{X}=\frac1n\sum X_i$ a mintaátlag.
      \item $(X_1-\bar{X},\dots,X_n-\bar{X})$ eloszlása nem függ $\theta$-tól és később kiszámoljuk, 
      hogy független $S$-től így $e=(1,\dots,1)$ jelöléssel
      \begin{displaymath}
        \P[\theta]{X\in H|S}
        =\P[\theta]{X-\bar{X}e+xe \in H}|_{x=S/n}
        =\P[0]{X-\bar{X}e+xe\in H}|_{x=S/n}
      \end{displaymath}
   \end{itemize}
\end{frame}



\begin{frame}{Neymann faktorizációs tétel}

  \begin{theorem}
    $\cP=\set*{\P[\theta]}{\theta\in\Theta}$ dominált mértékcsalád az 
    $(\mfX,\B(\mfX))$ minta téren, $\lambda$ domináló mértékkel. 
    
    Az $S$  statisztika pontosan akkor elégséges, ha az 
    $f_\theta=\frac{d\P[\theta]}{d\lambda}$ 
    sűrűségek
    \begin{displaymath}
      f_{\theta}(x)=g_{\theta}(S(x))h(x)  \tag{\hbox{${*}$}}
    \end{displaymath}
    alakban is megadhatóak, 
    alkalmas $g_\theta,h$ mérhető függvényekkel.
  \end{theorem}
  \begin{itemize}
    \item Először $\lambda=\P[0]\in\cP$-ra ellenőrizzük, hogy $S$ pontosan 
    akkor elégséges, ha a $\frac{d\P[\theta]}{d\P[0]}=g_\theta\circ S$ alakú, azaz
    a sűrűségeknek van $S$ mérhető változata.
    \item $\cP'$ a $\cP$ konvex keverékeiből álló család  
    \begin{displaymath}
      \cP'=\set*{\sum\nolimits_i c_i\P[i]}{c_i\geq0,\,\sum\nolimits_i c_i=1,\,\P[i]\in\cP}  
    \end{displaymath}
    Megmutatjuk, hogy $S$ pontosan akkor elégséges $\cP$-re nézve, ha $\cP'$-re az, 
    továbbá létezik $\P[0]\in \cP'$ ami, $\cP'$-t dominálja (Halmos--Savage tétel) 
    és $(*)$ pontosan akkor teljesül $\cP$-re, ha $\cP'$-re. .
    % \item $\cP'$-re az elsőként vizsgált eset alkalmazható.
    \item Ha $\lambda$ tetszőleges domináló mérték, akkor 
    \begin{displaymath}
      \text{$S$ egs.}\implies\frac{d\P[\theta]}{d\lambda}
      =\frac{d\P[\theta]}{d\P[0]}\frac{d\P[0]}{d\lambda}
      =(g_\theta\circ S) h,
      \quad\text{$(*)$}\implies
      \frac{d\P[\theta]}{d\P[0]}=\frac{d\P[\theta]}{d\lambda}:\frac{d\P[0]}{d\lambda}
      =\frac{(g_\theta\circ S) h}{(g_0\circ S) h}
      =\tilde g_\theta\circ S\implies \text{$S$ egs.}.
    \end{displaymath}
  \end{itemize}
  
  % \begin{itemize}
  %   \item Tegyük fel, hogy $S$ elégséges.  Ekkor  tetszőleges $H\in\B(\mfX)$-re
  %   \begin{displaymath}
  %     \P[\theta]{X\in H}
  %     =\E[\theta]{\P[\theta]{X\in H|S}}
  %     =\E[\theta]{\P[0]{X\in H|S}}
  %     =\E[0]{\P[0]{X\in H|S}f_\theta(X)}.
  %     %=\E[0]{\I{X\in H}\E[0]{f_{\theta}(X)|S}}
  %   \end{displaymath}
  %   Itt $\E[0](f_{\theta}(X)|S)=g_{\theta}(S)$ és 
  %   \begin{displaymath}
  %     \P[\theta]{X\in H}=\E[0]{\I{X\in H}\E[0]{f_{\theta}(X)|S}}=\int_H g_\theta(S(x)) \P[0](d x)
  %   \end{displaymath}
  %   azaz $g_\theta$ $S$ mérhető változata a $\frac{d\P[\theta]}{d\P[0]}$ sűrűségfüggvénynek.
  % \end{itemize}

\end{frame}
\end{document}

\begin{frame}{Neymann faktorizációs tétel, $\lambda=\P[0]$}
  \begin{theorem}
    $\cP=\set*{\P[\theta]}{\theta\in\Theta}$ dominált mértékcsalád az 
    $(\mfX,\B(\mfX))$ minta téren, $\P[0]\in\cP$ domináló mértékkel. 
    
    Az $S$  statisztika pontosan akkor elégséges, ha az 
    $f_\theta=\frac{d\P[\theta]}{d\P[0]}$ 
    sűrűségeknek van $\sigma(S)$ mérhető változata.
  \end{theorem}
  \begin{itemize}
    \item $S$ elégséges, ekkor $\P[\theta]{X\in H|S}=\P[0]{X\in H|S}$.
    \item Teljes valószínűség tétellel
    \begin{displaymath}
      \P[\theta]{X\in H}
      =\E[\theta]{\P[\theta]{X\in H|S}}
      =\E[0]{\frac{d\P[\theta]|_{\sigma(S)}}{d\P[0]|_{\sigma(S)}}\E[0]{\I{X\in H}|S}}
      =\E[0]{\I{X\in H}\frac{d\P[\theta]|_{\sigma(S)}}{d\P[0]|_{\sigma(S)}}}
    \end{displaymath}
    \item Legyen $g_\theta(S)=\frac{d\P[\theta]|_{\sigma(S)}}{d\P[0]|_{\sigma(S)}}$
    \begin{displaymath}
      \P[\theta]{X\in H}=\E[0]{\I{X\in H}g_\theta(S)}=\int_H g_\theta(S)d\P[0],
    \end{displaymath}
    azaz $g_\theta\circ S$  a $\frac{d\P[\theta]}{d\P[0]}$ Radon-Nikodym derivált egy változata.
    \item A fordított irányt korábban leellenőriztük a Bayes szabály felhasználásával.
  \end{itemize}
\end{frame}

\begin{frame}{$\cP$ és $\cP'$}
  $\cP$ mértékcsalád a mintatéren
  \begin{displaymath}
    \cP'=\set*{\sum\nolimits_i c_i\P[i]}{c_i\geq0,\,\sum\nolimits_i c_i=1,\,\P[i]\in\cP}  
  \end{displaymath}
  \begin{proposition}
    $S$ pontosan akkor elégséges $\cP$-re nézve, ha $\cP'$-re elégséges.
  \end{proposition}
  \begin{itemize}
    \item Legyen $H$ rögzített és $q(S)=\P{X\in H|S}$ a $\cP$-re vonatkozó közös változat. 
    \item Tetszőleges $\P\in\cP'$-re $\P=\sum_{i } c_i\P[i]$ alakú és 
    \begin{displaymath}
      \P{X\in H, S\in A}
      =\sum_{i} c_i\int \P[i]{X\in H|S}\I{S\in A}d\P[i]
      =\sum_i c_i\int q(S) \I{S\in A}d\P[i]
      =\int q(S)\I{S\in A}d\P
    \end{displaymath} 
    \item $q(S)$ $S$ mérhető és teljesíti a parciális átlagolási tulajdonságot $q(S)=\P{X\in H|S}$
    minden $\P\in\cP'$-re.
    \item Másik irányhoz elég észrevenni, hogy ha $S$ elégséges egy mértékcsaládra, 
    akkor tetszőleges részcsaládra is az és $\cP\subset\cP'$.
  \end{itemize}
\end{frame}

\begin{frame}{$\cP$ és $\cP'$}
  $\cP$ %=\set*{\P[\theta]}{\theta\in \Theta}$ 
  mértékcsalád a mintatéren $\lambda$ domináló mértékkel.
  \begin{displaymath}
    \cP'=\set*{\sum\nolimits_i c_i\P[i]}{c_i\geq0,\,\sum\nolimits_i c_i=1,\,\P[i]\in\cP}  
  \end{displaymath}
  Ha $\P=\sum_i c_i\P[i]\in\cP'$, akkor
  \begin{displaymath}
    \frac{d\P}{d\lambda} = \sum_{i} c_i\frac{d\P[i]}{d\lambda}
  \end{displaymath}

  \begin{proposition}
    $\lambda $ domináló mérték $\cP$-hez, $S$ statisztika. Ha
    \begin{displaymath}
      \frac{d\P[\theta]}{d\lambda}=(g_{\theta}\circ S)h,\quad \text{minden $\P[\theta]\in\cP$-re},
    \end{displaymath} 
    akkor ugyanez igaz minden $\P\in\cP'$-re.
  \end{proposition}
\end{frame}

\begin{frame}{Halmos--Savage tétel}
  \begin{theorem}
    $\cP$ konvex keverésre zárt, dominált mértékcsalád $(\mfX,\B(\mfX))$-en. 
    
    Ekkor létezik $\P[0]\in\cP$, ami dominálja $\cP$-t.
  \end{theorem}
  \begin{itemize}
    \item $\lambda$ $\sigma$-véges domináló mérték. Feltehető, hogy véges mérték. Ugyanis,
     ha $\mfX=\cup_n A_n$ és $\lambda(A_n)<\infty$, akkor 
     $c_n>0$, $\sum_n c_n\lambda(A_n)<\infty$ esetén
     \begin{displaymath}
       \lambda'(H)=\sum_n c_n \lambda(A_n\cap H)=\int_H \sum_n c_n\I[A_n] d\lambda
     \end{displaymath}
     véges mérték és $\lambda$, $\lambda'$ ekvivalensek és $\lambda'$ véges.
     \item $\lambda$ véges domináló mérték. Legyen
     \begin{displaymath}
      c=\sup\set*{\lambda(f>0)}{f=\frac{d\P}{d\lambda},\,\P\in\cP}
     \end{displaymath}
     \item Elég megmutatni, hogy $c$ maximum, 
     azaz létezik $\P[0]\in\cP$, amire $c=\lambda(\frac{d\P[0]}{d\lambda}>0)$.     
  \end{itemize}
  
\end{frame}

\begin{frame}{Halmos--Savage tétel igazolása I.}
  \begin{lemma}
    $\cP$ konvex keverésre zárt, véges $\lambda$ mértékkel dominált mértékcsalád 
    $(\mfX,\B(\mfX))$-en. Ha $\P[0]\in\cP$-re
    \begin{displaymath}
      \lambda\zjel*{\frac{d\P[0]}{d\lambda}>0}=c=\sup\set*{\lambda(f>0)}{f=\frac{d\P}{d\lambda},\,\P\in\cP}
    \end{displaymath}
    Ekkor $\P[0]$ dominálja $\cP$-t.
  \end{lemma}
  \begin{itemize}
    \item Tetszőleges $\P\in\cP$-re $\P'=\frac12(\P+\P[0])\in\cP$ a konvex keverésre zártság miatt.
    \item $f=\frac{d\P}{d\lambda}$, $f_0=\frac{d\P[0]}{d\lambda}$, 
    ekkor $2\frac{d\P'}{d\lambda}=f+f_0$ és 
    \begin{displaymath}
      c\geq \lambda((f+f_0)>0)\geq \lambda(f_0>0)=c, 
      \quad\text{mert $\event*{f+f_0>0}=\event*{f>0}\cup\event*{f_0>0}$}.
    \end{displaymath}
    \item $\lambda(\event*{f>0}\setminus\event*{f_0>0})=0$ és 
    \begin{displaymath}
      \P{X\in H}
      =\int_H f d\lambda
      =\int_H f\I{f_0>0}d\lambda
      =\int_H \frac{f}{f_0}\I{f_0>0}f_0d\lambda
      =\int_H \frac{f}{f_0}\I{f_0>0}d\P[0]
    \end{displaymath} 
  \end{itemize}
\end{frame}

\begin{frame}{Halmos--Savage tétel igazolása II.}
  \begin{lemma}
    $\cP$ konvex keverésre zárt, véges $\lambda$ mértékkel dominált mértékcsalád 
    $(\mfX,\B(\mfX))$-en. Ekkor létezik $\P[0]\in\cP$, amire
    \begin{displaymath}
      \lambda\zjel*{\frac{d\P[0]}{d\lambda}>0}=c=\sup\set*{\lambda(f>0)}{f=\frac{d\P}{d\lambda},\,\P\in\cP}
    \end{displaymath}
  \end{lemma}
  \begin{itemize}
    \item Legyen $\P[n]\in\cP$ olyan, amire az $f_n=\frac{d\P[n]}{d\lambda}$ jelöléssel
    \begin{displaymath}
      \lambda(f_n>0)\geq c-\frac1n
    \end{displaymath}
    \item $\P[0]=\sum_{n=1}^{\infty} 2^{-n} \P[n]\in\cP$, ekkor 
    $f=\frac{d\P[0]}{d\lambda}=\sum_n 2^{-n}f_n$ és 
    \begin{displaymath}
      \event*{f>0}=\cup_n\event*{f_n>0},\quad c = \sup_n\lambda(f_n>0)\leq \lambda(f>0)\leq c.
    \end{displaymath}
  \end{itemize}
\end{frame}

\begin{frame}[<*>]{Neymann--faktorizációs tétel összefoglalás}
  \begin{theorem}
    $\cP=\set*{\P[\theta]}{\theta\in\Theta}$ dominált mértékcsalád az 
    $(\mfX,\B(\mfX))$ minta téren, $\lambda$ domináló mértékkel. 
    
    Az $S$  statisztika pontosan akkor elégséges, ha az 
    $f_\theta=\frac{d\P[\theta]}{d\lambda}$ 
    sűrűségek
    \begin{displaymath}
      f_{\theta}(x)=g_{\theta}(S(x))h(x)  %\tag{$*$}
    \end{displaymath}
    alakban is megadhatóak, 
    alkalmas $g_\theta,h$ mérhető függvényekkel.
  \end{theorem}
  \begin{itemize}
    \item Feltehető, hogy $\cP$ zárt a konvex keverésre.
    \item Ha $\cP$ zárt a konvex keverésre, akkor a domináló mérték választható $\P[0]\in\cP$-nek is.
  \end{itemize}
  \begin{lemma}
    $\P[0]\in\cP$ domináló mérték $\cP$-hez. Ekkor $S$ pontosan akkor elégséges, ha $\frac{d\P}{d\P[0]}$-nak 
    van $\sigma(S)$ mérhető változata minden $\P\in\cP$-re.
  \end{lemma}
\end{frame}

\end{document}
 