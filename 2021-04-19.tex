\def\option{}
\providecommand{\option}{handout}
\documentclass[aspectratio=169,notheorems,9pt,\option]{beamer}

\usepackage{marvosym}
%\usepackage[magyar]{babel}
\usepackage{polyglossia}
\setmainlanguage{magyar}

% \usepackage{pvdefs}
\usepackage{etoolbox,mathtools}

%\makeatletter
\def\ifbrace{\@ifnextchar{\bgroup}}

\@ifundefined{texexp}{%
  \let\texexp\exp
  \def\exp{%
    \texexp\ifbrace{\cbr*}{}%
  }
}\relax
 
\def\isparam#1{\ifbrace#1\relax}

\DeclarePairedDelimiter{\abs}{\lvert}{\rvert}
\DeclarePairedDelimiter{\zjel}{(}{)}
\DeclarePairedDelimiter\br\lbrack\rbrack
\DeclarePairedDelimiter\event\lbrace\rbrace
\DeclarePairedDelimiter{\norm}{\|}{\|}
\let\smallset\event
\let\cbr\event
\DeclarePairedDelimiterX\set[2]\{\}{%
  #1\nonscript\::\allowbreak\nonscript\:\mathopen{}#2%
}

\let\PEfont\mathbb
\providecommand\given{}

\def\redefgiven#1{%
  \renewcommand\given{%
    \nonscript\:%
    #1\vert%
    \allowbreak%
    \nonscript\:%
    \mathopen{}%
  }%
}
\DeclarePairedDelimiterX\PEzjel[1](){\redefgiven{\delimsize}#1}
\def\aftersuperscript#1^#2{%
  ^{#2}#1%
}
\newcommand{\PE}[1][]{%
  \PEfont{\Pe}%
  \ifblank{#1}{}{_{#1}}%
  \@ifnextchar{^}{%
    \aftersuperscript{\ifbrace{\PEzjel*}{}}%
    }{%
      \ifbrace{\PEzjel*}{}%
    }%
  %\ifbrace{\PEzjel*}{}%
}
\newcommand{\E}{\def\Pe{E}\PE}
\newcommand{\D}{\def\Pe{D}\PE}
\renewcommand{\P}{\def\Pe{P}\PE}

\def\cov{\operatorname{cov}}
\let\eps\varepsilon
\def\ito/{It\^o}
\let\Ito=\ito
\def\eqinlaw{\stackrel{d}{=}}

\def\strip#1{%
    \expandafter\stripi\detokenize{#1}\relax
}
\def\stripi#1#2\relax{%
    \def\stripi##1##2\relax{%
    \ifx##1#1\else\expandafter##1\fi##2%
    }%
}
\expandafter\stripi\detokenize{\ }\relax

\def\strip#1{%
    \expandafter\stripi\string#1\relax
}
\def\stripi#1#2\relax{%
    \def\stripi##1##2\relax{%
    \ifx##1#1\else\expandafter##1\fi##2%
    }%
}
\expandafter\stripi\string\ \relax

\DeclareListParser{\dolist}{}

\def\defprefix#1#2{
    \def\do##1{\csdef{#1\strip##1}{{#2{##1}}}}%
    \ifbrace\dolist\do
}

\defprefix{c}{\mathcal}{XAPIDLCSGTHEN}
\defprefix{b}{\bar}{hX}
\defprefix{t}{\tilde}{STQbCf\lambda \Omega \sigma }
\defprefix{}{\mathcal}{BF}
\defprefix{mf}{\mathfrak}{X}

\def\sign{\operatorname{sign}}
\def\real{\mathbb{R}}
\def\Z{\mathbb{Z}}
\def\Q{\mathbb{Q}}

\def\Isymb{{\mathbf 1}}
\def\substr#1{_{\zjel{#1}}}
\newcommand{\I}[1][]{%
  \Isymb\ifblank{#1}{\substr}{_{#1}}%
}

\def\Isymb{{\mathbb 1}}

\def\I{\@ifstar\I@star\I@}
\def\I@star{%
  %\message{I@star arg:|#1|^^J}%
  \Isymb\ifbrace{_}{}% } %{_{#1}}%
  }
\newcommand{\I@}[2][\zjel]{%
  \Isymb_{#1{#2}}% } %{_{#1}}%
}



\def\interior{\operatorname{int}}
\def\tg{\operatorname{tg}}
\DeclarePairedDelimiterX{\zfrac}[2](){\frac{#1}{#2}}
\newcommand\independent{\protect\mathpalette{\protect\independenT}{\perp}}
\def\independenT#1#2{\mathrel{\rlap{$#1#2$}\mkern2mu{#1#2}}}


\newtheorem{df}{Definíció}
\newtheorem{theorem}{Tétel}
\newtheorem{lemma}{Lemma}
\newtheorem{proposition}{Állítás}
\newtheorem{corollary}{Következmény}



\def\diag{\operatorname{diag}}
\def\Id{\mathbf{I}}
\let\PEfont\mathbb
\def\npto{\buildrel p\over \nrightarrow}
\def\ndto{\buildrel d\over \nrightarrow}
%\def\dto{\buildrel d\over \rightarrow}
\def\dto{\stackrel{d}{\to}}
\def\pto{\stackrel{p}{\to}}
\def\tr{\operatorname{Tr}}
\def\arctg{\operatorname{arctg}}
\let\nto\nrightarrow

\def\Leb{\operatorname{Leb}}


\def\readrest#1\enddate{}
\def\setdate@#1-#2-#3\relax{\def\lecturedate{#1.#2.#3.}\readrest}
\def\setdate{\expandafter\setdate@\jobname\relax-\relax-\relax\enddate}
\setdate

% \usetheme{Bruno}
\usecolortheme{Bruno}
% \usefonttheme{Bruno}
% \useinnertheme{Bruno}
\useoutertheme{Bruno}

\newrobustcmd{\continue}{%
  \onslide<\value{beamerpauses}->\relax
}


\beamerdefaultoverlayspecification{<+->}

\mode<presentation>

\RequirePackage[math-style=TeX, bold-style=upright]{unicode-math}
\setmathfont{Latin Modern Math}[Scale=1]
\setsansfont[Ligatures=TeX, ItalicFont={Fira Sans Italic}, BoldFont={Fira Sans SemiBold}, Scale=0.9]{Fira Sans Book}
\usepackage[mathcal]{euscript}

% Title page
\setbeamerfont{title}{
    size=\LARGE,
    shape=\bfseries
}

\setbeamerfont{subtitle}{
    size=\large,
    shape=\normalfont
}

\setbeamerfont{author}{
    size=\normalsize,
    shape=\normalfont
}

\setbeamerfont{details}{
    size=\footnotesize,
    shape=\normalfont
}

% Slide title
\setbeamerfont{frametitle}{size=\LARGE}
\setbeamerfont{framesubtitle}{
    size=\normalsize,
    %shape=\normalfont\scshape
}

% Blocks
\setbeamerfont{block title}{
    size=\normalfont,
    shape=\strut
}

\setbeamerfont{blockdef}{
    size=\large,
    shape=\bfseries
}

% Description
\setbeamerfont{description item}{shape=\bfseries}


% Footer information line with title, author and slide number
\setbeamertemplate{footline}{%
    \leavevmode%
    \hbox{%
        \usebeamerfont{footline}%
        \begin{beamercolorbox}[
            wd=\textwidth,
            ht=3ex,
            dp=1.25ex
        ]{footline}%
            \hspace{1cm}%
            \insertshorttitle%
            \hspace{2em}%
            \insertdate%
            \hfill%
            \insertframenumber{} / \inserttotalframenumber%
            \hspace{1cm}
        \end{beamercolorbox}%
    }
    \vskip0pt%
}

%% Bruno Beamer theme — Version 1.0
%% A lightweight Beamer theme inspired from the Metropolis theme
%%
%% Written in 2017-2019 by:
%% — Rémi Cérès <remiceres@msn.com>
%% — Mattéo Delabre <bonjour@matteodelabre.me>
%%
%% This work is released under the CC0 1.0 Universal license. See the
%% accompanying LICENSE file for details. To the extent possible under
%% law, Rémi Cérès and Mattéo Delabre have waived all copyright and
%% related or neighboring rights to the Bruno Beamer theme. This work
%% is published from France.

% \mode<presentation>

\setbeamertemplate{sections/subsections in toc}[square]
\setbeamertemplate{itemize item}[square]
\setbeamertemplate{itemize subitem}[circle]

% Allow multi-slide frames
\setbeamertemplate{frametitle continuation}{}

% Horizontal separator
\setbeamertemplate{separator}{%
    \usebeamercolor{separator}%
    \textcolor{fg}{\rule{.7\textwidth}{.5pt}}%
}

% Separate authors with newlines
\renewcommand{\beamer@andtitle}{\\}

% Title page. If you want to add a background image on the title slide,
% use the \background command to set the path to that image. Otherwise,
% no image will be included
\RequirePackage{tikz}
\usetikzlibrary{fadings}

\newcommand{\background}[1]%
    {\newcommand{\bruno@background}{#1}}


%% \newcommand\insertcaption{}
\newcommand{\backgroundcaption}[2][white]{\def\bruno@caption{{#1}{#2}}}

\newcommand{\insertcaption}{%
    \ifdefined
    \bruno@caption{%
        \expandafter\textcolor\bruno@caption
    }%
    \fi
}


\tikzfading[
    name=title page picture fading,
    left color=transparent!0,
    right color=transparent!100,
]

\setbeamertemplate{title page}{%
    \begin{minipage}{.7\textwidth}
        \raggedright
        \usebeamerfont{title}
        \inserttitle

        \usebeamertemplate{separator}

        \usebeamerfont{author}
        \vspace{2em}
        \insertauthor

        \vspace{2em}
        \usebeamerfont{details}
        \insertinstitute

        \insertdate
    \end{minipage}

    % Include faded image on the right if defined
    \ifdefined\bruno@background
    \begin{tikzpicture}[remember picture, overlay]
        % Crop image to a trapezium on the right
        \clip (current page.south east)
           -- (current page.north east)
           -- ++(-5.1, 0)
           -- ++(-3, -\paperheight)
           -- cycle;

        % Include background image
        \node[
            anchor=south east,
            inner sep=0,
            outer sep=0
        ] at (current page.south east) {
            \includegraphics[height=\paperheight]
                {\bruno@background}
        };

        \node[anchor=south east,
            inner sep=0,
            outer sep=15pt,
        ] at (current page.south east){\insertcaption};

        % Add a slight shadow
        \fill[
            black, path fading=title page picture fading,
            fading angle=-15
        ]
            (current page.south east)
            rectangle
            ++(-10, \dimexpr\paperheight+1cm);

        
    \end{tikzpicture}
    \fi
}

\mode<all>
\AtBeginDocument{%
  \let\phi\varphi
  \let\theta\vartheta
  \let\setminus\smallsetminus
}

\title{Statisztika előadás}
\date{\lecturedate}
\background{images/young-Ronald-Fisher.jpg}
\backgroundcaption{Ronald Fisher (1913)}


\begin{document}

\maketitle

\begin{frame}[<*>]{Többdimenziós normális eloszlás, emlékeztető}
  \begin{itemize}
  \item $X$ egy dimenziós normális változó, ha $X\eqinlaw \sigma
    Z+\mu$, ahol $Z\sim N (0,1)$ és $\sigma\geq0$, $\mu\in\real$.
  \item Elfajult eloszlás ($\sigma=0$) is normális.
  \item Karakterisztikus függvény
    \begin{displaymath}
      \phi_{Z} (t)= \E{e^{itZ}}=e^{-\frac12 t^2},\quad
      \phi_{\sigma Z+\mu} (t)=
      \E{e^{it (\sigma Z+\mu)}}=
      e^{it\mu}\phi_{Z} (t\sigma)=e^{it\mu-\frac12t^2\sigma^2}.
    \end{displaymath}
  % \item $X$ $d$-dimenziós vektorváltozó karakterisztikus függvénye
  %   \begin{displaymath}\textstyle
  %     \phi_X:\real^d\to\C,\quad\phi_X (t)=\E{\exp{i\sum_j t_j
  %         X_j}}=\E{e^{i t\cdot X}}
  %   \end{displaymath}
  %   Vektorváltozóra is van inverziós formula, azaz $X$
  %   karakterisztikus függvénye  $X$ eloszlását meghatározza.
  % \item Következmény. $X$ eloszlását a lineáris kombinációk $\sum_j
  %   a_jX_j$ ($a\in\real^d$) eloszlásai meghatározzák.
    
  \item $X$ $d$--dimenziós normális vektorváltozó, ha minden
    $a\in\real^d$-re $\sum_j a_jX_j$ egy dimenziós normális.
  \item Ha $Z_1,\dots,Z_d$ független standard normálisok, akkor $Z=
  (Z_1,\dots,Z_d)$. Ekkor $Z$ $d$--dimenziós  standard normális vektorváltozó.
  \item $X$ $d$--dimenziós normális $\mu$ várható érték 
  vektorral és $\Sigma$ kovariancia mátrixszal. Ekkor $X\eqinlaw AZ+\mu$, 
  ahol $Z$ $d$-dimenziós standard normális és $AA^T=\Sigma$.
  \end{itemize}
\end{frame}

% \begin{frame}{Többdimenziós normális eloszlás II.}

%   \begin{itemize}
%   \item Példa. Ha $Z_1,\dots,Z_d$ független standard normálisok, akkor $Z=
%     (Z_1,\dots,Z_d)$. Ekkor $Z$ normális vektorváltozó.
%     \begin{displaymath}
%       \phi_Z (t)=\E{e^{it\cdot Z}}=\E{\prod e^{i t_jZ_j}}=\prod
%       e^{-\frac12 t_j^2}=e^{-\frac12\norm*{t}^2}
%     \end{displaymath}
%     Ha $a\in\real^d$, akkor
%     \begin{displaymath}
%       \phi_{a\cdot Z} (t)=\E{e^{it a\cdot Z}}=\phi_{Z} (ta)=e^{-\frac12t^2\norm*{a}^2}
%     \end{displaymath}
%     Azaz $Z$ $d$ dimenziós normális.
%     $Z$-t \textbf{$d$--dimenziós standard normális} változónak
%     hívjuk.
%   \item
%     Ha $Z$ $d$ dimenziós standard normális változó és
%     $A\in\real^{k\times d}$, $m\in\real^k$, akkor $X=AZ+m$ $k$-dimenziós
%     normális, hiszen $a\in\real^k$-ra
%     \begin{displaymath}
%       a\cdot X= (A^Ta)\cdot Z+a\cdot m\quad\text{egy dimenziós normális}
%     \end{displaymath}    
%   \item $Z$ $d$-dimenziós normális $X=AZ+m$ várható érték vektora $m$,
%     kovariancia mátrixa $AA^T$.
%   \item Tetszőleges pozitív szemidefinit $\Sigma$  mátrix
%     faktorizálható $AA^T$ alakban, sőt $A$ alsó háromszög mátrixnak is
%     választható  (Cholesky  felbontás).
%   \end{itemize}
% \end{frame}

% \begin{frame}{Többdimenziós normális eloszlás III.}
%   \begin{itemize}
%     \item Ha $X= (X_1,\dots,X_d)$ $d$--dimenziós normális vektor, akkor
%     minden $j$-re $X_j$ egy dimenziós normális, vagyis $X_j\in L^2$.
%   \item Jelölje $m$, $\Sigma$ az $X$ várható érték vektorát, és
%     kovariancia mátrixát. Ekkor $t\in\real^d$ esetén $Y=\sum_j
%     t_jX_j\sim N (t\cdot m,t^T\Sigma t)$
%     \begin{displaymath}
%       %\E{Y}=\sum_j t_j m_j=t\cdot m,\quad D^2 (Y)=t^T\Sigma t,\quad
%       \phi_X(t)=\E{e^{i t\cdot X}}=\phi_{t\cdot X} (1)=e^{i t\cdot m-\frac12t^T\Sigma t}.
%     \end{displaymath}
%   \item Többdimenziós normális eloszlást a várható érték vektor és a
%     kovariancia mátrix meghatározza.

%     Jelölés $N (\mu,\Sigma)$.
%   \end{itemize}
% \end{frame}

\begin{frame}{Többdimenziós normális sűrűségfüggvény}
  \begin{itemize}
  \item $Z$ $d$--dimenziós standard normális.
    \begin{displaymath}
      f_Z (t)=\prod f_{Z_j} (t_j)=\prod \frac{1}{\sqrt{2\pi}}
      \exp{-\frac12 t_j^2}= (2\pi)^{-n/2} \exp{-\frac12\norm*{t}^2}
    \end{displaymath}
  \item Ha $A$ invertálható, $m\in\real^d$, $X=AZ+m$, akkor
    \begin{displaymath}
      f_X (t)=f_Z (A^{-1} (t-m))\abs{\det A^{-1}}
      =(2\pi)^{-n/2}\exp{-\frac12\norm*{A^{-1} (t-m)}^2}\abs{\det A^{-1}}
    \end{displaymath}
    $X\sim N (m,\Sigma=AA^T)$
    \begin{displaymath}
      f_X (t)=
      \frac{1}{(2\pi)^{d/2}(\det\Sigma)^{1/2}}\exp{-\frac12 (t-m)^T\Sigma^{-1} (t-m)}
    \end{displaymath}
  \item Ha $U^T=U^{-1}$, vagyis $U$ ortonormált mátrix, akkor $UZ$ is
    $d$-dimenziós standard normális.
  \item Ha $\Sigma$ nem invertálható, azaz nem teljes rangú, akkor az
    eloszlás nem abszolút folytonos a $d$--dimenziós Lebesgue mértékre
    nézve.

    Ha $a\perp \im\Sigma$, akkor $\D^2{a^TX}=a^T\Sigma a=0$ és
    $a^TX=a^Tm$ egy valószínűséggel.

    Megmutatható, hogy $\P{X\in\im\Sigma+m}=1$.
  \end{itemize}
\end{frame}

\begin{frame}{Többdimenziós normális, függetlenség és korrelálatlanság}
  \begin{proposition}
    $X=(X_1,X_2)^T$ normális vektor változó $X_1$ $d_1$, $X_2$ $d_2$
    dimenziós.

    $\cov (X_1,X_2)=0$ pontosan akkor, ha $X_1\independent X_2$.
  \end{proposition}
  \begin{itemize}
  \item Ha $X_1\independent X_2$, akkor $\cov (X_1,X_2)=0$. Elég a
    másik iránnyal foglalkozni
  \item $X$ kovariancia mátrixa $\Sigma$.
    \begin{displaymath}
      \Sigma=
      \begin{pmatrix}
        \Sigma_{11}&0\\0&\Sigma_{22}
      \end{pmatrix},
      \quad \Sigma_{11}=A_1A_1^T,\quad \Sigma_{22}=A_2A_2^T,\quad
      A=
      \begin{pmatrix}
        A_{1}&0\\0&A_{2}
      \end{pmatrix},
      \quad
      \Sigma=AA^T
    \end{displaymath}
   
  \item Ha $Z=(Z_1,Z_2)^T$ standard normális, akkor
    \begin{displaymath}
      X\eqinlaw AZ+\mu=\
      \begin{pmatrix}
        A_1Z_1+\mu_1\\
        A_2Z_2+\mu_2
      \end{pmatrix}
    \end{displaymath}
  \item $A_1Z_1+\mu_1\independent A_2Z_2+\mu_2$ azaz $X_1\independent X_2$.
  \end{itemize}
  \continue
  Ha $X=(X_1,\dots,X_r)$ normális vektor változó, $X_i$ 
  $d_i$ dimenziós és a keresztkovarianciák eltűnnek, akkor 
  $X_1,\dots,X_r$ független vektorváltozók.
\end{frame}

% \begin{frame}{Példa}
%   \begin{itemize}
%   \item $Z$ standard normális, $\xi\independent Z$ és
%     $\P{\xi=\pm1}=\frac12$.
%   \item $Z_1=Z$, $Z_2=\xi Z$. $Z_1$, $Z_2$ korrelálatlan standard
%     normálisok, de $Z_1$, $Z_2$ nem független. (HF.)
%   \item $\cov (X_1,X_2)=0\iff X_1\independent X_2$ állításban
%     $X=(X_1,X_2)^T$ normalitása fontos. $X_1$ normális, $X_2$ normális
%     nem elég.
%   \end{itemize}
% \end{frame}

\begin{frame}{Következmények}
  \begin{proposition}
    $X=(X_1,X_2)^T$ normális vektor változó $X_1$ $d_1$, $X_2$ $d_2$
    dimenziós. $\cov (X_1,X_2)=0$ pontosan akkor, ha $X_1\independent X_2$.
  \end{proposition}
  Következmények:
  \begin{itemize}
  \item $X\sim N(\mu,\Sigma)$, ekkor $AX+a\independent BX+b$ pontosan
    akkor ha $A\Sigma B^T=0$.
  \item $X= (X_1,X_2)^T$, $X\sim N (\mu,\Sigma)$,
    ekkor $X_1-\Sigma_{12}\Sigma_{22}^{-1}X_2\independent X_2$, hiszen
    \begin{displaymath}
      \cov (X_1 -\Sigma_{12}\Sigma_{22}^{-1}X_2,X_2)=0
    \end{displaymath}
  \item
    \begin{displaymath}
      \P{X_1\in H|X_2}=
      \left.\P{X_1-\Sigma_{12}\Sigma_{22}^{-1}X_2+y\in H}\right|_{y=\Sigma_{12}\Sigma_{22}^{-1}X_2}
    \end{displaymath}
    Azaz $X_1$ feltételes eloszlása $X_2$ feltétel mellett normális,
    $N (\mu_{1|2},\Sigma_{11|2})$, ahol
    \begin{align*}
      \mu_{1|2}&=\E{X_1-\Sigma_{12}\Sigma_{22}^{-1}X_2}+\Sigma_{12}\Sigma_{22}^{-1}X_2=
      \mu_1+\Sigma_{12}\Sigma_{22}^{-1}(X_2-\mu_2)
      \\
      \Sigma_{11|2}&=\cov (X_1-\Sigma_{12}\Sigma_{22}^{-1}X_2,X_1-\Sigma_{12}\Sigma_{22}^{-1}X_2)=\Sigma_{11}-\Sigma_{12}\Sigma_{22}^{-1}\Sigma_{21}
    \end{align*}
  \end{itemize}
\end{frame}

\begin{frame}{$\chi^2$ eloszlás, vetületek eloszlása}
  \begin{df}
    $Z$ $n$-dimenziós standard normális. $\norm{Z}^2=\sum_jZ_j^2$
    eloszlását $n$ szabadságfokú khi-négyzet eloszlásnak
    nevezzük. Jelölés $\chi^2_n$.
  \end{df}
   $\chi^2_n=\Gamma_{\frac n2,\frac12}$.
   \begin{proposition}
     $Z$ $n$-dimenziós standard normális,
     $\real^n=\cL_1\oplus\cdots\oplus\cL_r$, $\cL_i$-k merőleges
     alterek, $P_i$ a merőleges vetítés $\cL_i$-re.

     Ekkor $X^{(i)}=P_iZ\sim N (0,P_i)$, $X^{(1)},\dots,X^{(r)}$
     függetlenek és $\norm*{X^{(i)}}^2\sim\chi^2_{\dim\cL_i}$
   \end{proposition}
   \continue
   \begin{displaymath}
     \cov (X^{(i)},X^{(j)})=P_i\cov (Z,Z)P_j^T=P_iP_j=P_i\delta_{i,j}
   \end{displaymath}

   Legyen $d=\dim\cL_i$ és $e_1,\dots,e_d$ ortonormált bázis
   $\cL_i$-ben.

   Ekkor $X^{(i)}=\sum_k (e_k\cdot Z)e_k$.

   $(e_1\cdot Z,\dots,e_d\cdot Z)$  független $N (0,1)$-ek.

   $\norm*{X^{(i)}}^2=\sum_k (e_k\cdot Z)^2\sim\chi^2_{d}$.
  
\end{frame}

\begin{frame}{Fisher-Bartlett tétel} 
  \begin{theorem}
    $X_1,\dots,X_n$ független $N (\mu,\sigma^2)$ változók.
    \begin{displaymath}
      \bar{X}=\frac1n\sum_i X_i,\quad {s^*_n}^2=\frac1{n-1}\sum_i (X_i-\bar{X})^2
    \end{displaymath}
    $\bar{X}$ a mintaátlag, ${s^*_n}^2$ a korrigált tapasztalati
    szórásnégyzet.

    Ekkor $\bar{X}\sim N (\mu,\frac{\sigma^2}n)$, $
    {s^*_n}^2\sim\frac{\sigma^2}{n-1}\chi^2_{n-1}$ és
    $\bar{X}\independent {s^*_n}^2$.
  \end{theorem}
  \begin{itemize}
  \item
    Feltehető $\sigma>0$. $Z_i=\frac{X_i-\mu}{\sigma}$, $Z_i$-k
    független $N (0,1)$-ek, $\bar{X}=\sigma\bar{Z}+\mu$,
    ${s^*_n}^2 (X)=\sigma^2{s^*_n}^2 (Z)$

    
  \item Elég $\mu=0$, $\sigma^2=1$-re ellenőrizni.
  
    
  \item Legyen $e= \frac1{\sqrt n}(1,\dots,1)$, $\cL_1=\real e$,
    $\cL_2=\cL_1^\perp$, $P_1,P_2$ a projekciók.
  \item $P_1Z$ és $P_2Z$ függetlenek.
  \item $P_1Z= (e\cdot Z) e = (\bar{Z},\dots,\bar{Z})^T$.
  \item $\norm*{P_2Z}^2=\sum_j (Z_j-\bar{Z})^2= (n-1){s^*_n}^2 (Z)\sim\chi^2_{n-1}$
  \end{itemize}
  \continue
  A függetlenség Basu tételével is megkapható. HF. Hogyan? 
\end{frame} 

\begin{frame}{Fisher-Bartlett tétel megfordítása I.} 
  \begin{theorem}
    $n\geq 2$, $X_1,\dots,X_n$ független, azonos eloszlásúak.

    Ha $\xi=\sum_kX_k$ és $\eta={s^*_n}^2 (X)=\frac1{n-1}\sum_k (X_k-\bar{X})^2$
    független, akkor az $X_i$-k normális eloszlásúak.
  \end{theorem}
  \begin{itemize}
  \item Feltesszük, hogy az $X$-ek szórásnégyzete $\sigma^2$ véges.
  \item Feltehető, $\E{X_k}=0$, $\sigma^2=1$. Legyen  $\phi (t)=\E{e^{it X_1}}$.
  \item Egyfelől $\E{\eta}=\E{(s_n^*)^2}=\sigma^2=1$ miatt
    \begin{displaymath}
      \E{\eta e^{it\xi}}=\E{\eta}\E{e^{it\xi}}=\phi^n (t)
    \end{displaymath}
     
  \item Másfelől, ha $e= (1,\dots,1)^T$,
    \begin{align*}
      \eta&=\frac1{n-1}\norm*{X-\bar{X}e}^2=\frac1{n-1}\zjel{\norm*{X}^2-(\bar{X})^2\norm*{e}^2}\\
          &=\frac1{n-1}\zjel{\sum_k X_k^2-\frac{1}n\sum_k
            X_k^2-\frac{2}{n}\sum_{k<\ell} X_kX_\ell}=\frac1n\sum_kX_k^2-\frac{2}{n (n-1)}\sum_{k<\ell}X_kX_\ell
    \end{align*}
  \end{itemize}
\end{frame}

\begin{frame}{Fisher-Bartlett tétel megfordítása II.} 
  \begin{theorem}
    $n\geq 2$, $X_1,\dots,X_n$ független, azonos eloszlásúak.

    Ha $\xi=\sum_kX_k$ és $\eta={s^*_n}^2 (X)=\frac1{n-1}\sum_k (X_k-\bar{X})^2$
    független, akkor az $X_i$-k normális eloszlásúak.
  \end{theorem}
  \begin{itemize}
  \item $\E{X_k}=0$, $\E{X_k^2}=1$, $\phi (t)=\E{e^{it X_1}}$.
  \item Ekkor $k\neq\ell$ mellett
    \begin{align*}
      \E{X_ke^{it\xi}}&=\E{X_ke^{it X_k}}
                        \E{e^{it\sum_{\ell\neq k}X_\ell}}
                        = (-i)\phi'(t)\phi^{n-1}(t)\\
      \E{X^2_ke^{it\xi}}&=\E{X^2_ke^{it X_k}}
                          \E{e^{it\sum_{\ell\neq k}X_\ell}}=
                          -\phi''(t)\phi^{n-1}(t)\\
      \E{X_kX_\ell e^{it\xi}}&=\E{X_ke^{it X_k}}\E{X_\ell e^{itX_\ell}}\E{e^{it\sum_{r\neq
                        k,\ell}X_r}}=- (\phi'(t))^2\phi^{n-2}(t)      
    \end{align*}
  \item Láttuk, hogy
    \begin{displaymath}
      \phi^n (t)=\E{\eta e^{it\xi}}
      =\frac1n\sum_k\E{X_k^2e^{it\xi}}
      -\frac2{n (n-1)}\sum_{k<\ell}\E{X_kX_\ell e^{it\xi}}
      =-\phi'' (t)\phi^{n-1} (t)- (\phi' (t))^2\phi^{n-2} (t)
    \end{displaymath}
     
  \item Átosztás után,
    \begin{displaymath}
      -1=\frac{\phi''}{\phi}+\zfrac*{\phi'}{\phi}^2=(\ln\phi)''
      \quad\implies\quad \ln\phi (t)=-\tfrac12t^2
    \end{displaymath}
  \end{itemize}
\end{frame}

\begin{frame}{$t$ eloszlás}

  \begin{itemize}
  \item $X_i=\sigma Z_i+\mu$, ($i=1,\dots,n$) $N (\mu,\sigma^2)$
    eloszlású minta. $H_0:\mu=\mu_0$ esetén
    \begin{displaymath}
      T=\sqrt{n}\frac{\bar{X}-\mu_0}{s^*_n}=\sqrt{n}\frac{\sigma \bar{Z}}{\sigma\sqrt{\frac1{n-1}\sum_i (Z_i-\bar{Z})^2}}=\frac{\sqrt{n}\bar{Z}}{\sqrt{\frac1{n-1}\sum_i (Z_i-\bar{Z})^2}}
    \end{displaymath}
  \item A Fisher-Bartlett tétel alapján
    \begin{displaymath}
      \sum Z_i\independent\sum (Z_i-\bar{Z})^2\quad\text{és}\quad \sum
      Z_i\sim N (0,n),\quad \sum (Z_i-\bar{Z})^2\sim \chi^2_{n-1}
    \end{displaymath}
  \item Azaz $H_0$ mellett
    \begin{displaymath}
      T=\sqrt{n}\frac{\bar{X}-\mu_0}{s^*_n}\eqinlaw\frac{Z_1}{\sqrt{\frac{1}{n-1}\sum_{i=2}^n Z_i^2}}
    \end{displaymath}
    Ezt hívják $t_{n-1}$ eloszlásnak ($n-1$ szabadságfokú $t$
    eloszlás).
  \end{itemize}
\end{frame}

\begin{frame}{$t$ eloszlás}
  \begin{df}
    $Z_0,\dots,Z_n\sim N (0,1)$ függetlenek.
    \begin{displaymath}
      \frac{Z_0}{\sqrt{\frac1n\sum_{i=1}^n Z_i^2}}
    \end{displaymath}
    eloszlását $t_n$ eloszlásnak ($n$ szabadságfokú $t$ eloszlás) hívjuk.
  \end{df}
  \begin{itemize}
  \item $t_1$ azonos a Cauchy eloszlással, ugyanis nem más, mint
    \begin{displaymath}
      \frac{Z_1}{\abs{Z_2}}=\frac{R\sin\phi}{R\abs{\cos\phi}}\eqinlaw
      \frac{\sin\phi}{\cos\phi}=\tg\phi,\quad\text{ahol
        $R^2=Z_1^2+Z_2^2\sim\exp (1/2)$ és $\phi=\arg (Z_2,Z_1)\sim U (0,2\pi)$}
    \end{displaymath}
  \item Általános $n$-re (sűrűségfüggvény transzformációs formulával)
    a $t_n$ eloszlás sűrűségfüggvénye (HF.)
    \begin{displaymath}
      \frac{1}{\sqrt{\pi}}\frac{\Gamma\zfrac{n+1}2}{\Gamma\zfrac n2} \zjel{1+\tfrac{x^2}n}^{-\frac{n+1}2}
    \end{displaymath}
  \item $n\to\infty$ esetén $t_n\wto N (0,1)$, ugyanis a nevező 1-hez tart.
  \end{itemize}
  
\end{frame}

\begin{frame}{Klasszikus próbák. $u$-próba}
  \begin{itemize}
  \item $X_1,\dots,X_n$ minta $N (\mu,\sigma^2)$ eloszlásból, $\sigma$
    ismert, $\mu$ ismeretlen.
  \item Esetek:

    $H_0:\mu=\mu_0$, $H_1:\mu<\mu_0$ egyoldali ellenhipotézis

    $H_0:\mu=\mu_0$, $H_1:\mu>\mu_0$ egyoldali ellenhipotézis
    
    $H_0:\mu=\mu_0$, $H_1:\mu\neq\mu_0$ kétoldali ellenhipotézis
 
  \item Próba statisztika:
    \begin{displaymath}
      T (X)=\sqrt{n}\frac{\bar{X}-\mu_0}{\sigma}
    \end{displaymath}
  \item $T$ eloszlása $H_0$ mellett $N (0,1)$
  \item $u_\alpha=\Phi^{-1} (1-\alpha)$, azaz
    $\P{Z>u_\alpha}=\alpha$, ha $Z\sim N (0,1)$
  \item $\alpha$ terjedelmű próba. Kritikus tartomány:

    $H_1:\mu<\mu_0$ esetén $T<-u_\alpha$

    $H_1:\mu>\mu_0$ esetén $T>u_\alpha$

    $H_1:\mu\neq \mu_0$ esetén $\abs*{T}>u_{\alpha/2}$

  \end{itemize}
\end{frame}

\begin{frame}{Klasszikus próbák. $t$-próba}
  \begin{itemize}
  \item $X_1,\dots,X_n$ minta $N (\mu,\sigma^2)$ eloszlásból, $\mu$, $\sigma$ ismeretlen.
  \item Esetek:

    $H_0:\mu=\mu_0$, $H_1:\mu<\mu_0$ egyoldali ellenhipotézis

    $H_0:\mu=\mu_0$, $H_1:\mu>\mu_0$ egyoldali ellenhipotézis
    
    $H_0:\mu=\mu_0$, $H_1:\mu\neq\mu_0$ kétoldali ellenhipotézis
 
  \item Próba statisztika:
    \begin{displaymath}
      T (X)=\sqrt{n}\frac{\bar{X}-\mu_0}{s_n^*},\quad\text{ahol}\quad
      {s_n^*}^2=\frac{1}{n-1}\sum_{i} (X_i-\bar{X})^2\quad\text{a
        korrigált tapasztalati szórásnégyzet}
    \end{displaymath}
  \item $T$ eloszlása $H_0$ mellett $t_{n-1}$ Student--féle $t$
    eloszlás $n-1$ szabadságfokkal.
  \item $t_\alpha$ kritikus érték, az $1-\alpha$ szinthez tartozó kvantilis, azaz
    $\P{T<t_\alpha}=1-\alpha$, ha $T\sim t_{n-1}$
  \item $\alpha$ terjedelmű próba. Kritikus tartomány:

    $H_1:\mu<\mu_0$ esetén $T<-t_\alpha$

    $H_1:\mu>\mu_0$ esetén $T>t_\alpha$

    $H_1:\mu\neq \mu_0$ esetén $\abs*{T}>t_{\alpha/2}$

  \end{itemize}
\end{frame}


\begin{frame}{Kétmintás $u$--próba}
  \begin{itemize}
  \item $X_1,\dots,X_n\sim N (\mu_1,\sigma_1^2)$, $Y_1,\dots, Y_m\sim N
    (\mu_2,\sigma_2^2)$ független minták.
  \item $\sigma_1,\sigma_2$
    ismert, $\mu_1,\mu_2$ ismeretlen.
  \item Esetek:

    $H_0:\mu_1=\mu_2$, $H_1:\mu_1<\mu_2$  (vagy   $H_1:\mu_1>\mu_2$) egyoldali ellenhipotézis

   
    $H_0:\mu_1=\mu_2$, $H_1:\mu_1\neq\mu_2$ kétoldali ellenhipotézis
 
  \item Próba statisztika:
    \begin{displaymath}
      T (X,Y)=\frac{\bar{X}-\bar{Y}}{\sqrt{\frac{\sigma_1^2}{n}+\frac{\sigma_2^2}{m}}}
    \end{displaymath}
  \item $T$ eloszlása $H_0$ mellett $N (0,1)$
  \item $u_\alpha=\Phi^{-1} (1-\alpha)$, azaz
    $\P{Z>u_\alpha}=\alpha$, ha $Z\sim N (0,1)$
  \item $\alpha$ terjedelmű próba. Kritikus tartomány:

    $H_1:\mu_1<\mu_2$ esetén $T<-u_\alpha$

    $H_1:\mu_1>\mu_2$ esetén $T>u_\alpha$ 

    $H_1:\mu_1\neq \mu_2$ esetén $\abs*{T}>u_{\alpha/2}$
  \item Ezek a próbák egyenletesen
    legerősebbek  a torzítatlan próbák között.
  \end{itemize}
 
\end{frame}

\begin{frame}{Kétmintás $t$--próba}
  \begin{itemize}
  \item $X_1,\dots,X_n\sim N (\mu_1,\sigma^2)$,
    $Y_1,\dots, Y_m\sim N(\mu_2,\sigma^2)$ független minták. $\sigma,\mu_1,\mu_2$
    ismeretlen. \textbf{A két minta szórása azonos!}
  \item Esetek:

    $H_0:\mu_1=\mu_2$, $H_1:\mu_1<\mu_2$ (vagy $H_1:\mu_1>\mu_2$) egyoldali ellenhipotézis

    $H_0:\mu_1=\mu_2$, $H_1:\mu_1\neq\mu_2$ kétoldali ellenhipotézis
 
  \item Próba statisztika:
    \begin{displaymath}
      T (X,Y)=\frac1{\sqrt{\frac1n+\frac1m}}\cdot\frac{\bar{X}-\bar{Y}}{\sqrt{\frac{(n-1)(s_X^*)^2+ (m-1)(s_Y^*)^2}{n+m-2}}}
    \end{displaymath}
    vagy más alakban
    \begin{displaymath}
      T(X,Y)=\frac{\bar{X}-\bar{Y}}{\hat{\sigma}},\quad\text{ahol}\quad
      \hat\sigma^2=\zjel{\tfrac{1}{n}+\tfrac{1}m}\cdot\frac{\sum
        (X_i-\bar{X})^2+\sum (Y_i-\bar{Y})^2}{n+m-2}
    \end{displaymath}
  \item $T$ eloszlása $H_0$ mellett $t_{n+m-2}$
  \item  $t_\alpha$ kritikus érték, az $1-\alpha$ szinthez tartozó kvantilis, azaz
    $\P{T<t_\alpha}=1-\alpha$, ha $T\sim t_{n+m-2}$
  \item $\alpha$ terjedelmű próba. Kritikus tartomány:

    $H_1:\mu_1>\mu_2$ esetén $T>t_\alpha$ ($H_1:\mu_1<\mu_2$ esetén $T<-t_\alpha$)

    $H_1:\mu_1\neq \mu_2$ esetén $\abs*{T}>t_{\alpha/2}$
  \item Ezek a próbák egyenletesen
    legerősebbek  a torzítatlan próbák között.
  \end{itemize}
\end{frame}

\begin{frame}{$F$--próba a szórások egyezésére}
  \begin{itemize}
  \item $X_1,\dots,X_n\sim N (\mu_1,\sigma_1^2)$, $Y_1,\dots, Y_m\sim N
    (\mu_2,\sigma_2^2)$ minta. $\sigma_1\sigma_2,\mu_1,\mu_2$
    ismeretlen. 
  \item Esetek:

    $H_0:\sigma^2_1=\sigma^2_2$, $H_1:\sigma^2_1>\sigma^2_2$  egyoldali ellenhipotézis

    $H_0:\sigma^2_1=\sigma^2_2$, $H_1:\sigma^2_1\neq\sigma^2_2$ kétoldali ellenhipotézis
    
  \item Próba statisztika:
    \begin{displaymath}
      T = T(X,Y)=\frac {{s_X^*}^2}{{s_Y^*}^2}
      %{\sqrt{\frac1n+\frac1m}}\cdot\frac{\bar{X}-\bar{Y}}{\frac{(n-1)(s_X^*)^2+ (m-1)(s_Y^2)^2}{n+m-2}}
    \end{displaymath}
  \item $T$ eloszlása $H_0$ mellett
    $F_{a,b}=\frac{\chi^2_{a}/a}{\chi^2_{b}/b}$, $a=n-1$,
    $b=m-1$. $F_{a,b}$ neve $a,b$ szabadságfokú %khi--négyzet
    F eloszlás. 
  \item  Az $f_\alpha (a,b)$ kritikus érték, az $1-\alpha$ szinthez tartozó kvantilis, azaz
    $\P{T<f_\alpha}=1-\alpha$, ha $T\sim F_{a,b}$
  \item $\alpha$ terjedelmű próba. Kritikus tartomány:

    $H_1:\sigma^2_1>\sigma^2_2$ esetén $T> f_\alpha (a,b)$, 

    $H_1:\sigma^2_1\neq \sigma^2_2$ esetén $T\notin (1/f_{\alpha/2} (b,a),f_{\alpha/2} (a,b))$
  \item $a=b$ esetén a próba egyenletesen
    legerősebb  a torzítatlan próbák között. 
    
    Ha $a\neq b$, akkor a megadott elutasítási 
    tartomány nem a legerősebb próbát adja, de ezt szokás használni.
  \end{itemize}
\end{frame}

\begin{frame}{$F$ eloszlás}
  \begin{itemize}
  \item Legyen $X,Y$ független, $X\sim \chi^2_a=\Gamma_{a/2,1/2}$,
    $Y\sim\chi^2_b=\Gamma_{b/2,1/2}$.
    
  \item $F_{a,b}$ nem más, mint $(X/a)/(Y/b)$ eloszlása, vagyis
    ha $Z\sim F_{a,b}$, akkor
    \begin{displaymath}
      \frac{b}{b+aZ}=\frac{Y}{X+Y}\sim\beta (\tfrac b2,\tfrac a2),
    \end{displaymath}
    
  \item Ha $\beta\sim\beta (\tfrac b2,\tfrac a2)$, akkor
    \begin{displaymath}
      \frac b a\zjel*{\frac1\beta-1}\sim F_{a,b}
    \end{displaymath}
  \item Ha $b\to\infty$, akkor $aF_{a,b}\wto \chi^2_a$ 
  \end{itemize}
\end{frame}

\begin{frame}{Diszkrét illeszkedés vizsgálat, $\chi^2$-próbák}
  \begin{itemize}
  \item $X_1,\dots,X_n$ $n$ elemű minta.
    $X_k\in\smallset{1,\dots,r}$. 
  \item $\P[\pi]{X_k=i}=\pi_i$, $\pi\in \Delta=\set{\pi}{\pi_i\geq
      0,\sum\pi_i=1}$.
  \item $p\in \Delta$,
    $H_0:\pi=p$, $H_1:\pi\neq p$.

  \item $f_\pi(x)=\prod_{k=1}^n \P{X_k=x_k}=\prod_{k=1}^n \pi_{x_k}=\prod_{i=1}^{r} \pi_i^{\sum_{k}\I{X_k=i}}$.
  
  \item Jelölje $N_i=\sum_k \I{X_k=i}$, $i=1,\dots,r$ az egyes értékek
    gyakoriságát. Maximum likelihood becslés $\pi$-re: $\ell(\pi)=\sum N_i\ln \pi_i$.

    Feltételes szélsőérték ($\Delta$-n maximalizálunk):
    $\ell'(\hat\pi)\perp \Delta$, azaz $\hat\pi_i=N_i/n$ a relatív gyakoriság.
  \item 
    A próba statisztika
    \begin{displaymath}
      T_n=T_n (X)=\sum_{i=1}^r
      \frac{(N_i-np_i)^2}{np_i}=n\sum\frac{(\hat \pi_i-p_i)^2}{p_i}
    \end{displaymath}
  \item $n\to\infty$ esetén $H_0$ mellett $T_n\dto \chi^2_{r-1}$, míg
    $H_1$ mellett $T_n\to\infty$. (később belátjuk.)
    
  \item Kritikus tartomány. $T_n>c_\alpha$, ahol $c_\alpha$ a
    $\chi^2_{r-1}$ eloszlás kritikus értéke.
  \item Nem egzakt próba, mert $T_n$ $H_0$ alatti eloszlásának
    közelítéséből számoljuk a kritikus értéket.
  \end{itemize}
\end{frame}

\begin{frame}{Speciális eset $r=2$}
  
  \begin{itemize}
    \item $r=2$ esetén $p_2=1-p_1$, $N_2=n-N_1$, így
    \begin{align*}
      T_n&=\frac{(N_1-np_1)^2}{np_1}+\frac{(N_2-np_2)^2}{np_2}
      \\
         &=
           \frac{(N_1-np_1)^2}n \zjel*{\frac1{p_1}+\frac1{1-p_1}}
      \\
         &
           =\zfrac*{N_1-np_1}{\sqrt{np_1(1-p_1)}}^2,\\
    \end{align*}
    \item  
    \begin{displaymath}
      \frac{N_1-np_1}{\sqrt{np_1 (1-p_1)}}\dto N (0,1)
      \quad\implies\quad
      T_n\dto \chi^2_1\quad\text{ha $n\to\infty$}
    \end{displaymath}
    \item $r=2$ esetén lényegében az $u$-próbát kapjuk vissza,
    $H_0$ mellett, nagy $n$-re $N_1=\sum_{k=1}^n\I{X_k=1}$ közelítőleg normális eloszlású 
    $np_1$ várható értékkel és $n(p_1(1-p_1))$ szórásnégyzettel.

    A $H_0: \pi_1=p_1$ nullhipotézist a $H_1: \pi_1\neq p_1$ ellenhipotézis
    ellenében elutasítjuk, ha
    \begin{displaymath}
      \frac{N_1-np_1}{\sqrt{np_1 (1-p_1)}}>u_{\alpha/2},\quad
      \text{másképp írva}\quad 
      T_n>u^2_{\alpha/2}
    \end{displaymath}
  \end{itemize}
\end{frame}


\begin{frame}{$T_n$ határeloszlása I.}
  \begin{itemize}
  \item  Legyen $Y_{k}=(\I{X_k=1},\dots,\I{X_k=r})$, ekkor
    $Y_1,\dots,Y_n$  független azonos eloszlású változók.
  \item $\E[\pi]{Y_k}=\pi$ és
    $\cov_\pi
    (\I{X_k=i},\I{X_k=j})=\P[\pi]{X_k=i,\,X_k=j}-\pi_i\pi_j=\delta_{i,j}\pi_i-\pi_i\pi_j$.
  \item A centrális határeloszlás tételből $H_0$, azaz $\pi=p$  mellett 
    \begin{displaymath}
      S_n=\frac{\sum_{k=1}^n Y_k-p} {\sqrt{n}}\to N (0,\Sigma (Y)),
      \quad\text{ahol}\quad
      \sum_{k=1}^n Y_k-p=(N_1-n p_1,\dots,N_r-n p_r)
    \end{displaymath}
    Így $T_n=\sum_{i=1}^r\zfrac{N_i-np_i}{\sqrt{np_i}}^2$ nem más mint
    \begin{displaymath}
      T_n=\norm{D S_n}^2,
    \quad\text{ahol}\quad
    D =\diag(p_1^{-1/2},\dots,p_r^{-1/2}),\quad
    D S_n\dto N (0,D\Sigma (Y)D)
  \end{displaymath}
  Itt
  \begin{displaymath}
    (D\Sigma (Y)D)_{i,j}=\frac{\delta_{i,j}p_i-p_ip_j}{\sqrt{p_ip_j}}=\delta_{i,j}-\sqrt{p_i}\sqrt{p_j}
  \end
  {displaymath}
  Ha $e= (p_1^{1/2},\dots,p_r^{1/2})$, akkor $\norm{e}_2=\sum_i p_i=1$
  \begin{displaymath}
    (D\Sigma (Y)D)=I-e e^T
  \end{displaymath}
  \end{itemize}
\end{frame}

\begin{frame}{$T_n$ határeloszlása II.}
  \begin{itemize}
  \item  Legyen $Y_{k}=(\I{X_k=1},\dots,\I{X_k=r})$, ekkor
    $Y_1,\dots,Y_n$  független azonos eloszlású változók.
  \item A centrális határeloszlás tételből $H_0$, azaz $\pi=p$  mellett 
    \begin{displaymath}
      D S_n=\zjel*{\frac{N_1-np_1}{\sqrt{np_1}},\dots,\frac{N_r-np_r}{\sqrt{np_r}}}
      % =\frac{\sum_{k=1}^n Y_k-p} {\sqrt{n}}
      \dto N (0,I-e e^T),
      \quad\text{ahol}\quad e=(p_1^{1/2},\dots,p_r^{1/2})
    \end{displaymath}
  \item $I-ee^T$ merőleges vetítés $\real e$ ortokomplementerére, így
    $N (0,I-ee^T)$ nem más mint $(I-ee^T)Z$ eloszlása, ahol $Z$ $r$
    dimenziós standard normális. 
  \item Innen
    \begin{displaymath}
      T_n=\norm*{D S_n}^2\dto \norm{(I-ee^T)Z}^2\sim\chi^2_{r-1}
    \end{displaymath}
    Emlékeztető. Ha $\xi_n\dto\xi$, és $h$ folytonos, akkor $h(\xi_n)\dto h(\xi)$.
  \end{itemize}
\end{frame}
\end{document}
