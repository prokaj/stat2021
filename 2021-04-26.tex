%\def\option{}
\providecommand{\option}{handout}
\documentclass[aspectratio=169,notheorems,9pt,\option]{beamer}

\usepackage{marvosym}
%\usepackage[magyar]{babel}
\usepackage{polyglossia}
\setmainlanguage{magyar}

% \usepackage{pvdefs}
\usepackage{etoolbox,mathtools}

%\makeatletter
\def\ifbrace{\@ifnextchar{\bgroup}}

\@ifundefined{texexp}{%
  \let\texexp\exp
  \def\exp{%
    \texexp\ifbrace{\cbr*}{}%
  }
}\relax
 
\def\isparam#1{\ifbrace#1\relax}

\DeclarePairedDelimiter{\abs}{\lvert}{\rvert}
\DeclarePairedDelimiter{\zjel}{(}{)}
\DeclarePairedDelimiter\br\lbrack\rbrack
\DeclarePairedDelimiter\event\lbrace\rbrace
\DeclarePairedDelimiter{\norm}{\|}{\|}
\let\smallset\event
\let\cbr\event
\DeclarePairedDelimiterX\set[2]\{\}{%
  #1\nonscript\::\allowbreak\nonscript\:\mathopen{}#2%
}

\let\PEfont\mathbb
\providecommand\given{}

\def\redefgiven#1{%
  \renewcommand\given{%
    \nonscript\:%
    #1\vert%
    \allowbreak%
    \nonscript\:%
    \mathopen{}%
  }%
}
\DeclarePairedDelimiterX\PEzjel[1](){\redefgiven{\delimsize}#1}
\def\aftersuperscript#1^#2{%
  ^{#2}#1%
}
\newcommand{\PE}[1][]{%
  \PEfont{\Pe}%
  \ifblank{#1}{}{_{#1}}%
  \@ifnextchar{^}{%
    \aftersuperscript{\ifbrace{\PEzjel*}{}}%
    }{%
      \ifbrace{\PEzjel*}{}%
    }%
  %\ifbrace{\PEzjel*}{}%
}
\newcommand{\E}{\def\Pe{E}\PE}
\newcommand{\D}{\def\Pe{D}\PE}
\renewcommand{\P}{\def\Pe{P}\PE}

\def\cov{\operatorname{cov}}
\let\eps\varepsilon
\def\ito/{It\^o}
\let\Ito=\ito
\def\eqinlaw{\stackrel{d}{=}}

\def\strip#1{%
    \expandafter\stripi\detokenize{#1}\relax
}
\def\stripi#1#2\relax{%
    \def\stripi##1##2\relax{%
    \ifx##1#1\else\expandafter##1\fi##2%
    }%
}
\expandafter\stripi\detokenize{\ }\relax

\def\strip#1{%
    \expandafter\stripi\string#1\relax
}
\def\stripi#1#2\relax{%
    \def\stripi##1##2\relax{%
    \ifx##1#1\else\expandafter##1\fi##2%
    }%
}
\expandafter\stripi\string\ \relax

\DeclareListParser{\dolist}{}

\def\defprefix#1#2{
    \def\do##1{\csdef{#1\strip##1}{{#2{##1}}}}%
    \ifbrace\dolist\do
}

\defprefix{c}{\mathcal}{XAPIDLCSGTHEN}
\defprefix{b}{\bar}{hX}
\defprefix{t}{\tilde}{STQbCf\lambda \Omega \sigma }
\defprefix{}{\mathcal}{BF}
\defprefix{mf}{\mathfrak}{X}

\def\sign{\operatorname{sign}}
\def\real{\mathbb{R}}
\def\Z{\mathbb{Z}}
\def\Q{\mathbb{Q}}

\def\Isymb{{\mathbf 1}}
\def\substr#1{_{\zjel{#1}}}
\newcommand{\I}[1][]{%
  \Isymb\ifblank{#1}{\substr}{_{#1}}%
}

\def\Isymb{{\mathbb 1}}

\def\I{\@ifstar\I@star\I@}
\def\I@star{%
  %\message{I@star arg:|#1|^^J}%
  \Isymb\ifbrace{_}{}% } %{_{#1}}%
  }
\newcommand{\I@}[2][\zjel]{%
  \Isymb_{#1{#2}}% } %{_{#1}}%
}



\def\interior{\operatorname{int}}
\def\tg{\operatorname{tg}}
\DeclarePairedDelimiterX{\zfrac}[2](){\frac{#1}{#2}}
\newcommand\independent{\protect\mathpalette{\protect\independenT}{\perp}}
\def\independenT#1#2{\mathrel{\rlap{$#1#2$}\mkern2mu{#1#2}}}


\newtheorem{df}{Definíció}
\newtheorem{theorem}{Tétel}
\newtheorem{lemma}{Lemma}
\newtheorem{proposition}{Állítás}
\newtheorem{corollary}{Következmény}



\def\diag{\operatorname{diag}}
\def\Id{\mathbf{I}}
\let\PEfont\mathbb
\def\npto{\buildrel p\over \nrightarrow}
\def\ndto{\buildrel d\over \nrightarrow}
%\def\dto{\buildrel d\over \rightarrow}
\def\dto{\stackrel{d}{\to}}
\def\pto{\stackrel{p}{\to}}
\def\tr{\operatorname{Tr}}
\def\arctg{\operatorname{arctg}}
\let\nto\nrightarrow

\def\Leb{\operatorname{Leb}}


\def\readrest#1\enddate{}
\def\setdate@#1-#2-#3\relax{\def\lecturedate{#1.#2.#3.}\readrest}
\def\setdate{\expandafter\setdate@\jobname\relax-\relax-\relax\enddate}
\setdate

% \usetheme{Bruno}
\usecolortheme{Bruno}
% \usefonttheme{Bruno}
% \useinnertheme{Bruno}
\useoutertheme{Bruno}

\newrobustcmd{\continue}{%
  \onslide<\value{beamerpauses}->\relax
}


\beamerdefaultoverlayspecification{<+->}

\mode<presentation>

\RequirePackage[math-style=TeX, bold-style=upright]{unicode-math}
\setmathfont{Latin Modern Math}[Scale=1]
\setsansfont[Ligatures=TeX, ItalicFont={Fira Sans Italic}, BoldFont={Fira Sans SemiBold}, Scale=0.9]{Fira Sans Book}
\usepackage[mathcal]{euscript}

% Title page
\setbeamerfont{title}{
    size=\LARGE,
    shape=\bfseries
}

\setbeamerfont{subtitle}{
    size=\large,
    shape=\normalfont
}

\setbeamerfont{author}{
    size=\normalsize,
    shape=\normalfont
}

\setbeamerfont{details}{
    size=\footnotesize,
    shape=\normalfont
}

% Slide title
\setbeamerfont{frametitle}{size=\LARGE}
\setbeamerfont{framesubtitle}{
    size=\normalsize,
    %shape=\normalfont\scshape
}

% Blocks
\setbeamerfont{block title}{
    size=\normalfont,
    shape=\strut
}

\setbeamerfont{blockdef}{
    size=\large,
    shape=\bfseries
}

% Description
\setbeamerfont{description item}{shape=\bfseries}


% Footer information line with title, author and slide number
\setbeamertemplate{footline}{%
    \leavevmode%
    \hbox{%
        \usebeamerfont{footline}%
        \begin{beamercolorbox}[
            wd=\textwidth,
            ht=3ex,
            dp=1.25ex
        ]{footline}%
            \hspace{1cm}%
            \insertshorttitle%
            \hspace{2em}%
            \insertdate%
            \hfill%
            \insertframenumber{} / \inserttotalframenumber%
            \hspace{1cm}
        \end{beamercolorbox}%
    }
    \vskip0pt%
}

%% Bruno Beamer theme — Version 1.0
%% A lightweight Beamer theme inspired from the Metropolis theme
%%
%% Written in 2017-2019 by:
%% — Rémi Cérès <remiceres@msn.com>
%% — Mattéo Delabre <bonjour@matteodelabre.me>
%%
%% This work is released under the CC0 1.0 Universal license. See the
%% accompanying LICENSE file for details. To the extent possible under
%% law, Rémi Cérès and Mattéo Delabre have waived all copyright and
%% related or neighboring rights to the Bruno Beamer theme. This work
%% is published from France.

% \mode<presentation>

\setbeamertemplate{sections/subsections in toc}[square]
\setbeamertemplate{itemize item}[square]
\setbeamertemplate{itemize subitem}[circle]

% Allow multi-slide frames
\setbeamertemplate{frametitle continuation}{}

% Horizontal separator
\setbeamertemplate{separator}{%
    \usebeamercolor{separator}%
    \textcolor{fg}{\rule{.7\textwidth}{.5pt}}%
}

% Separate authors with newlines
\renewcommand{\beamer@andtitle}{\\}

% Title page. If you want to add a background image on the title slide,
% use the \background command to set the path to that image. Otherwise,
% no image will be included
\RequirePackage{tikz}
\usetikzlibrary{fadings}

\newcommand{\background}[1]%
    {\newcommand{\bruno@background}{#1}}


%% \newcommand\insertcaption{}
\newcommand{\backgroundcaption}[2][white]{\def\bruno@caption{{#1}{#2}}}

\newcommand{\insertcaption}{%
    \ifdefined
    \bruno@caption{%
        \expandafter\textcolor\bruno@caption
    }%
    \fi
}


\tikzfading[
    name=title page picture fading,
    left color=transparent!0,
    right color=transparent!100,
]

\setbeamertemplate{title page}{%
    \begin{minipage}{.7\textwidth}
        \raggedright
        \usebeamerfont{title}
        \inserttitle

        \usebeamertemplate{separator}

        \usebeamerfont{author}
        \vspace{2em}
        \insertauthor

        \vspace{2em}
        \usebeamerfont{details}
        \insertinstitute

        \insertdate
    \end{minipage}

    % Include faded image on the right if defined
    \ifdefined\bruno@background
    \begin{tikzpicture}[remember picture, overlay]
        % Crop image to a trapezium on the right
        \clip (current page.south east)
           -- (current page.north east)
           -- ++(-5.1, 0)
           -- ++(-3, -\paperheight)
           -- cycle;

        % Include background image
        \node[
            anchor=south east,
            inner sep=0,
            outer sep=0
        ] at (current page.south east) {
            \includegraphics[height=\paperheight]
                {\bruno@background}
        };

        \node[anchor=south east,
            inner sep=0,
            outer sep=15pt,
        ] at (current page.south east){\insertcaption};

        % Add a slight shadow
        \fill[
            black, path fading=title page picture fading,
            fading angle=-15
        ]
            (current page.south east)
            rectangle
            ++(-10, \dimexpr\paperheight+1cm);

        
    \end{tikzpicture}
    \fi
}

\mode<all>
\AtBeginDocument{%
  \let\phi\varphi
  \let\theta\vartheta
  \let\setminus\smallsetminus
}

\title{Statisztika előadás}
\date{\lecturedate}
\background{images/young-Ronald-Fisher.jpg}
\backgroundcaption{Ronald Fisher (1913)}


\begin{document}

\maketitle

\begin{frame}{Diszkrét illeszkedés vizsgálat, $\chi^2$-próbák}
  \begin{itemize}
    \item $X_1,\dots,X_n$ $n$ elemű minta.
    $X_k\in\smallset{1,\dots,r}$. 
    \item $\P[\pi]{X_k=i}=\pi_i$, 
    $\pi\in \Delta=\set{\pi}{\pi_i>0,\sum\pi_i=1}$.
    \item $p\in \Delta$,
    $H_0:\pi=p$, $H_1:\pi\neq p$.
    
    \item $f_\pi(x)=\prod_{k=1}^n \P{X_k=x_k}=\prod_{k=1}^n \pi_{x_k}=\prod_{i=1}^{r} \pi_i^{\sum_{k}\I{X_k=i}}$.
    
    \item Jelölje $N_i=\sum_k \I{X_k=i}$, $i=1,\dots,r$ az egyes értékek
    gyakoriságát. Maximum likelihood becslés $\pi$-re: $\ell(\pi)=\sum N_i\ln \pi_i$.
    
    Feltételes szélsőérték ($\Delta$-n maximalizálunk):
    $\ell'(\hat\pi)\perp \Delta$, azaz $\hat\pi_i=N_i/n$ a relatív gyakoriság.
    \item 
    A próba statisztika
    \begin{displaymath}
      T_n=T_n (X)=\sum_{i=1}^r
      \frac{(N_i-np_i)^2}{np_i}=n\sum\frac{(\hat \pi_i-p_i)^2}{p_i}
    \end{displaymath}
    \item $n\to\infty$ esetén $H_0$ mellett $T_n\dto \chi^2_{r-1}$, míg
    $H_1$ mellett $T_n\to\infty$. (később belátjuk.)
    
    \item Kritikus tartomány. $T_n>c_\alpha$, ahol $c_\alpha$ a
    $\chi^2_{r-1}$ eloszlás kritikus értéke.
    \item Nem egzakt próba, mert $T_n$ $H_0$ alatti eloszlásának
    közelítéséből számoljuk a kritikus értéket.
  \end{itemize}
\end{frame}

\begin{frame}{Speciális eset $r=2$}
  
  \begin{itemize}
    \item $r=2$ esetén $p_2=1-p_1$, $N_2=n-N_1$, így
    \begin{align*}
      T_n&=\frac{(N_1-np_1)^2}{np_1}+\frac{(N_2-np_2)^2}{np_2}
      \\
      &=
      \frac{(N_1-np_1)^2}n \zjel*{\frac1{p_1}+\frac1{1-p_1}}
      \\
      &
      =\zfrac*{N_1-np_1}{\sqrt{np_1(1-p_1)}}^2,\\
    \end{align*}
    \item  
    \begin{displaymath}
      \frac{N_1-np_1}{\sqrt{np_1 (1-p_1)}}\dto N (0,1)
      \quad\implies\quad
      T_n\dto \chi^2_1\quad\text{ha $n\to\infty$}
    \end{displaymath}
    \item $r=2$ esetén lényegében az $u$-próbát kapjuk vissza,
    $H_0$ mellett, nagy $n$-re $N_1=\sum_{k=1}^n\I{X_k=1}$ közelítőleg normális eloszlású 
    $np_1$ várható értékkel és $n(p_1(1-p_1))$ szórásnégyzettel.
    
    A $H_0: \pi_1=p_1$ nullhipotézist a $H_1: \pi_1\neq p_1$ ellenhipotézis
    ellenében elutasítjuk, ha
    \begin{displaymath}
      \abs*{\frac{N_1-np_1}{\sqrt{np_1 (1-p_1)}}}>u_{\alpha/2},\quad
      \text{másképp írva}\quad 
      T_n>u^2_{\alpha/2}
    \end{displaymath}
  \end{itemize}
\end{frame}


\begin{frame}{$T_n$ határeloszlása I.}
  \begin{itemize}
    \item  Legyen $Y_{k}=(\I{X_k=1},\dots,\I{X_k=r})$, ekkor
    $Y_1,\dots,Y_n$  független azonos eloszlású változók.
    \item $\E[\pi]{Y_k}=\pi$ és
    $\cov_\pi
    (\I{X_k=i},\I{X_k=j})=\P[\pi]{X_k=i,\,X_k=j}-\pi_i\pi_j=\delta_{i,j}\pi_i-\pi_i\pi_j$.
    \item A centrális határeloszlás tételből $H_0$, azaz $\pi=p$  mellett 
    \begin{displaymath}
      S_n=\frac{\sum_{k=1}^n Y_k-p} {\sqrt{n}}\to N (0,\Sigma (Y)),
      \quad\text{ahol}\quad
      \sum_{k=1}^n (Y_k-p)=(N_1-n p_1,\dots,N_r-n p_r)
    \end{displaymath}
    Így $T_n=\sum_{i=1}^r\zfrac{N_i-np_i}{\sqrt{np_i}}^2$ nem más mint
    \begin{displaymath}
      T_n=\norm{D S_n}^2,
      \quad\text{ahol}\quad
      D =\diag(p_1^{-1/2},\dots,p_r^{-1/2}),\quad
      D S_n\dto N (0,D\Sigma (Y)D)
    \end{displaymath}
    Itt
    \begin{displaymath}
      (D\Sigma (Y)D)_{i,j}
      =\frac{\delta_{i,j}p_i-p_ip_j}{\sqrt{p_ip_j}}
      =\delta_{i,j}-\sqrt{p_i}\sqrt{p_j}
    \end{displaymath}
    Ha $e= (p_1^{1/2},\dots,p_r^{1/2})$, akkor $\norm{e}_2=\sum_i p_i=1$
    \begin{displaymath}
      (D\Sigma (Y)D)=I-e e^T
    \end{displaymath}
  \end{itemize}
\end{frame}

\begin{frame}{$T_n$ határeloszlása II.}
  \begin{itemize}
    \item  Legyen $Y_{k}=(\I{X_k=1},\dots,\I{X_k=r})$, ekkor
    $Y_1,\dots,Y_n$  független azonos eloszlású változók.
    \item A centrális határeloszlás tételből $H_0$, azaz $\pi=p$  mellett 
    \begin{displaymath}
      D S_n=\zjel*{\frac{N_1-np_1}{\sqrt{np_1}},\dots,\frac{N_r-np_r}{\sqrt{np_r}}}
      \dto N (0,I-e e^T),
      \quad\text{ahol}\quad e=(p_1^{1/2},\dots,p_r^{1/2})
    \end{displaymath}
    \item $I-ee^T$ merőleges vetítés $\real e$ ortokomplementerére, így
    $N (0,I-ee^T)$ nem más mint $(I-ee^T)Z$ eloszlása, ahol $Z$ $r$
    dimenziós standard normális. 
    \item Innen
    \begin{displaymath}
      T_n=\norm*{D S_n}^2\dto \norm{(I-ee^T)Z}^2\sim\chi^2_{r-1}
    \end{displaymath}
    Emlékeztető. Ha $\xi_n\dto\xi$, és $h$ folytonos, akkor $h(\xi_n)\dto h(\xi)$.
  \end{itemize}
\end{frame}


\begin{frame}{Emlékeztető}
  
  \begin{itemize}
    \item Egy megfigyelés sűrűségfüggvénye $f_\theta$, loglikelihoodja
    $\ell (\theta)$. $\theta\in\Theta$, ahol $\Theta\subset\real^p$ nyílt.
    \item Ha $X= (X_1,X_2,\dots)$ független megfigyelések az $f_\theta$
    eloszlásból, akkor legyen
    \begin{displaymath}
      \ell_n (\theta) = \ell_n (\theta,X)=\sum_{i=1}^n \ell (\theta,X_i)
    \end{displaymath}
    \item $\hat\theta_n$ $\theta$ ML becslése az első $n$ megfigyelés alapján. 
    Likelihood egyenlet: $\ell_n'(\hat\theta_n)=0$ 
    \item $\xi_n=\frac1{\sqrt{n}}\ell_n'(\theta)\dto N(0, I(\theta))$, ha $\E[\theta]{\ell'(\theta)}=0$ 
    és $I(\theta)$ létezik és véges, pl. $(R)$  esetén.  
    \item Ha $(RR)$ is teljesül, pl. exponenciális családban, vagy annak részcsaládjában, akkor
    \begin{displaymath}
      -\xi_n=-\frac1{\sqrt{n}}\ell_n' (\theta)=\frac1{\sqrt{n}}\zjel*{\ell_n'(\hat\theta_n)-\ell'_n(\theta)}=
      \zjel*{\frac1n\ell''_n(\theta)+R_n}\sqrt{n}(\hat\theta_n-\theta)      
    \end{displaymath}
    ahol $\frac1n \ell_n''(\theta)\to\E{\ell''(\theta)}=-I(\theta)$ és $R_n\to0$. Ebből %kaptuk, hogy
    $\sqrt{n}(\hat\theta_n-\theta)= I(\theta)^{-1}\xi_n+o_p(1)$ %\dto N(0,I^{-1}(\theta))$
    \item 
    \begin{align*}
      \ell_n(\hat\theta_n)-\ell_n(\theta) 
      &=
      \frac1{\sqrt n}\ell_n'(\theta)\sqrt{n}(\hat\theta_n-\theta)
      +\frac12\sqrt{n}(\hat\theta_n-\theta)^T\frac1n \ell''_n(\theta)\sqrt{n}(\hat\theta_n-\theta)+o_p(1)\\
      &=\frac12\xi_n^T I(\theta)^{-1}\xi_n+o_p(1)=\frac12\abs{I^{-1/2}(\theta)\xi_n}^2+o_p(1)\dto \frac12 \chi^2_p.
    \end{align*}
  \end{itemize}
\end{frame}

\begin{frame}
  \begin{proposition}
    $\Theta\subset\real^p$ $\Xi\subset\real^q$ nyílt,  $h:\Xi\to\Theta$ sima és $h'$ teljes rangú.
    $\set{\P[\theta]}{\theta\in\Theta}$ exponenciális eloszlás család. 
    $X_1,X_2,\dots $ független megfigyelések $\P[h(\tau)]$ eloszlásból $\theta=h(\tau)$, 
    $\ell_n$ az első $n$ megfigyelés loglikelihoodja,
    $\hat\tau_n=\arg\max_{\tau\in\Xi} \ell_n(h(\tau))$  a $\tau$ maximum likelihood 
    becslése az első $n$ megfigyelés alapján.
    
    Ekkor 
    \begin{displaymath}
      \ell_n(h(\hat\tau_n))-\ell_n(\theta))=\tfrac12\abs{P I^{-1/2}(\theta)\xi_n}^2+o(1) 
    \end{displaymath}
    ahol $\xi_n=\frac1{\sqrt{n}}\ell'_n(\theta)$ és $P$ merőleges vetítés $I^{1/2}(\theta)h'(\tau)$ képterére.
  \end{proposition}
  \begin{itemize}
    \item $\set{\P[h(\tau)]}{\tau\in \Xi}$-re a korábbi számolás alapján
    \begin{displaymath}
      \ell_n(h(\hat\tau_n))-\ell_n(h(\tau))=\tfrac12 (\xi_n')^T J(\tau)^{-1}\xi'_n,
      \quad\text{ahol}\quad
      (\xi_n')^T=\frac1{\sqrt{n}}\frac{\partial}{\partial\tau}\ell_n(h(\tau))=\frac1{\sqrt{n}}\ell_n'(\theta) h'(\tau)
    \end{displaymath}
    \item $H=h'(\tau)$ jelöléssel $J(\tau)=H^T I(\theta) H$ és 
    \begin{displaymath}
      \ell_n(h(\hat\tau_n))-\ell_n(\theta)=\tfrac12 \xi_n^T H (H^T I(\theta) H)^{-1} H^T \xi_n+o(1)
    \end{displaymath}
    \item $A=I^{1/2}H$ jelöléssel  $I^{1/2}(\theta)H (H^T I(\theta) H)^{-1} H^T I^{1/2}(\theta)=A(A^T A)^{-1} A^T=P$
    merőleges vetítés $A=I^{1/2} H$ képterére és 
    \begin{displaymath}
      \ell_n(h(\hat\tau_n))-\ell_n(\theta)
      =\tfrac12 \xi_n^T I^{-1/2} I^{1/2}H (H^T I(\theta) H)^{-1} H^T I^{1/2} I^{-1/2} \xi_n+o(1)
      =\tfrac12\abs{PI^{-1/2}\xi_n}^2 + o(1)
    \end{displaymath}
  \end{itemize}
  
\end{frame}

\begin{frame}{Likelihood hányados határeloszlása}
  \begin{proposition}
    \begin{itemize}[<*>]
      \item $\cP=\set{\P[\theta]}{\theta\in\Theta}$ exponenciális család $\Theta\subset \real^p$ nyílt. 
      \item $\Xi\subset \real^q$ nyílt. $h:\Xi\to\Theta$, $h$ sima, $H=h'$, $H (\tau)$ rangja $q$.
      \item $X_1,X_2,\dots$ független megfigyelések $\P[\theta]$ eloszlásból, $\theta=h (\tau)$
      \item $\ell_n (\theta) =\sum_{k=1}^n\ell (\theta,X_k)$
      \item $\hat\theta_n=\arg\max_{\theta\in\Theta}\ell_n (\theta)$ ML becslés $\theta$-ra.
      $\hat\tau_n=\arg\max_{\tau\in \Xi}\ell_n (h (\tau))$  ML becslés $\tau$-ra.
    \end{itemize}
    Ekkor
    \begin{displaymath}
      2 \zjel*{\ell_n (\hat\theta_n)-\ell_n(h(\hat\tau_n))} \dto\chi^2_{p-q}\quad\text{ha $n\to\infty$}.
    \end{displaymath}
  \end{proposition}
  \continue
  Legyen $\xi_n=\frac1{\sqrt{n}} \ell_n'(\theta)$ és $P$ merőleges vetítés $I^{1/2}h'(\tau)$ képterére.
  \begin{align*}
    2\zjel*{\ell_n (\hat\theta_n)-\ell_n(h(\hat\tau_n))}&=
    2\zjel*{\ell_n (\hat\theta_n)-\ell_n(\theta)}-2\zjel*{\ell_n(h(\hat\tau_n))-\ell_n(\theta)}\\
    &=
    \abs{I^{-1/2}(\theta)\xi_n}^2-\abs{PI^{-1/2}(\theta)\xi_n}^2+o(1)
    =\abs{(\Id_p-P)I^{-1/2}\xi_n}^2+o(1)\dto\chi^2_{p-q},
  \end{align*}
  mert $I^{-1/2}(\theta)\xi_n\dto N(0,\Id_p)$ és $\Id_p-P$ merőleges vetítés egy $p-q$ dimenziós altérre.
\end{frame}

\begin{frame}{Diszkrét illeszkedés vizsgálat, $T_n$ határeloszlása}
  \begin{itemize}
    \item $\P[\pi]{X_k=i}=\pi_i=\exp{\ln p_i}$, $i=1,\dots,r$.
    Egy mintaelem eloszlása exponenciális családot alkot $T(x)=(\I{x=1},\dots,\I{x=r})$ (one-hot encoding), 
    a természetes paraméterezés $(\ln p_1,\dots,\ln p_r)$,
    azaz $\Theta'=\set{t\in\real^r}{\sum_{i=1}^r e^{t_i}=1}$. Ez egy $r-1$ dimenziós sokaság $\real^r$-ben. 
    \item Egy lehetséges paraméterezés $\Theta=\real^{r-1}=\real^{r-1}\times\smallset{0}$, 
    \begin{displaymath}
      p_i=\frac{e^{\theta_i}}{1+\sum_{j<r} e^{\theta_j}},
      \quad i<r\quad\text{és}\quad p_r=\frac1{1+\sum_{j<r} e^{\theta_j}},
      \quad
      \ell(\theta,x)=\sum_{i<r}\theta_i\I{x=i}-\ln(1+\sum\nolimits_{j<r} e^{\theta_j}).
    \end{displaymath}
    azaz $b(\theta)=-\ln p_r$ és $\ln p_i=\theta_i-b(\theta)$ ($\theta_r=0$). 
    $b'(\theta)=(p_1(\theta),\dots,p_{r-1}(\theta))$.
    \item Ezzel a paraméterezéssel és a $\hat p_{n,i}=\frac1n\sum_{k\leq n}\I{X_k=i}$ jelöléssel
    \begin{displaymath}
      \ell_n(\theta)=\ln p(\theta) \cdot \sum\nolimits_{k=1}^n T(X_k)=n \ln p(\theta)\cdot \hat p_n  
    \end{displaymath}
    $p(\hat\theta_n)=\hat p_n$, mert $\hat p_n$ a maximum likelihood becslés.
    \item 
    \begin{displaymath}
      \ell_n(\hat\theta_n)-\ell_n(\theta^*) =
      n\zjel*{\ln p(\hat\theta_n)\cdot\hat p_n-\ln p(\theta^*)\cdot \hat p_n}
      =n\sum\nolimits_{i=1}^r \hat p_{n,i}\ln\zfrac*{\hat p_{n,i}}{p_i}.
    \end{displaymath}
    
  \end{itemize}
\end{frame}


\begin{frame}
  \begin{proposition}
    $p,q$ két eloszlás $\smallset{1,\dots,r}$-en. Ekkor
    \begin{displaymath}
      \textstyle \sum_i q_i\log\frac {q_i}{p_i}=\tfrac12\sum_i\frac
      {(q_i-p_i)^2}{p_i}\zjel{1+\alpha\max_{i}\abs{\frac{q_i-p_i}{p_i}}}\quad\text{ahol $\alpha\in[-1,1]$.}
    \end{displaymath}
  \end{proposition}
  \begin{itemize}
    \item $f (x)=x\log x$-re $f (1)=0$, $f' (1)=1$ és $f''
    (x)=1/x$, mert $(x\log x)'=1+\log x$, $(x\log x)''=1/x$.
    \begin{align*}
      x\log x = f (x)&=f (1)+f' (1) (x-1)
      +\tfrac12 f'' (\xi) (x-1)^2\\
      &= x-1+\tfrac1{2\xi} (x-1)^2
      \quad\text{ahol $\xi$ alkalmas
      közbülső pont $1$ és $x$ között}%\xi\in (0,x)
    \end{align*}
    \item Ha a $\xi$ pont  $1$ és $x$ között van, akkor
    $\frac1\xi-1=\alpha\abs{\frac1x-1}$ valamely $\alpha\in[-1,1]$-re.
    
    \item
    \begin{align*}
      \sum\nolimits_i q_i\log\tfrac{q_i}{p_i}
      &=\sum p_if(\tfrac{q_i}{p_i})=
      \sum\nolimits_i p_i(\tfrac{q_i}{p_i}-1)+
      \tfrac12\sum\nolimits_ip_i(\tfrac{q_i}{p_i}-1)^2\zjel{1+\alpha_i\abs{\tfrac{q_i}{p_i}-1}}\\
      &=
      \zjel*{\tfrac12\sum\nolimits_i \tfrac{(q_i-p_i)^2}{p_i}}
      \zjel*{1+\alpha\max\abs{\tfrac{q_i-p_i}{p_i}}}\quad\alpha\in[-1,1]
    \end{align*}
    \item $\hat p\to p$ esetén   $\max\abs{\tfrac{\hat p_i-p_i}{p_i}}\to0$
    \item $2\zjel{\ell_n(\hat p)-\ell_n(p)}
    =n\sum_i \hat p_{n,i}\ln\frac{\hat p_{n,i}}{p_i}
    =n\sum_i\frac{(\hat p_{n,i}-p_i)^2}{p_i}+o(1)\dto\chi^2_{r-1}$.
  \end{itemize}
  
\end{frame}



\begin{frame}{Függetlenség vizsgálat $\chi^2$-próbával}
  \begin{itemize}
    \item Két szempont szerint osztályozunk, azaz a megfigyelések
    $(X_k,Y_k)$, $k=1,\dots,n$.
    \item $X_k\in\smallset{1,\dots,r}$, $Y_k\in\smallset{1,\dots,s}$.
    
    \item Kérdés: független-e $X$ és $Y$? Azaz ha az együttes eloszlás
    $\pi$, $\pi_{i,\cdot}=\sum_j \pi_{i,j}$, $\pi_{\cdot,j}=\sum_i
    \pi_{i,j}$ akkor $H_0:\pi_{i,j}=\pi_{i,\cdot}\pi_{\cdot,j}$,
    $H_1:\exists i,j,\,\,\pi_{i,j}\neq\pi_{i,\cdot}\pi_{\cdot,j}$.
    
    \item Gyakoriságok $n_{i,j}=\sum_{k}\I{X_k=i,Y_k=j}$,
    $n_{\cdot,j}=\sum_{i}n_{i,j}$, $n_{i,\cdot}=\sum_j n_{i,j}$
    \item 
    \begin{displaymath}
      T_n=\sum_{i=1}^r\sum_{j=1}^s
      \frac{(n_{i,j}-n\frac{n_{i,\cdot}n_{\cdot,j}}{n^2})^2}{n\frac{n_{i,\cdot}n_{\cdot,j}}{n^2}}
      =n
      \sum_{i=1}^r\sum_{j=1}^s
      \frac{(n_{i,j}-\frac{n_{i,\cdot}n_{\cdot,j}}{n})^2}{n_{i,\cdot}n_{\cdot,j}}
      \dto\chi^2_{(r-1) (s-1)}
    \end{displaymath}
    
  \end{itemize}
\end{frame}

\begin{frame}{$T_n$ határeloszlása}
  \begin{itemize}
    \item $\zfrac{n_{i,j}}{n}_{i,j}$ a $(\pi_{i,j})_{i,j}$ eloszlás
    maximum likelihood becslése az összes lehetséges eloszlás
    $rs-1$ dimenziós családjában.
    \item $\frac{n_{i,\cdot}n_{\cdot,j}}{n^2}$ a maximum likelihood
    becslés a szorzat alakú eloszlások $(r-1)+(s-1)$ dimenziós
    családjában. 
    \item A likelihood hányados logaritmusa közelítőleg $rs-1- (r-1) - (s-1)= (r-1)
    (s-1)$ szabadságokú $\chi^2$ eloszlású.
    \item A likelihood hányados logaritmusa
    \begin{align*}
      2\zjel*{\sum_{i,j}n_{i,j}\log\frac{n_{i,j}}n-\sum_{i,j}n_{i,j}\log\frac{n_{i,\cdot}n_{\cdot,j}}{n^2}}
      &=
      2n\sum_{i,j}\hat p_{i,j}\log\frac{\hat p_{i,j}}{\hat p_{i,\cdot}\hat p_{\cdot,j}}\\
      &=n\sum_{i,j}\frac{(\hat p_{i,j}-\hat p_{i,\cdot}\hat
      p_{\cdot,j})^2}{\hat p_{i,\cdot}\hat p_{\cdot,j}}+o (1)
      \\
      &=
      n\sum_{i=1}^r\frac{(n_{i,j}-\frac{n_{i,\cdot}n_{\cdot,j}}{n})^2}{n_{i,\cdot}n_{\cdot,j}}=T_n+o (1)
    \end{align*}
  \end{itemize}
  
\end{frame}

\begin{frame}{Homogenitás vizsgálat $\chi^2$-próbával}
  \begin{itemize}
    \item $X_1,\dots,X_n$, $X_i\in\smallset{1,\dots,r}$, $\pi$
    eloszlásból.
    \item $Y_1,\dots,Y_m$, $Y_i\in\smallset{1,\dots,r}$, $\rho$
    eloszlásból.
    
    \item Gyakoriságok: $n_i=\sum_k\I{X_k=i}$, $m_i=\sum_k\I{Y_k=i}$.
    \item Kérdés $H_0:\pi=\rho$, $H_1:\pi\neq\rho$.
    \item Öntsük össze a megfigyeléseket és $Z_i=1$ ha az
    $i$. megfigyelés az első mintából, $Z_i=2$ ha a második mintából
    származik.
    \item Új kérdés független-e $X,Z$?
    
    \item
    \begin{displaymath}
      n_{i,j}=
      \begin{cases}
        n_i&j=1\\m_i&j=2
      \end{cases},
      \quad
      n_{i,\cdot}=n_i+m_i,\quad
      n_{\cdot,1}=n,n_{\cdot,2}=m
    \end{displaymath}
    \item
    \begin{align*}
      T_n&=(n+m)\sum_{i,j}\frac{(n_{i,j}-\frac{n_{i,\cdot}n_{\cdot,j}}{n+m})^2}{n_{i,\cdot}n_{\cdot,j}}\\
      &= (n+m)
      \sum_{i}\frac{(n_{i}-\frac{(n_i+m_i)n}{n+m})^2}{(n_i+m_i)n}+%\\
      (n+m)
      \sum_{i}\frac{(m_{i}-\frac{(n_i+m_i)m}{n+m})^2}{(n_i+m_i)m}
      % &
      =nm\sum_{i}\frac{(\frac{n_{i}}n-\frac{m_i}m)^2}{n_i+m_i}\quad\dto\chi^2_{r-1}
    \end{align*}
    
  \end{itemize}
\end{frame}

\begin{frame}{Becsléses illeszkedés vizsgálat}
  \begin{itemize}
    \item Példa. $X_1,X_2,\dots,X_n$ mintáról szeretnénk eldönteni, hogy
    Poisson eloszlásból származik-e.
    
    
    \item  A mintából maximum likelihood módszerrel megbecsüljük a
    $\lambda$ paramétert, majd a megfigyeléseket osztályokba soroljuk
    és $\chi^2$ próbát végzünk.
    \begin{displaymath}
      T_n=\sum_{i}\frac{(n_i-np_i(\hat\lambda))^2}{np_i (\hat\lambda)}
    \end{displaymath}
    
    \item $T_n$ eloszlása közelítőleg $\chi^2$ a szabadságfok $r-1-p$, ahol
    $p$ a becsült paraméterek száma.
    
    \item A konkrét feladatban $p=1$, $T_n$ közelítőleg $r-2$ szabadságfokú.
  \end{itemize}
\end{frame}

\begin{frame}{Megjegyzések $\chi^2$ próbákhoz}
  \begin{itemize}
    \item A próbastatisztika eloszlását a határeloszlással közelíti. 
    \item Nagy mintára használható.
    \item Minden cellában legyen néhány megfigyelés.
    \item Ökölszabály: minden  cella gyakoriság  %Gyakorlatban
    legalább 4-6, ha ez nem teljesül, akkor cella összevonás.
  \end{itemize}
  
\end{frame}

\begin{frame}{Folytonos illeszkedés vizsgálat}
  \begin{itemize}
    \item $X_1,X_2,\dots,X_n$ minta  $F$ folytonos egy dimenziós
    eloszlásból.
    \item $F_0$ adott folytonos eloszlás függvény.
    \item $H_0: F=F_0$, lehetséges alternatívák
    
    $H_1:F\geq F_0$, $F \neq F_0$ %(egyoldali ellenhipotézis).
    
    $H^*_1:F\neq F_0$ %(egyoldali ellenhipotézis).
    \item Statisztikák:
    \begin{displaymath}
      D_n^+=\sup_t F_n (t)-F_0 (t),
      \quad 
      D_n=\sup_t\abs*{F_n (t)-F_0 (t)}
      \quad\text{Kolmogorov--Szmirnov}
    \end{displaymath}
    \begin{displaymath}
      \omega^2_n=\int_{\real} (F_n (t)-F_0 (t))^2 dF_0 (t),
      \quad\text{Cramer--von Mises}
    \end{displaymath}
  \end{itemize}
  \begin{proposition}
    A Kolmogorov--Szmirnov és Cramer--von Mises statisztikák
    eloszlásmentesek, azaz $H_0$ fennállása esetén eloszlásuk nem függ $F_0$-tól.
  \end{proposition}
\end{frame}

\begin{frame}{Kolmogorov--Szmirnov statisztika eloszlásmentessége}
  \begin{itemize}
    \item $X_1,\dots,X_n$ $n$--elemű minta $F_0$ folytonos
    eloszlásfüggvényből.
    \item $X_0^*=-\infty<X_1^*<X_2^*<\dots<X_n^*<X_{n+1}^*=\infty$ egy valószínűséggel, mert $i\neq
    j$-re $\P{X_i=X_j}=\E{\P{X_i=X_j|X_j}}=\E{\P{X_i=y}|_{y=X_j}}=0$.
    \item $(X_i^*,X_{i+1}^*)$-n $F_n$ konstans, $F_0$ monoton,
    \begin{displaymath}
      D_n^+ = \sup_{t}F_n (t)-F_0 (t)=\max_{1\leq i\leq n} (F_n (X_i)-F_0 (X_i))\vee(F_n (X_i+)-F_0 (X_i))
    \end{displaymath}
    
    \item $F_n (X_i^*)=\frac{i-1}n$ és $F_n (X_i^*+)=\frac{i}n$. Azaz
    \begin{displaymath}
      D_n^+=\max_i(\tfrac{i-1}n-F_0 (X_i^*))\vee(\tfrac{i}n-F_0 (X_i^*))
    \end{displaymath}
    \item $U_i=F_0 (X_i)$ $(0,1)$-en egyenletes eloszlású minta, $F_0
    (X_i^*)=U_i^*$
    \item $D_n^+$ eloszlása tetszőleges $F_0$ esetén ugyanaz.
    \item Hasonlóan $D_n=\sup_t\abs*{F_n (t)-F (t)}$-re.
  \end{itemize}
  
\end{frame}

\begin{frame}{Cramer--von Mises statisztika eloszlásmentes}
  \begin{itemize}
    \item $X_1,\dots,X_n$ $n$--elemű minta $F_0$ folytonos
    eloszlásfüggvényből.
    \begin{displaymath}
      \omega_n^2=\int (F_n (t)-F_0 (t))^2dF_0 (t)
    \end{displaymath}
    \item Ha $X_0$ a mintától független $F_0$ eloszlású
    \begin{displaymath}
      \omega_n^2=\E{(F_n (X_0)-F_0 (X_0))^2|X_1,\dots,X_n}
    \end{displaymath}
    \item $U_0=F_0 (X_0)$, $U_i=F_0 (X_i)$ $i=1,\dots,n$.
    \begin{displaymath}
      F^U_n (t)=\frac1n\sum_{i=1}^n \I{U_i<t}
    \end{displaymath}
    \item Mivel $\P{U_0\in\smallset{U_1,\dots,U_n}}=0$, ezért $F_n
    (X_0)=F^{U}_n (U_0)$ és
    \begin{displaymath}
      \omega_n^2=\E{(F^U_n (U_0)-U_0)^2|U_1,\dots,U_n}
    \end{displaymath}
    azaz $\omega_n^2$ eloszlása minden $F_0$ esetén ugyanaz.
  \end{itemize}
  
\end{frame}

\begin{frame}{Folytonos illeszkedés vizsgálat}
  \begin{itemize}
    \item $X_1,X_2,\dots,X_n$ minta  $F$ folytonos egy dimenziós
    eloszlásból.
    \item $F_0$ adott folytonos eloszlás függvény.
    \item $H_0: F=F_0$, lehetséges alternatívák
    
    $H_1:F\geq F_0$, $F \neq F_0$.
    
    $H^*_1:F\neq F_0$.
    \item Statisztikák:
    \begin{displaymath}
      D_n^+=\sup_t F_n (t)-F_0 (t),\quad D_n=\sup_t\abs*{F_n (t)-F_0 (t)}\quad\text{Kolmogorov--Szmirnov}
    \end{displaymath}
    \begin{displaymath}
      \omega^2_n=\int_{\real} (F_n (t)-F_0 (t))^2 dF_0 (t),
      \quad\text{Cramer--von Mises}
    \end{displaymath}
    
    \item Kolmogorov--Szmirnov próba kritikus tartománya
    
    $H_1$ esetében $D_n^+>c$ esetén elvetjük a nullhipotézist
    
    $H^*_1$ esetében $D_n>c$ esetén elvetjük a nullhipotézist
    
    \item Cramer--von Mises próba:  $\omega^2_n>c$ esetén vetjük el $H_0$-t.
    
    \item Nem túl nagy $n$-re léteznek táblázatok. 
    \item A skálázott próba statisztikának létezik határeloszlása,
    általában a pontos eloszlást a határeloszlással közelítjük.
  \end{itemize}
\end{frame}

\begin{frame}{Skálázott próbastatisztikák határeloszlása}
  \begin{itemize}
    \item
    \begin{displaymath}
      D_n^+=\sup_t F_n (t)-F_0 (t),\quad D_n=\sup_t\abs*{F_n (t)-F_0 (t)}
    \end{displaymath}
    \item
    \begin{displaymath}
      \lim_{n\to\infty}\P{\sqrt{n}D_n^+>x}
      =e^{-2x^2}=\frac{\phi (2x)}{\phi (0)},
    \end{displaymath}
    ahol $\phi$ a standard normális sűrűség függvény.
    
    Más szóval $n(D_n^+)^2\dto\exp{2}$.
    \item
    \begin{displaymath}
      \lim_{n\to\infty}\P{\sqrt{n}D_n<x} =
      \sum_{k\in \Z} (-1)^{k} e^{-2k^2x^2}=\sum_{k\in \Z} (-1)^{k}\frac{\phi (2kx)}{\phi (0)}
    \end{displaymath}
    \item
    \begin{displaymath}
      n\omega_n^2  \dto\sum_{k=1}^\infty
      \frac{Z_k^2}{k^2\pi^2},\quad\text{ahol $Z_1,Z_2,\dots$ független $N (0,1)$-ek.}
    \end{displaymath}
  \end{itemize}
\end{frame}


\begin{frame}{Brown híd, mese}
  \begin{itemize}
    \item Legyen $U_1,\dots,U_n$ $(0,1)$e-n egyenletes minta.
    \begin{displaymath}
      X^{n}=(n^{-1/2} \sum_{i=1}^n(\I{U_i\leq t}-t))_{t\in[0,1]}=\sqrt{n}\zjel*{F_n(t)-t}_{t\in[0,1]}
    \end{displaymath}
    valószínűségi változók paraméterezett családja, azaz folyamat.
    \item $t\mapsto X^{n}_t (\omega)$ jobbról folytonos, balról
    határértékkel rendelkező függvény (trajektória).
    \begin{displaymath}
      D[0,1]=\set{w:[0,1]\to\real}{\text{$w$ jobbról folytonos, balról
      határértékkel rendelkezik}}
    \end{displaymath}
    \item $D[0,1]$-en megadható olyan metrika, amivel $D[0,1]$ teljes
    szeparábilis metrikus tér. $X^{n}$ 
    véletlen elem ebből a térből.
    
    \item Megmutatható, hogy $X^n$ eloszlásban konvergens, az
    eloszlásbeli limesz neve Brown híd.
    
    \item $(B_t)_{t\in[0,1]}$ Brown híd, ha folytonos trajektóriájú Gauss
    folyamat, $\E{B_t}=0$, $\cov (B_t,B_s)=s\wedge t-st$.
    \item $\phi (w)=\max_s w (s)$, $\max_s \abs*{w (s)}$, $\int_0^1w^2
    (s)ds$ folytonos funkcionálok $D[0,1]$-en.
    
    
    \item
    \begin{displaymath}
      \sqrt{n}D_n^+\dto \max_{s\in[0,1]} B_s,
      \quad\sqrt{n}D_n\dto \max_{s\in[0,1]}\abs*{B_s},
      \quad n\omega_n^2\dto \int_0^1B_s^2ds,\quad\text{ahol $B$ Brown híd}
    \end{displaymath}
    
  \end{itemize}
\end{frame}
\end{document}
