\documentclass[11pt,oneside,a4paper,final
]{memoir}%
\usepackage[scale=.8]{geometry}


\chapterstyle{lyhne}%{bringhurst}
\pagestyle{ruled}

\let\ttdef\ttdefault
\AtBeginDocument{\global\let\ttdefault=\ttdef}

%\usepackage{palatino}
%\usepackage[utf8]{inputenc}

\usepackage[mathcal]{euscript}

\usepackage{graphicx}                           
\RequirePackage{amsthm,amsmath,natbib,amssymb,soul,bbold}

\usepackage{color}

\usepackage[notcite,notref]{showkeys}
\usepackage{polyglossia}
\setdefaultlanguage{hungarian}

\theoremstyle{plain}
\newtheorem{theorem}{Tétel}

\newtheorem{proposition}{Állítás}
\newtheorem{lemma}{Lemma}
\newtheorem{claim}{Állítás}
\newtheorem{corollary}{Következmény} 
\newtheorem{problem}{Kérdés}


\theoremstyle{definition} 
\newtheorem{df}{Definíció} 

\newtheorem*{example}{Példa} 
\newtheorem{conjecture}{Sejtés}
\newtheorem{condition}{Feltétel}


\theoremstyle{remark} 
\newtheorem{remark}{Megjegyzés}

%%%%%%%%%%%%%%%%%%%%%%%%%%%%%
% \usepackage{pvdefs}
\usepackage{etoolbox,mathtools}

\makeatletter
\def\ifbrace{\@ifnextchar{\bgroup}}

\@ifundefined{texexp}{%
  \let\texexp\exp
  \def\exp{%
    \texexp\ifbrace{\cbr*}{}%
  }
}\relax
 
\def\isparam#1{\ifbrace#1\relax}

\DeclarePairedDelimiter{\abs}{\lvert}{\rvert}
\DeclarePairedDelimiter{\zjel}{(}{)}
\DeclarePairedDelimiter\br\lbrack\rbrack
\DeclarePairedDelimiter\event\lbrace\rbrace
\DeclarePairedDelimiter{\norm}{\|}{\|}
\let\smallset\event
\let\cbr\event
\DeclarePairedDelimiterX\set[2]\{\}{%
  #1\nonscript\::\allowbreak\nonscript\:\mathopen{}#2%
}

\let\PEfont\mathbb
\providecommand\given{}

\def\redefgiven#1{%
  \renewcommand\given{%
    \nonscript\:%
    #1\vert%
    \allowbreak%
    \nonscript\:%
    \mathopen{}%
  }%
}
\DeclarePairedDelimiterX\PEzjel[1](){\redefgiven{\delimsize}#1}
\def\aftersuperscript#1^#2{%
  ^{#2}#1%
}
\newcommand{\PE}[1][]{%
  \PEfont{\Pe}%
  \ifblank{#1}{}{_{#1}}%
  \@ifnextchar{^}{%
    \aftersuperscript{\ifbrace{\PEzjel*}{}}%
    }{%
      \ifbrace{\PEzjel*}{}%
    }%
  %\ifbrace{\PEzjel*}{}%
}
\newcommand{\E}{\def\Pe{E}\PE}
\newcommand{\D}{\def\Pe{D}\PE}
\renewcommand{\P}{\def\Pe{P}\PE}

\def\cov{\operatorname{cov}}
\let\eps\varepsilon
\def\ito/{It\^o}
\let\Ito=\ito
\def\eqinlaw{\stackrel{d}{=}}


\DeclareListParser{\dolist}{}


\def\stripdef@end#1{\egroup\def#1}

\def\stripdef#1#2#3{%
  \bgroup
  \escapechar=-1\relax
  \expandafter\stripdef@end\csname#1\string#2#3\endcsname
}

\def\defprefix#1#2{
    \def\do##1{\stripdef{#1}{##1}{}{{#2{##1}}}}%
    \ifbrace\dolist\do
}



\def\defpostfix#1#2{
    \def\do##1{\stripdef{}{##1}{#1}{{#2{##1}}}}%
    \ifbrace\dolist\do
}


\defprefix{c}{\mathcal}{XAPIDLCSGTHENU}
\defprefix{b}{\bar}{hX\theta\Theta\psi,\phi}
\defprefix{t}{\tilde}{STQbCfT\lambda\Omega\sigma\beta}
\defprefix{}{\mathcal}{BF}
\defprefix{mf}{\mathfrak}{X}

\def\sign{\operatorname{sign}}
\def\real{\mathbb{R}}
\def\C{\mathbb{C}}
\def\Z{\mathbb{Z}}
\def\Q{\mathbb{Q}}

\def\Isymb{{\mathbf 1}}
\def\substr#1{_{\zjel{#1}}}
\newcommand{\I}[1][]{%
  \Isymb\ifblank{#1}{\substr}{_{#1}}%
}

\def\Isymb{{\mathbb 1}}

\def\I{\@ifstar\I@star\I@}
\def\I@star{%
  %\message{I@star arg:|#1|^^J}%
  \Isymb\ifbrace{_}{}% } %{_{#1}}%
  }
\newcommand{\I@}[2][\zjel]{%
  \Isymb_{#1{#2}}% } %{_{#1}}%
}



\def\interior{\operatorname{int}}
\def\tg{\operatorname{tg}}
\DeclarePairedDelimiterX{\zfrac}[2](){\frac{#1}{#2}}
\DeclarePairedDelimiterX{\ip}[2]\langle\rangle{#1, #2}

\newcommand\independent{\protect\mathpalette{\protect\independenT}{\perp}}
\def\independenT#1#2{\mathrel{\rlap{$#1#2$}\mkern2mu{#1#2}}}





\def\diag{\operatorname{diag}}
\def\Id{\mathbf{I}}
\let\PEfont\mathbb
\def\npto{\buildrel p\over \nrightarrow}
\def\ndto{\buildrel d\over \nrightarrow}
%\def\dto{\buildrel d\over \rightarrow}
\def\defto#1{\stackrel{#1}{\to}}
\defpostfix{to}{\defto}{dpw}
% \def\dto{\stackrel{d}{\to}}
% \def\pto{\stackrel{p}{\to}}
% \def\pto{\stackrel{w}{\to}}

\def\tr{\operatorname{Tr}}
\def\arctg{\operatorname{arctg}}
\let\nto\nrightarrow

\def\Leb{\operatorname{Leb}}
\def\im{\operatorname{im}}

\makeatother

\defprefix{b}{\bar}{\phi}


\begin{document}
\let\phi\varphi
\let\theta\vartheta
\setmainlanguage{hungarian}

\chapter*{Emlékeztető. Exponenciális család, próba, terjedelem, erő, torzítatlan próba}
\begin{df}
  Az $X_1,\dots,X_n$ minta eloszlásainak családja exponenciális családot alkot, ha alkalmas paraméterezés 
  és domináló mérték mellett a sűrűségfüggvények
  \begin{displaymath}
    f_\theta(x)=\exp{\theta T(x)-b(\theta)},\quad\theta\in\Theta\subset\real^p
  \end{displaymath}
  alakúak.
\end{df}
Ha az eloszláscsalád exponenciális, akkor a kitevőben megjelenő statisztika elégséges, ha a paramétertér belseje nem üres, 
akkor teljes is.

\begin{lemma}
  Ha a minta eloszlásainak a családja exponenciális, akkor $T(X)$ eloszlásainak a családja is az.
\end{lemma}
\begin{proof}
  Legyen $\lambda$ az a domináló mérték amire nézve a sűrűségek a megadott alakúak. Ekkor 
  $\lambda\circ T^{-1}$ ($\sigma$-véges) mérték $\real^p$ Borel
  halmazain és 
  \begin{displaymath}
    \P[\theta]{T\in H}
    =\int \I{T(x)\in H} \exp{(\theta T(x)-b(\theta)} \lambda(d x)
    =\int \I{t\in H} \exp{\theta t-b(\theta)} (\lambda\circ T{-1})(d t)
  \end{displaymath}
  Azaz $\P[\theta]\circ T^{-1}$ abszolút folytonos $\lambda\circ T^{-1}$-re nézve és a sűrűségek 
  \begin{displaymath}
    f_{T;\theta}(t)=\exp{\theta t - b(\theta)}\quad \theta\in\Theta.
  \end{displaymath}
  alakúak.
\end{proof}

\begin{example}
  $X_1,\dots,X_n$ $N(\mu,\sigma^2)$ eloszlású minta, $\mu\in\real$, $\sigma^2>0$. Ekkor
  \begin{displaymath}
    f_{\mu,\sigma^2}(x)=\prod_{i=1}^n \frac{1}{\sqrt{2\pi}\sigma}\exp{-\frac12\cdot\frac{(x_i-\mu)^2}{\sigma^2}}
    =C\exp{-\frac{1}{2\sigma^2}\sum_i x_i^2+\frac{\mu}{\sigma^2}\sum_i x_i-\zjel*{\frac{n\mu^2}{2\sigma^2}-\frac{n}2\ln\sigma^2}}.
  \end{displaymath}
  Azaz ha $\theta_1=\frac{-n}{2\sigma^2}$, $\theta_2=\frac{\mu n}{\sigma^2}$, akkor a $\Theta=(-\infty,0)\times\real$ és 
  $T(x)=(\frac1n\sum_i x_i^2,\frac1n\sum_i x_i)$. Ekkor $T$ felírható a mintaátlaggal és a tapasztalati szórásnégyzettel:
  $s^2_n=T_1-T_2^2$ a tapasztalati szórásnégyzet, amire $s_n^2/sigma^2\sim \chi^2_{n-1}$
  és $T_2=\bar{X}\sim N(\mu,\frac1n\sigma^2)$.
  Innen  a sűrűségfüggvény transzformációs formulával 
  \begin{align*}
    f_{T_1,T_2}(s,t)
    & =f_{T_2,T_1-T^2}(t,s-t^2)\\
    & =
    \frac1{\sqrt{2\pi\sigma^2/n}}\exp{-\frac n2\cdot\frac{t-\mu}{\sigma^2}}
    \frac{1}{\Gamma(n-1)}\zfrac*{n}{2\sigma^2}^{n-1} (s-t^2)^{n-2} \I{s-t^2>0} \exp{-\frac{n}{2\sigma^2}(s-t^2)}\\
    &=\sqrt{\frac n{2\pi\sigma^2}} \frac1{(n-2)!}\zfrac*{n}{2\sigma^2}^{n-1} (s-t^2)^{n-2}\I{t<\sqrt{s}} 
    \exp{\frac {n\mu}{\sigma^2} t-\frac{n}{2\sigma^2}s -\frac{n\mu^2}{2\sigma^2}}\\
    &=C(s-t^2)^{n-2}\I{t<\sqrt{s}}\exp{\theta_1 s+\theta_2 t-b(\theta)}
  \end{align*}
  Azaz $T$ eloszlása exponenciális családot, a domináló mérték sűrűségfüggvénye a Lebesgue mértékre nézve 
  $C(s-t^2)^{n-2}\I{t<\sqrt{s}}$. Itt érdemes észrevenni azt is, hogy $T_1,T_2$ sűrűsége szorzat alakú, 
  de a domináló mérték nem az és $T_1,T_2$ nem független.
\qed
\end{example}

$\Theta=\Theta_0\cup^*\Theta_1$, $H_0:\theta\in\Theta_0$, $H_1:\theta\in\Theta_1$. $\phi:\mfX\to[0,1]$ a döntési függvény.
$\psi(\theta)=\E[\theta]{\phi(X)}$. Ekkor $\psi|_{\Theta_1}$ az erőfüggvény, 
$\alpha(\phi)=\sup_{\theta\in\Theta_0}\psi(\theta)$ a terjedelem. $\phi$ torzítatlan próba, ha 
$\alpha(\phi)=\sup_{\alpha\in\Theta_0}\psi(\theta)\leq \inf_{\theta\in\Theta_1}\psi(\theta)$. 
Azonos terjedelmű próbákat az erőfüggvény alapján hasonlítjuk össze.

Csak azt az esetet nézzük, amikor az eloszláscsalád exponenciális és $\Theta$ nyílt. Ekkor teljesül $(R)$ és $\psi$ folytonos, 
sőt folytonosan differenciálható. Legyen $\Theta^*=\bar{\Theta_0}\cap \bar{\Theta_1}$, ez a $\Theta_0$ határa $\Theta$-ban.
$\psi$ folytonossága miatt, ha $\phi$ torzítatlan $\alpha$ terjedelmű próba, 
akkor $\psi(\theta)=\alpha$ minden $\theta\in\Theta*$-ra.

Végül $\psi(\theta)=\E[\theta]{\phi(X)}=\E[\theta]{\bphi(T)}$, 
ahol $\bphi=\E[\theta]{\phi(X)|T}$ a feltételes várható érték közös változata. Azaz mindig elegendő $T$ eloszlásainak a 
családját tekinteni és $T$-n alapuló próbát keresni.


\chapter*{Egyenletesen legerősebb torzítatlan próba kétoldali ellenhipotézis mellett}

\begin{df}
  A $\phi$ döntési függvényű próba torzítatlan, ha 
  $\sup_{\theta\in\Theta_0}\E[\theta]{\phi}\leq\inf_{\theta\in\Theta_1}\E[\theta]{\phi}$.
\end{df}

$\cP=\set{\P[\theta]}{\theta\in\Theta}$ exponenciális család, $\Theta\subset\real$ nyílt intervallum. 
$H_0:\theta=\theta_0$ $H_1:\theta\neq\theta_0$.
A domináló mérték lehet $\P[\theta_0]$ és átparaméterezéssel ($\theta$ helyett $\theta-\theta_0$-at 
használva) elérhető, hogy $H_0: \theta=0$ és $H_1:\theta\neq0$. Ekkor 
\begin{displaymath}
  f_\theta(x)=\exp{\theta T(x)-b(\theta)}
\end{displaymath}
Az is feltehető, hogy $b'(0)=0$. Ha nem így lenne, akkor $T(x)-b'(0)$-t használva $T$ helyett és $b$-t
ennek megfelelően módosítva ez elérhető. Ekkor persze $\ell'(0)=T$. 

\begin{example}%\label{ex:1}
  $X_1,\dots,X_n$ minta az $N(\mu,\sigma^2)$ eloszlásból, $\sigma$ ismert. $H_0:\mu=\mu_0$, $H_1:\mu\neq\mu_0$.
  \begin{displaymath}
    f_\mu(x)=\prod_{i=1}^n \frac1{\sqrt{2\pi}\sigma}e^{-\frac12\cdot\frac{(x_i-\mu)^2}{\sigma^2}}.
  \end{displaymath}
  \begin{displaymath}
    \frac{f_\mu}{f_{\mu_0}}(x)= \exp{\frac{\mu-\mu_0}{\sigma^2}\sum_i x_i-\frac n{2\sigma^2}\zjel{\mu^2-\mu_0^2}}
  \end{displaymath}
  Itt $\theta=\frac{\mu-\mu_0}{\sigma^2}$, $\ell'(0,x)=T(x)=\sum_i (x_i-\mu_0)$ és 
  $b(\theta)=\frac{n}{2\sigma^2}(\mu-\mu_0)^2=\frac{n}{2}\sigma^2\theta^2$. 
  Az átparaméterezés után $H_0: \theta=0$ és $H_1:\theta\neq0$.
\end{example}
\begin{lemma}
  Ha $\phi$ torzítatlan próba, akkor $\E[0]{\phi \ell'(0)}=0$.
\end{lemma}
\begin{proof}
  $\psi(\theta)=\E[\theta]{\phi}$ folytonosan differenciálható, hiszen $\phi$ korlátos és 
  a gyenge regularitási feltétel teljesül és $\psi'(\theta)=\E[\theta]{\phi\ell'(\theta)}$.

  A torzítatlanság miatt $\psi$ minimumhelye $0$, ezért $0=\psi'(0)=\E[0]{\psi\ell'(0)}$.
\end{proof}


\begin{lemma}
  Legyen $\psi=\I{\ell'(0)\notin[c,d]}$ alakú. Ha $\psi$ torzítatlan, 
  akkor egyenletesen legerősebb  a vele azonos terjedelmű torzítatlan próbák között.
\end{lemma}
\begin{proof}
  $\bphi$ tetszőleges torzítatlan próba, $\phi$-vel azonos terjedelemmel.
  $\E[0]{\I{\ell'(0)\in[c,d]}\ell'(0)}=0$ miatt $c<0<d$. 
  
  $\theta_1<0$ esetén tekintsük a $H_0: f=f_0$, $H_1^*: f=pf_{\theta_1}+(1-p)f_{\theta_2}$ 
  egyszerű hipotézisvizsgálati feladatot. Erre a Neyman-Pearson lemma alkalmazható és 
  később belátjuk, hogy $\theta_2,p$ 
  alkalmas választása mellett, ennek a feladatnak a legerősebb megoldása éppen $\phi$. 

  Ekkor $p\E[\theta_1]{\phi-\bphi}+(1-p)\E[\theta_2]{\phi-\bphi}\geq0$. 
  Itt persze $p$ függ $\theta_2$-től és az derül ki, 
  hogy ha $\theta_2\to0$ akkor $p(\theta_2)/\theta_2$
  véges pozitív határértékkel rendelkezik. Átrendezés után
  \begin{displaymath}
    \E[\theta_1]{\phi-\bphi}\geq 
    \lim_{\theta_2\to0}
    (1-p(\theta_2))\frac{\theta_2}{p(\theta_2)}
    \frac{\E[\theta_2]{\phi-\bphi}-\E[0]{\phi-\bphi}}{\theta_2}
    =\lim_{\theta_2\to0}\frac{\theta_2}{p(\theta_2)}\E[0]{(\phi-\bphi)\ell'(0)}=0
  \end{displaymath}
  Ebből kapjuk, hogy $\phi$ ereje a $\theta_1<0$ pontban legalább akkora,
  mint $\bphi$-é.

  Hasonló módon kapható meg, hogy $\phi$ ereje tetszőleges $\theta>0$ pontban 
  is legalább akkora, mint $\bphi$-é.
\end{proof}

A fenti lemmában a következő összefüggést használtuk.
\begin{lemma}
  $\theta_1<0<\theta_2$. Ekkor létezik $p=p(\theta_2)\in(0,1)$ úgy, hogy a $H_0:f=f_0$, 
  $H_1^*: f=f_{*}=pf_{\theta_1}+(1-p)f_{\theta_2}$ egyszerű hipotézis vizsgálati feladatra 
  $\phi$ a likelihood hányados próba, azaz $\phi(x)=\I{\ell'(0)\notin[c,d]}
  =\I{pf_{\theta_1}+(1-p)f_{\theta_2}>cf_0}$ alakú.
  A $\lim_{\theta_2\to0} \frac{p(\theta_2)}{\theta_2}$ limesz létezik és véges.
\end{lemma}
\begin{proof}
  Ahogy fent már észrevettük feltehető, hogy $f_0=1$ és $\ell'(0,x)=T(x)$ a sűrűségek 
  felírásában a kitevőben megjelenő statisztika. Így
  \begin{displaymath}
    \frac{f_*}{f_0}(x)=f_*(x)=\zjel*{pe^{\theta_1 t-b(\theta_1)}+(1-p)e^{\theta_2 t-b(\theta_2)}}|_{t=T(x)}=g(T(x))
  \end{displaymath}
  Azaz a hányados $T(x)$ konvex függvénye, és ha $g(c)=g(d)$, akkor $\event{f_*>g(c)=g(d)}=\event{T\notin[c,d]}$.
  Ez utóbbi abból adódik, hogy $\lim g(t)=\infty$, ha $\abs{t}\to\infty$. 

  Azt kell tehát csak meggondolni, hogy $p$ alkalmas választásával $g(c)=g(d)$:
  %, ami a következő összefüggéssel ekvivalens:
  \begin{displaymath}
    pe^{\theta_1 c-b(\theta_1)}+(1-p)e^{\theta_2 c-b(\theta_2)}
    =pe^{\theta_1 d-b(\theta_1)}+(1-p)e^{\theta_2 d-b(\theta_2)}
  \end{displaymath}
  Átrendezés után
  \begin{displaymath}
    \frac{p}{1-p} = \frac{e^{\theta_2 d}-e^{\theta_2 c}}{e^{\theta_1 c}-e^{\theta_1d}} e^{b(\theta_1)-b(\theta_2)}
  \end{displaymath}
  Mivel $\theta_1<0<\theta_2$ a jobboldali kifejezés pozitív és  $p=1/(1+(1-p)/p)$ 
  \begin{displaymath}
    p(\theta_2)=\frac{1}{1+\frac{e^{\theta_1 c}-e^{\theta_1d}}{e^{\theta_2 d}-e^{\theta_2 c}} e^{b(\theta_2)-b(\theta_1)}}
    =\frac{e^{\theta_2 d}-e^{\theta_2 c}} 
    {e^{\theta_2 d}-e^{\theta_2 c}+(e^{\theta_1 c}-e^{\theta_1d})(e^{b(\theta_2)-b(\theta_1)})}
  \end{displaymath}
  Innen $p(\theta_2)\to0$ ha $\theta_2\to0$ és  L'Hospital szabállyal
  \begin{displaymath}
    \lim_{\theta_2\to0}\frac{p(\theta_2)}{\theta_2}
    =\frac{d-c}{d-c+(e^{\theta_1 c}-e^{\theta_1d})e^{b(0)-b(\theta_1)}b'(0)}=1.
  \end{displaymath}
  Az hogy a limesz pont egy lett azon múlt, hogy olyan paraméterezést választottunk, ami mellett $T=\ell'(0)$, 
  ami nulla várható értékű $\P[0]$ alatt.
\end{proof}
\begin{example}
  $X_1,\dots,X_n$ minta az $N(\mu,\sigma^2)$ eloszlásból, $\sigma$ ismert. 
  Láttuk, hogy $\theta=\frac{\mu-\mu_0}{\sigma^2}$, $\ell'(0,x)=T(x)=\sum_i (x_i-\mu_0)$ és 
  $b(\theta)=\frac{n}{2\sigma^2}(\mu-\mu_0)^2=\frac{n}{2}\sigma^2\theta^2$ paraméterezéssel a $H_0:\mu=\mu_0$ $H_1:\mu\neq \mu_0$
  a $H_0:\theta=0$, $H_1:\theta\neq 0$ alakot ölti és a sűrűségfüggvények
  \begin{displaymath}
    f_\theta(x)=\exp{\theta T(x)-b(\theta)}
  \end{displaymath}
  alakot öltik, ahol $T(x)=\sum_i x_i-\mu_0$.
  
  $T=\ell'(0)$ eloszlása $H_0$ mellett normális nulla várható értékkel, $\sigma$ szórással. 
  Ha a $\phi=\I{\abs{T}>c}$, akkor $\phi$ torzítatlan próba, 
  mert $\psi(\theta)=\E[\theta]{\phi}$ folytonosan deriválható, $\psi'(\theta)=\E[\theta]{\phi \ell'(\theta)}$.
  Itt $\ell'(\theta)=T-b'(\theta)$, ami nulla várható értékű normális eloszlású $\sigma$ szórással.   
  \qed
\end{example}


\chapter*{Próbák exponenciális családban zavaró paraméterek mellett}

Azt az esetet vizsgáljuk, amikor a sűrűségek
\begin{displaymath}
  f_{\theta,\tau}(x)=\exp{\theta T(x)+\tau S(x)-b(\theta,\tau)}
\end{displaymath}
alakúak és a hipotézisek a $\theta$-ra vonatkoznak, $H_0:\theta=0$, 
$H_1:\theta>0$ (egyoldali ellenhipotézis), vagy $H_1:\theta\neq0$ 
(kétoldali ellenhipotézis).

\begin{example}
  $t$-próba esetén $\mu,\sigma$ ismeretlen, a hipotézisek $\mu$-re vonatkoznak, $H_0:\mu=0$.
  \begin{displaymath}
    f_{\mu,\sigma}
    =C\exp{-\frac{1}{2\sigma^2}\sum_i (x_i-\mu)^2}
    =C\exp{-\frac1{2\sigma^2}\sum_i x_i^2+\frac{\mu}{\sigma_2}\sum_i x_i-\frac{n\mu^2}{2\sigma^2}}
  \end{displaymath}
  Itt  pl. lehet $\theta=\frac{n\mu}{\sigma^2}$, 
  $\tau=-\frac{1}{2\sigma^2}$ és $T(x)=\bar{x}$, $S(x)=\sum_i x_i^2$. Ezzel a 
  sűrűségeket a fenti alakba írtuk.
\end{example}
\begin{example}
  Kétmintás $u$ próba esetén $\mu_1,\mu_2$ ismeretlen, $\sigma_1,\sigma_2$ ismert. A
  hipotézisek a $\mu_1-\mu_2$ különbségről szólnak.
  \begin{align*}
    f_{\mu}(x,y)
    &=C\exp{-\frac12\zjel{\sum_i \frac{(x_i-\mu_1)^2}{\sigma_1^2} +
    \sum_j \frac{(y_j-\mu_2)^2}{\sigma_2^2}}}\\
    &=
    h(x,y,\sigma_1,\sigma_2)\exp{\frac{\mu_1}{\sigma_1^2}\sum_i x_i+\frac{\mu_2}{\sigma_2^2}\sum_j y_j-b(\mu_1,\mu_2,\sigma_1,\sigma_2)}\\
    &=h\exp{(\mu_1-\mu_2)
    \zjel*{\frac1{2\sigma_1^2}\sum_i x_i-\frac1{2\sigma_2^2}\sum_j y_j}
    +(\mu_1+\mu_2)
    \zjel*{\frac1{2\sigma_1^2}\sum_i x_i+\frac1{2\sigma_2^2}\sum_j y_j}-b}
  \end{align*}
  Azaz $\theta=\mu_1-\mu_2$, $\tau=\mu_1+\mu_2$ és                                                                            
  $T(x,y)=\frac{n}{2\sigma_1^2}\bar{x}-\frac{m}{2\sigma_2^2}\bar{y}$ 
  és $S(x,y)=\frac{n}{2\sigma_1^2}\bar{x}+\frac{m}{2\sigma_2^2}\bar{y}$ választással 
  a sűrűségfüggvények a fenti alakba írhatóak ($n$, $m$ az $X$ ill. $Y$ minta elemszáma). 
  A $h$ szorzótényező a domináló mértékbe beolvasztható.
\end{example}

\begin{example}
  Tegyük fel, hogy  szórások egyezését vizsgáljuk $F$ próbával, de $\mu_1=\mu_2=0$ ismert.
  Ekkor
  \begin{align*}
    f_{\sigma_1,\sigma_2}(x,y)
    &=C\exp{-\frac1{2\sigma^2_1}\sum_i x_i^2-\frac1{2\sigma_2^2}\sum_j y_j^2}\\
    &=C\exp{-\frac1{2\sigma^2_1\sigma_2^2}
    \zjel*{\sigma_2^2\sum_i x_i^2+\sigma_1^2\sum_j y_j^2}}\\
    &=C\exp{-\frac1{4\sigma^2_1\sigma_2^2}
    \zjel*{(\sigma_2^2-\sigma^2_1)\zjel{\sum_i x_i^2-\sum_j y_j^2}+
    (\sigma_2^2+\sigma^2_1)\zjel{\sum_i x_i^2+\sum_j y_j^2)}}}
    %&=C\exp{-\frac14\zjel{\zjel{\frac1{\sigma_1^2}-\frac1{\sigma_2^2}(\sum_i x_i^2-\sum_j y_j^2)+
    %}}
  \end{align*}
  Azaz  pl. a $\theta=\frac{1}{4}\cdot\zjel{\frac1{\sigma_2^2}-\frac1{\sigma_1^2}}$,
  $\tau=\frac{1}{4}\cdot\zjel{\frac1{\sigma_1^2}+\frac1{\sigma_2^2}}$ és 
  $T(x)=\sum_i x_i^2-\sum_j y_j^2$, $S(x)= \sum_i x_i^2+\sum_j y_j^2$ választással a sűrűségek most is 
  a fenti alakba írhatóak.
\end{example}

Feltehető, hogy $(0,0)\in\Theta$ és a domináló mérték $\P[(0,0)]$. 
$(T,S)$ együttes eloszlásainak a családja dominált, 
pl. $\P[(0,0)]\circ (T,S)^{-1}$ domináló mértékkel és erre a domináló 
mértékre nézve a sűrűségek:
\begin{displaymath}
  f_{T,S;(\theta,\tau)}(t,s)=e^{\theta t+\tau s-b(\theta,\tau)}
\end{displaymath}
Valóban
\begin{displaymath}
  \P[(\theta\tau)]{(T,S)\in H}
  =\E[(0,0)]{e^{\theta T+\tau S-b(\theta,\tau)}\I{(T,S)\in H}}=
  \int \I{(t,s)\in H}e^{\theta t+\tau s-b(\theta,\tau)} (\P[(0,0)]\circ(T,S)^{-1})(dt,ds)
\end{displaymath}


Legyen $T$ 
feltételes eloszlása $S$-re nézve a $\P[(0,0)]$ mérték alatt $Q_{T|S}$. 
Ez a feltételes eloszlás  reguláris változata, 
$Q_{T|S}(H,S)=\P[(0,0)]{T\in H\given S}$.
A faktorizációs tételhez hasonlóan:
\begin{align*}
  \P[(\theta,\tau)]{T\in H,\,S\in A}
  &=\int e^{\theta t+\tau s- b(\theta,\tau)} (\P[(0,0)]\circ(T,S)^{-1})(d t,d s)\\
  &=\int \int e^{\theta t} Q_{T|S}(d t,s) e^{\tau s-b(\theta,\tau)} (\P[0,0]\circ S^{-1})(d s)\\
  &=\int \int e^{\theta t-b_1(\theta,s)} Q_{T|S}(d t,s)  e^{\tau s-b_2(\tau)} (\P[0,0]\circ S^{-1})(d s)
\end{align*}
Itt $b_1(\theta,s)$ a normalizáló konstans $b_1(\theta,s)=-\ln\int e^{\theta t} Q_{T|S}(d t,s)$ és $b_2(\tau)$ 
a normalizáló konstans $S$ eloszlásában $b_2(\tau)=-ln\int e^{s\tau} (\P[0,0]\circ S^{-1})(d s)=\E[(0,0)]{e^{\tau S}}$.

Azt kaptuk, hogy az $S$ feltétel mellett $T$ eloszlásainak családja egy paraméteres exponenciális családot alkot 
(a paraméter $\theta$). Az is kiderült, hogy $S$ eloszlásainak a családja is exponenciális családot alkot, a paraméter $\tau$. 
Ha itt a paramétertér $\set{\tau}{\exists \theta, \,(\theta,\tau)\in\Theta}$ belseje nem üres, 
akkor $S$ eloszlásainak családja teljes.

\begin{example}
  $t$-próba esetén $\mu,\sigma$ ismeretlen, a hipotézisek $\mu$-re vonatkoznak, $H_0:\mu=0$.
  \begin{displaymath}
    f_{\mu,\sigma}(x)
    =C\exp{-\frac1{2\sigma^2}\sum_i x_i^2+\frac{\mu}{\sigma_2}\sum_i x_i-\frac{n\mu^2}{2\sigma^2}}
    =C \exp{\theta T(x)+\tau S(x)-b(\theta,\tau)}
  \end{displaymath}
  ahol $\theta=\frac{n\mu}{\sigma^2}$, 
  $\tau=-\frac{n}{2\sigma^2}$ és $T(x)=\frac1n\sum x_i=\bar{x}$, $S(x)=\frac1n\sum_i x_i^2=\bar{x^2}$. 

  Itt $s_n^2= S-T^2$ független $T$-től és $n(S-T^2)\sim \sigma^2 \chi^2_{n-1}$, míg $T\sim N(\mu,\frac1n \sigma^2)$. 
  Innen  a sűrűségfüggvény transzformációs formulával 
  \begin{align*}
    f_{T,S}(t,s)
    & =f_{T,S-T^2}(t,s-t^2)\\
    & =
    \frac1{\sqrt{2\pi\sigma^2/n}}\exp{-\frac n2\cdot\frac{t-\mu}{\sigma^2}}
    \frac{1}{\Gamma(n-1)}\zfrac*{n}{2\sigma^2}^{n-1} (s-t^2)^{n-2} \I{s-t^2>0} \exp{-\frac{n}{2\sigma^2}(s-t^2)}\\
    &=\sqrt{\frac n{2\pi\sigma^2}} \frac1{(n-2)!}\zfrac*{n}{2\sigma^2}^{n-1} (s-t^2)^{n-2}\I{t<\sqrt{s}} 
    \exp{\frac {n\mu}{\sigma^2} t-\frac{n}{2\sigma^2}s -\frac{n\mu^2}{2\sigma^2}}
  \end{align*}
  Ebből a feltételes sűrűségfüggvény
  \begin{displaymath}
    f_{T|S}(t|s)= C (s-t^2)^{n-2}\I{0<t<\sqrt{s}} e^{\frac{n\mu}{\sigma^2} t}= C(\theta,s) (s-t^2)^{n-2}e^{\theta t}\I{0<t<\sqrt{s}}.
  \end{displaymath}
  Ez pontosan olyan, mint amit az általános esetre kiszámoltunk, 
  ugyanis $(s-t^2)^{n-2}\I{0<t<\sqrt{s}}$ beolvasztható a domináló mértékbe.
\end{example}

$T$ $S$-re vonatkozó feltételes eloszlásainak családja egyparaméteres családot alkot a sűrűségek 
$Q_{T|S}(\cdot,s)$ domináló mértékre nézve $e^{\theta t-b_1(\theta,s)}$ alakúak. 
Tetszőleges $\phi=\phi(T,S)$ alakú döntési függvény esetén 
$\E[(\theta,\tau)]{\phi(T,S)}=\E[(\theta,\tau)]{\E[(\theta,\tau)]{\phi(T,S)|S}}
=\E[(0,\tau)]{\E[\theta,0]{\phi(T,S)|S}}$ hiszen $S$ eloszlása nem függ $\theta$-tól és $T$ feltételes eloszlása 
nem függ $\tau$-tól. 

 
Legyen $\phi=\phi(T,S)$ torzítatlan próba $\alpha$ terjedelemmel és $\Theta_0$ határa $\Theta$-ban $\Theta_0$, azaz 
tetszőleges $\tau$-ra és és $\delta>0$-ra létezik $(\theta',\tau')\in\Theta$, amire $(0,\tau)$ távolsága 
$(\theta,\tau)$-tól legfeljebb $\delta$. Ekkor $\E[(0,\tau)]{\phi(T,S)}\leq\alpha$, mert a terjedelem $\alpha$, másfelől 
$\psi(\theta,\tau)=\E[(\theta,\tau)]{\phi}$ folytonossága és a próba torzítatlansága miatt legalább $\alpha$. Így
\begin{displaymath}
  \E[(0,\tau)](\phi(T,S))=\alpha,\quad \text{minden $\tau$-ra}
\end{displaymath}
a $g(S)=\E[(0,\tau)]{\phi(T,S)|S}-\alpha =\E[(0,0)]{\phi(T,S)|S}-\alpha$ korlátos statisztikára $\E[(0,\tau)]{g(S)}=0$. 
Mivel $S$ eloszlásainak a családja exponenciális családot alkot (a paraméterterének belseje nem üres) ezért $g(S)=0$ és 
$\int \phi(t,s)Q_{T|S}(dt,s)=\alpha$. Azt kaptuk, hogy $\phi(T,S)$ a feltételes feladatra nézve is $\alpha$ terjedelmű. 
A feltételes feladatra, egy oldali ellenhipotézis esetén, a $\phi(T,S)=\I{T>c(S)}$ döntési függvény a legerősebb próba, 
ha $c(S)$ olyan, hogy a kapott próba $\alpha$ terjedelmű. 
A $t$ próba feladatánál $T/\sqrt{S-T^2}$ eloszlása mindegyik $(0,\tau)$ mellett ugyanaz, méghozzá
Student-féle $t$ eloszlású $n-1$ 
szabadságokkal, $T$-ben monoton nő,  ezért $T>c(S)$ valójában $T/\sqrt{S-T^2}>t_\alpha$. 

Összefoglalva tetszőleges $\alpha$ terjedelmű torzítatlan $\phi(T,S)$ próba esetén, $\phi(T,S)$  az 
$S$ feltétel melletti feladatra is 
$\alpha$ terjedelmű. A feltételes feladatra $\phi(T,S)$ nem lehet erősebb, mint a $t$-próba ami  
$\bphi(T,S)=\I{T/\sqrt{T-S^2}>t_\alpha}$ alakú. Ezért 
\begin{displaymath}
  \E[(\theta,\tau)]{\phi(T,S)}=\E[(0,\tau)]{\E[\theta,0]{\phi(T,S)|S}}
  \leq \E[(0,\tau)]{\E[\theta,0]{\bphi(T,S)|S}}=\E[(\theta,\tau)]{\bphi(T,S)}.
\end{displaymath}




\end{document}