

\makeatletter
\def\readrest#1\enddate{}
\def\setdate@#1-#2-#3\relax{\def\lecturedate{#1.#2.#3.}\readrest}
\def\setdate{\expandafter\setdate@\jobname\relax-\relax-\relax\enddate}

\makeatother

\setdate


\usepackage{marvosym}
%\usepackage[magyar]{babel}
\usepackage{polyglossia}
\setmainlanguage{magyar}

% \usetheme{Bruno}

\usecolortheme{Bruno}
% \usefonttheme{Bruno}
% \useinnertheme{Bruno}
\useoutertheme{Bruno}

\usepackage{pvdefs}
\newtheorem{df}{Definíció}
\newtheorem{theorem}{Tétel}
\newtheorem{lemma}{Lemma}
\newtheorem{proposition}{Állítás}
\newtheorem{corollary}{Következmény}

\defcx{XAPIDLCSGTHEN}
\deftx{STQbCf\lambda\Omega\sigma}
\defmf{X}

\def\diag{\operatorname{diag}}
\def\Id{\mathbf{I}}
\let\PEfont\mathbb
\def\npto{\buildrel p\over \nrightarrow}
\def\ndto{\buildrel d\over \nrightarrow}
\def\dto{\buildrel d\over \rightarrow}
\def\dto{\stackrel{d}{\to}}
\let\nto\nrightarrow

\def\Leb{\operatorname{Leb}}

\newrobustcmd{\continue}{%
  \onslide<\value{beamerpauses}->\relax
}


\beamerdefaultoverlayspecification{<+->}

\mode<presentation>

\RequirePackage[math-style=TeX, bold-style=upright]{unicode-math}
\setmathfont{Latin Modern Math}[Scale=1]
% \setsansfont{Kurier Cond}
\setsansfont[Ligatures=TeX, ItalicFont={Fira Sans Italic}, BoldFont={Fira Sans SemiBold}, Scale=0.9]{Fira Sans Book}
%\setsansfont{Ubuntu}

%\usepackage{mathspec}
%\setmathsfont{lm}
%(Digits,Latin,Greek)
%[Numbers={Lining,Proportional}]{lm}
%\setmathsfont(Digits,Greek)
%[Uppercase=Plain,Lowercase=Regular,Scale=MatchLowercase]
%{GFS Porson}


% Title page
\setbeamerfont{title}{
    size=\LARGE,
    shape=\bfseries
}

\setbeamerfont{subtitle}{
    size=\large,
    shape=\normalfont
}

\setbeamerfont{author}{
    size=\normalsize,
    shape=\normalfont
}

\setbeamerfont{details}{
    size=\footnotesize,
    shape=\normalfont
}

% Slide title
\setbeamerfont{frametitle}{size=\LARGE}
\setbeamerfont{framesubtitle}{
    size=\normalsize,
    %shape=\normalfont\scshape
}

% Blocks
\setbeamerfont{block title}{
    size=\normalfont,
    shape=\strut
}

\setbeamerfont{blockdef}{
    size=\large,
    shape=\bfseries
}

% Description
\setbeamerfont{description item}{shape=\bfseries}


% Footer information line with title, author and slide number
\setbeamertemplate{footline}{%
    \leavevmode%
    \hbox{%
        \usebeamerfont{footline}%
        \begin{beamercolorbox}[
            wd=\textwidth,
            ht=3ex,
            dp=1.25ex
        ]{footline}%
            \hspace{1cm}%
            \insertshorttitle%
            \hspace{2em}%
            \insertdate%
            \hfill%
            \insertframenumber{} / \inserttotalframenumber%
            \hspace{1cm}
        \end{beamercolorbox}%
    }
    \vskip0pt%
}

%% Bruno Beamer theme — Version 1.0
%% A lightweight Beamer theme inspired from the Metropolis theme
%%
%% Written in 2017-2019 by:
%% — Rémi Cérès <remiceres@msn.com>
%% — Mattéo Delabre <bonjour@matteodelabre.me>
%%
%% This work is released under the CC0 1.0 Universal license. See the
%% accompanying LICENSE file for details. To the extent possible under
%% law, Rémi Cérès and Mattéo Delabre have waived all copyright and
%% related or neighboring rights to the Bruno Beamer theme. This work
%% is published from France.

% \mode<presentation>

\setbeamertemplate{sections/subsections in toc}[square]
\setbeamertemplate{itemize item}[square]
\setbeamertemplate{itemize subitem}[circle]

% Allow multi-slide frames
\setbeamertemplate{frametitle continuation}{}

% Horizontal separator
\setbeamertemplate{separator}{%
    \usebeamercolor{separator}%
    \textcolor{fg}{\rule{.7\textwidth}{.5pt}}%
}

% Separate authors with newlines
\renewcommand{\beamer@andtitle}{\\}

% Title page. If you want to add a background image on the title slide,
% use the \background command to set the path to that image. Otherwise,
% no image will be included
\RequirePackage{tikz}
\usetikzlibrary{fadings}
\newcommand{\background}[1]%
    {\newcommand{\bruno@background}{#1}}


%% \newcommand\insertcaption{}
\newcommand{\backgroundcaption}[2][white]{\def\bruno@caption{{#1}{#2}}}

\newcommand{\insertcaption}{\ifdefined\bruno@caption{\expandafter\textcolor\bruno@caption}\fi}

\tikzfading[
    name=title page picture fading,
    left color=transparent!0,
    right color=transparent!100,
]

\setbeamertemplate{title page}{%
    \begin{minipage}{.7\textwidth}
        \raggedright
        \usebeamerfont{title}
        \inserttitle

        \usebeamertemplate{separator}

        \usebeamerfont{author}
        \vspace{2em}
        \insertauthor

        \vspace{2em}
        \usebeamerfont{details}
        \insertinstitute

        \insertdate
    \end{minipage}

    % Include faded image on the right if defined
    \ifdefined\bruno@background
    \begin{tikzpicture}[remember picture, overlay]
        % Crop image to a trapezium on the right
        \clip (current page.south east)
           -- (current page.north east)
           -- ++(-5.1, 0)
           -- ++(-3, -\paperheight)
           -- cycle;

        % Include background image
        \node[
            anchor=south east,
            inner sep=0,
            outer sep=0
        ] at (current page.south east) {
            \includegraphics[height=\paperheight]
                {\bruno@background}
        };

        \node[anchor=south east,
            inner sep=0,
            outer sep=15pt,
        ] at (current page.south east){\insertcaption};

        % Add a slight shadow
        \fill[
            black, path fading=title page picture fading,
            fading angle=-15
        ]
            (current page.south east)
            rectangle
            ++(-10, \dimexpr\paperheight+1cm);

        
    \end{tikzpicture}
    \fi
}

\mode<all>
\AtBeginDocument{%
  \let\phi\varphi
  \let\theta\vartheta
  \let\setminus\smallsetminus
}


\title{Statisztika előadás}
\date{\lecturedate}
\background{images/young_Ronald_Fisher.jpg}
\backgroundcaption{Ronald Fisher (1913)}
