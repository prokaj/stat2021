%\def\option{}
\providecommand{\option}{handout}
\documentclass[aspectratio=169,notheorems,9pt,\option]{beamer}



\makeatletter
\def\readrest#1\enddate{}
\def\setdate@#1-#2-#3\relax{\def\lecturedate{#1.#2.#3.}\readrest}
\def\setdate{\expandafter\setdate@\jobname\relax-\relax-\relax\enddate}

\makeatother

\setdate


\usepackage{marvosym}
%\usepackage[magyar]{babel}
\usepackage{polyglossia}
\setmainlanguage{magyar}

% \usetheme{Bruno}

\usecolortheme{Bruno}
% \usefonttheme{Bruno}
% \useinnertheme{Bruno}
\useoutertheme{Bruno}

\usepackage{pvdefs}
\newtheorem{df}{Definíció}
\newtheorem{theorem}{Tétel}
\newtheorem{lemma}{Lemma}
\newtheorem{proposition}{Állítás}
\newtheorem{corollary}{Következmény}

\defcx{XAPIDLCSGTHEN}
\deftx{STQbCf\lambda\Omega\sigma}
\defmf{X}

\def\diag{\operatorname{diag}}
\def\Id{\mathbf{I}}
\let\PEfont\mathbb
\def\npto{\buildrel p\over \nrightarrow}
\def\ndto{\buildrel d\over \nrightarrow}
\def\dto{\buildrel d\over \rightarrow}
\def\dto{\stackrel{d}{\to}}
\let\nto\nrightarrow

\def\Leb{\operatorname{Leb}}

\newrobustcmd{\continue}{%
  \onslide<\value{beamerpauses}->\relax
}


\beamerdefaultoverlayspecification{<+->}

\mode<presentation>

\RequirePackage[math-style=TeX, bold-style=upright]{unicode-math}
\setmathfont{Latin Modern Math}[Scale=1]
% \setsansfont{Kurier Cond}
\setsansfont[Ligatures=TeX, ItalicFont={Fira Sans Italic}, BoldFont={Fira Sans SemiBold}, Scale=0.9]{Fira Sans Book}
%\setsansfont{Ubuntu}

%\usepackage{mathspec}
%\setmathsfont{lm}
%(Digits,Latin,Greek)
%[Numbers={Lining,Proportional}]{lm}
%\setmathsfont(Digits,Greek)
%[Uppercase=Plain,Lowercase=Regular,Scale=MatchLowercase]
%{GFS Porson}


% Title page
\setbeamerfont{title}{
    size=\LARGE,
    shape=\bfseries
}

\setbeamerfont{subtitle}{
    size=\large,
    shape=\normalfont
}

\setbeamerfont{author}{
    size=\normalsize,
    shape=\normalfont
}

\setbeamerfont{details}{
    size=\footnotesize,
    shape=\normalfont
}

% Slide title
\setbeamerfont{frametitle}{size=\LARGE}
\setbeamerfont{framesubtitle}{
    size=\normalsize,
    %shape=\normalfont\scshape
}

% Blocks
\setbeamerfont{block title}{
    size=\normalfont,
    shape=\strut
}

\setbeamerfont{blockdef}{
    size=\large,
    shape=\bfseries
}

% Description
\setbeamerfont{description item}{shape=\bfseries}


% Footer information line with title, author and slide number
\setbeamertemplate{footline}{%
    \leavevmode%
    \hbox{%
        \usebeamerfont{footline}%
        \begin{beamercolorbox}[
            wd=\textwidth,
            ht=3ex,
            dp=1.25ex
        ]{footline}%
            \hspace{1cm}%
            \insertshorttitle%
            \hspace{2em}%
            \insertdate%
            \hfill%
            \insertframenumber{} / \inserttotalframenumber%
            \hspace{1cm}
        \end{beamercolorbox}%
    }
    \vskip0pt%
}

%% Bruno Beamer theme — Version 1.0
%% A lightweight Beamer theme inspired from the Metropolis theme
%%
%% Written in 2017-2019 by:
%% — Rémi Cérès <remiceres@msn.com>
%% — Mattéo Delabre <bonjour@matteodelabre.me>
%%
%% This work is released under the CC0 1.0 Universal license. See the
%% accompanying LICENSE file for details. To the extent possible under
%% law, Rémi Cérès and Mattéo Delabre have waived all copyright and
%% related or neighboring rights to the Bruno Beamer theme. This work
%% is published from France.

% \mode<presentation>

\setbeamertemplate{sections/subsections in toc}[square]
\setbeamertemplate{itemize item}[square]
\setbeamertemplate{itemize subitem}[circle]

% Allow multi-slide frames
\setbeamertemplate{frametitle continuation}{}

% Horizontal separator
\setbeamertemplate{separator}{%
    \usebeamercolor{separator}%
    \textcolor{fg}{\rule{.7\textwidth}{.5pt}}%
}

% Separate authors with newlines
\renewcommand{\beamer@andtitle}{\\}

% Title page. If you want to add a background image on the title slide,
% use the \background command to set the path to that image. Otherwise,
% no image will be included
\RequirePackage{tikz}
\usetikzlibrary{fadings}
\newcommand{\background}[1]%
    {\newcommand{\bruno@background}{#1}}


%% \newcommand\insertcaption{}
\newcommand{\backgroundcaption}[2][white]{\def\bruno@caption{{#1}{#2}}}

\newcommand{\insertcaption}{\ifdefined\bruno@caption{\expandafter\textcolor\bruno@caption}\fi}

\tikzfading[
    name=title page picture fading,
    left color=transparent!0,
    right color=transparent!100,
]

\setbeamertemplate{title page}{%
    \begin{minipage}{.7\textwidth}
        \raggedright
        \usebeamerfont{title}
        \inserttitle

        \usebeamertemplate{separator}

        \usebeamerfont{author}
        \vspace{2em}
        \insertauthor

        \vspace{2em}
        \usebeamerfont{details}
        \insertinstitute

        \insertdate
    \end{minipage}

    % Include faded image on the right if defined
    \ifdefined\bruno@background
    \begin{tikzpicture}[remember picture, overlay]
        % Crop image to a trapezium on the right
        \clip (current page.south east)
           -- (current page.north east)
           -- ++(-5.1, 0)
           -- ++(-3, -\paperheight)
           -- cycle;

        % Include background image
        \node[
            anchor=south east,
            inner sep=0,
            outer sep=0
        ] at (current page.south east) {
            \includegraphics[height=\paperheight]
                {\bruno@background}
        };

        \node[anchor=south east,
            inner sep=0,
            outer sep=15pt,
        ] at (current page.south east){\insertcaption};

        % Add a slight shadow
        \fill[
            black, path fading=title page picture fading,
            fading angle=-15
        ]
            (current page.south east)
            rectangle
            ++(-10, \dimexpr\paperheight+1cm);

        
    \end{tikzpicture}
    \fi
}

\mode<all>
\AtBeginDocument{%
  \let\phi\varphi
  \let\theta\vartheta
  \let\setminus\smallsetminus
}


\title{Statisztika előadás}
\date{\lecturedate}
\background{images/young_Ronald_Fisher.jpg}
\backgroundcaption{Ronald Fisher (1913)}


\begin{document}

\maketitle

\begin{frame}{Emlékeztető}
  $\cP=\set{\P[\theta]}{\theta\in\Theta}$, $\Theta\subset\real^p$ nyílt, eloszláscsalád az $\mfX$ mintatéren, 
  $\lambda$ domináló mértékkel.

  $\ell:\Theta\times\mfX\to\real$, $\ell(\theta)=\ell(\theta,x)=\ln f_\theta(x)$ a loglikelihood függvény, 
  $f_\theta=\frac{d\P[\theta]}{d\lambda}$. 
  
  \begin{df}[Fisher információ]
    $I(\theta)=\E[\theta]{\ell'(\theta)^T\ell'(\theta)}\in\real^{p\times p}$
    a minta Fisher információja, ha a formula értelmes.
  \end{df}

  \begin{df}[Exponenciális család]
    $\cP$ exponenciális családot alkot, ha alkalmas $\lambda$ és paraméterezés 
  mellett  $\ell(\theta,x)=\theta T(x)-b(\theta)$ alakú. 
  \end{df}
  
  \begin{df}[Gyenge regularitási feltétel, $(R)$]
    $\lambda$ m.m. $x$-re $\theta\mapsto \sqrt{f_\theta(x)}\in C^1$.
    $I(\theta)$ véges, $\theta$-ban folytonos, és $I(\theta)$ nem szinguláris ($\det I(\theta)>0$).
  \end{df}
  
  \begin{theorem}
    $(R)$ + $\theta\mapsto\E[\theta]{\abs{T}^2}$ lokálisan korlátos 
    $\implies$  $g(\theta)=\E[\theta]{T}$ folytonosan deriválható és $g'(\theta)=\E[\theta]{T\ell'(\theta)}$.
  \end{theorem}
\end{frame}

\begin{frame}{Gyenge regularitási feltétel exponenciális családban}
  \begin{proposition}[Emlékeztető]
    Ha $Y$ $p$-dimenziós vektorváltozó és $Y\in L^2$, akkor $\P{Y\in \E{Y}+\im\Sigma(Y)}=1$ 
  \end{proposition}
  $a\in\real^p$-re $\D^2{a^T Y}=a^T\Sigma(Y)a$ $\implies$ 
  $\forall a\perp \im\Sigma(Y)$ vektorra $\D^2{a^T Y}=0$ és $\P{a^T (Y-\E{Y})=0}=1$. 
  
  $a_1,\dots,a_r$ $(\im\Sigma(Y))^{\perp}$  bázisa, %ortokomplementerében, 
  $\P{a_i^T (Y-\E{Y})=0,\,i=1,\dots,r}=1$ 
  és így $\P{Y-\E{Y}\in \im\Sigma(Y)}=1$.

  \begin{itemize}
  \item 
   %Tegyük fel, hogy 
    $\cP=\set{f_\theta d\lambda}{\theta\in\Theta}$ exponenciális családot alkot és $\Theta\subset\real^p$ nyílt. 
    \begin{displaymath}
      f_\theta(x)=\exp{\theta T(x)-b(\theta)},\quad \sqrt{f_\theta(x)}=\exp{\tfrac12\theta T(x)-\tfrac12b(\theta)}
    \end{displaymath}
    Itt $b:\Theta\to\real$ %(\theta)$ 
    sima, azaz $\theta\mapsto \sqrt{f_\theta}(x)$ folytonosan differenciálható.
    
    \item $I(\theta)=b''(\theta)=\Sigma_\theta(T)$, azaz $I(\theta)$ véges értékű és folytonos.
    
    \item A $\P[\theta]$ eloszlások ekvivalensek, ugyanez igaz $T$ eloszlásaira is.  
    % $T$ eloszlásainak a családja abszolút folytonos a $\P[\theta_0]\circ T^{-1}$ mértékre nézve. $T$ sűrűségfüggvénye
    % \begin{displaymath}
    %   \frac{d\P[\theta]\circ T^{-1}}{d\P[0]\circ T^{-1}}=\exp{(\theta-\theta_0)T(x)-(b(\theta)-b(\theta_0))}
    % \end{displaymath}
    Tetszőleges $V$ affin altérre
    $\P[\theta_0]{T\in V}=1$ $\iff$ $\P[\theta]{T\in V}=1$ minden $\theta\in\Theta$-ra.  
    \item $(R)$ teljesül, ha $T$ a $\real^p$ nem egy valódi affin altérből veszi fel az értékeit, 
    pl. ha a paraméterezés egyértelmű, akkor $(R)$ teljesül.
    \item Ha $T\in V\subset\real^p$ valódi $V$ affin altérrel,
     akkor alacsonyabb dimenziós paraméterezést is választhatunk és $(R)$ teljesül. 
  \end{itemize}  
  \continue
  Azaz, ha exponenciális családunk van nyílt paramétertérrel, akkor $(R)$ lényegében mindig teljesül.
\end{frame}

\begin{frame}{További példák}
  \begin{itemize}
    \item $X$ Cauchy eloszlás eltolás paraméteres családjából származó $n$ elemű minta. 
    $X$ eloszlásai nem alkotnak exponenciális családot, de
    \begin{displaymath}
      \sqrt{f_\theta(x)}=\prod \pi^{-1/2} (1+(x_i-\theta)^2)^{-1/2}
    \end{displaymath}
    folytonosan deriválható, és 
    \begin{displaymath}
      \hphantom{\hspace{-2em}}
      \ell(\theta,x)=c+\sum_{i} \frac{-1}{2}\ln(1+(x_i-\theta)^2),
      \quad
      \ell'(\theta,x)=\sum_i \frac{x_i-\theta}{1+(x_i-\theta)^2}
    \end{displaymath}
    \begin{displaymath}
      \E[\theta]{\ell'(\theta)} = n\int_{\real} \frac{x}{\pi (1+x^2)^2} d x=0
      \quad
      I(\theta) = \D[\theta]^2{\ell'(\theta)}
      =n\int_{\real} \frac{x^2}{(1+x^2)^2}\frac{1}{\pi(1+x^2)}d x<\infty 
    \end{displaymath}
    $I(\theta)$ véges, folytonos és nem nulla. 
    $X$ eloszlásainak családja teljesíti a gyenge regularitási feltételt.
  \item $X$ a $(0,\theta)$ intervallumon egyenletes eloszlású, $\theta>0$. 
  Ekkor  $\P[\theta]{X<y}=1\wedge \frac{y}{\theta}$
  \begin{displaymath}
    %\P[\theta]{X<y}=1\wedge \frac{y}{\theta},
    %\quad
    \partial_\theta\P[\theta]{X<y}=\frac{-y}{\theta^2}\I{y<\theta},
    \quad 
    \E[\theta]{\I{X<y}\ell'(\theta)}
    =\int_0^\infty \I{x<y} \partial_\theta f_\theta(x) d x 
    = \int_0^\theta \I{x<y}\frac{-1}{\theta^2} d x
    = \frac{-y\wedge\theta}{\theta^2} 
  \end{displaymath}
  $X$ eloszlásainak a családja nem teljesíti $(R)$-et.
  \end{itemize}  
\end{frame}

\begin{frame}{Független minták Fisher információja összeadódik}
    \begin{df}[$(R)$, azaz gyenge regularitási feltétel]
      $\cP=\set{f_\theta d\lambda}{\theta\in\Theta}$ dominált mértékcsalád, $\Theta\subset\real^p$ nyílt.

      $\lambda$ m.m. $x$-re $\theta\mapsto \sqrt{f_\theta(x)}\in C^1$. % folytonosan differenciálható.
    %és 
    $I(\theta)$ véges, $\theta$-ban folytonos, és $I(\theta)$ nem szinguláris ($\det I(\theta)>0$).
      % \begin{enumerate}[<*>]
      %   \item $\lambda$ majdnem minden $x$-re $\theta\mapsto \sqrt{f_\theta(x)}$ folytonosan differenciálható.
      %   \item $I(\theta)$ véges $\theta$-ban folytonos, nem szinguláris ($\det I(\theta)>0$).
      % \end{enumerate}
    \end{df}
    \begin{proposition}
      $X$, $Y$ minden $\theta\in\Theta$ mellett függetlenek és eloszlásaikra teljesül $(R)$.
  
      Ekkor $(X,Y)$ együttes eloszlásaira is teljesül $(R)$ és $I_{X,Y}(\theta)=I_{X}(\theta)+I_{Y}(\theta)$.
    \end{proposition}
    \begin{itemize}
      \item $X$ eloszlásainak domináló mértéke $\lambda_X$, $Y$-ra $\lambda_Y$. Ekkor $(X,Y)$ eloszlásait 
      $\lambda_X\otimes\lambda_Y$ dominálja. 
      \item $f_{(X,Y),\theta}(x,y)=f_{X,\theta}(x)f_{Y,\theta}(y)$. Ha $\sqrt{f_{X,\theta}(x)}$ $\sqrt{f_{Y,\theta}(y)}$ 
      folytonosan deriválható $\theta$-ban $\lambda_X$ majdnem minden $x$-re ill. $\lambda_Y$ majdnem minden $y$-ra, akkor 
      $\sqrt{f_{X,Y,\theta}(x,y)}$ is folytonosan differenciálható $\lambda_X\otimes\lambda_Y$ m.m $(x,y)$-ra. 
      \item $\ln f_{X,Y,\theta}(x,y)=\ln f_{X,\theta}(x)+\ln f_{Y,\theta}(y)$. 
      Ebből $I_{X,Y}(\theta)=I_X(\theta)+I_{Y}(\theta)$ következik. 
      \item $I_{X,Y}(\theta)$ folytonos, véges értékű.
      \item $I_{X,Y}\geq I_{X}$, amiből $\ker I_{X+Y}\subset \ker I_X=\smallset{0}$, 
      $I_{X,Y}$ is invertálható, minden $\theta$-ra.
    \end{itemize}
  \end{frame}
  
  \begin{frame}{$S(X)$ Fisher információja nem lehet $X$ információ tartalmánál nagyobb}
    \begin{proposition}
      $X$ eloszlásainak a családjára teljesül $(R)$. $S$ statisztika. Ekkor $S$ eloszlásainak a családja is dominált.
  
      Ha $S$ eloszlásainak családjára is teljesül $(R)$, akkor $I_{S(X)}(\theta)\leq I_X(\theta)$.
    \end{proposition}
    \begin{itemize}
      %\item $(R)$ nem függ a domináló mértéktől, $\P[0]\in \cP'$ választható. 
      %($\cP'=\set{\sum c_i\cP[\theta_i]}{c_i\geq0,\,\sum c_i=1}$). 
      \item $\lambda\circ S^{-1}$ domináló mérték $S$ eloszlásaihoz: 
      \begin{displaymath}
        \P[\theta]{S\in H}
        =\int_{\mfX} f_\theta(x) \I{S(x)\in H} \lambda (d x)
        % =\E[0]{\frac{d\P[\theta]|_{\sigma(S)}}{d\P[0]|_{\sigma(S)}}\I{S\in H}}
        =\int_H g_\theta(s) \lambda\circ S^{-1}(d s), 
        \quad g_\theta(S)= \frac{d\P[\theta]|_{\sigma(S)}} {d\lambda|_{\sigma(S)}}. 
      \end{displaymath}
      \item Ha $S$ eloszlásai is teljesítik $(R)$-et, $\P[\theta]{S\in H}$-re adott mindkét formulában az 
      integrálon belül lehet deriválni $\theta$ szerint.
      $\ell_X(\theta,x)=\ln f_{\theta}(x)$ 
      és $\ell_S(\theta,s)=\ln g_\theta(s)$ jelöléssel
      \begin{displaymath}
        %\partial_\theta \P[\theta]{S\in H}
        %=
        \int_{\mfX} \partial_\theta  f_{\theta}(x)\I{S(x)\in H} \lambda(d x)
        =\E[\theta]{\I{S\in H} \ell_X'(\theta,X)}
        =\int_{H} g_\theta'(s) \lambda\circ S^{-1}(d s)
        =\E[\theta]{\I{S\in H}\ell'_S(\theta,S)}
      \end{displaymath}
      \item $\E[\theta]{\ell'_X(\theta,X)|S}=\ell'_S(\theta,S)$ és a teljes szórásnégyzet tétel szerint 
      $I_S(\theta)\leq I_X(\theta)$.
    \end{itemize}
  \end{frame}
  
  \begin{frame}{Elégséges statisztika Fisher információja}
    \begin{proposition}
      $X$ minta eloszlásainak családjára teljesül $(R)$, $S$ elégséges statisztika. 
      Ekkor $S$ eloszlásainak a családjára is teljesül $(R)$ és $I_{S(X)}(\theta)=I_{X}(\theta)$.
    \end{proposition}
    \begin{itemize}
      \item  $(R)$ nem függ a domináló mértéktől, $\P[0]\in \cP'$ választható. 
      ($\cP'=\set{\sum c_i\P[\theta_i]}{c_i\geq0,\,\sum c_i=1}$). 
      \item $S$ elégséges, ezért $\P[0]$-ra vonatkozó sűrűségek $f_{\theta}(x)=g_{\theta}(S(x))$ alakúak.
      \item $g_\theta$ $S$ eloszlásának a sűrűsége a $\P[0]\circ S^{-1}$ domináló mértékre nézve.
      \item $\sqrt{f_{\theta}(x)}=\sqrt{g_\theta(S(x))}$, azaz ha $(R)$ teljesül, 
      akkor $\sqrt{g_\theta(s)}$ folytonosan deriválható $\P[0]\circ S^{-1}$ m.m. $s$-re. 
      \item $\ell_X'(\theta,X)=\ell_{S}'(\theta,S)$, amiből $I_X(\theta)=I_{S}(\theta)$ és $(R)$ $I$-re vonatkozó 
      előírásai $I_S$-re is igazak.
    \end{itemize}
  \end{frame}
  
  \begin{frame}{Fisher információ és elégséges statisztika}
    \begin{proposition}
      $(R)$ teljesül $X$ és az $S$ statisztika  eloszlásainak családjára is.
       $f_\theta>0$ a mintatéren minden $\theta\in\Theta$-ra és $\Theta$ összefüggő nyílt.
      
       Ha $I_X(\theta)=I_{S(X)}(\theta)$, akkor $S$ elégséges.
    \end{proposition}
    \begin{itemize}
      \item $\P[0]=\P[\theta_0]$ $\theta_0\in \Theta$ választható domináló mértéknek, $f_{\theta_0}\equiv1$.
      \item Láttuk, hogy $\E[\theta]{(\partial_\theta \ln f_\theta)(X)|S }=(\partial_\theta \ln g_\theta)(S)$, 
      ahol $g_\theta$ $S$ eloszlásainak sűrűsége $\P[\theta_0]\circ S^{-1}$-re.
      \item A feltétel szerint $I_X=I_S$, ami csak akkor lehet, ha 
      $\partial_\theta\ln f_\theta(x)=\partial_\theta \ln g_\theta(S(x))$ minden $\theta$-ra $\P[0]$ m.m..
      \item $\partial_\theta\ln f_\theta(x)$, $\partial_\theta\ln g_\theta(x)$ $\P[0]$ m.m. 
      $x$-re folytonos $\theta$-ban $(R)$ miatt. 
      \item $\P[0]{ \forall\theta\in\Theta,\,\partial_\theta\ln f_\theta(x)=\partial_\theta \ln g_\theta(S(x))}=1$.
      \item $\ln f_\theta(x)-\ln g_\theta(S(x))$ majdnem minden $x$-re 
      $\theta$-ban konstans, azaz $f_\theta(x)=g_\theta(S(x))h(x)$ alakú, ahol $h(x)=f_{\theta_0}(x)/g_{\theta_0(S(x))}$.
      % \begin{displaymath}
      %   \ln f_{\theta_1}(x)-\ln f_{\theta_0} (x) = \int_{\gamma} \partial_\theta \ln f_\theta(x) d\theta=
      %   \int_\gamma \partial_\theta \ln g_\theta (S(x))d\theta= \ln g_{\theta_1}(S(x))-\ln g_{\theta_0}(S(x))
      % \end{displaymath}
      % \begin{displaymath}
      %   f_\theta(x)=\frac{g_\theta(S(x))}{g_{\theta_0}(S(x))}
      % \end{displaymath}
      % és 
      $S$ elégséges a faktorizációs tétel miatt.
    \end{itemize}
  \end{frame}
  
  
  \begin{frame}{Összefoglalás}
    Ha az $X$, $Y$ minta és $S(X)$ eloszlásainak családja teljesíti $(R)$-et, akkor
    \begin{itemize}
      \item $X$, $Y$ függetlenek $I_{X,Y}(\theta)=I_X(\theta)+I_Y(\theta)$, 
      pl. $n$ elemű mintára $I_n(\theta) = nI_1(\theta)$.
      \item $I_{S(X)}(\theta)\leq I_X(\theta)$
      \item $I_{S(X)}(\theta)=I_X(\theta)$, ha $S$ elégséges.
      \item Ha $I_{S(X)}(\theta)=I_X(\theta)$, és $f_\theta>0$ $\mfX$-en, akkor $S$ elégséges.
    \end{itemize}
  \end{frame}
  
  \begin{frame}{Becslési módszerek}
    \begin{itemize}
      
    %\item Tapasztalati becslések
    \item Momentum módszer: válasszuk azt a paramétert, ami az első néhány tapasztalati momentumra illeszkedik.
    \item Maximum likelihood becslés: válasszuk azt a paramétert, ami a minta likelihoodját maximalizálja.
    \item Bayes becslés: válasszuk azt a paramétert ami az ,,a priori'' rizikót minimalizálja (részletek később).
    \end{itemize}
  \end{frame}

  \begin{frame}{Momentum módszer indikátor mintára}
    ,,\textit{Válasszuk azt a paramétert, ami az első néhány tapasztalati momentumra illeszkedik}''
    \begin{itemize} 
      \item $X_1,\dots,X_n$ indikátor minta ismeretlen $\theta\in[0,1]$ paraméterrel. 
      \item $\theta=\E[\theta]{X_1}$
      \item $S=\sum X_i$ a mintaösszeg, akkor $\frac1n S$ a mintaátlag a várható érték ,,tapasztalati'' becslése.
      \item $\hat{\theta}_{MM}=\frac1n S$.
    \end{itemize}
    \continue
    Itt a $\phi(\theta)=\E[\theta]{X_1}$ jelöléssel a 
    $\phi(\hat{\theta}_{MM})=\frac1n\sum X_i$ összefüggést oldottuk meg.
    
  \end{frame}

  \begin{frame}{Maximum likelihood elv indikátor mintára}
    ,,\textit{Válasszuk az a paramétert, ami a megfigyelés likelihoodját $f_\theta(x)$-et maximalizálja.}
    \begin{itemize}
      \item $X_1,\dots,X_n$ indikátor minta ismeretlen $\theta\in[0,1]$ paraméterrel. 
      
      $\hat{\theta}_{ML}=\arg\max_\theta f_{\theta}(x)$.
      
      \item $f_\theta(x)=\theta^{S(x)}(1-\theta)^{n-S(x)}$, ahol $S(x)=\sum x_i$. 
      % \item A maximumhely létezik és egyértelmű.
      \item Ha $S=0$, vagy $S=n$, akkor a $\hat{\theta}=\frac1n S$.
      \item Ha $0<S<n$ akkor áttérhetünk a loglikelihoodra:
      Likelihood egyenlet:
      \begin{displaymath}
        0=\ell'(\theta,x)=\frac{S}{\theta}-\frac{n-S}{1-\theta}\iff \theta=\frac{1}n S
      \end{displaymath}
      $\hat\theta_{ML}=\frac1n S$ a mintaátlag.
    \end{itemize}
    \continue
    Megjegyzések:
    \begin{itemize}
      \item A paraméter teret $[0,1]$-nek választottuk, hogy akkor is legyen maximumhely, ha $S(X)=0$ vagy $S(X)=n$.
      \item A maximum helyet gyakran a likelihood egyenlet megoldásával keressük.
    \end{itemize}
\end{frame}

\begin{frame}{Bayes módszer indikátor mintára}
  \begin{itemize}
    \item Adatok nélkül minden paraméter érték ,,egyformán valószínű''.
    Az \textbf{a priori eloszlás} a  $\Theta=(0,1)$ paramétertéren egyenletes, $q\equiv1$ a sűrűségfüggvénye. 
    (Más eloszlást is választhatunk).
    \item $\theta$ is valószínűségi változó, $\Omega=\Theta\times\mfX$ a valószínűségi mező alaphalmaza. 
    $\theta:\Omega\to\Theta$ az első, $X:\Omega\to\mfX$ a második koordináta leképezés.
    \begin{displaymath}
      \P{H}=\int_{\Theta}\int_{\mfX} \I{(t,x)\in H} \P[t]{dx}q(t) d t=\int_{H} f_t(x) q(t) d t\otimes\lambda(dx).  
    \end{displaymath}
    ahol $\lambda$ a számláló mérték $\mfX$-en, $dt$ a ,,Lebesgue mérték''.
    \item Négyzetes veszteség esetén $R_T(\theta)=\E[\theta]{(\theta-T)^2}=\E{(\theta-T)^2|\theta}$ a rizikó függvény, 
    $R_T=\E{R_T(\theta)}=\E{(\theta-T)^2}$ az \textbf{a priori rizikó}. 
    Ez az átlagos veszteség, ha a $T$ statisztikát használjuk becslésként.
    \item $T$ csak a megfigyeléstől függhet  (azaz $\sigma(X)$ mérhető), ezért 
    az a priori rizikót $T(X)=\E{\theta\given X}$ minimalizálja. 
    $(\theta,X)$ együttes sűrűségfüggvénye ($d t\times\lambda(d x)$-re nézve) $f_t(x) q(t)$,
    a feltételes sűrűségfüggvény:
    \begin{displaymath}
      f_{\theta|X}(t|x)=\frac{f_{\theta,X}(t,x)}{f_{X}(x)}=C(x) t^{\sum x_i} (1-t)^{n-\sum x_i}
    \end{displaymath}
    % \begin{align*}
    %   \P{\theta\in A |X=x}
    %   =\frac{\P{\theta\in A,\, X=x}}{\P{X=x}}
    %   &=\frac{\int_A t^{S(x)}(1-t)^{n-S(x)}d t}{\int_{(0,1)}t^{S(x)}(1-t)^{n-S(x)}} d t\\
    %   &= \int_A C(\alpha,\beta) t^{\alpha-1}(1-t)^{\beta-1}d t|_{\alpha=S(X)+1,\beta=n-S(X)+1} d t
    % \end{align*}
    Azaz $\theta$ $X$-re vonatkozó feltételes eloszlása $\text{Béta}(S+1,n+1-S)$, ahol $S$ a mintaösszeg. 
    % (Itt valójában a feltételes sűrűségfüggvény formuláját számoltuk ki újra.)
    \item $\E{\theta|X}=\frac{S+1}{n+2}$, $S$ A mintaösszeg. (Laplace simítás).
    \item Ha $g(\theta)$-t szeretnénk becsülni, akkor $\E{g(\theta)|X}$ lenne a Bayes becslés.
  \end{itemize}
\end{frame}

\begin{frame}{Tapasztalati eloszlás}
  \begin{df}[Tapasztalati eloszlás, tapasztalati eloszlásfüggvény]
      $X_1,\dots, X_n$ minta, $P_n^*=\frac1n \sum_{i}\delta_{X_i}$ a
      tapasztalati eloszlás, $F_n^*(t)=P_n^*((-\infty,t))=\frac1n\sum_i\I{X_i<t}$ a tapasztalati eloszlásfüggvény. 
  \end{df}
  \begin{itemize}
      \item $P_n^*$ véletlen valószínűségi mérték $F_n^*$ véletlen függvény (folyamat).
      \item $\mu_n,\mu$ valószínűségi mértékek $(\real,\B(\real))$-en. 
      $\mu_n\wto \mu$, ha minden $h$ folytonos korlátos függvényre $\int h d\mu_n\to\int h d\mu$.

      $\mu_n\wto\mu$ pontosan akkor, ha $\mu_n((-\infty,t))\to \mu((-\infty,t))$ minden olyan $t$-re, amire $\mu(\smallset{t})=0$.

      % \item $F_n^*(t)=P_n^*((-\infty,t))=\frac1n\sum_i \I{X_i<t}$. $F_n^*$ a tapasztalati eloszlás függvény.
      \item $F_n^*(t)\to F(t)$ egy valószínűséggel, ahol $F$ a mintaelemek közös eloszlásfüggvénye. Ez a nagy számok törvénye a 
      $\I{X_1<t},\I{X_2<t},\dots$ iid sorozatra. 
  \end{itemize}
  \continue
  Több is igaz.
  \begin{theorem}[Glivenko-Cantelli tétel (statisztika alaptétele)]
      $\sup_{t\in\real}\abs{F_n^*(t)-F(t)}\to0$ egy valószínűséggel.
  \end{theorem}
  Ebből $P_n^*\wto P$ egy valószínűséggel következik.
\end{frame}

\begin{frame}{Tapasztalati becslések}
  \begin{df}
    $\cP$ eloszlások egy családja, amely tartalmazza a véges sok
    pontra koncentrált eloszlásokat. $\phi$ a $\cP$-n értelmezett
    funkcionál.
    %
    $\phi (P)$ \textbf{tapasztalati becslése} $\phi (P_n^*)$, ahol $P_n^*$ a
    tapasztalati eloszlás. 
  \end{df}

  \begin{df}
    %$\set{P_\theta}{\theta\in\Theta}$ eloszláscsalád $(\real,\B(\real))$-en. 
    $X_1,X_2,\dots$ iid. minta a $P$ eloszlásból,
    $\phi(P)$ becslése $n$ elemű mintából $T_n=T_n (X_1,\dots,X_n)$.

    A $T_n$ %becsléssorozat 
    konzisztens, ha $T_n\pto\phi(P)$, és erősen
    konzisztens, ha $T_n\to\phi(P)$ egy valószínűséggel.
  \end{df}

  \begin{itemize}
  \item Példa. $N (\mu,1)$ eloszlásból származó $n$ elemű minta esetén kézenfekvő a paramétert a
    mintaátlaggal becsülni. 
  %\item 
    Itt a paraméter a várható érték, ami véges tartójú eloszlásokra is
    létezik. %A becslés a tapasztalati eloszlásból számolt várható érték.
    \begin{displaymath}
      \cP=\set{P}{\text{$\mu$ valószínűségi Borel mérték $\real$-en, 
      $\smallint \abs{x} P(dx)<\infty$}},\quad \phi(P)=\smallint x P(d x). 
    \end{displaymath}
    $P_n^*\in\cP$ és $\phi(P_n^*)=\int x \zjel{\frac1n\sum \delta_{X_i}(d x)}=\frac1n\sum_i X_i$.

    Ebben a példában $\phi(P_n^*)\to\phi(P)$, de $\phi$ nem gyengén folytonos $\cP$-n,
    sőt $N(\mu,1)$-ben sem.

  \item A minta a $P$ eloszlásból származik, $\phi$ gyengén folytonos a $P$ pontban. 
    Ekkor $\phi$ tapasztalati becslése erősen
    konzisztens, hiszen $P_n^*\wto P$ a Glivenko-Cantelli tétel szerint.
    
    Pl. Ha mintaelemek közös eloszlásának mediánja egyértelmű, akkor 
    a tapasztalati medián erősen konzisztens a mediánra.
  \end{itemize}
  
\end{frame}

\begin{frame}[<*>]{Momentum módszer}
  \begin{itemize}
  \item $\cP=\set{\P[\theta]}{\theta\in\Theta}$,  $\Theta\subset\real^p$. 
  %$\theta$-t akarjuk becsülni.
  \item Legyen %$g:\mfX _1\to\real^p$, és 
    $\phi(\theta)=\E[\theta] (g (X_1))$.
    Tegyük fel, hogy $\phi:\Theta\to T\subset\real^p$ kölcsönösen egyértelmű és oda-vissza
    folytonos. %, $T$ nyílt. 
  \item $\psi=\phi (\theta)$  tapasztalati becslése
    \begin{displaymath}
      \hat\psi_n=\frac1n \sum_{i=1}^n g (X_i)
    \end{displaymath}
  \item $\hat\psi_n\to \psi$ a nagy számok törvénye szerint. 
    Ha $n$ elég nagy és $\psi\in \interior T$, akkor $\hat\psi\in T$ és
    $\hat\theta=\phi^{-1} (\hat\psi)$ értelmes. 
  \item $\hat\psi_n\to\psi$ egy valószínűséggel és $\phi^{-1}$ folytonossága
    miatt $\hat\theta_n$ erősen konzisztens, de általában nem torzítatlan.

    $\hat\psi_n$ torzítatlan $\psi$-re, de $\phi^{-1}$ általában nem
    lineáris.
    
  \item Klasszikus momentum módszernél $g (x)=(x,x^2,\dots,x^p)$.
  \end{itemize}
\end{frame}

\begin{frame}{Példák}
  \begin{itemize}
    \item $X_1,\dots,X_n$ a $(-\theta,3\theta)$ intervallumon egyenletes eloszlásból származó minta, $\theta>0$.
    
    $\E[\theta]{X_1}=\theta$. 
    
    $\hat\theta_{MM}$ a mintaátlag. Minden $n$-re pozitív valószínűséggel $\hat\theta_{MM}<0$.
    
    \item $X_1,\dots,X_n$ $\text{Béta}(\alpha,\beta)$, $\alpha,\beta>0$, eloszlásból származó $n$ elemű minta.
    \begin{displaymath}
      (\E{X_1},\E{X_1^2})=\phi(\alpha,\beta),
       \quad\text{ahol}\quad 
       \phi(\alpha,\beta)=\zjel*{\frac{\alpha}{\alpha+\beta},\frac{\alpha(\alpha+1)}{(\alpha+\beta)(\alpha+\beta+1)}}
    \end{displaymath}

    $\hat\alpha_{MM},\hat\beta_{MM}$ a $\phi(\hat\alpha,\hat\beta)=(\bar{X},\bar{X^2})$ egyenlet megoldása. HF.
  \end{itemize}
  
\end{frame}

\begin{frame}{Glivenko--Cantelli tétel}
  \begin{theorem}[Statisztika alaptétele]
    $X_1,X_2,\dots$ független azonos eloszlású változók, a közös
    eloszlásfüggvény $F$.
    % A tapasztalati eloszlásfüggvény
    % \begin{displaymath}
    %  F_n (t)=\frac1n\sum_{i=1}^n\I{X_i<t}
    %\end{displaymath}
    Ekkor
    \begin{displaymath}
      \lim_{n\to\infty}\sup_t \abs*{F_n (t)-F (t)}=0\quad
      \text{egy valószínűséggel, ahol $F_n(t)=\frac1n\sum\nolimits_{i=1}^n \I{X_i<t}$} % a tapasztalati eloszlásfüggvény}
    \end{displaymath}
  \end{theorem}
  \continue
  Megjegyzések.
  \begin{itemize}
  \item Vektor értékű megfigyelésekre is igaz. $X_i$ $\real^d$
    értékű. Ilyenkor $t\in\real^d$ és $X_i<t$ ha minden koordinátára
    igaz az egyenlőtlenség.
  \item Balról folytonosság miatt $\sup_{t}\abs*{F_n (t)-F (t)}=\sup_{t\in\Q}\abs*{F_n (t)-F
      (t)}$, azaz a szuprémum valószínűségi változót definiál.
  \item Rögzített $t$-re $F_n (t)$ független $F (t)$ várható értékű
    indikátorok átlaga. A nagy számok törvényéből $F_n (t)\to F (t)$ egy
    valószínűséggel.
    
  \item CHT: rögzített $t$-re $\sqrt{n}(F_n (t)-F (t))\dto N (0,F (t) (1-F (t)))$.
  \end{itemize}
\end{frame}

\begin{frame}{Glivenko--Cantelli igazolása}
  \begin{theorem}[Statisztika alaptétele]
    $X_1,X_2,\dots$ független azonos eloszlású változók, a közös
    eloszlás $P$ eloszlásfüggvény $F$.
    Ekkor
    \begin{displaymath}
      \lim_{n\to\infty}\sup_t \abs*{F_n (t)-F (t)}=0\quad
      \text{egy valószínűséggel, ahol $F_n(t)=\frac1n\sum\nolimits_{i=1}^n \I{X_i<t}$}
    \end{displaymath}
  \end{theorem}
  \begin{itemize}
    \item $k\geq 1$.  $t_i=\inf\set{t\in\real}{F(t)\geq \frac ik}$, $i=1,\dots,k-1$, $t_0=-\infty$, $t_k=\infty$ és
    $\cS(k)=\set{(-\infty,t_i),(-\infty,t_i]\cap\real}{i=0,\dots k}$ véges halmaz rendszer.
    
    \item Ha $t\in\real$ létezik $A,B\in \cS$, hogy $A\subset(-\infty,t)\subset B$ 
    és %$%P(A)\leq F(t)\leq 
    $P(B)\leq P(A)+\frac1k$. 
    
    Ha $t\in \smallset{t_1,\dots,t_k}$, akkor $A=B=(-\infty,t)$ jó, különben létezik $i$ hogy $t_i<t<t_{i+1}$ és 
    $A=(-\infty,t_i]$, $B=(-\infty,t_{i+1})$ jó ($P(B)=F(t_{i+1})\leq \frac{i+1}k$ és $P(A)=F(t_i+)\geq \frac ik$).
    \continue
    \begin{displaymath}
      F_n(t)-F(t)\leq P_n^*(B)-P(B)+P(B)-P(A),\quad F(t)-F_n(t)\leq P(B)-P(A)+P(A)-P^*_n(A)
    \end{displaymath}
    \pause
    \item A nagy számok törvénye alapján $P_n^*(H)\to P(H)$ minden $H\in\cC(k)$-ra
    \begin{displaymath}
      \limsup_{n\to\infty} \sup_{t\in\real}\abs{F_n(t)-F(t)}
      \leq \inf_k\limsup_{n\to\infty}\zjel*{\max_{H\in \cS(k)}\abs{P_n^*(H)-P(H)}+\tfrac1k}=\inf_k \tfrac1k=0
    \end{displaymath}
  \end{itemize}  
\end{frame}

\begin{frame}{Glivenko--Cantelli tétel igazolása}.
  \begin{df}
    $P$ valószínűségi mérték $(\mfX ,\B)$-n.
    $\cC\subset\B$ halmazrendszer végesen approximálható, ha  minde $\eps>0$-ra létezik
    $\cS (\eps)\subset\B$ véges rendszer úgy, hogy minden $C\in\cC$-re, 
    létezik $A,B\in\cS (\eps)$, amire $A\subset C\subset B$ és
    $P(B)\le  P(A)+\eps$.
  \end{df}
  \continue
  A skaláreset azon múlt, hogy $\cC=\set{(-\infty,t)}{t\in\real}$ végesen approximálható, továbbá
  \begin{proposition}
    Ha $\cC$ végesen approximálható, akkor
      $\sup_{H\in\cC}\abs*{P (H)-P^*_n (H)}\leq
      \eps + \max_{A\in\cS (\eps)}  \abs*{P(A)-P^*_n (A)}$
  \end{proposition}
  és a tapasztalati eloszlásra $P_n^*(A)\to P(A)$ minden $A$-ra egy valószínűséggel.
  
  \continue
  A vektor esethez csak annyit kell hozzátenni, hogy
  \begin{proposition}
    Ha $\cC',\cC''$ végesen approximálható, akkor
     $\cC' (\cap)\cC''=\set{C'\cap C''}{C'\in\cC',\,C''\in\cC''}$ is végesen approximálható.
  \end{proposition}  
\end{frame}

\begin{frame}{Glivenko-Cantelli másképp, vázlat}
  $X_1,X_2,\dots $ iid változók, közös $F$ eloszlásfüggvénnyel. 
  $(X'_n)_{n\geq 1}\eqinlaw (X_n)_{n\geq1}$ és $(X'_n)_{n\geq 1}$ független $(X_n)_{n\geq1}$-től.
  $F_n(t)=\frac1n\sum_{i\leq n}\I{X_i<t}$, $F'_n(t)=\frac1n\sum_{i\leq n}\I{X'_i<t}$

  \begin{proposition}[Első randomizálás]
    %\begin{displaymath}
      $\P{\sup_t \abs{F_n(t)-F(t)}>2\eps}\leq 2\P{\sup_t \abs{F_n(t)-F'_n(t)}>\eps}$, ha $n>2 \eps^{-2}$.
      %\quad\text{ha $n>8 \eps^{-2}$}.
    %\end{displaymath}
  \end{proposition}
  \begin{proposition}[Második randomizálás]
    \begin{displaymath}
      \P{\sup \abs{F_n-F'_n}>\eps}
      =2\P{\sup_t F_n-F'_n >\eps}
      =2\P{\sup\nolimits_t\tfrac1n \sum \xi_i\zjel{\I{X_i<t}-\I{X_i'<t}}>\eps} 
    \end{displaymath}
    ahol $\xi_1,\dots,\xi_n$ független véletlen előjelek, egymástól és az $(X_k)$, $(X'_k)$ változóktól is ($\P{\xi=\pm1}=1/2$). 
  \end{proposition}
  \begin{proposition}[Hoeffding egyenlőtlenség alkalmazása]
    $\P{\sup\nolimits_t \sum \xi_i\zjel{\I{X_i<t}-\I{X_i'<t}}> n\eps\given X,X'}\leq (2n+1)\exp{-\frac12 n\eps^2}$
  \end{proposition}
  \continue
  Összefoglalva: Létezik $c_1,c_2>0$, hogy $\P{\sup\abs{F_n-F}>\eps}\leq c_1ne^{-c_2n\eps^2}$ ha $n>8\eps^{-2}$.
  
  Innen $\sup\abs{F_n-F}\to0$ egy valószínűséggel a Borel--Cantelli lemma alkalmazásával.
 
\end{frame}

\begin{frame}{Első randomizálás részletei}
  $X_1,X_2,\dots $ iid változók, közös $F$ eloszlásfüggvénnyel. 
  $(X'_n)_{n\geq 1}\eqinlaw (X_n)_{n\geq1}$ és $(X'_n)_{n\geq 1}$ független $(X_n)_{n\geq1}$-től.
  $F_n(t)=\frac1n\sum_{i\leq n}\I{X_i<t}$, $F'_n(t)=\frac1n\sum_{i\leq n}\I{X'_i<t}$

  \begin{proposition}[Első randomizálás]
    %\begin{displaymath}
      $\P{\sup_t \abs{F_n(t)-F(t)}>2\eps}\leq 2\P{\sup_t \abs{F_n(t)-F'_n(t)}>\eps}$ ha $n>2 \eps^{-2}$.
      %\quad\text{ha $n>8 \eps^{-2}$}.
    %\end{displaymath}
  \end{proposition}
  \begin{itemize}
    \item Legyen $T=T(X)$ olyan változó, hogy $\abs{F_n-F} (T)>2\eps$, 
    a $\event{\sup\abs{F_n-F}>2\eps}$ eseményen.
    \begin{align*}
      &\P{\sup\abs{F_n-F}>2\eps}
      =\P{\abs{F_{n}-F}(T)>2\eps}\\
      &=\P{\abs{F_{n}-F}(T)>2\eps,\,\abs{F'_{n}-F}(T)>\eps}
      +\P{\abs{F_{n}-F}(T)>2\eps,\,\abs{F'_{n}-F}(T)\leq\eps} 
      %\\
      %&\leq\P{\abs{F_{n}(T)-F(T)}>\eps}\sup\nolimits_{t}\P{\abs{F'_{n}(t)-F(t)}>\tfrac12\eps}
      %+\P{\sup \abs{F_{n}-F_n'}>\tfrac12\eps}
    \end{align*}
    \item 
    % ahol azt használtuk, hogy
    \begin{align*}
      \P{\abs{F_{n}-F}(T)>2\eps,\,\abs{F'_{n}-F}(T)>\eps\given(X_n)_{n\geq1}}
      &=\I{\abs{F_{n}-F}(T)>2\eps}\P{\abs{F'_{n}(t)-F(t)}>\eps}|_{t=T}\\
      &\leq\I{\abs{F_{n}(T)-F(T)}>2\eps}\sup_t \P{\abs{F'_{n}(t)-F(t)}>\eps}
    \end{align*}
    és $\P{\abs{F'_{n}(t)-F(t)}>\eps}\leq \D^2{F_n'(t)} \eps^{-2}\leq \frac14n^{-1}\eps^{-2}<\frac12$, ha $n>2\eps^{-2}$.
    \item $\P{\sup\abs{F_n-F}>2\eps}\leq \frac12\P{\sup\abs{F_n-F}>2\eps}+\P{\sup\abs{F_n-F_n'}>\eps}$.
  \end{itemize}
\end{frame}

\begin{frame}{Második randomizálás részletei}
  \begin{proposition}[Második randomizálás]
    \begin{displaymath}
      \P{\sup \abs{F_n-F'_n}>\eps}
      =2\P{\sup\nolimits_t F_n-F'_n >\eps}
      =2\P{\sup\nolimits_t\tfrac1n \sum \xi_i\zjel{\I{X_i<t}-\I{X_i'<t}}>\eps} 
    \end{displaymath}
    ahol $\xi_1,\dots,\xi_n$ független előjelek, egymástól és az $(X_k)$, $(X'_k)$ változóktól is. 
  \end{proposition}
  \begin{itemize}
    \item $\sup F_n-F_n'\eqinlaw\sup F_n'-F_n$ $\implies$ $\P{F_n-F'_n>\eps}=\P{F_n-F'_n<-\eps}$.
    \item Legyen $S=\sup_t  \tfrac1n \sum \xi_i\zjel{\I{X_i<t}-\I{X_i'<t}}$. 
    \begin{displaymath}
      \P{S>\eps|\xi_1,\dots,\xi_n}=\P{\sup F_n-F_n'>\eps},
    \end{displaymath}
    mert $\xi_1,\dots,\xi_n$ hatása csak annyi, hogy ha $\xi_i=-1$, akkor $X_i$ és $X_i'$ szerepet cserél.
    \continue
    
    Formálisan. Ha $\sup F_n-F_n'=g(X,X')$, akkor $S=g(T(\xi, X,X'))$, ahol 
    \begin{displaymath}
      T(\xi,X,X')=(Y,Y')\quad\text{és}\quad (Y_i,Y'_i)=\I{\xi_i=1}(X_i,X_i')+\I{\xi_i=-1}(X'_i,X_i)
    \end{displaymath}
    $T(x,X,X')\eqinlaw (X,X')$ minden $x\in\smallset{-1,1}^n$ előjelsorozatra, vagyis 
    \begin{displaymath}
      \P{S>\eps\given\xi}=\P{g(T(\xi,X,X'))>\eps\given\xi}=\P{g(T(x,X,X'))>\eps}|_{x=\xi}=\P{g(X,X')>\eps}
    \end{displaymath}
  \end{itemize}  
\end{frame}

\begin{frame}{Hoeffding egyenlőtlenség}
  \begin{proposition}
    $Z_i$ független nulla várható értékű változók, $\abs{Z_i}\leq 1$, $\eps>0$. 
    Ekkor $\P{\sum Z_i> n\eps}\leq e^{-\frac12n\eps^2}$
  \end{proposition}
  \begin{itemize}
    \item 
    \begin{displaymath}
      \P{\sum Z_i>n\eps}\leq \inf_{t>0}\E{e^{t \sum_i Z_i}} e^{-t n\eps} %=\inf_{\lambda>0}\prod\E{e^{\lambda}}
    \end{displaymath}
    \item Függetlenség miatt $\E{e^{t\sum Z_i}}=\prod\E{e^{t Z_i}}$
    \item $Z=Z_i$-vel: $g(t)=\E{e^{t Z}}$, $g$ a $Z$ momentum generáló függvénye.
    $\ln g(0)=0$, $(\ln g)'(0)=\frac{g'(0)}{g(0)}=\E{Z}=0$ és 
    \begin{displaymath}
      (\ln g)''(s)=\frac{g''(s)}{g'(s)}-\zfrac*{g'(s)}{g(s)}^2=\E{Z^2\frac{e^{sZ}}{g(s)}}-\E^2{Z\frac{e^{sZ}}{g(s)}}=\D[s]^2{Z},
      \quad\text{ahol $\frac{d\P[s]}{d\P}=\frac{e^{sZ}}{g(s)}$}.
    \end{displaymath}
    \item $\D[s]^2{Z}\leq \E[s]{Z^2}\leq1$, így
    \begin{displaymath}
      \ln g(t)=\int_0^t (\ln g)'(s) d s=\int_0^t \int_0^s (\ln g)''(u)d u d s\leq \tfrac12 t^2 .
    \end{displaymath}
    \item 
    \begin{displaymath}
      \P{\sum Z_i>n\eps}
      \leq \inf_{t>0} e^{\frac{n}2\zjel{t^2-2t\eps}}
      =\inf_{t>0} e^{\frac n2\zjel{(t-\eps)^2-\eps^2}}=e^{-\frac 12 n\eps^2}.
    \end{displaymath}
  \end{itemize}
\end{frame}

\begin{frame}{Hoeffding egyenlőtlenség alkalmazása}
  \begin{proposition}[Hoeffding egyenlőtlenség alkalmazása]
    $\P{\sup\nolimits_t \sum \xi_i\zjel{\I{X_i<t}-\I{X_i'<t}}> n\eps\given X,X'}\leq (2n+1)\exp{-\frac12 n\eps^2}$
  \end{proposition}
  \begin{itemize}
    \item Az $X_1,\dots,X_n,X_1',\dots,X'_n$ pontok legfeljebb $2n+1$ részre osztják a számegyenest. 
    \item $\sum \xi_i\zjel{\I{X_i<t}-\I{X_i'<t}}$ az egyes felosztás darabokon konstans. 
    \item $X,X'$ alapján mindegyikből válaszunk a felosztás darabokból egy-egy $t_k=t_k(X,X')$ pontot.
    \begin{align*}
      &\P{\sup\nolimits_t \sum \xi_i\zjel{\I{X_i<t}-\I{X_i'<t}}> n\eps\given X,X'}\\
      &\leq 
      \sum_{k} 
      \P{\sum \xi_i\zjel{\I{X_i<t_k(X,X')}-\I{X_i'<t_k(X,X')}}> n\eps\given X,X'}\\
      &=
      \sum_k\P{\sum \xi_i\zjel{\I{x_i<t_k(x,x')}-\I{x_i'<t_k(x,x')}}> n\eps}|_{(x,x')=(X,X')}
    \end{align*}
    \item $Z_i=\xi_i\zjel{\I{x_i<t_k(x,x')}-\I{x_i'<t_k(x,x')}}$, 
    $i=1,\dots,n$ független változók, $\abs{Z_i}\leq1$ 
    és $\E{Z_i}=0$.
    \item Hoeffding egyenlőtlenség alapján a felső becslés mindegyik tagja legfeljebb $e^{-\frac12 n\eps^2}$.
  \end{itemize}
\end{frame}

\begin{frame}{Glivenko--Cantelli tétel, vektorváltozó esete. Részletek I}
  \begin{proposition}
    Ha $\cC$ végesen approximálható, akkor
      $\sup_{H\in\cC}\abs*{P (H)-P^*_n (H)}\leq
      \eps + \max_{A\in\cS (\eps)}  \abs*{P(A)-P^*_n (A)}$
  \end{proposition}
  $H\in\cC$, $A,B\in\cS (\eps)$ $A\subset H\subset B$
    és $P (B\setminus A)\leq\eps$.
    \begin{align*}
      P (H)-P_n^* (H)
      &\leq P (B)-P_n^* (A)\\
      &=P (B\setminus A)+P(A)-P_n^* (A_1)\\
      &\leq \eps+\max_{A\in\cS}  \abs*{P(A)-P^*_n (A)}\\
      P_n^* (H)-P (H)
      &\leq
        P_n^* (B)-Q (A)\\
      &=P (B\setminus A)+P_n^*(B)-P (B)\\
      &\leq \eps+\max_{A\in\cS}  \abs*{P(A)-P^*_n (A)}
    \end{align*}
\end{frame}

\begin{frame}{Glivenko--Cantelli tétel, vektorváltozó esete. Részletek II.}
  \begin{proposition}
    Ha $\cC$ végesen approximálható, akkor
    \begin{displaymath}\textstyle
      \sup_{H\in\cC}\abs*{P (H)-P^*_n (H)}\leq
      \eps + \max_{A\in\cS (\eps)}  \abs*{P(A)-P^*_n (A)}
      % \sup_{}\to0,\quad\text{egy valószínűséggel.}
    \end{displaymath}
  \end{proposition}
  \begin{corollary}
    Ha $\cC$ végesen approximálható, akkor
    \begin{displaymath}\textstyle
      \sup_{H\in\cC}\abs*{P(H)-P^*_n (H)} %=\max_{A\in\cS}  \abs*{Q (A)-Q^*_n (A)}
    \to0,\quad\text{egy valószínűséggel.}
    \end{displaymath}
  \end{corollary}
  \continue
  \begin{displaymath}
    \limsup_{n\to\infty}\sup_{H\in\cC}\abs*{P (H)-P^*_n (H)}
    \leq \eps+\lim_{n\to\infty}\max_{A\in\cS (\eps)}  \abs*{P (A)-P^*_n (A)}=\eps
  \end{displaymath}
  Ez minden $\eps>0$-ra igaz, azaz a $\limsup$ nulla és a limesz is az.
\end{frame}

\begin{frame}{Glivenko--Cantelli tétel, vektorváltozó esete. Részletek III.}
  \begin{proposition}
    Ha $\cC',\cC''$ végesen approximálható, akkor
    $\cC' (\cap)\cC''=\set{C'\cap C''}{C'\in\cC',\,C''\in\cC''}$
    is végesen approximálható.
    %\begin{displaymath}\textstyle
    %  \cC' (\cap)\cC''=\set{C'\cap C''}{C'\in\cC',\,C''\in\cC''}
    %  ,\quad\text{$\cC' (\cap)\cC''$ is végesen approximálható}
    %\end{displaymath}
  \end{proposition}

  \begin{itemize}
  \item $\cS' (\eps/2)$, $\cS'' (\eps/2)$ véges approximáló
    halmazrendszerek $\cC'$ ill. $\cC''$-höz.
  \item $C'\in\cC'$-höz létezik $A'_1,A_2'\in\cS' (\eps/2)$, amire
    $A'_1\subset C'\subset A_2'$ és $Q (A_2'\setminus A_1')<\eps/2$.
  \item Hasonlóan $C''\in\cC''$-höz létezik $A''_1,A_2''\in\cS''
    (\eps/2)$.
  \item
    \begin{displaymath}
      A_1=A_1'\cap A_1''\subset C'\cap C''\subset A_2=A_2'\cap A_2'',\quad
      A_2\setminus A_1\subset A_2'\setminus A_1' \cup  A_2''\setminus
      A_1'',\quad
      Q (A_2\setminus A_1)<\eps
    \end{displaymath}
  \item $\set{A'\cap A''}{A'\in\cS (\eps/2),\,A''\in\cS (\eps/2)}$
    véges $\eps$ approximáló rendszer $\cC' (\cap)\cC''$-höz.
  \item Hasonló állítás igaz unióval és komplementerrel.
  \end{itemize}

\end{frame}

\begin{frame}{Glivenko--Cantelli tétel, vektorváltozó esete. Részletek IV.}
  \begin{proposition}
    $\cC=\set{\prod_{i=1}^r (-\infty,t_i)}{t\in\real^{r}}$ végesen approximálható.
  \end{proposition}
  \begin{itemize}
  \item Skalár esetet, $r=1$ láttuk, ha $P$ eloszlás $(\real,\B(\real))$-en, akkor alkalmas $(t_i)_{i=0}^{k}$ osztópontokkal
   $\cS_1=\set{(-\infty,t_i),(-\infty,t_i]\cap\real}{i=0,\dots k}$ véges $1/k$-approximáló rendszer 
   $\cC_1=\set{(-\infty,t)}{t\in\real}$-hez.
  \item $\cC_i=\set{\real^{i-1}\times (-\infty,t)\times\real^{n-i}}{t\in\real}$ 
  lényegében az egy dimenziós eset, azaz végesen approximálható, és $\cC=(\cap)_{i=1}^r \cC_i$.
  \end{itemize}
\end{frame}

\end{document}

\begin{frame}{Maximum likelihood becslés}
  $\cP$ dominált mértékcsalád, $f_\theta$ a sűrűségfüggvény,
  $\ell (\theta)$ a loglikelihood.
  \begin{df}
    A $\theta$ paraméter maximum likelihood becslése $\hat\theta (X)$,
    ha $f_{\hat{\theta}} (X)=\sup_{\theta}f_{\theta} (X)$.
  \end{df}
  Nem biztos, hogy létezik, nem biztos, hogy egyértelmű  
  \begin{itemize}
  \item $U (\theta,\theta+1)$ eloszláscsalád. $f_\theta
    (x)=\I{X_n^*-1\leq \theta\leq X_1^*}$, azaz ha  $T (X)\in
    [X_n^*-1,X_1^*]$, akkor $T$ maximum likelihood becslése
    $\theta$-nak.

  \item $f (x)$ sűrűségfüggvény $\real$-en.
    $$f_{\mu,\sigma}
    (x)=\frac{1}\sigma f \zfrac*{x-\mu}{\sigma}
    $$
    $\Theta=\real\times (0,\infty)$. $c\in\real$, $f (c)>0$, $x$ a
    megfigyelt érték
    \begin{displaymath}
      \sup_{\mu,\sigma}\frac{1}{\sigma}f \zfrac*{x-\mu}{\sigma}\geq
      \lim_{\sigma\to0}\frac{1}{\sigma} f\zfrac*{x- (x+c\sigma)}{\sigma}=\infty
    \end{displaymath}
  \end{itemize}  
\end{frame}

\begin{frame}{Maximum likelihood becslés}
  $\psi=g(\theta)$ becslése maximum likelihood elvvel. Ha $g$ kölcsönösen
  egyértelmű, akkor csak átparamétereztük a családot,
  $\hat\psi=g(\hat\theta)$.

  Ha $g$ nem injektív, akkor az indukált likelihoodot szokták
  maximalizálni
  \begin{displaymath}
    f^*_\psi (x)=\sup\set{f_\theta (x)}{g (\theta)=\psi}
  \end{displaymath}

  \begin{proposition}
    \begin{itemize}
    \item Ha $T$ elégséges statisztika és létezik ML becslés, akkor
      van olyan is, amelyik $T$ függvénye.
    \item Ha $\hat\theta$ a $\theta$ ML becslése, akkor $g
      (\hat\theta)$ a $g (\theta)$ ML becslése.
    \end{itemize}
  \end{proposition}
  \begin{itemize}
  \item Ha a maximum hely egyértelmű, akkor a faktorizációs tétel
    miatt $f_\theta (x)=g_\theta (T (x)) h (x)$ csak $T (x)$-en
    keresztül függ a megfigyeléstől. Ha nem egyértelmű, akkor arra
    kell figyelni, hogy $T (x)$ függvényében válasszunk.
    
    pl. $U (\theta,\theta+1)$ esetén elégséges statisztika
    $T (X) =(X_1^*,X_n^*)$. $\hat\theta(=X_n^*-1+X_1^*)/2$ ML becslés $T$
    függvényei között,
    \begin{displaymath}
      \hat\theta+\sin (X_1)\frac{X_1^*-(X_n^*-1)}2
    \end{displaymath}
    pedig nem (feltéve, hogy a minta elemszáma legalább 2).
  \end{itemize}  
\end{frame}

 
  
\begin{frame}{Bayes becslés, bevezetés}
    $\cP=\set{\P[\theta]}{\theta\in\Theta}$ eloszláscsalád. 
    $g(\theta)$-t szeretnénk becsülni.
    \begin{displaymath}
      L (\theta,T)=(g(\theta)-T)^2,\quad R_T(\theta)=\E[\theta] (L (\theta,T))
    \end{displaymath}
    $L (\theta,T)$ a veszteség, amit a $T$ becslés okoz, $R_T (\theta)$
    az átlagos veszteség, más néven rizikó.
  
    Bayes becslésnél a cél a
    \begin{displaymath}
      R_T (Q)=\int_\Theta R_T (\theta) Q (d\theta)\quad \text{apriori rizikó}
    \end{displaymath}
    minimalizálása.
  
    $Q$ gyakran valószínűség eloszlás. Ilyenkor úgy gondolhatunk a feladatra,
    hogy $\theta$ is valószínűségi változó, melynek eloszlása (a priori eloszlás)
    $Q$.
    $\P[t]$ a minta feltételes eloszlása a $\theta=t$ feltétel mellett.
  
    \begin{displaymath}
      R_T (Q)=\E{L (\theta,T)}=\E{(g (\theta)-T (X))^2}\geq \E{(g (\theta)-\E{g (\theta)|X})^2}
    \end{displaymath}
  
    A Bayes becslés $g (\theta)$ feltételes várható értéke a mintára nézve.
\end{frame}
  
\end{document}
  
       