%\def\option{}
\providecommand{\option}{handout}
\documentclass[aspectratio=169,notheorems,9pt,\option]{beamer}

\usepackage{marvosym}
%\usepackage[magyar]{babel}
\usepackage{polyglossia}
\setmainlanguage{magyar}

% \usepackage{pvdefs}
\usepackage{etoolbox,mathtools}

%\makeatletter
\def\ifbrace{\@ifnextchar{\bgroup}}

\@ifundefined{texexp}{%
  \let\texexp\exp
  \def\exp{%
    \texexp\ifbrace{\cbr*}{}%
  }
}\relax
 
\def\isparam#1{\ifbrace#1\relax}

\DeclarePairedDelimiter{\abs}{\lvert}{\rvert}
\DeclarePairedDelimiter{\zjel}{(}{)}
\DeclarePairedDelimiter\br\lbrack\rbrack
\DeclarePairedDelimiter\event\lbrace\rbrace
\DeclarePairedDelimiter{\norm}{\|}{\|}
\let\smallset\event
\let\cbr\event
\DeclarePairedDelimiterX\set[2]\{\}{%
  #1\nonscript\::\allowbreak\nonscript\:\mathopen{}#2%
}

\let\PEfont\mathbb
\providecommand\given{}

\def\redefgiven#1{%
  \renewcommand\given{%
    \nonscript\:%
    #1\vert%
    \allowbreak%
    \nonscript\:%
    \mathopen{}%
  }%
}
\DeclarePairedDelimiterX\PEzjel[1](){\redefgiven{\delimsize}#1}
\def\aftersuperscript#1^#2{%
  ^{#2}#1%
}
\newcommand{\PE}[1][]{%
  \PEfont{\Pe}%
  \ifblank{#1}{}{_{#1}}%
  \@ifnextchar{^}{%
    \aftersuperscript{\ifbrace{\PEzjel*}{}}%
    }{%
      \ifbrace{\PEzjel*}{}%
    }%
  %\ifbrace{\PEzjel*}{}%
}
\newcommand{\E}{\def\Pe{E}\PE}
\newcommand{\D}{\def\Pe{D}\PE}
\renewcommand{\P}{\def\Pe{P}\PE}

\def\cov{\operatorname{cov}}
\let\eps\varepsilon
\def\ito/{It\^o}
\let\Ito=\ito
\def\eqinlaw{\stackrel{d}{=}}

\def\strip#1{%
    \expandafter\stripi\detokenize{#1}\relax
}
\def\stripi#1#2\relax{%
    \def\stripi##1##2\relax{%
    \ifx##1#1\else\expandafter##1\fi##2%
    }%
}
\expandafter\stripi\detokenize{\ }\relax

\def\strip#1{%
    \expandafter\stripi\string#1\relax
}
\def\stripi#1#2\relax{%
    \def\stripi##1##2\relax{%
    \ifx##1#1\else\expandafter##1\fi##2%
    }%
}
\expandafter\stripi\string\ \relax

\DeclareListParser{\dolist}{}

\def\defprefix#1#2{
    \def\do##1{\csdef{#1\strip##1}{{#2{##1}}}}%
    \ifbrace\dolist\do
}

\defprefix{c}{\mathcal}{XAPIDLCSGTHEN}
\defprefix{b}{\bar}{hX}
\defprefix{t}{\tilde}{STQbCf\lambda \Omega \sigma }
\defprefix{}{\mathcal}{BF}
\defprefix{mf}{\mathfrak}{X}

\def\sign{\operatorname{sign}}
\def\real{\mathbb{R}}
\def\Z{\mathbb{Z}}
\def\Q{\mathbb{Q}}

\def\Isymb{{\mathbf 1}}
\def\substr#1{_{\zjel{#1}}}
\newcommand{\I}[1][]{%
  \Isymb\ifblank{#1}{\substr}{_{#1}}%
}

\def\Isymb{{\mathbb 1}}

\def\I{\@ifstar\I@star\I@}
\def\I@star{%
  %\message{I@star arg:|#1|^^J}%
  \Isymb\ifbrace{_}{}% } %{_{#1}}%
  }
\newcommand{\I@}[2][\zjel]{%
  \Isymb_{#1{#2}}% } %{_{#1}}%
}



\def\interior{\operatorname{int}}
\def\tg{\operatorname{tg}}
\DeclarePairedDelimiterX{\zfrac}[2](){\frac{#1}{#2}}
\newcommand\independent{\protect\mathpalette{\protect\independenT}{\perp}}
\def\independenT#1#2{\mathrel{\rlap{$#1#2$}\mkern2mu{#1#2}}}


\newtheorem{df}{Definíció}
\newtheorem{theorem}{Tétel}
\newtheorem{lemma}{Lemma}
\newtheorem{proposition}{Állítás}
\newtheorem{corollary}{Következmény}



\def\diag{\operatorname{diag}}
\def\Id{\mathbf{I}}
\let\PEfont\mathbb
\def\npto{\buildrel p\over \nrightarrow}
\def\ndto{\buildrel d\over \nrightarrow}
%\def\dto{\buildrel d\over \rightarrow}
\def\dto{\stackrel{d}{\to}}
\def\pto{\stackrel{p}{\to}}
\def\tr{\operatorname{Tr}}
\def\arctg{\operatorname{arctg}}
\let\nto\nrightarrow

\def\Leb{\operatorname{Leb}}


\def\readrest#1\enddate{}
\def\setdate@#1-#2-#3\relax{\def\lecturedate{#1.#2.#3.}\readrest}
\def\setdate{\expandafter\setdate@\jobname\relax-\relax-\relax\enddate}
\setdate

% \usetheme{Bruno}
\usecolortheme{Bruno}
% \usefonttheme{Bruno}
% \useinnertheme{Bruno}
\useoutertheme{Bruno}

\newrobustcmd{\continue}{%
  \onslide<\value{beamerpauses}->\relax
}


\beamerdefaultoverlayspecification{<+->}

\mode<presentation>

\RequirePackage[math-style=TeX, bold-style=upright]{unicode-math}
\setmathfont{Latin Modern Math}[Scale=1]
\setsansfont[Ligatures=TeX, ItalicFont={Fira Sans Italic}, BoldFont={Fira Sans SemiBold}, Scale=0.9]{Fira Sans Book}
\usepackage[mathcal]{euscript}

% Title page
\setbeamerfont{title}{
    size=\LARGE,
    shape=\bfseries
}

\setbeamerfont{subtitle}{
    size=\large,
    shape=\normalfont
}

\setbeamerfont{author}{
    size=\normalsize,
    shape=\normalfont
}

\setbeamerfont{details}{
    size=\footnotesize,
    shape=\normalfont
}

% Slide title
\setbeamerfont{frametitle}{size=\LARGE}
\setbeamerfont{framesubtitle}{
    size=\normalsize,
    %shape=\normalfont\scshape
}

% Blocks
\setbeamerfont{block title}{
    size=\normalfont,
    shape=\strut
}

\setbeamerfont{blockdef}{
    size=\large,
    shape=\bfseries
}

% Description
\setbeamerfont{description item}{shape=\bfseries}


% Footer information line with title, author and slide number
\setbeamertemplate{footline}{%
    \leavevmode%
    \hbox{%
        \usebeamerfont{footline}%
        \begin{beamercolorbox}[
            wd=\textwidth,
            ht=3ex,
            dp=1.25ex
        ]{footline}%
            \hspace{1cm}%
            \insertshorttitle%
            \hspace{2em}%
            \insertdate%
            \hfill%
            \insertframenumber{} / \inserttotalframenumber%
            \hspace{1cm}
        \end{beamercolorbox}%
    }
    \vskip0pt%
}

%% Bruno Beamer theme — Version 1.0
%% A lightweight Beamer theme inspired from the Metropolis theme
%%
%% Written in 2017-2019 by:
%% — Rémi Cérès <remiceres@msn.com>
%% — Mattéo Delabre <bonjour@matteodelabre.me>
%%
%% This work is released under the CC0 1.0 Universal license. See the
%% accompanying LICENSE file for details. To the extent possible under
%% law, Rémi Cérès and Mattéo Delabre have waived all copyright and
%% related or neighboring rights to the Bruno Beamer theme. This work
%% is published from France.

% \mode<presentation>

\setbeamertemplate{sections/subsections in toc}[square]
\setbeamertemplate{itemize item}[square]
\setbeamertemplate{itemize subitem}[circle]

% Allow multi-slide frames
\setbeamertemplate{frametitle continuation}{}

% Horizontal separator
\setbeamertemplate{separator}{%
    \usebeamercolor{separator}%
    \textcolor{fg}{\rule{.7\textwidth}{.5pt}}%
}

% Separate authors with newlines
\renewcommand{\beamer@andtitle}{\\}

% Title page. If you want to add a background image on the title slide,
% use the \background command to set the path to that image. Otherwise,
% no image will be included
\RequirePackage{tikz}
\usetikzlibrary{fadings}

\newcommand{\background}[1]%
    {\newcommand{\bruno@background}{#1}}


%% \newcommand\insertcaption{}
\newcommand{\backgroundcaption}[2][white]{\def\bruno@caption{{#1}{#2}}}

\newcommand{\insertcaption}{%
    \ifdefined
    \bruno@caption{%
        \expandafter\textcolor\bruno@caption
    }%
    \fi
}


\tikzfading[
    name=title page picture fading,
    left color=transparent!0,
    right color=transparent!100,
]

\setbeamertemplate{title page}{%
    \begin{minipage}{.7\textwidth}
        \raggedright
        \usebeamerfont{title}
        \inserttitle

        \usebeamertemplate{separator}

        \usebeamerfont{author}
        \vspace{2em}
        \insertauthor

        \vspace{2em}
        \usebeamerfont{details}
        \insertinstitute

        \insertdate
    \end{minipage}

    % Include faded image on the right if defined
    \ifdefined\bruno@background
    \begin{tikzpicture}[remember picture, overlay]
        % Crop image to a trapezium on the right
        \clip (current page.south east)
           -- (current page.north east)
           -- ++(-5.1, 0)
           -- ++(-3, -\paperheight)
           -- cycle;

        % Include background image
        \node[
            anchor=south east,
            inner sep=0,
            outer sep=0
        ] at (current page.south east) {
            \includegraphics[height=\paperheight]
                {\bruno@background}
        };

        \node[anchor=south east,
            inner sep=0,
            outer sep=15pt,
        ] at (current page.south east){\insertcaption};

        % Add a slight shadow
        \fill[
            black, path fading=title page picture fading,
            fading angle=-15
        ]
            (current page.south east)
            rectangle
            ++(-10, \dimexpr\paperheight+1cm);

        
    \end{tikzpicture}
    \fi
}

\mode<all>
\AtBeginDocument{%
  \let\phi\varphi
  \let\theta\vartheta
  \let\setminus\smallsetminus
}

\title{Statisztika előadás}
\date{\lecturedate}
\background{images/young-Ronald-Fisher.jpg}
\backgroundcaption{Ronald Fisher (1913)}


\begin{document}

\maketitle

\begin{frame}{Példák}
    \begin{itemize}
      \item Ha $\cP$ exponenciális családot alkot $\Theta\subset\real^p$ nyílt. 
      \begin{displaymath}
        f_\theta(x)=\exp{\theta T(x)-b(\theta)},\quad \sqrt{f_\theta(x)}=\exp{\tfrac12\theta T(x)-\tfrac12b(\theta)}
      \end{displaymath}
      ahol $b(\theta)$ sima $\Theta$-n, azaz $\theta\mapsto \sqrt{f_\theta}(x)$ folytonosan differenciálható.
      
      $I(\theta)=b''(\theta)=\Sigma_\theta(T)$, azaz $I(\theta)$ véges értékű és folytonos, 
      $\det I(\theta)>0$ teljesül, ha $T$ a $\real^p$ nem egy valódi affin altérből veszi fel az értékeit.
      \item $X$ Cauchy eloszlás eltolás paraméteres családjából származó $n$ elemű minta. 
      $X$ eloszlásai nem alkotnak exponenciális családot, de
      \begin{displaymath}
        \sqrt{f_\theta(x)}=\prod \pi^{-1/2} (1+(x_i-\theta)^2)^{-1/2}
      \end{displaymath}
      folytonosan deriválható, és 
      \begin{displaymath}
        \hphantom{\hspace{-2em}}
        \ell(\theta,x)=c+\sum_{i} \frac{-1}{2}\ln(1+(x_i-\theta)^2),
        \quad
        \ell'(\theta,x)=\sum_i \frac{x_i-\theta}{1+(x_i-\theta)^2}
        %\quad 
        %\E[\theta]{\ell'(\theta)}=\E[0]{\ell'(0)}=\sum_i \int_{\real} \frac{x_i}{\pi (1+x_i^2)^2} d x_i=0
        %\quad
        % I(\theta) = \E[\theta]{(\ell'(\theta))^2}
      %   =n\int_{\real} \frac{x^2}{(1+x^2)^2}\frac{1}{\pi(1+x^2)}d x=\frac{n}{4}<\infty 
      \end{displaymath}
      \begin{displaymath}
        \E[\theta]{\ell'(\theta)} = n\int_{\real} \frac{x}{\pi (1+x^2)^2} d x=0
        \quad
        I(\theta) = \D[\theta]^2{\ell'(\theta)}
        =n\int_{\real} \frac{x^2}{(1+x^2)^2}\frac{1}{\pi(1+x^2)}d x<\infty 
      \end{displaymath}
      $I(\theta)$ véges, folytonos és nem nulla. 
      $X$ eloszlásainak családja teljesíti a gyenge regularitási feltételt.
      %$\ell'(\theta)$ eloszlása 
    \end{itemize}  
  \end{frame}
  
  \begin{frame}{Független minták Fisher információja összeadódik}
    \begin{df}[$(R)$, azaz gyenge regularitási feltétel]
      $\cP=\set{f_\theta d\lambda}{\theta\in\Theta}$ dominált mértékcsalád, $\Theta\subset\real^p$ nyílt.
      \begin{enumerate}[<*>]
        \item $\lambda$ majdnem minden $x$-re $\theta\mapsto \sqrt{f_\theta(x)}$ folytonosan differenciálható.
        \item $I(\theta)$ véges $\theta$-ban folytonos, nem szinguláris ($\det I(\theta)>0$).
      \end{enumerate}
    \end{df}
    \begin{proposition}
      $X$, $Y$ minden $\theta\in\Theta$ mellett függetlenek és eloszlásaikra teljesül $(R)$.
  
      Ekkor $(X,Y)$ együttes eloszlásaira is teljesül $(R)$ és $I_{X,Y}(\theta)=I_{X}(\theta)+I_{Y}(\theta)$.
    \end{proposition}
    \begin{itemize}
      \item $X$ eloszlásainak domináló mértéke $\lambda_X$, $Y$-ra $\lambda_Y$. Ekkor $(X,Y)$ eloszlásait 
      $\lambda_X\otimes\lambda_Y$ dominálja. 
      \item $f_{(X,Y),\theta}(x,y)=f_{X,\theta}(x)f_{Y,\theta}(y)$. Ha $\sqrt{f_{X,\theta}(x)}$ $\sqrt{f_{Y,\theta}(y)}$ 
      folytonosan deriválható $\theta$-ban $\lambda_X$ majdnem minden $x$-re ill. $\lambda_Y$ majdnem minden $y$-ra, akkor 
      $\sqrt{f_{X,Y,\theta}(x,y)}$ is folytonosan differenciálható $\lambda_X\otimes\lambda_Y$ m.m $(x,y)$-ra. 
      \item $\ln f_{X,Y,\theta}(x,y)=\ln f_{X,\theta}(x)+\ln f_{Y,\theta}(y)$. 
      Ebből $I_{X+Y}(\theta)=I_X(\theta)+I_{Y}(\theta)$ következik. 
      \item $I_{X+Y}(\theta)$ folytonos, véges értékű.
      \item $I_{X+Y}\geq I_{X}$, amiből $\ker I_{X+Y}\subset \ker I_X=\smallset{0}$, 
      $I_{X+Y}$ is invertálható, minden $\theta$-ra.
    \end{itemize}
  \end{frame}
  
  \begin{frame}{$S(X)$ Fisher információja nem lehet $X$ információ tartalmánál nagyobb}
    \begin{proposition}
      $X$ eloszlásainak a családjára teljesül $(R)$. $S$ statisztika. Ekkor $S$ eloszlásainak a családja is dominált.
  
      Ha $S$ eloszlásainak családjára is teljesül $(R)$, akkor $I_{S(X)}(\theta)\leq I_X(\theta)$.
    \end{proposition}
    \begin{itemize}
      %\item $(R)$ nem függ a domináló mértéktől, $\P[0]\in \cP'$ választható. 
      %($\cP'=\set{\sum c_i\cP[\theta_i]}{c_i\geq0,\,\sum c_i=1}$). 
      \item $\lambda\circ S^{-1}$ domináló mérték $S$ eloszlásaihoz: 
      \begin{displaymath}
        \P[\theta]{S\in H}
        =\int_{\mfX} f_\theta(x) \I{S(x)\in H} \lambda (d x)
        % =\E[0]{\frac{d\P[\theta]|_{\sigma(S)}}{d\P[0]|_{\sigma(S)}}\I{S\in H}}
        =\int_H g_\theta(s) \lambda\circ S^{-1}(d s), 
        \quad g_\theta(S)= \frac{d\P[\theta]|_{\sigma(S)}} {d\lambda|_{\sigma(S)}}. 
      \end{displaymath}
      \item Ha $S$ eloszlásai is teljesítik $(R)$-et, $\P[\theta]{S\in H}$-re adott mindkét formulában az 
      integrálon belül lehet deriválni $\theta$ szerint.
      $\ell_X(\theta,x)=\ln f_{\theta}(x)$ 
      és $\ell_S(\theta,s)=\ln g_\theta(s)$ jelöléssel
      \begin{displaymath}
        %\partial_\theta \P[\theta]{S\in H}
        %=
        \int_{\mfX} \partial_\theta  f_{\theta}(x)\I{S(x)\in H} \lambda(d x)
        =\E[\theta]{\I{S\in H} \ell_X'(\theta,X)}
        =\int_{H} g_\theta'(s) \lambda\circ S^{-1}(d s)
        =\E[\theta]{\I{S\in H}\ell'_S(\theta,S)}
      \end{displaymath}
      \item $\E[\theta]{\ell'_X(\theta,X)|S}=\ell'_S(\theta,S)$ és a teljes szórásnégyzet tétel szerint 
      $I_S(\theta)\leq I_X(\theta)$.
    \end{itemize}
  \end{frame}
  
  \begin{frame}{Elégséges statisztika Fisher információja}
    \begin{proposition}
      $X$ minta eloszlásainak családjára teljesül $(R)$, $S$ elégséges statisztika. 
      Ekkor $S$ eloszlásainak a családjára is teljesül $(R)$ és $I_{S(X)}(\theta)=I_{X}(\theta)$.
    \end{proposition}
    \begin{itemize}
      \item  $(R)$ nem függ a domináló mértéktől, $\P[0]\in \cP'$ választható. 
      ($\cP'=\set{\sum c_i\cP[\theta_i]}{c_i\geq0,\,\sum c_i=1}$). 
      \item $S$ elégséges, ezért $\P[0]$-ra vonatkozó sűrűségek $f_{\theta}(x)=g_{\theta}(S(x))$ alakúak.
      \item $g_\theta$ $S$ eloszlásának a sűrűsége a $\P[0]\circ S^{-1}$ domináló mértékre nézve.
      \item $\sqrt{f_{\theta}(x)}=\sqrt{g_\theta(S(x))}$, azaz ha $(R)$ teljesül, 
      akkor $\sqrt{g_\theta(s)}$ folytonosan deriválható $\P[0]\circ S^{-1}$ m.m. $s$-re. 
      \item $\ell_X'(\theta,X)=\ell_{S}'(\theta,S)$, amiből $I_X(\theta)=I_{S}(\theta)$ és $(R)$ $I$-re vonatkozó 
      előírásai $I_S$-re is igazak.
    \end{itemize}
  \end{frame}
  
  \begin{frame}{Fisher információ és elégséges statisztika}
    \begin{proposition}
      $(R)$ teljesül $X$ és az $S$ statisztika  eloszlásainak családjára is.
       $f_\theta>0$ a mintatéren minden $\theta\in\Theta$-ra és $\Theta$ összefüggő nyílt.
      
       Ha $I_X(\theta)=I_{S(X)}(\theta)$, akkor $S$ elégséges.
    \end{proposition}
    \begin{itemize}
      \item $\P[0]=\P[\theta_0]$ $\theta_0\in \Theta$ választható domináló mértéknek, $f_{\theta_0}\equiv1$.
      \item Láttuk, hogy $\E[\theta]{(\partial_\theta \ln f_\theta)(X)|S }=\E{(\partial_\theta \ln g_\theta)(S)}$, 
      ahol $g_\theta$ $S$ eloszlásainak sűrűsége $\P[\theta_0]\circ S^{-1}$-re.
      \item A feltétel szerint $I_X=I_S$, ami csak akkor lehet, ha 
      $\partial_\theta\ln f_\theta(x)=\partial_\theta g_\theta(S(x))$ minden $\theta$-ra $\P[0]$ m.m..
      \item $\partial_\theta\ln f_\theta(x)$, $\partial_\theta\ln g_\theta(x)$ $\P[0]$ m.m. 
      $x$-re folytonos $\theta$-ban $(R)$ miatt. 
      \item $\P[0]{ \forall\theta\in\Theta,\,\partial_\theta\ln f_\theta(x)=\partial_\theta g_\theta(S(x))}=1$.
      \item 
      \begin{displaymath}
        \ln f_{\theta_1}(x)-\ln f_{\theta_0} (x) = \int_{\gamma} \partial_\theta \ln f_\theta(x) d\theta=
        \int_\gamma \partial_\theta \ln g_\theta (S(x))d\theta= \ln g_{\theta_1}(S(x))-\ln g_{\theta_0}(S(x))
      \end{displaymath}
      \begin{displaymath}
        f_\theta(x)=\frac{g_\theta(S(x))}{g_{\theta_0}(S(x))}
      \end{displaymath}
      azaz $S$ elégséges a faktorizációs tétel miatt.
    \end{itemize}
  \end{frame}
  \end{document}
  \begin{frame}{Összefoglalás}
    
  \end{frame}
  
  \begin{frame}{Becslési Módszerek}
    \begin{itemize}
      
    \item Tapasztalati becslések
    \item Momentum módszer
    \item Maximum likelihood becslés
    \item Bayes becslés
    \end{itemize}
  \end{frame}
  
  \begin{frame}{Néhány fogalom}
      \begin{df}[Tapasztalati eloszlás]
        $X_1,\dots, X_n$ minta, $P_n^*=\frac1n \sum_{i}\delta_{X_i}$ a
        tapasztalati eloszlás.
      \end{df}
  
      Ki fogjuk számolni, hogy ha $X_1,X_2,\dots$ a $P$ eloszlásból
      származó független megfigyelések, akkor
      $P_n^*\dto P$ egy valószínűséggel.
      (Glivenko-Cantelli tétel)
  
      \begin{df}
        $\cP=\set{\P[\theta]}{\theta\in
          \Theta}$,
        $X_1,X_2,\dots$ iid. minta,
        $T_n=T_n (X_1,\dots,X_n)$ 
        $g (\theta)$ becslése $n$ elemű mintából.
  
        A $T_n$
        becsléssorozat konzisztens, ha $T_n\pto g (\theta)$, és erősen
        konzisztens ha $T_n\to g (\theta)$ egy valószínűséggel.
      \end{df}
  
  \end{frame}
  
  \begin{frame}{Tapasztalati becslések}
    \begin{df}
        $\cP$ eloszlások egy családja, amely tartalmazza a véges sok
        pontra koncentrált eloszlásokat. $\phi$ a $\cP$-n értelmezett
        funkcionál.
  
        $\phi (P)$ tapasztalati becslése $\phi (P_n^*)$, ahol $P_n^*$ a
        tapasztalati eloszlás. 
      \end{df}
  
    \begin{itemize}
    \item Példa. $N (\mu,1)$ eloszláscsaládnál kézenfekvő a paramétert a
      mintaátlaggal becsülni.
    \item A paraméter a várható érték, ami véges tartójú eloszlásokra is
      létezik. A becslés a tapasztalati eloszlásból számolt várható érték.
    
    \item Ha a minta a $P$ eloszlásból származik, akkor $P_n^*\dto P$
      egy valószínűséggel, azaz ha $\phi$ a gyenge konvergenciára 
      nézve folytonos, akkor $\phi$ tapasztalati becslése erősen
      konzisztens
      
    \item pl. tapasztalati medián, ha egyértelmű, erősen konzisztens a mediánra.
    \end{itemize}
    
  \end{frame}
  
  \begin{frame}{Momentum módszer}
    \begin{itemize}
    \item $\cP=\set{\P[\theta]}{\theta\in\Theta}$, $\theta$-t akarjuk
      becsülni.
    \item Legyen $g:\mathfrak X_1\to\real^p$, és $\phi
      (\theta)=\E[\theta] (g (X_1))$.
      Tegyük fel, hogy $\phi:\Theta\to T$ kölcsönösen egyértelmű és oda-vissza
      folytonos, $T$ nyílt. 
    \item $\psi=\phi (\theta)$  tapasztalati becslése
      \begin{displaymath}
        \hat\psi_n=\frac1n \sum_i g (X_i)
      \end{displaymath}
    \item Ha $n$ elég nagy, akkor $\hat\psi\in T$ és
      $\hat\theta=\phi^{-1} (\hat\psi)$ értelmes. 
    \item $\hat\psi_n\to\psi$ egy valószínűséggel és $\phi^{-1}$ folytonossága
      miatt $\hat\theta_n$ erősen konzisztens, de általában nem torzítatlan.
  
      $\hat\psi_n$ torzítatlan $\psi$, de $\phi^{-1}$ általában nem
      lineáris.
      
    \item Klasszikus momentum módszernél $g (x)=(x_1^1,\dots,x_1^p)$.
    \end{itemize}
  
  \end{frame}
   
  \begin{frame}{Példa}
    
  \end{frame}
  
  \begin{frame}{Maximum likelihood becslés}
    \begin{itemize}
    \item $\cP$ dominált mértékcsalád, $f_\theta$ a sűrűségfüggvény,
      $\ell (\theta)$ a loglikelihood.
      \begin{df}
        A $\theta$ paraméter maximum likelihood becslése $\hat\theta (X)$,
        ha $f_{\hat{\theta}} (X)=\sup_{\theta}f_{\theta} (X)$.
      \end{df}
  
    \item Nem biztos, hogy létezik, nem biztos, hogy egyértelmű.
  
      pl. $U (\theta,\theta+1)$ eloszláscsalád. $f_\theta
      (x)=\I{X_n^*-1\leq \theta\leq X_1^*}$, azaz ha  $T (X)\in
      [X_n^*-1,X_1^*]$, akkor $T$ maximum likelihood becslése
      $\theta$-nak.
  
      pl. $f (x)$ sűrűségfüggvény $\real$-en.
      $$f_{\mu,\sigma}
      (x)=\frac{1}\sigma f \zfrac*{x-\mu}{\sigma}
      $$
      $\Theta=\real\times (0,\infty)$. $c\in\real$, $f (c)>0$, $x$ a
      megfigyelt érték
      \begin{displaymath}
        \sup_{\mu,\sigma}\frac{1}{\sigma}f \zfrac*{x-\mu}{\sigma}\geq
        \lim_{\sigma\to0}\frac{1}{\sigma} f\zfrac*{x- (x+c\sigma)}{\sigma}=\infty
      \end{displaymath}
    \end{itemize}  
  \end{frame}
  
  \begin{frame}{Maximum likelihood becslés}
    $\psi=g(\theta)$ becslése maximum likelihood elvvel. Ha $g$ kölcsönösen
    egyértelmű, akkor csak átparamétereztük a családot,
    $\hat\psi=g(\hat\theta)$.
  
    Ha $g$ nem injektív, akkor az indukált likelihoodot szokták
    maximalizálni
    \begin{displaymath}
      f^*_\psi (x)=\sup\set{f_\theta (x)}{g (\theta)=\psi}
    \end{displaymath}
  
    \begin{proposition}
      \begin{itemize}
      \item Ha $T$ elégséges statisztika és létezik ML becslés, akkor
        van olyan is, amelyik $T$ függvénye.
      \item Ha $\hat\theta$ a $\theta$ ML becslése, akkor $g
        (\hat\theta)$ a $g (\theta)$ ML becslése.
      \end{itemize}
    \end{proposition}
    \begin{itemize}
    \item Ha a maximum hely egyértelmű, akkor a faktorizációs tétel
      miatt $f_\theta (x)=g_\theta (T (x)) h (x)$ csak $T (x)$-en
      keresztül függ a megfigyeléstől. Ha nem egyértelmű, akkor arra
      kell figyelni, hogy $T (x)$ függvényében válasszunk.
      
      pl. $U (\theta,\theta+1)$ esetén elégséges statisztika
      $T (X) =(X_1^*,X_n^*)$. $\hat\theta(=X_n^*-1+X_1^*)/2$ ML becslés $T$
      függvényei között,
      \begin{displaymath}
        \hat\theta+\sin (X_1)\frac{X_1^*-(X_n^*-1)}2
      \end{displaymath}
      pedig nem (feltéve, hogy a minta elemszáma legalább 2).
    \end{itemize}  
  \end{frame}
  
  \begin{frame}{Bayes becslés, bevezetés}
    $\cP=\set{\P[\theta]}{\theta\in\Theta}$ eloszláscsalád. $g
    (\theta)$-t szeretnénk becsülni.
    \begin{displaymath}
      L (\theta,T)=(g(\theta)-T)^2,\quad R_T(\theta)=\E[\theta] (L (\theta,T))
    \end{displaymath}
    $L (\theta,T)$ a veszteség, amit a $T$ becslés okoz, $R_T (\theta)$
    az átlagos veszteség, más néven rizikó.
  
    Bayes becslésnél a cél a
    \begin{displaymath}
      R_T (Q)=\int_\Theta R_T (\theta) Q (d\theta)\quad \text{apriori rizikó}
    \end{displaymath}
    minimalizálása.
  
    $Q$ gyakran valószínűség eloszlás. Ilyenkor úgy gondolhatunk a feladatra,
    hogy $\theta$ is valószínűségi változó, melynek eloszlása (a priori eloszlás)
    $Q$.
    $\P[t]$ a minta feltételes eloszlása a $\theta=t$ feltétel mellett.
  
    \begin{displaymath}
      R_T (Q)=\E{L (\theta,T)}=\E{(g (\theta)-T (X))^2}\geq \E{(g (\theta)-\E{g (\theta)|X})^2}
    \end{displaymath}
  
    A Bayes becslés $g (\theta)$ feltételes várható értéke a mintára nézve.
  \end{frame}
  
  \end{document}
  
    