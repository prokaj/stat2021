%\def\option{}
\providecommand{\option}{handout}
\documentclass[aspectratio=169,notheorems,9pt,\option]{beamer}



\makeatletter
\def\readrest#1\enddate{}
\def\setdate@#1-#2-#3\relax{\def\lecturedate{#1.#2.#3.}\readrest}
\def\setdate{\expandafter\setdate@\jobname\relax-\relax-\relax\enddate}

\makeatother

\setdate


\usepackage{marvosym}
%\usepackage[magyar]{babel}
\usepackage{polyglossia}
\setmainlanguage{magyar}

% \usetheme{Bruno}

\usecolortheme{Bruno}
% \usefonttheme{Bruno}
% \useinnertheme{Bruno}
\useoutertheme{Bruno}

\usepackage{pvdefs}
\newtheorem{df}{Definíció}
\newtheorem{theorem}{Tétel}
\newtheorem{lemma}{Lemma}
\newtheorem{proposition}{Állítás}
\newtheorem{corollary}{Következmény}

\defcx{XAPIDLCSGTHEN}
\deftx{STQbCf\lambda\Omega\sigma}
\defmf{X}

\def\diag{\operatorname{diag}}
\def\Id{\mathbf{I}}
\let\PEfont\mathbb
\def\npto{\buildrel p\over \nrightarrow}
\def\ndto{\buildrel d\over \nrightarrow}
\def\dto{\buildrel d\over \rightarrow}
\def\dto{\stackrel{d}{\to}}
\let\nto\nrightarrow

\def\Leb{\operatorname{Leb}}

\newrobustcmd{\continue}{%
  \onslide<\value{beamerpauses}->\relax
}


\beamerdefaultoverlayspecification{<+->}

\mode<presentation>

\RequirePackage[math-style=TeX, bold-style=upright]{unicode-math}
\setmathfont{Latin Modern Math}[Scale=1]
% \setsansfont{Kurier Cond}
\setsansfont[Ligatures=TeX, ItalicFont={Fira Sans Italic}, BoldFont={Fira Sans SemiBold}, Scale=0.9]{Fira Sans Book}
%\setsansfont{Ubuntu}

%\usepackage{mathspec}
%\setmathsfont{lm}
%(Digits,Latin,Greek)
%[Numbers={Lining,Proportional}]{lm}
%\setmathsfont(Digits,Greek)
%[Uppercase=Plain,Lowercase=Regular,Scale=MatchLowercase]
%{GFS Porson}


% Title page
\setbeamerfont{title}{
    size=\LARGE,
    shape=\bfseries
}

\setbeamerfont{subtitle}{
    size=\large,
    shape=\normalfont
}

\setbeamerfont{author}{
    size=\normalsize,
    shape=\normalfont
}

\setbeamerfont{details}{
    size=\footnotesize,
    shape=\normalfont
}

% Slide title
\setbeamerfont{frametitle}{size=\LARGE}
\setbeamerfont{framesubtitle}{
    size=\normalsize,
    %shape=\normalfont\scshape
}

% Blocks
\setbeamerfont{block title}{
    size=\normalfont,
    shape=\strut
}

\setbeamerfont{blockdef}{
    size=\large,
    shape=\bfseries
}

% Description
\setbeamerfont{description item}{shape=\bfseries}


% Footer information line with title, author and slide number
\setbeamertemplate{footline}{%
    \leavevmode%
    \hbox{%
        \usebeamerfont{footline}%
        \begin{beamercolorbox}[
            wd=\textwidth,
            ht=3ex,
            dp=1.25ex
        ]{footline}%
            \hspace{1cm}%
            \insertshorttitle%
            \hspace{2em}%
            \insertdate%
            \hfill%
            \insertframenumber{} / \inserttotalframenumber%
            \hspace{1cm}
        \end{beamercolorbox}%
    }
    \vskip0pt%
}

%% Bruno Beamer theme — Version 1.0
%% A lightweight Beamer theme inspired from the Metropolis theme
%%
%% Written in 2017-2019 by:
%% — Rémi Cérès <remiceres@msn.com>
%% — Mattéo Delabre <bonjour@matteodelabre.me>
%%
%% This work is released under the CC0 1.0 Universal license. See the
%% accompanying LICENSE file for details. To the extent possible under
%% law, Rémi Cérès and Mattéo Delabre have waived all copyright and
%% related or neighboring rights to the Bruno Beamer theme. This work
%% is published from France.

% \mode<presentation>

\setbeamertemplate{sections/subsections in toc}[square]
\setbeamertemplate{itemize item}[square]
\setbeamertemplate{itemize subitem}[circle]

% Allow multi-slide frames
\setbeamertemplate{frametitle continuation}{}

% Horizontal separator
\setbeamertemplate{separator}{%
    \usebeamercolor{separator}%
    \textcolor{fg}{\rule{.7\textwidth}{.5pt}}%
}

% Separate authors with newlines
\renewcommand{\beamer@andtitle}{\\}

% Title page. If you want to add a background image on the title slide,
% use the \background command to set the path to that image. Otherwise,
% no image will be included
\RequirePackage{tikz}
\usetikzlibrary{fadings}
\newcommand{\background}[1]%
    {\newcommand{\bruno@background}{#1}}


%% \newcommand\insertcaption{}
\newcommand{\backgroundcaption}[2][white]{\def\bruno@caption{{#1}{#2}}}

\newcommand{\insertcaption}{\ifdefined\bruno@caption{\expandafter\textcolor\bruno@caption}\fi}

\tikzfading[
    name=title page picture fading,
    left color=transparent!0,
    right color=transparent!100,
]

\setbeamertemplate{title page}{%
    \begin{minipage}{.7\textwidth}
        \raggedright
        \usebeamerfont{title}
        \inserttitle

        \usebeamertemplate{separator}

        \usebeamerfont{author}
        \vspace{2em}
        \insertauthor

        \vspace{2em}
        \usebeamerfont{details}
        \insertinstitute

        \insertdate
    \end{minipage}

    % Include faded image on the right if defined
    \ifdefined\bruno@background
    \begin{tikzpicture}[remember picture, overlay]
        % Crop image to a trapezium on the right
        \clip (current page.south east)
           -- (current page.north east)
           -- ++(-5.1, 0)
           -- ++(-3, -\paperheight)
           -- cycle;

        % Include background image
        \node[
            anchor=south east,
            inner sep=0,
            outer sep=0
        ] at (current page.south east) {
            \includegraphics[height=\paperheight]
                {\bruno@background}
        };

        \node[anchor=south east,
            inner sep=0,
            outer sep=15pt,
        ] at (current page.south east){\insertcaption};

        % Add a slight shadow
        \fill[
            black, path fading=title page picture fading,
            fading angle=-15
        ]
            (current page.south east)
            rectangle
            ++(-10, \dimexpr\paperheight+1cm);

        
    \end{tikzpicture}
    \fi
}

\mode<all>
\AtBeginDocument{%
  \let\phi\varphi
  \let\theta\vartheta
  \let\setminus\smallsetminus
}


\title{Statisztika előadás}
\date{\lecturedate}
\background{images/young_Ronald_Fisher.jpg}
\backgroundcaption{Ronald Fisher (1913)}


\begin{document}

\maketitle

\begin{frame}{Rásonyi Miklós habilitációs előadása}
\vfill
  \begin{center}
    \textbf{Centrális határeloszlás tétel}
    
    \vspace{1cm}

    2021. április 16. (péntek), 10 óra 30 perc
\end{center}
\vfill

\url{https://zoom.us/j/99761822815?pwd=dTdKM3l5M1NpTER6T0JLbW5MNVhiZz09}{}\\
%\begin{verbatim}
Meeting ID: 997 6182 2815\\
Passcode: 286612
%\end{verbatim}
\end{frame}

\begin{frame}{Emlékeztető}
  $\cP=\set{\P[\theta]}{\theta\in \Theta}$, exponenciális család, $\Theta$ konvex, nyílt. $f_\theta$ jelöli 
  \textbf{egy mintaelem} sűrűségfüggvényét.
  \begin{theorem}
    $X_1,X_2,\dots$ független azonos eloszlású megfigyelések $f_{\theta_0}$ ($\theta_0\in\Theta$) 
    közös sűrűségfüggvényű eloszlásból. $\hat\theta_n$ a $\theta$ ML becslése az első $n$ megfigyelés alapján.
    %
    Ekkor $\hat\theta_n\to\theta_0$ és $\sqrt{n}\zjel{\hat\theta_n-\theta_0}\dto N(0, I(\theta_0)^{-1})$.
  \end{theorem}
  \textbf{Az ML becslés (szép eloszláscsaládban) konzisztens és aszimptotikusan normális}.
\end{frame}

\begin{frame}[<*>]{Általánosítás, konzisztencia}

    $X_1,X_2,\dots$ független azonos eloszlású megfigyelések $f_{\theta_0}$  közös sűrűségfüggvénnyel. 
    $\Theta\subset\real^p$ nyílt. $f_\theta$, $\theta\in\Theta$ teljesíti a gyenge regularitási feltételt.
  
    Alkalmas feltételek mellett a következő lépések igazak.
   \begin{itemize}
  
      \item Elég kicsi $\delta>0$-ra  
        \begin{displaymath}
          S_n = \sup_{\abs{t}\leq \delta} 
          \tfrac1n\zjel{\ell_n(\theta_0+t)-\ell_n(\theta_0)}+ \tfrac14 t^T I(\theta_0) t,
          \quad 
          \text{jelöléssel}
          \quad
          \lim_{n\to\infty} S_n\leq 0\quad\text{egy valószínűséggel.}
        \end{displaymath}
      \item $I(\theta_0)>0$, ezért $\frac14 t^T I(\theta_0) t \geq c\abs{t}^2$ alkalmas $c>0$-val.
      \item Jelölje $\hat\theta_n$ a loglikehood $\ell_n$ maximum helyét 
      a $\set{\theta}{\abs{\theta-\theta_0}\leq\delta}$ halmazon.  
        \begin{displaymath}
          0\leq \tfrac1n\zjel{\ell_n(\hat\theta_n)-\ell_n(\theta_0)}
          \leq S_n-c\abs{\hat\theta_n-\theta_0}^2 %\frac14 t^T I(\theta_0) t
          \quad\text{amiből $\abs{\hat\theta_n-\theta_0}^2\leq S_n/c\to 0$}.
        \end{displaymath}
      \item Elég nagy $n$-től kezdve $\abs{\hat\theta_n-\theta_0}<\delta$ és $\ell_n'(\hat\theta_n)=0$.
      
      Ez adja az ML becslés konzisztenciáját.
    \end{itemize}
  \end{frame}
  
  \begin{frame}[<*>]{Általánosítás, aszimptotikus normalitás}
  
    $X_1,X_2,\dots$ független azonos eloszlású megfigyelések $f_{\theta_0}$  közös sűrűségfüggvénnyel. 
    $\Theta\subset\real^p$ nyílt. $f_\theta$, $\theta\in\Theta$ teljesíti a gyenge regularitási feltételt.
  
  Alkalmas feltételek mellett a következő lépések igazak.
  \begin{itemize}
    \item CHT alapján:
      \begin{displaymath}
        n^{-1/2}\ell'_n(\theta_0)
        =n^{-1/2}\sum_{i=1}^n \ell'(\theta_0,X_i)\dto N(0,I(\theta_0))  
      \end{displaymath}
    \item $\hat\theta_n=\arg\max_{\abs{\theta-\theta_0}\leq \delta} \ell_n(\theta)$. 
    Ekkor $n^{-1/2}\ell'_n(\hat\theta_n)\to0$
      \begin{displaymath}
        n^{-1/2}\zjel{\ell'(\hat\theta_n)-\ell'(\theta_0)}
        =\zjel*{\frac{1}{n}\ell_n''(\theta_0)+R_n}\sqrt{n}(\hat\theta_n-\theta_0)\dto N(0,I(\theta_0))
      \end{displaymath}
      Itt $R_n\pto0$ és 
      $\frac1n\ell_n''(\theta_0)\to \E[\theta_0]{\ell''(\theta_0, X_1)}=\alert{-I(\theta_0)}$.
    \item $I(\theta_0)$ invertálható. Létezik $K_n$ sorozat, 
    hogy $K_n\zjel*{\frac{1}{n}\ell_n''(\theta_0)+R_n}=\Id_p$ ha $n$ elég nagy. 
    
    Ebből $K_n\to I^{-1}(\theta_0)$ és 
      \begin{displaymath}
        \sqrt{n}(\hat\theta_n-\theta_0)-K_n n^{-1/2}\zjel{\ell'(\hat\theta_n)-\ell'(\theta_0)}\to0\implies
        \sqrt{n}(\hat\theta_n-\theta_0)\dto N(0,I^{-1}(\theta_0))
        %\implies
        %\sqrt{n}(\hat\theta_n-\theta_0)\dto N(0,I^{-1}(\theta_0)).
      \end{displaymath}
      hiszen $K_n n^{-1/2}\zjel{\ell'(\hat\theta_n)-\ell'(\theta_0)}\dto N(0,I^{-1}(\theta_0))$.
  \end{itemize}
\end{frame}
  
\begin{frame}{Erős regularitási feltétel}
  \begin{df}[$(RR)$]
      $(R)$ + $\theta\mapsto\ell(\theta,x)\in C^2$ 
      és $\cP$ m.m. $x$-re és majorálható ,,egyenletesen integrálható''
      $M$-mel. Azaz
      \begin{itemize}
      \item $\norm{\partial_\theta^2\ell (\theta,x)}\leq M (x)$,
      \item $\sup_\theta\E[\theta]{M^2}<\infty$. 
      \end{itemize}
  \end{df}
    
  \begin{theorem}
      $X_1,X_2,\dots$ iid sorozat $f_{\theta_0}$ sűrűségfüggvényből. 
      $(RR)$  teljesül a $\set{f_\theta}{\theta\in\Theta}$ eloszláscsaládra.
  
      $\ell_n(\theta) = \sum_{i=1}^n \ell_1(\theta,X_i)$ az $n$ elemű minta loglikelihoodja.
      \begin{itemize}
      \item Az $\ell'_n(\theta)=0$ likelihood egyenletnek létezik olyan $\hat\theta_n$ gyöke,
        amely lokális maximum és erősen konzisztens
      \item Erre a gyökre  
      $\sqrt{n} (\hat\theta_n-\theta_0)\dto N(0,I^{-1} (\theta_0))$ 
      \end{itemize}
  \end{theorem}  
    $(RR)$-t elég lokálisan megkövetelni.
\end{frame}
  
\begin{frame}{Erős regularitási feltétel következményei I.}
  \begin{df}[RR]
      (R) + $\theta\mapsto\ell(\theta,x)\in C^2$ $\mathcal P$
      m.m. $x$-re és majorálható ,,egyenletesen integrálható''
      $M$-mel. Azaz
      \begin{itemize}
      \item $\norm{\partial_\theta^2\ell (\theta,x)}\leq M (x)$,
      \item $\sup_\theta\E[\theta]{M^2}<\infty$. 
      \end{itemize}
  \end{df}
    
  \begin{lemma}
      (RR) esetén
      \begin{displaymath}
        \eta (\eps)
        = \E[\theta]{\sup_{\abs{t}<\eps}\norm{\ell''(\theta+t)-\ell''(\theta)}}
        \to0\quad\text{ha $\eps\to0$}
      \end{displaymath}
  \end{lemma}
    Dominált konvergencia tétel:
    \begin{displaymath}
      \sup_{\abs{t}\leq\eps}\norm*{\ell''(\theta+t,x)-\ell''(\theta,x)}\leq 2M (x)
    \end{displaymath}
    és $t\mapsto \ell''(t,x)$ folytonos. 
    
\end{frame}
  
\begin{frame}{Erős regularitási feltétel következményei II.}
    \begin{df}[RR]
      (R) + $\theta\mapsto\ell(\theta,x)\in C^2$ $\mathcal P$
      m.m. $x$-re és %majorálható ,,egyenletesen integrálható'' $M$-mel. Azaz
      $\norm{\partial_\theta^2\ell (\theta,x)}\leq M (x)$,
      $\sup_\theta\E[\theta]{M^2}<\infty$. 
    \end{df}
    
    \begin{lemma}
      (RR) esetén $\theta\mapsto \E[\theta]{\ell''(\theta)}$ folytonos.
    \end{lemma}
    \begin{itemize}
      \item $\theta_n\to \theta$
      \begin{displaymath}
        \E[\theta_n]{\ell''(\theta_n)}-\E[\theta]{\ell''(\theta)}=\E[\theta]{\ell''(\theta_n)-\ell''(\theta)}+
        \int \ell''(\theta_n,x)\zjel{f_{\theta_n}(x)-f_{\theta}(x)}dx
      \end{displaymath}
      \item $\ell''(\theta_n)\to \ell''(\theta)$ $\cP$ m.m. és $2M$ integrálható majoráns $\implies$
         $ \lim_{n\to\infty}\E[\theta]{\ell'(\theta_n)-\ell'(\theta)}=0$
      \item Levágás $+$ Scheffé tétel
      \begin{align*}
        \norm*{\smallint \ell''(\theta_n,x)\zjel{f_{\theta_n}(x)-f_{\theta}(x)}dx}
        &\leq 
        \smallint \norm*{\ell''(\theta_n,x)}\abs{f_{\theta_n}(x)-f_{\theta}(x)}dx
        \leq \smallint M(x)\abs{f_{\theta_n}(x)-f_{\theta}(x)} dx\\
        &\leq K\smallint \abs{f_{\theta_n}(x)-f_{\theta}(x)} dx +\E[\theta_n]{M\I{M>K}}+\E[\theta]{M\I{M>K}}
      \end{align*}
      Első tag nullához tart  minden rögzített $K$-ra a Scheffé tétel miatt.
      $\sup_\theta \E[\theta]{M\I{M>K}}\leq \frac1 K \sup_{\theta}\E[\theta] {M^2}\to0$ ha $K\to\infty$.
      $(RR)$ miatt.
    \end{itemize}
\end{frame}
  
\begin{frame}{Erős regularitási feltétel következményei III.}
    % \begin{df}[RR]
    %   (R) + $\theta\mapsto\ell(\theta,x)\in C^2$ $\mathcal P$
    %   m.m. $x$-re és %majorálható ,,egyenletesen integrálható'' $M$-mel. Azaz
    %   %\begin{itemize}
    %   %\item 
    %   $\norm{\partial_\theta^2\ell (\theta,x)}\leq M (x)$,
    %   %\item 
    %   $\sup_\theta\E[\theta]{M^2}<\infty$. 
    %   %\end{itemize}
    % \end{df}
    
    \begin{lemma}
      (RR) esetén $\theta\mapsto \E[\theta]{\ell''(\theta)}$ folytonos.
    \end{lemma}
    \begin{corollary}
      (RR) esetén % $\int \partial_\theta^2 f_\theta(x) dx=0$ és 
      $\E[\theta]{\ell''(\theta)}=-I(\theta)$.
    \end{corollary}
    \begin{itemize}
      \item $(R)$ miatt $\int \partial_\theta f_\theta(x)dx=0$ minden $\theta\in\Theta$-ra.
      \item $e\in\real^p$ egység vektor
      \begin{displaymath}
        0 = \int \partial_\theta f_{\theta_1+t  e}(x)-\partial_\theta f_{\theta_1}(x)d x=
        \int_0^1 \int \partial^2_\theta f_{\theta_1+u te}(x) d x d u t e
      \end{displaymath}
      \item 
      \begin{displaymath}
        \int \partial^2_\theta f_{\theta}(x) dx=\E[\theta]{\ell''(\theta)}+I(\theta),
        \quad\text{hiszen}\quad
        (\ln f)'' =\frac{f''}{f}-\frac{(f')^T f'}{f^2}
      \end{displaymath}
      \item $t\to0$ mellett, minden $e\in\real^p$ egységvektorra
      \begin{displaymath}
        0 = \int_0^1   \E[\theta]{\ell''(\theta+ut e)}+I(\theta+u t e) d u e
        \to  (\E[\theta]{\ell''(\theta)}+I(\theta)) e 
        \implies \E[\theta]{\ell''(\theta)}+I(\theta)=0.
      \end{displaymath}
    \end{itemize}
\end{frame}
  
  
\begin{frame}{Maximum likelihood becslés aszimptotikus tulajdonságai}
    \begin{theorem}
      (RR) esetén a  likelihood egyenletnek létezik olyan $\hat\theta_n$ gyöke,
        amely lokális maximum és erősen konzisztens
    \end{theorem}
    \begin{itemize}
      \item $L_n(\theta)=\frac1n\ell_n(\theta)=\frac1n\sum_{i=1}^n\ell (\theta,X_i)$. % Ha $\norm{h}=\eps$
      \begin{displaymath}
        L _n(\theta+t)-L_n (\theta)=L'_n (\theta) t+t^T\zjel*{\tfrac12 L_n''(\theta) + R_n(t)} t
      \end{displaymath}
      \item Hibatag:  
      \begin{displaymath}
        R_n(t)=\int_0^1\int_0^v L''_n (\theta+u t)-L_n''(\theta) d u d v
      \end{displaymath}
      és %$\norm*{h}\leq\eps$
      \begin{multline*}
        \sup_{t:\norm*{t}\leq \delta}\norm*{R_n(t)}\leq
        \frac12\sup_{\norm*{t}\leq\delta}\norm{L_n''(\theta+t)-L_n''(\theta)}
        \leq
        \frac12\frac1n\sum_{i=1}^n\sup_{\abs*{t}\leq\delta}
        \norm{\partial^2\ell(t+\theta,X_i)-\partial^2\ell(\theta,X_i)}\\
        \\
        \to
        \frac12\E[\theta]{\sup_{\norm*{t}\leq\delta}
          \norm{\partial^2_\theta\ell(t+\theta,X_1)-\partial^2_\theta\ell (\theta,X_1)}}
        =\frac12\eta(\delta),\quad\text{ahol $(RR)$ miatt $\lim_{\delta\to0}\eta (\eps)=0$.}
      \end{multline*}
    \end{itemize}  
\end{frame}
  
\begin{frame}{Maximum likelihood becslés aszimptotikus tulajdonságai}
    \begin{theorem}
      (RR) esetén a  likelihood egyenletnek létezik olyan $\hat\theta_n$ gyöke,
        amely lokális maximum és erősen konzisztens
    \end{theorem}
    $L_n(\theta)=\frac1n\sum_{i=1}^n\ell (\theta,X_i)$. %Ha $\norm{h}=\eps$
    \begin{displaymath}
      \sup_{\abs{t}\leq \delta} L_n (\theta+t)-L (\theta)+\tfrac14 t^T I(\theta) t
      = \sup _{\abs{t}\leq \delta} L_n' (\theta) t+t^T\zjel{\tfrac12L_n''(\theta)+\tfrac14 I(\theta) +R_n(t)}t
    \end{displaymath}
  
    \begin{displaymath}
      \tfrac12L_n''(\theta)+\tfrac14 I(\theta)\to \tfrac12\E[\theta]{\ell''(\theta)}+\tfrac14I(\theta)=-\tfrac14I(\theta)
    \end{displaymath}
    \continue
    Ha $\delta$ olyan kicsi, hogy $\tfrac12\eta(\delta)\Id_p<\tfrac14 I(\theta)$, akkor
    \begin{displaymath}
      \lim_{n\to\infty} \sup_{\abs{t}\leq \delta} t^T\zjel{\tfrac12L_n''(\theta)+\tfrac14 I(\theta) +R_n(t)}t\leq 0
    \end{displaymath}
    míg
    \begin{displaymath}
      L_n' (\theta) t\to \E[\theta]{\ell'(\theta)} t=0
    \end{displaymath}
\end{frame}
  
  
  
\begin{frame}{Maximum likelihood becslés aszimptotikus tulajdonságai}
    \begin{theorem}
      (RR) esetén
      \begin{itemize}
      \item A likelihood egyenletnek létezik olyan $\hat\theta_n$ gyöke,
        amely lokális maximum és erősen konzisztens
      \item Erre a gyökre
        $\sqrt{n} (\hat\theta_n-\theta)\dto N (0,I^{-1} (\theta))$
        ($\P[\theta]$ alatt), vagyis $\hat\theta_n$ aszimptotikusan
        normális. %optimális.
      \end{itemize}
    \end{theorem}
  
    Legyen $\xi_n=\hat\theta_n-\theta$. Ekkor
    $\xi_n\to0$ egy valószínűséggel.
    Mivel $\hat\theta_n$ a likelihood egyenlet megoldása
    \begin{displaymath}
      0=L'_n (\hat\theta_n)
      =L'_n (\theta)+\zjel{ L''_n (\theta)+ R_n}\xi_n
    \end{displaymath}
    amiből
    \begin{displaymath}
      -\zjel{L''_n(\theta)+ R_n}\sqrt{n} \xi_n
      =\sqrt{n}L_n' (\theta)
    \end{displaymath}
    ahol nagy $n$-re $\sqrt{n}L_n' (\theta)$ eloszlása közel $N(0,I (\theta))$ a CHT miatt, 
    míg $-\zjel{L''_n(\theta)+ R_n}$ nagy valószínűséggel $I (\theta)$-hoz van közel a nagy
    számok erős törvénye miatt.
  
    Ebből $\sqrt{n}\xi_n$ eloszlása közel $N (0,I^{-1} (\theta))$, ha $n$ nagy.
\end{frame}
  
\begin{frame}{Maximum likelihood becslés aszimptotikus tulajdonságai}
    \begin{theorem}
      (RR) esetén
      \begin{itemize}
      \item A likelihood egyenletnek létezik olyan $\hat\theta_n$ gyöke,
        amely lokális maximum és erősen konzisztens
      \item Erre a gyökre
        $\sqrt{n} (\hat\theta_n-\theta)\dto N (0,I^{-1} (\theta))$
        ($\P[\theta]$ alatt), vagyis $\hat\theta_n$ aszimptotikusan
        optimális.
      \end{itemize}
    \end{theorem}
  
    \begin{displaymath}
      R_n=\int_0^1 L''(\theta+u \xi_n)-L''(\theta) d u
      \quad\text{amiből}\quad
      \limsup\norm{R_n}\leq \inf_{\delta>0} \lim_{n\to\infty}\sup_{\abs{t}
      \leq\abs{\delta}} \norm{L''_n(\theta+t)-L''_n(\theta)}=\inf_{\delta>0}\eta (\delta)=0
    \end{displaymath}
  
    \begin{displaymath}
      K_n =
      \begin{cases}
        -(L''_n(\theta)+R_n)^{-1}&\text{ha
          $(L''_n(\theta)+R_n)$ invertálható}\\
        \text{egységmátrix}&\text{különben}
      \end{cases}
    \end{displaymath}
    $I (\theta)$ invertálható
    \begin{displaymath}
      -(L''_n(\theta)+R_n)\to I (\theta)\quad\implies\quad
      \sqrt{n}\xi_n-K_n\sqrt{n}L'_n (\theta)\to0\quad\text{és}\quad K_n\to I (\theta)^{-1}
    \end{displaymath}
    Azaz $\sqrt{n}\xi_n$ és $K_n\sqrt{n}L'_n (\theta)$ eloszlásbeli limesze
    egyszerre létezik és megegyezik
    \begin{displaymath}
      \sqrt{n}L'_n (\theta)\dto N (0,I (\theta)),\quad\implies\quad
      K_n\sqrt n L'_n (\theta)\dto N (0,I^{-1} (\theta)I(\theta)I^{-1} (\theta))
      =N (0,I^{-1} (\theta))
    \end{displaymath}
      
\end{frame}

\begin{frame}{Példa, keverék eloszlás}
  \begin{itemize}
    \item Egy mintaelem sűrűségfüggvénye $f_{\btheta}(x)=\sum_k p_k g_{\theta_k}(x)$, ahol 
    $\btheta=(p,(\theta_k)_{k=1}^r)$ és $p\in\Delta_r=\set{p\in (0,1)^r}{\sum p_i=1}$, 
    $g_{\theta}$, $\theta\in\Theta$ exponenciális családot alkot. 
    \item Megmutatható, hogy $f_\btheta$, $\btheta\in\bTheta=\Delta\times\Theta^k$ 
    teljesíti az erős regularitási feltételt, minden olyan $\btheta=(p,(\theta_k)_{k=1}^r)$ pont körül, amire 
    $\theta_1,\dots,\theta_r$ mind különbözőek.
    \item A likelihood egyenletnek létezik olyan megoldása, ami konzisztens becsléssorozatot ad.
    \item A loglikelihood függvény deriváltjai:
    \begin{displaymath}  
    \begin{aligned}
      \partial_{p_k} \ln f_{\btheta}(x) & =\sum_{i=1}^n q_k(\btheta, x_i) \frac1{p_k}\\
      \partial_{\theta_k} \ln f_{\btheta}(x) & =\sum_{i=1}^n q_k(\btheta, x_i) \partial_{\theta_k} \ln g_{\theta_k}(x_i)
    \end{aligned}
    \quad\text{ahol}\quad
    q_{k,i}(\btheta) = \frac{p_k g_{\theta_k}(x)}{\sum_j p_j g_{\theta_j}(x)}
    \end{displaymath}
    \item A likelihood egyenlet:
    \begin{align*}
      \partial_{p_k} \ln f_{\btheta}(x) & =\sum_{i=1}^n q_{k,i}(\btheta) \frac1{p_k}=\lambda 
      \implies p_k = \frac{\sum_{i} q_{k,i}(\btheta)}{\sum_{i,j} q_{j,i}(\btheta))} ,
      \\
      \partial_{\theta_k} \ln f_{\btheta}(x) & =\sum_{i=1}^n q_{k,i}(\btheta) \partial_{\theta_k} \ln g_{\theta_k}(x_i)=0
    \end{align*}
  \end{itemize}
\end{frame}

\begin{frame}{EM algoritmus}
  \begin{itemize}
    \item Rekurzív eljárás: $\btheta^{(t)}$ a paraméter a $t$. lépés után.
    \begin{align*}
      q_{k,i}^{(t)}=q_{k,i}(\btheta^{(t)}) &= \frac{p_k g_{\theta^{(t)}_k}(x_i)}{\sum_j p_j g_{\theta^{(t)}_j}(x_i)} &
      \btheta^{(t+1)} &=\arg\max_{\btheta} \sum_{i,k} q_{k,i}^{(t)} \ln (p_k g_{\theta_k}(x_i))
    \end{align*}    
    Azaz
    \begin{displaymath}
       p^{(t+1)}_k = \frac{\sum_{i} q_{k,i}^{(t)}}{\sum_{i,k} q_{k,i}^{(t)}} ,
      \qquad  
      \sum_{i} q_{k,i}^{(t)} \partial_{\theta_k}\ln g_{\theta^{(t+1)}_k}(x_i)=0 
    \end{displaymath}
    Ha $\btheta^{(t)}$ konvergens, akkor $\btheta=\lim_{t\to\infty}\btheta^{(t)}$ megoldja a likelihood egyenletet.

    Ezt hívják EM algoritmusnak. $q$ kiszámítása az E-lépés (expectation) 
    $\btheta$ frissítése az M-lépés (maximization).
  \end{itemize}

  \begin{proposition}
    $\ell(\theta^{(t)})$ a  $t$ monoton növő függvénye. 
  \end{proposition}
\end{frame}


\begin{frame}{Az EM algoritmus minden lépése növeli  a likelihood értékét} 
  \begin{proposition}
    $\ell(\theta^{(t)})$ a  $t$ monoton növő függvénye. 
  \end{proposition}
  \begin{itemize}
    \item 
    \begin{align*}
      q_{k,i}^{(t)} &= \frac{p_k g_{\theta^{(t)}_k}(x_i)}{\sum_j p_j g_{\theta^{(t)}_j}(x_i)} &
      \btheta^{(t+1)} &=\arg\max_{\btheta} \sum_{i,k} q_{k,i}^{(t)} \ln (p_k g_{\theta_k}(x_i))
    \end{align*} 
    \item 
    \begin{displaymath}
      F(q,\btheta) = \sum_{k,i} q_{k,i} \ln (p_k g_{\theta_k}(x_i))= \sum_{k,i} q_{k,i}\ln q_{k,i}(\btheta)+\sum_i \ell(\btheta,x_i)
    \end{displaymath}
    Így 
    \begin{displaymath}
      \ell(\btheta^{(t+1)})-\ell(\btheta^{(t)}) = 
      F(q^{(t)},\btheta^{(t+1)})-F(q^{(t)},\btheta^{(t)}) - \sum_{k,i} q^{(t)}_{k,i}\ln\frac{q^{(t+1)}_{k,i}}{q^{(t)}_{k,i}}
    \end{displaymath}
    Mivel $-\ln$ konvex, ezért
    \begin{displaymath} 
      -\sum_{k} q^{(t)}_{k,i}\ln\frac{q^{(t+1)}_{k,i}}{q^{(t)}_{k,i}}\geq \ln\zjel*{\sum_k q^{(t+1)}_{k,i}}=0 
    \end{displaymath}
  \end{itemize}
\end{frame}

\begin{frame}{Becslések tulajdonságai I.}
  \begin{itemize}
  \item $\set{\P[\theta]}{\theta\in \cP}$ eloszláscsalád, $g (\theta)$
    értékét akarjuk becsülni, erre a  $T$ statisztikát használjuk.
    
  \item Egy $T$ becslés ,,jóságát'' mérhetjük az átlagos veszteséggel
    \begin{displaymath}
      R_T (\theta)=\E[\theta]{W (T,\theta)}
    \end{displaymath}
    $R_T$ neve: rizikó függvény. 
  \item $W (t,\theta)$ a veszteség, amit $g
    (\theta)$ a $t$ értékkel történő közelítésekor kapunk.

  \item Négyzetes
    veszteség: $W (t,\theta)=(g(\theta)-t)(g(\theta)-t)^T$ ha $g$ vektor
    értékű és $ %W (t,\theta)=
    (g(\theta)-t)^2$, ha $g$ skalár
    értékű.
    
  \item Általánosabban $W$ lehetne tetszőleges nem negatív, első
    változóban  konvex  függvény, amire $W(g (\theta),\theta)=0$.
  \end{itemize}
\end{frame}

\begin{frame}{Becslések tulajdonságai II.}
  \begin{itemize}
  \item $\set{\P[\theta]}{\theta\in \cP}$ eloszláscsalád, $g (\theta)$
    értékét akarjuk becsülni, erre a  $T$ statisztikát használjuk.
    
  \item Rizikó függvény %Egy $T$ becslés ,,jóságát'' mérhetjük az átlagos veszteséggel
    \begin{displaymath}
      R_T (\theta)=\E[\theta]{W (T,\theta)},\quad
      W (t,\theta)= (g (\theta)-t)^2\quad\text{négyzetes rizikó és veszteség}
    \end{displaymath}
    % $R_T$ neve: rizikó függvény. 
  \item $T$ torzítatlan, ha $\E[\theta]{T}=g(\theta)$, ha nem
    torzítatlan, akkor a torzítása (bias) $b (\theta)=\E[\theta]{T}-g (\theta)$.

  \item $T$ és $S$ ekvivalens becslések, ha $R_T (\theta) =R_S
    (\theta)$, minden $\theta$-ra. 
  \item $T$ egyenletesen jobb (nem rosszabb), mint $S$, ha  $R_T (\theta)\leq R_S
    (\theta)$, minden $\theta$-ra és nem ekvivalensek.
  \item Ha $\cD$ becslések (statisztikák) egy családja,
    pl. torzítatlan becslések, akkor $T\in\cD$ \textbf{megengedhető}
    ($\cD$-re nézve), ha nincs nála jobb becslés $\cD$-ben, azaz
    tetszőleges $S\in\cD$-re, $R_S\leq R_T$-ből $R_S=R_T$ következik.
  \item $T\in\cD$ \textbf{optimális}, ha $\cD$ minden eleménél jobb vagy
    ekvivalens,
    azaz $R_T\leq R_S$ minden $S\in\cD$-re.

    Ha $\cD$ a torzítatlan becslések családja, akkor az optimális
    becslés neve \textbf{hatásos} becslés és jobb helyett hatásosabbat
    mondunk.
  \item $T\in \cD$ \textbf{minimax}, $\sup_{\theta\in\Theta}R_T
    (\theta)\leq \sup_{\theta\in\Theta}R_S (\theta)$, minden $S\in\cD$-re.
  \end{itemize}
\end{frame}

\begin{frame}{Becslések tulajdonságai III.}
  \begin{itemize}
  \item $R_T (\theta)=\E[\theta] {W(T,\theta)}$,  pl. $R_T (\theta) =\E[\theta] {(T-g (\theta))^2}$
  %\item $T$ és $S$ ekvivalens, ha $R_T=R_S$ 
  \item $T\in\cD$ \textbf{megengedhető} ha 
    tetszőleges $S\in\cD$-re, $R_S\leq R_T$-ből $R_S=R_T$ következik.
  \item $T\in\cD$ \textbf{optimális}, $R_T=\min_{S\in\cD}R_S$
  \item $T\in \cD$ \textbf{minimax}, ha $\sup_{\theta\in\Theta}R_T
    (\theta)\leq \sup_{\theta\in\Theta}R_S (\theta)$, minden $S\in\cD$-re.
  \end{itemize}
  \begin{proposition}
    Ha $T$ optimális, akkor $T$ megengedhető és minimax.

    Ha $\cD$ konvex, $W$ első változójában szigorúan konvex, akkor az
    optimális becslés egyértelmű (ha létezik).
  \end{proposition}

  Ha $T_1,T_2$ optimális, akkor $R_{T_1}=R_{T_2}$. Ha $\cD$ konvex, akkor $T=\frac12
  (T_1+T_2)\in\cD$ és
  \begin{align*}
    R_T (\theta)
    &=\E[\theta]{W (\textstyle{\frac12 T_1+\frac12 T_2},\theta)}\\
    &\leq
    \frac12\E[\theta]{W (T_1,\theta)}+\frac12\E[\theta]{W (T_2,\theta)}
    =\frac12R_{T_1} (\theta)+\frac12R_{T_2} (\theta)=R_{T_1} (\theta)=R_{T_2} (\theta)
  \end{align*}
  Ha $\P[\theta]{T_1\neq T_2}>0$, akkor
  $\P[\theta]{W(T,\theta)<\frac12(
    W(T_1,\theta)+W(T_2,\theta))}>0$ és $R_T (\theta)<R_{T_1} (\theta)$.
\end{frame}

\begin{frame}{Emlékeztető, Bayes becslés}
  \begin{itemize}
  \item $\theta\mapsto \P[\theta] (A)$ mérhető minden $A\in\B
    (\mathfrak X)$-re, ekkor $\theta\mapsto R_T (\theta)$ mérhető.

    pl. dominált mérték család és $(\theta,x)\mapsto f_\theta (x)$
    mérhető.
  \item $Q$ valószínűségi mérték $\Theta$-n.
  \item $R_T (Q)=\int_\Theta R_T (t) Q (t)$ a priori rizikó
  \item $T$ Bayes becslés ($Q$-Bayes), ha $R_T (Q)\leq R_S (Q)$ minden
    $S$-re.
  \end{itemize}
\end{frame}

\begin{frame}{Emlékeztető, Bayes becslés}
  \begin{itemize}
  \item $\theta\mapsto \P[\theta] (A)$ mérhető minden $A\in\B
    (\mathfrak X)$-re, ekkor $\theta\mapsto R_T (\theta)$ mérhető.

  \item
    \begin{displaymath}
      \P{A}=\int_\Theta\int_{\mathfrak X} \I{(t,x)\in A}\P[t] (dx)Q
      (d t),\quad A\in\B (\Theta)\otimes\B (\mathfrak X)
    \end{displaymath}
    valószínűségi mérték $\Theta\times\mathfrak X$-en. $\theta (t,x)=t$, $X
    (t,x)=x$ valószínűségi váltózók a $(\Omega=\Theta\times\mathfrak X,\B
    (\Theta)\otimes\B (\mathfrak X),\P)$ valószínűségi mezőn. És
    \begin{displaymath}
      \P{X\in H|\theta}=\P[\theta] (H)
    \end{displaymath}
  \item
    \begin{displaymath}
      R_T (Q)=\int_\Theta R_T (t)Q(d t)
      =\int_\Theta\int_{\mathfrak X} W (T (x),t)\P[t] (d x)Q (d t)=
      \E{W(T,\theta)}
    \end{displaymath}
  \item Négyzetes veszteség esetén ($\E{g(\theta)^2}<\infty$)
    \begin{displaymath}
      \E{(T-g(\theta))^2}=\E{(T-\E{g (\theta)|X})^2} +
      \E{(g (\theta)-\E{g (\theta)|X})^2}
    \end{displaymath}
    a Bayes becslés $T (X)=\E{g(\theta)|X}$
    
  \item $\P{\theta\in H|X}$-nek létezik reguláris változata, ez az
    a posteriori eloszlás. Jelölés $Q^* (dt|x)$.
  \item A Bayes  becslés az a posteriori eloszlásból számolt várható érték.
  \end{itemize}
\end{frame}

\begin{frame}{Bayes becslés aszimptotikus normalitása}

  $f_\theta$, $\ell (\theta,x)$ egy mintaelem sűrűségfüggvénye,
  ill. loglikelihoodja. $X_1,X_2,\dots$ minta $\P[\theta_0]$ eloszlásból
  \begin{displaymath}
    L_n (\theta)=\frac1n\sum_{i=1}^n\ell (\theta,X_i)
  \end{displaymath}

  \begin{theorem}[Bizonyítás nélkül]
    \begin{itemize}
    \item (RR) teljesül
    \item Minden $\eps>0$-ra
      \begin{displaymath}
        \lim_{\delta\to0}\liminf_{n\to\infty}
        \P[\theta_0]^n{\sup_{\norm*{h}\geq\eps}L_n (\theta_0+h)\leq L_n (\theta_0) -\delta} = 1
      \end{displaymath}
    \item $Q$ abszolút folytonos és $q (\theta_0)>0$
    \item $\int_{\Theta} \norm*{t}q (t)dt<\infty$.
    \end{itemize}
    $\theta_n$ a Bayes becslés $n$ elemű mintából.  Ekkor
    $\sqrt n (\theta_n-\theta)\dto N (0,I (\theta_0)^{-1})$. 
  \end{theorem}
\end{frame}

\begin{frame}{Bayes becslés tulajdonságai I.}
  \begin{proposition}
    Ha $T$ Bayes becslés, akkor triviális esettől eltekintve nem torzítatlan.
  \end{proposition}
  Ha $T=\E{g (\theta)|X}=\E{g(\theta)|T}$ torzítatlan, akkor
  \begin{displaymath}
    \E{T|\theta}=\E[\theta]{T}=g (\theta)=\E{T|g (\theta)}
  \end{displaymath}

  \begin{lemma}
    Ha $X,Y\in L^1$, $\E{X|Y}=Y$ és $\E{Y|X}=X$, akkor $X=Y$ egy valószínűséggel
  \end{lemma}

  A Lemma alapján $T$ torzítatlan  Bayes becslésre
  $\P{g(\theta)=T (x)}=1$.

  Azaz $Q$ majdnem minden $t$-re $\P[t] (g(t)=T)=1$, vagyis létezik
  $\Theta'\subset \Theta$ $Q$ teljes mértékű halmaz,
  $\theta_1,\theta_2\in\Theta'$, $g(\theta_1)\neq g (\theta_2)$ esetén
  $\P[\theta_1]$ és $\P[\theta_2]$ diszjunkt halmazokra koncentrálódik.    
\end{frame}

\begin{frame}{Lemma bizonyítása}
  \begin{lemma}
    Ha $X,Y\in L^1$, $\E{X|Y}=Y$ és $\E{Y|X}=X$, akkor $X=Y$ egy valószínűséggel
  \end{lemma}
  \begin{itemize}
  \item Ugyanis, ha $X,Y\in L^2$, akkor
    \begin{displaymath}
      \E{(X-Y)^2}=\E{X^2}+\E{Y^2}-2\E{XY},\quad\E{XY}=\E{\E{XY|Y}}=\E{Y^2}=\E{X^2}
    \end{displaymath}
  \item 
    Ha $X,Y\in L^1$, akkor legyen $h (x)=x\arctg x$ szigorúan konvex
    (HF).
    $h (X),h (Y)\in L^1$ és a Jensen egyenlőtlenség alapján
    \begin{displaymath}
      \E{h (X)}\geq \E{h (\E{X|Y})}=\E{h (Y)}=\E{h (X)}
    \end{displaymath}
    % $h$ szigorúan konvex
    \begin{displaymath}
      h (X)> h (Y)+h' (Y) (X-Y)\quad\text{az $\event{X\neq Y}$
        eseményen}
      \quad\implies\quad \P{X=Y}=1
    \end{displaymath}
  \end{itemize}
\end{frame}

\begin{frame}{Bayes becslés tuladonságai II.}
  \begin{itemize}
  \item $X$ peremeloszlása $\P$ alatt a prediktív eloszlás
    \begin{displaymath}
      \P[Q] (H)=\int_{\Theta}\P[t] (X\in H) Q (dt)
  \end{displaymath}
  \item $R_T (\theta)=\E[\theta] {W(T,\theta)}$,  pl. $R_T (\theta) =\E[\theta] {(T-g (\theta))^2}$
  %\item $T$ és $S$ ekvivalens, ha $R_T=R_S$ 
  \item $T\in\cD$ \textbf{megengedhető} ha 
    tetszőleges $S\in\cD$-re, $R_S\leq R_T$-ből $R_S=R_T$ következik.
  %\item $T\in \cD$ \textbf{minimax}, ha $\sup_{\theta\in\Theta}R_T
  %  (\theta)\leq \sup_{\theta\in\Theta}R_S (\theta)$, minden $S\in\cD$-re.
  \end{itemize}

  \begin{proposition}
    Tegyük fel, hogy $\P[Q]\sim \cP$.
    Ekkor ha $T$ Bayes becslés, akkor megengedhető.
  \end{proposition}
  Legyen $S$ statisztika, amire $R_S\leq R_T$. Ekkor $S$ is Bayes
  becslés ($R_S (Q)\leq R_T (Q)$) és ezért $S=\E{g (\theta)|X}=T$,
  vagyis $\P{S=T}=1$. $S$ és $T$ a minta függvénye
  \begin{displaymath}
    \P[Q]{T=S}=1\quad\implies \quad \text{$T=S$ $\cP$ majdnem mindenütt}
  \end{displaymath}  
\end{frame}

\begin{frame}{Bayes becslés tuladonságai III.}
  \begin{itemize}
    
  % \item $X$ peremeloszlása $\P$ alatt a prediktív eloszlás
  %   \begin{displaymath}
  %     \P[Q] (H)=\int_{\Theta}\P[t] (X\in H) Q (dt)
  % \end{displaymath}
  \item $R_T (\theta)=\E[\theta] {W(T,\theta)}$,  pl. $R_T (\theta) =\E[\theta] {(T-g (\theta))^2}$
  %\item $T$ és $S$ ekvivalens, ha $R_T=R_S$ 
  %\item $T\in\cD$ \textbf{megengedhető} ha 
  %  tetszőleges $S\in\cD$-re, $R_S\leq R_T$-ből $R_S=R_T$ következik.
  \item $T\in \cD$ \textbf{minimax}, ha $\sup_{\theta\in\Theta}R_T
    (\theta)\leq \sup_{\theta\in\Theta}R_S (\theta)$, minden $S\in\cD$-re.
  \end{itemize}

  \begin{proposition}
    Ha a $T$ Bayes becslés rizikója konstans, akkor $T$ minimax becslés.
  \end{proposition}

  $S$ tetszőleges becslés
  \begin{displaymath}
    \sup_\theta R_S (\theta)\geq \int_{\Theta}R_S (t)Q (dt)=R_S
    (Q)\geq R_T (Q)=\int_{\Theta}R_T (t)Q (dt)=\sup_\theta R_T (\theta)
  \end{displaymath}

  \begin{proposition}
    Dominált mértékcsalád esetén,
    ha $S$ elégséges statisztika, akkor létezik Bayes becslés $S$
    függvényei között is.
  \end{proposition}
  \begin{displaymath}
    q^*(t|x)=\frac{g_\theta (S (x))h (x) q (t)}{\int_\Theta g_t (S
      (x))h (x)q (t)dt}=
    \frac{g_\theta (S (x)) q (t)}{\int_\Theta g_t (S (x))q (t)dt}
  \end{displaymath}
  Azaz az  a posteriori sűrűség csak $S (X)$-en keresztül függ $X$, a
  feltételes várható érték is ilyen.
\end{frame}

\begin{frame}{Minimax becslés indikátor mintából}
  \begin{itemize}
  \item $Q=\text{Béta} (\alpha,\beta)$
  \item $r_n (X)=\frac1n\sum_{i=1}^n X_i$,
    \begin{displaymath}
     \hat\theta=\frac{\alpha+S}{\alpha+\beta+n}=a+b r_n 
    \end{displaymath}
    ahol $b=\frac{n}{\alpha+\beta+n}$, $a=\frac{\alpha}{\alpha+\beta+n}$.
  \item
    \begin{align*}
      R_{\hat\theta}(\theta)
      &=\E[\theta]{\zjel*{a+br_n-\theta}^2}= (a-
        (1-b)\theta)^2+b^2\frac{\theta (1-\theta)}n\\
%      \gamma^2\zjel{\frac{\alpha}{\alpha+\beta}-\theta}^2+
%        (1-\gamma)^2D^2\zfrac Sn\\
%      &=
%        \gamma^2\zjel{\frac{\alpha}{\alpha+\beta}-\theta}^2+
%        (1-\gamma)^2\frac{\theta (1-\theta)}n\\
       &=\theta^2\zjel*{(1-b)^2-\tfrac{b^2}{n}}
       +\theta\zjel*{\tfrac{b^2}{n}-2a(1-b)}+a^2
    \end{align*}
  \item $R_T$ konstans, ha $b=\frac{\sqrt{n}}{1+\sqrt n}$, és
    $a=\frac12\frac{1}{1+\sqrt{n}}$ és
    $\alpha=n\frac{a}{b}=\frac{\sqrt{n}}2$ és 
    $\beta=\frac{n}{b}-\alpha-n=\sqrt{n}-\frac{\sqrt n}2=\frac{\sqrt{n}}2$.

  \item Ha $Q=\text{Béta} (\frac{\sqrt n}2,\frac{\sqrt n}2)$, 
      akkor a Bayes becslés, azaz
    \begin{displaymath}
      T (X)=\frac{\frac12\sqrt{n}+\sum X_i}{\sqrt{n}+n}
    \end{displaymath}
    minimax
  \end{itemize}
\end{frame}

\end{document}
