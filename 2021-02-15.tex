%\def\option{}
\providecommand{\option}{handout}
\documentclass[aspectratio=169,notheorems,9pt,\option]{beamer}



\makeatletter
\def\readrest#1\enddate{}
\def\setdate@#1-#2-#3\relax{\def\lecturedate{#1.#2.#3.}\readrest}
\def\setdate{\expandafter\setdate@\jobname\relax-\relax-\relax\enddate}

\makeatother

\setdate


\usepackage{marvosym}
%\usepackage[magyar]{babel}
\usepackage{polyglossia}
\setmainlanguage{magyar}

% \usetheme{Bruno}

\usecolortheme{Bruno}
% \usefonttheme{Bruno}
% \useinnertheme{Bruno}
\useoutertheme{Bruno}

\usepackage{pvdefs}
\newtheorem{df}{Definíció}
\newtheorem{theorem}{Tétel}
\newtheorem{lemma}{Lemma}
\newtheorem{proposition}{Állítás}
\newtheorem{corollary}{Következmény}

\defcx{XAPIDLCSGTHEN}
\deftx{STQbCf\lambda\Omega\sigma}
\defmf{X}

\def\diag{\operatorname{diag}}
\def\Id{\mathbf{I}}
\let\PEfont\mathbb
\def\npto{\buildrel p\over \nrightarrow}
\def\ndto{\buildrel d\over \nrightarrow}
\def\dto{\buildrel d\over \rightarrow}
\def\dto{\stackrel{d}{\to}}
\let\nto\nrightarrow

\def\Leb{\operatorname{Leb}}

\newrobustcmd{\continue}{%
  \onslide<\value{beamerpauses}->\relax
}


\beamerdefaultoverlayspecification{<+->}

\mode<presentation>

\RequirePackage[math-style=TeX, bold-style=upright]{unicode-math}
\setmathfont{Latin Modern Math}[Scale=1]
% \setsansfont{Kurier Cond}
\setsansfont[Ligatures=TeX, ItalicFont={Fira Sans Italic}, BoldFont={Fira Sans SemiBold}, Scale=0.9]{Fira Sans Book}
%\setsansfont{Ubuntu}

%\usepackage{mathspec}
%\setmathsfont{lm}
%(Digits,Latin,Greek)
%[Numbers={Lining,Proportional}]{lm}
%\setmathsfont(Digits,Greek)
%[Uppercase=Plain,Lowercase=Regular,Scale=MatchLowercase]
%{GFS Porson}


% Title page
\setbeamerfont{title}{
    size=\LARGE,
    shape=\bfseries
}

\setbeamerfont{subtitle}{
    size=\large,
    shape=\normalfont
}

\setbeamerfont{author}{
    size=\normalsize,
    shape=\normalfont
}

\setbeamerfont{details}{
    size=\footnotesize,
    shape=\normalfont
}

% Slide title
\setbeamerfont{frametitle}{size=\LARGE}
\setbeamerfont{framesubtitle}{
    size=\normalsize,
    %shape=\normalfont\scshape
}

% Blocks
\setbeamerfont{block title}{
    size=\normalfont,
    shape=\strut
}

\setbeamerfont{blockdef}{
    size=\large,
    shape=\bfseries
}

% Description
\setbeamerfont{description item}{shape=\bfseries}


% Footer information line with title, author and slide number
\setbeamertemplate{footline}{%
    \leavevmode%
    \hbox{%
        \usebeamerfont{footline}%
        \begin{beamercolorbox}[
            wd=\textwidth,
            ht=3ex,
            dp=1.25ex
        ]{footline}%
            \hspace{1cm}%
            \insertshorttitle%
            \hspace{2em}%
            \insertdate%
            \hfill%
            \insertframenumber{} / \inserttotalframenumber%
            \hspace{1cm}
        \end{beamercolorbox}%
    }
    \vskip0pt%
}

%% Bruno Beamer theme — Version 1.0
%% A lightweight Beamer theme inspired from the Metropolis theme
%%
%% Written in 2017-2019 by:
%% — Rémi Cérès <remiceres@msn.com>
%% — Mattéo Delabre <bonjour@matteodelabre.me>
%%
%% This work is released under the CC0 1.0 Universal license. See the
%% accompanying LICENSE file for details. To the extent possible under
%% law, Rémi Cérès and Mattéo Delabre have waived all copyright and
%% related or neighboring rights to the Bruno Beamer theme. This work
%% is published from France.

% \mode<presentation>

\setbeamertemplate{sections/subsections in toc}[square]
\setbeamertemplate{itemize item}[square]
\setbeamertemplate{itemize subitem}[circle]

% Allow multi-slide frames
\setbeamertemplate{frametitle continuation}{}

% Horizontal separator
\setbeamertemplate{separator}{%
    \usebeamercolor{separator}%
    \textcolor{fg}{\rule{.7\textwidth}{.5pt}}%
}

% Separate authors with newlines
\renewcommand{\beamer@andtitle}{\\}

% Title page. If you want to add a background image on the title slide,
% use the \background command to set the path to that image. Otherwise,
% no image will be included
\RequirePackage{tikz}
\usetikzlibrary{fadings}
\newcommand{\background}[1]%
    {\newcommand{\bruno@background}{#1}}


%% \newcommand\insertcaption{}
\newcommand{\backgroundcaption}[2][white]{\def\bruno@caption{{#1}{#2}}}

\newcommand{\insertcaption}{\ifdefined\bruno@caption{\expandafter\textcolor\bruno@caption}\fi}

\tikzfading[
    name=title page picture fading,
    left color=transparent!0,
    right color=transparent!100,
]

\setbeamertemplate{title page}{%
    \begin{minipage}{.7\textwidth}
        \raggedright
        \usebeamerfont{title}
        \inserttitle

        \usebeamertemplate{separator}

        \usebeamerfont{author}
        \vspace{2em}
        \insertauthor

        \vspace{2em}
        \usebeamerfont{details}
        \insertinstitute

        \insertdate
    \end{minipage}

    % Include faded image on the right if defined
    \ifdefined\bruno@background
    \begin{tikzpicture}[remember picture, overlay]
        % Crop image to a trapezium on the right
        \clip (current page.south east)
           -- (current page.north east)
           -- ++(-5.1, 0)
           -- ++(-3, -\paperheight)
           -- cycle;

        % Include background image
        \node[
            anchor=south east,
            inner sep=0,
            outer sep=0
        ] at (current page.south east) {
            \includegraphics[height=\paperheight]
                {\bruno@background}
        };

        \node[anchor=south east,
            inner sep=0,
            outer sep=15pt,
        ] at (current page.south east){\insertcaption};

        % Add a slight shadow
        \fill[
            black, path fading=title page picture fading,
            fading angle=-15
        ]
            (current page.south east)
            rectangle
            ++(-10, \dimexpr\paperheight+1cm);

        
    \end{tikzpicture}
    \fi
}

\mode<all>
\AtBeginDocument{%
  \let\phi\varphi
  \let\theta\vartheta
  \let\setminus\smallsetminus
}


\title{Statisztika előadás}
\date{\lecturedate}
\background{images/young_Ronald_Fisher.jpg}
\backgroundcaption{Ronald Fisher (1913)}


\begin{document}
\maketitle

\begin{frame}{Emlékeztető}
  $(\Omega,\cA,\cP=\set{\P[\theta]}{\theta\in\Theta})$ statisztikai mező. $\Omega=\mfX$.  
  \begin{df} 
    $S:\mfX\to\real^d$ \textbf{elégséges statisztika}, ha minden $A\in\B(\mfX)$-re $\P[\theta]{X\in A|S}$-nek 
    létezik $\theta$-tól független, közös változata.

    $\cS\subset\cA$ \textbf{elégséges $\sigma$-algebra}, ha minden $A\in\B(\mfX)$-re $\P[\theta]{X\in A|\cS}$-nek 
    létezik $\theta$-tól független, közös változata.
  \end{df}
  \begin{theorem}[Neymann faktorizációs tétel]
    $\cP=\set{\P[\theta]}{\theta\in\Theta}$ dominált mértékcsalád az 
    $(\mfX,\B(\mfX))$ minta téren, $\lambda$ domináló mértékkel. 
    
    Az $S$  statisztika pontosan akkor elégséges, ha az 
    $f_\theta=\frac{d\P[\theta]}{d\lambda}$ 
    sűrűségek
    \begin{displaymath}
      f_{\theta}(x)=g_{\theta}(S(x))h(x)  \tag{\hbox{${*}$}}
    \end{displaymath}
    alakban is megadhatóak, 
    alkalmas $g_\theta,h$ mérhető függvényekkel.
  \end{theorem}
  
\end{frame}

\begin{frame}{Neymann faktorizációs tétel}
  \begin{theorem}
    $\cP=\set{\P[\theta]}{\theta\in\Theta}$ dominált mértékcsalád az 
    $(\mfX,\B(\mfX))$ minta téren, $\lambda$ domináló mértékkel. 
    
    Az $S$  statisztika pontosan akkor elégséges, ha az 
    $f_\theta=\frac{d\P[\theta]}{d\lambda}$ 
    sűrűségek
    \begin{displaymath}
      f_{\theta}(x)=g_{\theta}(S(x))h(x)  \tag{\hbox{${*}$}}
    \end{displaymath}
    alakban is megadhatóak, 
    alkalmas $g_\theta,h$ mérhető függvényekkel.
  \end{theorem}
  \begin{itemize}
    \item Először $\lambda=\P[0]\in\cP$-ra ellenőrizzük, hogy $S$ pontosan 
    akkor elégséges, ha a $\frac{d\P[\theta]}{d\P[0]}=g_\theta\circ S$ alakú, azaz
    a sűrűségeknek van $S$ mérhető változata.
    \item $\cP'$ a $\cP$ konvex keverékeiből álló család  
    \begin{displaymath}
      \cP'=\set{\sum\nolimits_i c_i\P[i]}{c_i\geq0,\,\sum\nolimits_i c_i=1,\,\P[i]\in\cP}  
    \end{displaymath}
    Megmutatjuk, hogy $S$ pontosan akkor elégséges $\cP$-re nézve, ha $\cP'$-re az, 
    továbbá létezik $\P[0]\in \cP'$ ami, $\cP'$-t dominálja (Halmos--Savage tétel) 
    és $(*)$ pontosan akkor teljesül $\cP$-re, ha $\cP'$-re. .
    % \item $\cP'$-re az elsőként vizsgált eset alkalmazható.
    \item Ha $\lambda$ tetszőleges domináló mérték, akkor 
    \begin{displaymath}
      \text{$S$ egs.}\implies\frac{d\P[\theta]}{d\lambda}
      =\frac{d\P[\theta]}{d\P[0]}\frac{d\P[0]}{d\lambda}
      =(g_\theta\circ S) h,
      \quad\text{$(*)$}\implies
      \frac{d\P[\theta]}{d\P[0]}=\frac{d\P[\theta]}{d\lambda}:\frac{d\P[0]}{d\lambda}
      =\frac{(g_\theta\circ S) h}{(g_0\circ S) h}
      =\tilde g_\theta\circ S\implies \text{$S$ egs.}.
    \end{displaymath}
  \end{itemize}

\end{frame}

\begin{frame}{Neymann faktorizációs tétel, $\lambda=\P[0]$}
\begin{theorem}
  $\cP=\set{\P[\theta]}{\theta\in\Theta}$ dominált mértékcsalád az 
  $(\mfX,\B(\mfX))$ minta téren, $\P[0]\in\cP$ domináló mértékkel. 
  
  Az $S$  statisztika pontosan akkor elégséges, ha az 
  $f_\theta=\frac{d\P[\theta]}{d\P[0]}$ 
  sűrűségeknek van $\sigma(S)$ mérhető változata.
\end{theorem}
\begin{itemize}
  \item $S$ elégséges, ekkor $\P[\theta]{X\in H|S}=\P[0]{X\in H|S}$.
  \item Teljes valószínűség tétellel
  \begin{displaymath}
    \P[\theta]{X\in H}
    =\E[\theta]{\P[\theta]{X\in H|S}}
    =\E[0]{\frac{d\P[\theta]|_{\sigma(S)}}{d\P[0]|_{\sigma(S)}}\E[0]{\I{X\in H}|S}}
    =\E[0]{\I{X\in H}\frac{d\P[\theta]|_{\sigma(S)}}{d\P[0]|_{\sigma(S)}}}
  \end{displaymath}
  \item Legyen $g_\theta(S)=\frac{d\P[\theta]|_{\sigma(S)}}{d\P[0]|_{\sigma(S)}}$
  \begin{displaymath}
    \P[\theta]{X\in H}=\E[0]{\I{X\in H}g_\theta(S)}=\int_H g_\theta(S)d\P[0],
  \end{displaymath}
  azaz $g_\theta\circ S$  a $\frac{d\P[\theta]}{d\P[0]}$ Radon-Nikodym derivált egy változata.
  \item A fordított irányt korábban leellenőriztük a Bayes szabály felhasználásával.
\end{itemize}
\end{frame}

\begin{frame}{$\cP$ és $\cP'$}
$\cP$ mértékcsalád a mintatéren
\begin{displaymath}
  \cP'=\set{\sum\nolimits_i c_i\P[i]}{c_i\geq0,\,\sum\nolimits_i c_i=1,\,\P[i]\in\cP}  
\end{displaymath}
\begin{proposition}
  $S$ pontosan akkor elégséges $\cP$-re nézve, ha $\cP'$-re elégséges.
\end{proposition}
\begin{itemize}
  \item Legyen $H$ rögzített és $q(S)=\P{X\in H|S}$ a $\cP$-re vonatkozó közös változat. 
  \item Tetszőleges $\P\in\cP'$-re $\P=\sum_{i } c_i\P[i]$ alakú és 
  \begin{displaymath}
    \P{X\in H, S\in A}
    =\sum_{i} c_i\int \P[i]{X\in H|S}\I{S\in A}d\P[i]
    =\sum_i c_i\int q(S) \I{S\in A}d\P[i]
    =\int q(S)\I{S\in A}d\P
  \end{displaymath} 
  \item $q(S)$ $S$ mérhető és teljesíti a parciális átlagolási tulajdonságot $q(S)=\P{X\in H|S}$
  minden $\P\in\cP'$-re.
  \item Másik irányhoz elég észrevenni, hogy ha $S$ elégséges egy mértékcsaládra, 
  akkor tetszőleges részcsaládra is az és $\cP\subset\cP'$.
\end{itemize}
\end{frame}

\begin{frame}{$\cP$ és $\cP'$}
$\cP$ %=\set{\P[\theta]}{\theta\in \Theta}$ 
mértékcsalád a mintatéren $\lambda$ domináló mértékkel.
\begin{displaymath}
  \cP'=\set{\sum\nolimits_i c_i\P[i]}{c_i\geq0,\,\sum\nolimits_i c_i=1,\,\P[i]\in\cP}  
\end{displaymath}
Ha $\P=\sum_i c_i\P[i]\in\cP'$, akkor
\begin{displaymath}
  \frac{d\P}{d\lambda} = \sum_{i} c_i\frac{d\P[i]}{d\lambda}
\end{displaymath}

\begin{proposition}
  $\lambda $ domináló mérték $\cP$-hez, $S$ statisztika. Ha
  \begin{displaymath}
    \frac{d\P[\theta]}{d\lambda}=(g_{\theta}\circ S)h,\quad \text{minden $\P[\theta]\in\cP$-re},
  \end{displaymath} 
  akkor ugyanez igaz minden $\P\in\cP'$-re.
\end{proposition}
\end{frame}

\begin{frame}{Halmos--Savage tétel}
\begin{theorem}
  $\cP$ konvex keverésre zárt, dominált mértékcsalád $(\mfX,\B(\mfX))$-en. 
  
  Ekkor létezik $\P[0]\in\cP$, ami dominálja $\cP$-t.
\end{theorem}
\begin{itemize}
  \item $\lambda$ $\sigma$-véges domináló mérték. Feltehető, hogy véges mérték. Ugyanis,
   ha $\mfX=\cup_n A_n$ és $\lambda(A_n)<\infty$, akkor 
   $c_n>0$, $\sum_n c_n\lambda(A_n)<\infty$ esetén
   \begin{displaymath}
     \lambda'(H)=\sum_n c_n \lambda(A_n\cap H)=\int_H \sum_n c_n\I[A_n] d\lambda
   \end{displaymath}
   véges mérték és $\lambda$, $\lambda'$ ekvivalensek és $\lambda'$ véges.
   \item $\lambda$ véges domináló mérték. Legyen
   \begin{displaymath}
    c=\sup\set{\lambda(f>0)}{f=\frac{d\P}{d\lambda},\,\P\in\cP}
   \end{displaymath}
   \item Elég megmutatni, hogy $c$ maximum, 
   azaz létezik $\P[0]\in\cP$, amire $c=\lambda(\frac{d\P[0]}{d\lambda}>0)$.     
\end{itemize}

\end{frame}

\begin{frame}{Halmos--Savage tétel igazolása I.}
\begin{lemma}
  $\cP$ konvex keverésre zárt, véges $\lambda$ mértékkel dominált mértékcsalád 
  $(\mfX,\B(\mfX))$-en. Ha $\P[0]\in\cP$-re
  \begin{displaymath}
    \lambda\zjel{\frac{d\P[0]}{d\lambda}>0}=c=\sup\set{\lambda(f>0)}{f=\frac{d\P}{d\lambda},\,\P\in\cP}
  \end{displaymath}
  Ekkor $\P[0]$ dominálja $\cP$-t.
\end{lemma}
\begin{itemize}
  \item Tetszőleges $\P\in\cP$-re $\P'=\frac12(\P+\P[0])\in\cP$ a konvex keverésre zártság miatt.
  \item $f=\frac{d\P}{d\lambda}$, $f_0=\frac{d\P[0]}{d\lambda}$, 
  ekkor $2\frac{d\P'}{d\lambda}=f+f_0$ és 
  \begin{displaymath}
    c\geq \lambda((f+f_0)>0)\geq \lambda(f_0>0)=c, 
    \quad\text{mert $\event{f+f_0>0}=\event{f>0}\cup\event{f_0>0}$}.
  \end{displaymath}
  \item $\lambda(\event{f>0}\setminus\event{f_0>0})=0$ és 
  \begin{displaymath}
    \P{X\in H}
    =\int_H f d\lambda
    =\int_H f\I{f_0>0}d\lambda
    =\int_H \frac{f}{f_0}\I{f_0>0}f_0d\lambda
    =\int_H \frac{f}{f_0}\I{f_0>0}d\P[0]
  \end{displaymath} 
\end{itemize}
\end{frame}

\begin{frame}{Halmos--Savage tétel igazolása II.}
\begin{lemma}
  $\cP$ konvex keverésre zárt, véges $\lambda$ mértékkel dominált mértékcsalád 
  $(\mfX,\B(\mfX))$-en. Ekkor létezik $\P[0]\in\cP$, amire
  \begin{displaymath}
    \lambda\zjel{\frac{d\P[0]}{d\lambda}>0}=c=\sup\set{\lambda(f>0)}{f=\frac{d\P}{d\lambda},\,\P\in\cP}
  \end{displaymath}
\end{lemma}
\begin{itemize}
  \item Legyen $\P[n]\in\cP$ olyan, amire az $f_n=\frac{d\P[n]}{d\lambda}$ jelöléssel
  \begin{displaymath}
    \lambda(f_n>0)\geq c-\frac1n
  \end{displaymath}
  \item $\P[0]=\sum_{n=1}^{\infty} 2^{-n} \P[n]\in\cP$, ekkor 
  $f=\frac{d\P[0]}{d\lambda}=\sum_n 2^{-n}f_n$ és 
  \begin{displaymath}
    \event{f>0}=\cup_n\event{f_n>0},\quad c = \sup_n\lambda(f_n>0)\leq \lambda(f>0)\leq c.
  \end{displaymath}
\end{itemize}
\end{frame}

\begin{frame}[<*>]{Neymann--faktorizációs tétel összefoglalás}
\begin{theorem}
  $\cP=\set{\P[\theta]}{\theta\in\Theta}$ dominált mértékcsalád az 
  $(\mfX,\B(\mfX))$ minta téren, $\lambda$ domináló mértékkel. 
  
  Az $S$  statisztika pontosan akkor elégséges, ha az 
  $f_\theta=\frac{d\P[\theta]}{d\lambda}$ 
  sűrűségek
  \begin{displaymath}
    f_{\theta}(x)=g_{\theta}(S(x))h(x)  %\tag{$*$}
  \end{displaymath}
  alakban is megadhatóak, 
  alkalmas $g_\theta,h$ mérhető függvényekkel.
\end{theorem}
\begin{itemize}
  \item Feltehető, hogy $\cP$ zárt a konvex keverésre.
  \item Ha $\cP$ zárt a konvex keverésre, akkor a domináló mérték választható $\P[0]\in\cP$-nek is.
\end{itemize}
\begin{lemma}
  $\P[0]\in\cP$ domináló mérték $\cP$-hez. Ekkor $S$ pontosan akkor elégséges, ha $\frac{d\P}{d\P[0]}$-nak 
  van $\sigma(S)$ mérhető változata minden $\P\in\cP$-re.
\end{lemma}
\end{frame}

\begin{frame}{Emlékeztető}
  $\cP$ mértékcsalád, $\cP'=\set{\sum_{i} c_i\P[i]}{c_i\geq0,\,\sum c_i=1,\,\P[i]\in\cP}$.
  \begin{proposition}[Halmos--Savage tétel]
    Ha $\cP$ dominált, akkor van olyan $\P[0]\in\cP'$, ami dominálja $\cP'$-t és így $\cP$-t is.
  \end{proposition}
  \begin{proposition}
    $S$  pontosan akkor elégséges statisztika, 
    ha $\frac{d\P}{d\P[0]}$-nak létezik $\sigma(S)$ mérhető változata minden $\P\in\cP$-re.
 
    $\cS$  pontosan akkor elégséges $\sigma$-algebra, 
    ha $\frac{d\P}{d\P[0]}$-nak létezik $\cS$ mérhető változata minden $\P\in\cP$-re.
  \end{proposition}

  \begin{itemize}
    \item $\cN(\cP)=\set{N}{\P{N}=0,\,\forall \P\in\cP}$, $\cN(\P[0])=\set{N}{\P[0]{N}=0}$. 
    Ekkor $\cN(\cP)=\cN(\cP')=\cN(\P[0])$.
    \item Tetszőleges $\cS$ $\sigma$-algebrára, legyen 
    $\cS^*=\set{A\circ N}{A\in\cS,\,N\in\cN(\cP)}$ a $\cS$ teljessé tétele, vagy teljes burka.
    \item Belátjuk, hogy $\cS^*$ $\sigma$-algebra és 
     $\frac{d\P}{d\P[0]}$-nak pontosan akkor létezik $\cS$ mérhető változata
    ha $\cS^*$ mérhető.
  \end{itemize}

  \begin{proposition}
    $S$  pontosan akkor elégséges statisztika, 
    ha $\frac{d\P}{d\P[0]}$  $\sigma(S)^*$ mérhető minden $\P\in\cP$-re.
 
    $\cS$  pontosan akkor elégséges $\sigma$-algebra, 
    ha $\frac{d\P}{d\P[0]}$  $\cS^*$ mérhető minden $\P\in\cP$-re.
  \end{proposition}

\end{frame}

\begin{frame}{$\sigma$-algebra teljessé tétele}
  $A\circ B=(A\setminus B)\cup(B\setminus A)$ az $A$ és $B$ szimmetrikus differenciája.
  \begin{proposition}
    $(\Omega,\cA)$ mérhető tér. $\cS\subset\cA$ $\sigma$-algebra, 
    $\cN\subset\cA$ leszálló és $\sigma$-unió zárt. Ekkor
    $\cS^*=\set{S\circ N}{S\in\cS,\,N\in\cN}$ $\sigma$-algebra.
  \end{proposition}
  \begin{itemize}
    \item $(S\circ N)^c=S^c\circ N$,
    \item $N=(\cup_i (S_i\circ N_i))\circ (\cup_i S_i)\subset \cup N_i$, ezért $N\in\cN$ és
    $\cup_i (S_i\circ N_i)= (\cup_i S_i)\circ N\in\cS^*$, ($B\circ(A\circ B)=A$).
  \end{itemize}
\end{frame}

\begin{frame}{$\sigma$-algebra teljessé tétele, folyt.}
  % $A\circ B=(A\setminus B)\cup(B\setminus A)$ az $A$ és $B$ szimmetrikus differenciája.
  \begin{proposition}
    $(\Omega,\cA)$ mérhető tér. $\cS\subset\cA$ $\sigma$-algebra, 
    $\cN\subset\cA$ leszálló és $\sigma$-unió zárt. Ekkor
    $\cS^*=\set{S\circ N}{S\in\cS,\,N\in\cN}$ $\sigma$-algebra.
  \end{proposition}
  
  A továbbiakban $\cN=\cN(\cP)$ és $\cS^*=\sigma(\cS\cup\cN)$.
  Ha $\P[0]{T'=T}=1$, akkor %ugyanez minden $\P\in\cP$-re is igaz, és azt mondjuk, hogy 
  $T'$ a $T$ egy változata.
  \begin{proposition}
    $T$-nek pontosan akkor létezik $\cS$ mérhető változata, ha $T$ $\cS^*$ mérhető.
  \end{proposition}
  \begin{itemize}
    \item $\Rightarrow$: $T'$  a $T$ $\cS$ mérhető változat 
    $N=\event{T\in H}\circ\event{T'\in H}\subset \event{T\neq T'}\in\cN(\P[0])$ 
    így $N\in\cN(\P[0])$ 
    és $\event{T\in H}=\event{T'\in H}\circ N\in \cS^*$. 
    \item $\Leftarrow$: $T$ $\cS^*$ mérhető. $\event{T<q}=H_q\circ N_q$, 
    ahol $H_q\in\cS$ és $N_q\in\cN$.
    \pause
    Legyen $T'(x)=\inf\set{q\in\Q}{x\in H_q}$.
    \pause
    \begin{displaymath}
      \event{T'<c}=\cup_{q<c} H_q\in\cS\quad\implies\quad \text{$T'$ $\cS$ mérhető}
    \end{displaymath} 
    \pause
    \begin{displaymath}
      \event{T<c}\circ\event{T'<c}=(\cup_{q<c}H_q\circ N_q)\circ(\cup_{q<c}H_q)
      \subset\cup_q N_q\in\cN \quad\implies\quad\P[0]{T\neq T'}=0.
    \end{displaymath}
  \end{itemize}  
\end{frame}

% \begin{frame}{Elégséges $\sigma$-algebrák metszete nem feltétlenül elégséges}
  
% \end{frame}

\begin{frame}{Minimális elégségesség}
  \begin{df}
    $\cS$ minimális elégséges $\sigma$-algebra, ha tetszőleges  
    $\cS'$ elégséges $\sigma$-algebrára: $\cS^*\subset (\cS')^*$.

    $S$ minimális elégséges statisztika, ha $\sigma(S)$ minimális elégséges $\sigma$-algebra.
  \end{df}
  \continue
  % Megjegyzések:
  \begin{itemize}
    \item Ha $S$ minimális elégséges és $T$ elégséges, akkor $\sigma(S)\subset\sigma(S)^*\subset \sigma(T)^*$ 
    és $\P{S=g(T)}=1$ minden $\P\in\cP$-re, alkalmas mérhető $g$-vel.
    \item Ha $\cP$ dominált, akkor $\sigma(\set{\frac{d\P}{d\P[0]}}{\P\in\cP})$,
    minimális elégséges $\sigma$-algebra.
    \pause

    $\cS$ pontosan akkor egs, ha $\frac{d\P}{d\P[0]}\sim\cS^*$ minden $\P\in\cP$-re. 

    \item $\cP=\set{\P[n]}{n\geq1}$ megszámlálható mértékcsalád, akkor 
    dominált és van minimális elégséges statisztika.
    \pause

    $\P[0]=\sum_n 2^{-n}\P[n]$ domináló mérték.

    $\cS=\sigma(\set{\frac{d\P}{d\P[0]}}{\P\in\cP})$ megszámlálhatóan generált. 
    Generátorrendszer $\set{H_n}{n\geq1}$, $S=\sum_n 3^{-n} 2\I[H_n]$, $\cS=\sigma(S)$ 
    Ez HF. $\event{\cbr{3^{n-1} S}>1/2}=H_n$.

    \item Ha $\cP_0\subset\cP$, $\cN(\cP_0)=\cN(\cP)$, 
    továbbá $S$ elégséges $\cP$-re és minimális elégséges $\cP_0$-ra, akkor 
    minimális elégséges $\cP$-re is.
    \pause

    $\cS$ elégséges $\sigma$-algebra $\cP$-re $\implies$ $\cS$ egs. 
    $\cP_0$-ra, $\sigma(S)^*\subset \cS^*$,
    ahol $\cN(\cP_0)=\cN(\cP)$-vel tettük teljessé a 
    $\sigma$-algebrákat azaz $S$ minimális egs $\cP$-re.
  \end{itemize}
\end{frame}

\begin{frame}{Feltétel minimális elégségességre}
  %$\cP$ $\lambda$-val dominált.
  $\mfX'\subset\mfX$ teljes mértékű, ha a komplementere $\cN(\cP)$ eleme.
  \begin{theorem} $\cP=\set{\P[\theta]=f_\theta d\lambda}{\theta\in\Theta}$ megszámlálható
     mértékcsalád az $(\mfX,\B)$ mintatéren.
    \begin{enumerate}[<*>]
      \item Ha $S$ elégséges, akkor létezik $\mfX'$ teljes mértékű rész úgy, hogy $x,y\in\mfX'$-re, $S(x)=S(y)$ $\implies$  
      $f_\theta(x)/f_\theta(y)$ nem függ $\theta$-tól.
      \item $S$ pontosan akkor minimális elégséges, ha létezik $\mfX'$ teljes mértékű rész és $x,y\in\mfX'$, 
      $S(x)=S(y)$ $\iff$  $f_\theta(x)/f_\theta(y)$ nem függ $\theta$-tól.
    \end{enumerate}
  \end{theorem}
  1. rész
  \begin{itemize}
    \item $S$ elégséges. A Neymann faktorizációs tétel szerint
    $f_\theta\neq (g_\theta\circ S)h$ $\lambda$-nullmértékű
    és így $\P[\theta]$ nullmértékű is minden $\theta$-ra. 

    \item $\P[\theta]{h=0}=0$  minden $\theta$-ra.
    
    A megszámlálható sok kivételes halmaz uniójának komplementere legyen $\mfX'$.
    
    \item Ha $x,y\in\mfX'$ és $S(x)=S(y)$, akkor 
    % $S(x)=S(y)$ esetén 
    \begin{displaymath}
      \Theta_{x}=\set{\theta}{f_\theta(x)=g_\theta(S(x))h(x)=0}
      =\set{\theta}{f_\theta(y)=g_\theta(S(y))h(y)=0}
    \end{displaymath} 
    és $\Theta_x$-en
    % $g_\theta(S(x))$ kiesik, $\set{\theta}{f_\theta(x)=0}$ 
    $f_\theta(x)/f_\theta(y)=h(x)/h(y)$, ami nem függ $\theta$-tól.

  \end{itemize}
\end{frame}

\begin{frame}{Feltétel minimális elégségességre, folyt.}
  %$\cP$ $\lambda$-val dominált.
  $\mfX'\subset\mfX$ teljes mértékű, ha a komplementere $\cN(\cP)$ eleme.
  \begin{theorem} $\cP=\set{\P[\theta]=f_\theta d\lambda}{\theta\in\Theta}$ megszámlálható
     mértékcsalád az $(\mfX,\B)$ mintatéren.
    \begin{enumerate}[<*>]
      \item Ha $S$ elégséges, akkor létezik $\mfX'$ teljes mértékű rész úgy, hogy $x,y\in\mfX'$-re, $S(x)=S(y)$ $\implies$  
      $f_\theta(x)/f_\theta(y)$ nem függ $\theta$-tól.
      \item $S$ pontosan akkor minimális elégséges, ha létezik $\mfX'$ teljes mértékű rész és $x,y\in\mfX'$, 
      $S(x)=S(y)$ $\iff$  $f_\theta(x)/f_\theta(y)$ nem függ $\theta$-tól.
    \end{enumerate}
  \end{theorem}
  2. rész:
  \begin{itemize}
    \item %1. szerint $S$ elégséges. 
    $\cP$ megszámlálható $\implies$ 
    létezik $T$ minimális elégséges statisztika és 1. alkalmazható $T$-re.
    
    \item %$S$ és $T$ elégséges, létezik $\mfX'$ teljes mértékű része $\mfX$-nek, amire 
    \begin{displaymath}
      x,y\in\mfX',\quad T(x)=T(y)
      %\quad 
      \implies %\quad 
      \text{$\frac{f_\theta(x)}{f_\theta(y)}$ nem függ $\theta$-tól}
      %\quad
      \implies
      %\quad
      S(x)=S(y)
      %\quad
      \implies
      %\quad
      S(x)=g(T(x)),\quad\text{ha $x\in\mfX'$}.
    \end{displaymath}
    %$S(x)=g(T(x))$ ha $x\in\mfX'$ és 
    $S$ ,,egyszerűbb'' mint $T$, azaz $\sigma(S)^*\subset\sigma(T)^*$.
    \item Ugyanakkor $\sigma(T)^*\subset\sigma(S)^*$, mert $S$ elégséges és $T$ minimális elégséges.
  \end{itemize}
\end{frame}

\begin{frame}{Pontosítások}
  \begin{itemize}
    %\item Ha $f_\theta$ nulla is lehet, akkor $f_{\theta}(x)/f_{\theta}(y)$ nem függ $\theta$-tól jelentése:
    %valamely $h>0$ függvényre $f_\theta(x)h(y)=f_\theta(y)h(x)$ minden $\theta$-ra.
    \item $x,y\in \mfX'$-re $T(x)=T(y)$ $\implies$ $S(x)=S(y)$. Feltehető, hogy ez mindenhol teljesül.
    
    $\tS(x)=\tg(S(x))\I{x\in \mfX'}+2\I{x\notin\mfX'}$, 
    $\tT(x)=\tg(T(x))\I{x\in \mfX'}+2\I{x\notin\mfX'}$ módosítással:
    \begin{displaymath}
      \sigma(S)^*=\sigma(\tS)^*,\quad \sigma(T)^*=\sigma(\tT)^*,\quad 
      \tT(x)=\tT(y) \implies\tS(x)=\tS(y)
    \end{displaymath} 
    \item  Ha $S$, $T$ mérhető és $T(x)=T(y)\implies S(x)=S(y)$, akkor általában \textbf{nem} igaz, 
    hogy $\sigma(S)\subset\sigma(T)$. 

    A tétel bizonyításának vége azon múlik, hogy ha $T:(\mfX,\B(\mfX)\to([0,1],\B([0,1]))$ mérhető, 
    ahol $\mfX$ teljes szeparábilis metrikus tér, pl. $\real^n$, akkor 
    \begin{displaymath}
      \sigma(T)=\set{\event{T\in H}}{H\in\B(\real)}=\tsigma(T)=\set{A\in\B(\mfX)}{x\in A\implies [x]_T\subset A}
    \end{displaymath}
    ahol $[x]_T=\set{y}{Tx=Ty}$.
    
    \item A bizonyítás végén $T(x)=T(y)\implies S(x)=S(y)$ miatt $[x]_T\subset [x]_S$ és 
    \begin{displaymath}
      \tsigma(S)=\set{A\in\B(\mfX)}{x\in A\implies [x]_S\subset A}\subset
      \tsigma(T)=\set{A\in\B(\mfX)}{x\in A\implies [x]_T\subset A}.
    \end{displaymath}
  \end{itemize}
\end{frame}

\begin{frame}{Miért kell megszámlálható részcsaládra áttérni?}
  \begin{itemize}
    \item Egy elemű $X$ minta  az 
    \begin{displaymath}
      f_\theta(x) = \theta e^{-2\theta \abs{x}} ,\quad\theta\in\Theta=(0,\infty)
    \end{displaymath}
    sűrűségfüggvénnyel megadott eloszláscsaládból.
    \item $S=\abs{X}$ elégséges.
    \item 
    \begin{displaymath}
      \tf_\theta(x)=\begin{cases}
        \theta e^{-2\theta \abs{x}}& x\neq \theta\\
        1& x=\theta
      \end{cases}  
    \end{displaymath}
    \item $\tf_\theta$ választás mellett akarjuk $\abs{X}$ minimális elégségességét ellenőrizni.
    \begin{displaymath}
      \frac{\tf_\theta(x)}{\tf_\theta(y)}=\begin{cases}
        e^{-2\theta(\abs{x}-\abs{y})} & x\neq\theta\neq y\\
        e^{-2\theta\abs{x}} & x\neq\theta=y\\
        e^{2\theta\abs{y}} &x=\theta\neq y\\
        1 & x=\theta=y
      \end{cases}
    \end{displaymath}
    Milyen $x,y$ párokra teljesül, hogy a hányados nem függ $\theta$-tól?
    Csak akkor $x=y$, holott $S=\abs{X}$ elégséges.
    \item Ha csak megszámlálható $\Theta_0\subset \Theta$ részcsaládot veszünk, akkor 
    $\mfX'=\real\setminus\Theta_0$ teljes mértékű, és $x,y\in\mfX'$ mellett a módosításnak nincs hatása
    $\tf_\theta(x)/\tf_\theta(y)$ $\theta$-ban konstans pontosan akkor, ha $\abs{x}=\abs{y}$.
  \end{itemize}
  
\end{frame}

\begin{frame}{Példa, $U(\theta,\theta+1)$ egyenletes eloszláscsalád}
  \begin{itemize}
    \item $\Theta=\real$, domináló mérték a Lebesgue mérték. 
    Egy megfigyelés sűrűségfüggvénye $f_{\theta}(x)=\I{\theta<x<\theta+1}$.
    \item Minta tér: $\mfX=\real^n$, a minta sűrűségfüggvénye:
    \begin{displaymath}
      f_\theta(x)=\prod_i f_\theta(x_i)
      =\prod_i\I{\theta<x_i<\theta+1}
      =\I{\theta<\min x_i,\max x_i<\theta+1}
      =\I{\max x_i-1<\theta<\min x_i}
    \end{displaymath}
    \item $X_1^*=\min X_i$, $X_n^*\max X_i$, $S=(X_1^*,X_n^*)$ elégséges statisztika.
    \item Van-e egyszerűbb elégséges statisztika?
    \item $\Theta'=\Q\subset\Theta$ megszámlálható részcsalád, 
    $\cN(\set{\P[\theta]}{\theta\in\Theta'})=\cN(\Leb)=\cN(\set{\P[\theta]}{\theta\in \Theta})$. 
    Így ha $S$ minimális  elégséges a szűkebb családra nézve, akkor az eredetire is az.
    \item $\mfX'=\set{x\in\mfX}{\max x_i-1<\min x_i}$ teljes mértékű része $\mfX$-nek. $x,y\in\mfX'$-re
    \begin{displaymath}
      \frac{f_\theta(x)}{f_\theta(y)}=\frac{\I{\max x_i-1<\theta <\min x_i}}{\I{\max y_i-1<\theta<\min y_i}}  
    \end{displaymath}
    Milyen $x,y\in\mfX'$ párokra nem függ $\theta\in\Q$-tól? 
    \begin{displaymath}
      \set{\theta\in\Q}{f_{\theta}(x)\neq0}=\Q\cap(\max x_i-1,\min x_i)=
      \set{\theta\in\Q}{f_{\theta}(y)\neq0}=\Q\cap(\max y_i-1,\min y_i)
      %\I{\max x_i-1<\theta <\min x_i}h(y)=\I{\max y_i-1<\theta <\min y_i}h(x),\quad\forall\theta\in\Q
    \end{displaymath}
    azaz $\min x_i=\min y_i$ és $\max x_i=\max y_i$ azaz $S$ minimális elégséges.
    % Hányados nem függ $\theta$-tól, azzal ekvivalens, hogy $Sx=Sy$, vagyis $S$ minimális elégséges.
    \end{itemize}
\end{frame}

\begin{frame}{Példa, Cauchy eloszlás eltolásparaméteres családja}
  \begin{itemize}
    \item $\Theta=\real$, egy mintaelem sűrűségfüggvénye 
    $f_{\theta}(x)=\frac1\pi\frac{1}{1+(x-\theta)^2}$.
    \item Minta tér: $\mfX=\real^n$, a minta sűrűségfüggvénye:
    \begin{displaymath}
      f_\theta(x)=\prod_k f_\theta(x_k)
      =\frac1{\pi^n}\cdot\frac1{\prod_k(1+(x_k-\theta)^2)}
    \end{displaymath}
    A rendezett minta mindig elégséges statisztika. 
    Nem látszik hogyan lehetne ennél egyszerűbbet megadni.
    \item Elég $\Theta'=\Q$ részcsaládot nézni.
    \item 
    \begin{displaymath}
      \frac{f_\theta(x)}{f_\theta(y)}
      =\frac{\prod(1+(y_k-\theta)^2)}{\prod(1+(x_k-\theta)^2)} =\frac{p_y(\theta)}{p_x(\theta)}
    \end{displaymath}
    ahol $p_x(\theta)$ $2n$-ed fokú polinom $\theta$-ban, 1 főegyütthatóval. 
    Ha ez nem függ $\theta$-tól, akkor a hányados konstans egy 
    és $p_x(\theta)=p_y(\theta)$, $\theta\in Q$, de akkor a két polinom azonos.

    $p_x=p_y$ miatt azonosak a gyökök (multiplicitással együtt), 
    $p_x$ gyökei $\set{\pm i+x_k}{k=1,\dots n}$, így 
    \begin{displaymath}
      \set{\pm i+x_k}{k=1,\dots,n}=\set{\pm i+ y_k}{k=1,\dots,n}
      \implies\set{x_k}{k=1,\dots,n}=\set{y_k}{k=1,\dots,n}
    \end{displaymath}
    Azaz $x$ és $y$ legfeljebb a koordináták sorrendjében 
    térhet el, a rendezett minta minimális elégséges statisztika.
  \end{itemize}
\end{frame}

\begin{frame}{Példa, Indikátor minta}
  \begin{itemize}
    \item $\Theta=(0,1)$, $\mfX=\smallset{0,1}^n$, 
      $f_\theta(x)=\P[\theta]{X=x}=\theta^{s}(1-\theta)^{n-s}$, ahol $s=\sum x_i$.
    \item $S=\sum X_i$ elégséges. Van-e egyszerűbb elégséges statisztika?
    \pause
    \begin{displaymath}
      \frac{f_\theta(x)}{f_\theta(y)}=\zfrac{\theta}{1-\theta}^{\sum x_i-\sum y_i}  
    \end{displaymath}
    ez csak akkor nem függ $\theta$-tól, ha $\sum x_i=\sum y_i$, azaz a mintaösszeg minimális elégséges.
    \item Ugyanez másképp. 
    
    $\cS$ elégséges $\sigma$-algebra. $\P[\theta]{X\in A|\cS}$ nem függ $\theta$-tól, 
    $T_A$ jelöli a közös változatot.
    \item $S$ a mintaösszeg elégséges, $\E[\theta]{\I{X\in A}-T_A|S}=h(S)$ a közös változat. Ekkor 
    \begin{displaymath}
      \E[\theta]{h(S)}=\E[\theta]{\I{X\in A}-T_A}
      =\P[\theta]{A}-\E[\theta]{\E[\theta]{\I[A]|\cS}}=0.
    \end{displaymath}
    \item $S$ binomiális eloszlású $n$-renddel 
    és $\theta$ paraméterrel:
    \begin{displaymath}
      0=\E[\theta]{h(S)}=
      \sum_{k=0}^n h(k)\binom n k \theta^k(1-\theta)^{n-k}\quad\text{minden $\theta\in(0,1)$}.
    \end{displaymath}
    Így $h(k)=0$, $k=0,1,\dots,n$ és $0=h(S)=\E[\theta]{\I[A]-T_A|S}$.
    \item Ha $A\in\sigma(S)$, akkor $h(S)=\I[A]-\E{T_A|S}$ és $T_A=\I[A]$, vagyis $A\in\cS^*$ és $\sigma(S)\subset\cS^*$.
  \end{itemize}
\end{frame}

\begin{frame}{Teljesség}
  \begin{df}
    $\cP=\set{\P[\theta]}{\theta\in\Theta}$ eloszláscsalád, 
    $S$ statisztika. $S$ \textbf{teljes statisztika} (a $\cP$ eloszláscsaládra nézve), ha 
    \begin{displaymath}
      \E[\theta](h(S))=0,\quad\text{minden $\theta\in\Theta$-ra}\implies \P[\theta]{h=0}=1,\quad\text{minden $\theta$-ra}
    \end{displaymath} 
    Azaz $S$ teljes, ha az $S$ függvényei között csak a lényegében nulla változóra teljesül, hogy  
    az eloszlás család minden tagja mellett létezik és nulla a várható értéke.

    $S$ \textbf{korlátosan teljes}, ha az $S$ \textbf{korlátos} függvényei között csak a 
    lényegében nulla változóra teljesül, hogy  
    az eloszlás család minden tagja mellett létezik és nulla a várható értéke.
  \end{df}
  \begin{theorem}
    Ha $S$ korlátosan teljes és elégséges a $\cP$ eloszláscsaládra, akkor minimális elégséges. 
  \end{theorem}
  \begin{itemize}
    \item $A\in\sigma(S)$, $\cS$ tetszőleges elégséges $\sigma$-algebra, $T_A=\P[\theta]{X\in A|\cS}$ közös változat.
    \item $h(S)=\E[\theta]{\I[A]-T_A|S}=\I[A]-\E{T|S}$ korlátos statisztika, $\E[\theta]{h(S)}=\P[\theta]{A}-\P[\theta]{A}=0$.
    \item Teljesség miatt $h(S)=0$, azaz $\I[A]=\E{T_A|S}$ $\cP$ majdnem mindenütt. $0\leq T_A\leq 1$ miatt
    \begin{displaymath}
      \E[\theta]{(\I[A]-T_A)^2|S}=\E{\I[A]-2\I[A]T_A+T_A^2|S}= \E{T_A^2|S}-\I[A]\leq\E{T_A|S}-\I[A]=0.
    \end{displaymath}
    \item $\P[\theta]{\I[A]=T_A}=1$ minden $\theta$-ra, 
    azaz $\P[\theta]{A\circ \event{T_A=1}}=0$, így $A\in\cS^*$ és $\sigma(S)\subset\cS^*$.
  \end{itemize}  
\end{frame}


\begin{frame}{Példa. Nem minden minimális elégséges statisztika korlátosan teljes}
  \begin{itemize}
    \item $U(\theta,\theta+1)$, $\theta\in\real$ eloszlásból származó, $n\geq2$ elemű minta.
    \item $S=(X_1^*,X_n^*)$ minimális elégséges statisztika.
    \item $X_n^*-X_1^*$ eloszlása nem függ $\theta$-tól és $\E[\theta]{X_n^*-X_1^*}=\frac{n-1}{n+1}$.
    \item $h(S)=(X_n^*-X_1^*)\wedge1-\frac{n-1}{n+1}$ olyan korlátos függvénye $S$-nek, amire
    \begin{displaymath}
      \E[\theta]{h(S)}=0,\quad\forall \theta\in\real.
    \end{displaymath}
  \end{itemize}
\end{frame}

\begin{frame}{További példa teljes statisztikára}
  \begin{itemize}
    \item $N(\theta,1)$ eloszlásból származó $n$ elemű mintánk van.
    \item $S=\sum X_i\sim N(n\theta,n)$ elégséges statisztika.
    \item 
    \begin{displaymath}
      0=\E[\theta]{h(S)}=\int h(x)\frac{1}{\sqrt{2\pi n}} e^{-\frac{(x-n\theta)^2}{2n}}d x 
      =C(\theta)\int h(x)e^{-\frac{x^2}{2n}}e^{\theta x}d x\quad\forall\theta\in\real  
    \end{displaymath} 
  \end{itemize}
  \begin{theorem}
    $h:\real\to\real$ mérhető, $a<b$, 
    \begin{displaymath}
      \int h(x)e^{\theta x} d x,\quad\text{létezik és nulla minden $\theta\in(a,b)$-re}.
    \end{displaymath} 
    Ekkor $h=0$ Lebesgue majdnem mindenütt.
  \end{theorem}
  \begin{itemize}
    \item A tétel alapján, $h(x)e^{-\frac{x^2}{2n}}=0$ mm. és így $\P[\theta]{h(S)=0}=1$ minden $\theta$-ra
    \item $S$ teljes és elégséges, ezért minimális elégséges.
  \end{itemize}
\end{frame}

\begin{frame}{Momentum generáló függvény}  
  $X$ $p$-dimenziós vektor változó, $M_X(t)=\E{e^{t\cdot X}}$, $t\in\real^p$ az $X$ momentum generáló függvénye. 
  \begin{proposition}
    Ha $X$ momentum generáló függvénye véges az origó egy
    környezetében, akkor meghatározza $X$ eloszlását.
  \end{proposition}
  \continue
  Bizonyítás vázlat.
  \begin{itemize}   
  \item $X$ skalár változó. $h (z)=h (x+iy)=\E{e^{(x+iy) X}}$  a képzetes tengely körüli
    sávban definiált és véges. Itt deriválható (komplex értelemben),
    ezért a valós tengelyen felvett értékek (a momentum generáló függvény)
    meghatározzák a karakterisztikus függvényt, az pedig az eloszlást.
 
  \item $X$ vektor változó. $\alpha$ rögzített vektor, $Y=\alpha\cdot
    X$ skalár változó. $\E{e^{tY}}=\E{e^{(t\alpha)X}}$, azaz $Y$ eloszlását
    $X$ momentum generáló függvénye meghatározza.
  \item $X$ eloszlását az $\alpha\cdot X$ alakú változók eloszlása
    meghatározza, ugyanis a karakterisztikus függvényekre
    $\phi_{X} (\alpha)=\phi_{\alpha\cdot X} (1)$.
  \item Összefoglalva, a momentum generáló függvény meghatározza az
    egy dimenziós vetületek eloszlását, azok pedig $X$ eloszlását.
  \end{itemize}
  
\end{frame}

\begin{frame}%%{Elégséges feltétel teljességre}
  \begin{theorem}
    Legyen $\mu$ mérték $\B (\real^p)$-n, $h:\real^p\to\real$ mérhető és 
    \begin{displaymath}\textstyle
      A=\set{\alpha\in\real^p}{\int e^{\alpha \cdot x}h (x)\mu
        (dx)=0}%\quad\text{nem üres nyílt}
    \end{displaymath}
    Ha $A$-nak létezik belső pontja, akkor $h$ $\mu$ majdnem mindenütt nulla.
  \end{theorem}
  \begin{itemize}
  \item Legyen $\alpha_0\in\interior A$, és
    \begin{displaymath}
      \nu_{\pm} (H)=\int e^{\alpha_0x}\abs{h}_\pm (x) \mu (d x),\quad H\in\B(\real^p).
    \end{displaymath}
    $\nu_{+}$, $\nu_{-}$ véges Borel mértékek,  hiszen $\alpha_0\in A$.
  \item  Ha $\nu_{+} (\real^p)=\nu_{-} (\real^p) = 0$, akkor $h$ $\mu$ majdnem mindenütt
    nulla.
  \item Ha $\nu_{+}{\real^p}=\nu_{-} (\real^p)>0$, akkor konstanssal
    való szorzás után feltehető, hogy $\nu_{+},\nu_{-}$ valószínűségi
    mértékek. Azt akarjuk megmutatni, hogy ez az eset nem fordulhat
    elő. 
    
  \item A két mérték momentum generáló függvénye az origó egy
    környezetében azonos, hiszen
    \begin{displaymath}
      \int e^{t x}\nu_{+} (d x)-\int e^{t x}\nu_{-} (d x)=\int
      e^{(t+\alpha_0)x}h (x)\mu (d x)=0
          \quad\text{ha $\alpha_0+t\in A$.}
    \end{displaymath}
  \item Mivel a momentum generáló függvény az eloszlást meghatározza,
    $\nu_{+}=\nu_{-}$ és %$\nu_{+} (h>0)=\nu_{-} (h>0)=0$
    \begin{displaymath}
      \nu_{-} (h>0)=0=
      \int \I{h (x) >0} e^{\alpha_0 x}\abs{h}_+ (x)\mu (d x)=
      \nu_{+} (\real^p)=1,
      \quad\text{\alert{\Large\Lightning}}
    \end{displaymath}
    % Ez ellentmondás, abból indultunk ki, hogy feltehető, hogy $\nu_{+} (\real^p)=1$.
  \end{itemize}
\end{frame}

\begin{frame}{Exponenciális család}
  \begin{df}
    $\cP$ exponenciális eloszlás család, ha dominált és alkalmas 
    $\cP=\set{\P[\theta]}{\theta\in\Theta}$, $\Theta\subset\real^p$ paraméterezéssel 
    %$\set{\P[\theta]}{\theta\in\Theta}$ dominált
    %eloszláscsalád, $\Theta\subset\real^p$ nyílt. Ha 
    és alkalmas domináló mértékkel, $T:\mfX\to\real^p$ statisztikával 
    a sűrűségfüggvények alakja 
    \begin{displaymath}
      f_{\theta} (x) =\exp{\theta\cdot T(x)-b (\theta)}.\tag{*}
    \end{displaymath}
    %akkor \textbf{exponenciális családról} beszélünk.
  \end{df}
  \begin{itemize}
  \item $T$ elégséges statisztika (Neymann faktorizációs tétel)
  \item $T$ sőt minimális elégséges, ha $\Theta\subset\real^p$ lineáris burka $\real^p$, 
  azaz ha $\Theta\subset\real^p$ tartalmazza $\real^p$ egy (lineáris) bázisát.
  

  $T$ mindig választható minimális elégségesnek, elég $P T$-t venni, ahol $P$ merőleges 
  vetítés  $\Theta$ lineáris burkára.

  \item Ha $\Theta$ belseje nem üres, akkor $T$ teljes is.
  
  $\lambda$ a domináló mérték, amivel a sűrűség $(*)$ alakú. $\tlambda(H)=\lambda(T\in H)$, $H\in\B(\real^p)$.

  Ha $\E[\theta](h(T))$ létezik és nulla, akkor
  \begin{displaymath}
    0=\E[\theta]{h(T)}=\int_{\mfX} h\circ T \exp{\theta T-b(\theta)}d\lambda 
    =\int_{\real^p} h(t) \exp{\theta t-b(\theta)}\tlambda(d t)
    \implies \int_{\real^p} e^{\theta t} h(t)\tlambda(d t),\quad \forall \theta\in\Theta
  \end{displaymath}

  $h d\tlambda$ az azonosan nulla mérték. Tetszőleges $H\in\B(\real^p)$-re
  \begin{displaymath}
    0=\int_H h d\tlambda=\int_{\mfX} \I{T\in H} h\circ T d\lambda,\quad
    \forall  H\in\B(\real^p)
    \quad\implies\quad \lambda(h\circ T\neq0)=0. 
  \end{displaymath}
  $\P[\theta]{h(T)\neq 0}=0$, azaz $\P[\theta]{h(T)=0}=1$.
  % \item $T$-nek véges a  momentum generáló függvénye az origó egy
  %   nyílt környezetében, mert
  %   \begin{displaymath}
  %     M (\alpha)=\E[\theta]{e^{\alpha\cdot T}}=\int e^{\alpha\cdot T (x)} f_\theta
  %     (x)\lambda (d x)=\int e^{\alpha\cdot T (x)+\theta T (x)+b
  %       (\theta)}\lambda (d x) =
  %     e^{b (\theta)-b (\theta+\alpha)},\quad\text{ha $\theta,\theta+\alpha\in\Theta$}
  %   \end{displaymath}
  
  % \item Általános összefüggés, visszatérünk rá a következő dián, hogy
  %   ha a momentum generáló függvény véges az origó egy környezetében,
  %   akkor ott $C^\infty$ és a nullabeli deriváltak a momentumokat adják.
  % \item Ebből $\theta\mapsto \ell (\theta,x)=\theta T (x)+b (\theta)$ minden $x$-re sima és
  %   \begin{align*}
  %     %-b (\theta+\alpha)&= \log M (\alpha)-b (\theta)&
  %     -b' (\theta)&= \frac {M'}{M} (0)=\E[\theta]{T},&
  %     -b'' (\theta)&=\frac {M''M-(M')^2}{M^2} (0)=\D[\theta]^2{T}.
  %   \end{align*}
  %   $T$ momentumai végesek mindegyik $\set{\P[\theta]}{\theta\in \Theta}$)
  %   alatt.
  \end{itemize}
\end{frame}

\begin{frame}{Példák exponenciális családra}
  \begin{itemize}
    \item $n$ elemű minta $\exp(\theta)$, $\theta>0$ eloszlásból, $\mfX=(0,\infty)^n$, $\lambda$ a Lebesgue mérték.
    \begin{displaymath}
      f_\theta(x) = \theta^n\exp{-\theta\sum x_i},\quad T(x)=-\sum x_i,\quad b(\theta)=-n\ln\theta.
    \end{displaymath}
    \item $n$ elemű indikátor minta, $p\in(0,1)$ paraméterrel, $\mfX=\smallset{0,1}^n$, 
     $\lambda$ a számláló mérték
    \begin{align*}
      f_\theta(x)&=p^{\sum x_i}(1-p)^{n-\sum x_i}=\zfrac{p}{1-p}^{\sum x_i} (1-p)^n\\
      &=\exp{\theta T(x) +b(\theta)},\quad \theta=\ln\frac{p}{1-p}, \quad p =\frac{1}{1+e^{-\theta}},\quad 
      T(x)=\sum x_i,\quad 
      b(\theta)= n\ln(1+e^{\theta})
    \end{align*}
    \item $N(\mu,\sigma^2)$ eloszlásból származó $n$ elemű minta. $\mfX=\real^n$ $\mu\in\real$, $\sigma^2>0$. 
    Sűrűségfüggvény a Lebesgue mértékre.
    \begin{align*}
      f_{\mu,\sigma^2}(x)&=\frac{1}{(2\pi)^{n/2}\sigma^n}\exp{-\frac1{2\sigma^2}\sum(x_i-\mu)^2}\\
      &=\exp{-\frac{1}{2\sigma^2}\sum x_i^2+\frac{\mu}{\sigma^2}\sum x_i -\tilde b(\mu,\sigma^2)},
      \quad \theta=\zjel{\tfrac1{2\sigma^2},\tfrac\mu{\sigma^2}}, \quad T(x)=\zjel{-\sum x_i^2,\sum x_i}.
    \end{align*}
  \end{itemize}
\end{frame}

\begin{frame}{$T$ nem mindig teljes}
  \begin{itemize}
    \item $N(\mu,\sigma^2)$ eloszlásból származó minta $\mu=\sigma^2$  ismeretlen.
    
    Átparaméterezés után $\theta(\mu,\sigma^2)=(\tfrac1{2\sigma^2},\tfrac\mu{\sigma^2})$,
    $\Theta=\set{(\tau,1)}{\tau>0}$. $T(x)=(-\sum x_i^2,\sum x_i)$ minimális elégséges.

    \item $n=2$, 
    \begin{align*}
      \E{T_1}&= \E{-(X_1^2+X_2^2)}=-2(\sigma^2+\mu^2),\\
      \E{T_2}&= \E{(X_1+X_2)} =2\mu,\quad
      \E{T^2_2}=\E{(X_1+X_2)^2} = 2\sigma^2+(2\mu)^2
    \end{align*}
    \item 
    \begin{displaymath}
      \E{2T_1+T_2^2 + T_2} = -4(\sigma^2+\mu^2)+2\sigma^2+4\mu^2+2\mu=2\mu-2\sigma^2=0  
    \end{displaymath}
    $\E[\theta]{h(T)}=0$, minden $\theta=(\tau,1)$-re, ha 
    $h(t_1,t_2)=2t_1+t_2^2+ t_2$.   
    $\E[\theta]{h(T)}=0$ miden $\theta$-ra, de $\P[\theta]{h(T)=0}=0$.
    \begin{displaymath}
      h(T(X))=-(X_1-X_2)^2+2(X_1+X_2)
      %\P[\theta]{-X_1^2+X_1=c}=\P[\theta]{X_1\in\smallset{x_1,x_2}}=0, 
      %\quad x_{1,2}=\frac{1\pm\sqrt{1-4c}}{2} 
    \end{displaymath}
    Itt $(X_1-X_2,X_1+X_2)$ független nem elfajult normális változók, 
    így $\P[\theta]{h(T)=0}=0$ minden $\theta\in\Theta$-ra. 
    % \begin{displaymath}
    %   \P[\theta]{h(T)=0}=\P[\theta]{-X_1^2+X_1=\sum_{k=2}^n -X_k^2+X_k}=0.
    % \end{displaymath}

  \end{itemize}  
\end{frame}

\begin{frame}{Nem minden eloszláscsalád exponenciális}
  \begin{itemize}
    \item Ha $\cP$ exponenciális eloszláscsalád, akkor $\P\in\cP$ mértékek sűrűségek mindenhol pozitívak, 
    azaz ekvivalensek.

    $n$ elemű minta $U(\theta,\theta+1)$, $\theta\in\real$ eloszlásból.
    A kapott eloszláscsalád nem exponenciális.
    \item Cauchy eloszlás eltolás paraméteres családja. 
    
    Ha $X-\mu\sim \text{Cauchy}$, $\mu\in\real$ esetén, $X$ eloszlásainak a családja exponenciális családot alkot, 
    akkor $n$-elemű minta eloszlásainak a családja ismeretlen
    \begin{displaymath}
      \ln f_\mu(x)=\sum_i \ln f_\mu(x_i)=\theta(\mu)T_n(x)-nb(\theta(\mu))+\ln h(x), \quad T_n(x)=\sum_i T(x_i) 
    \end{displaymath}
    ahol $\theta,T:\real\to\real^p$. $\theta(0)=0$ feltehető, mert $\theta(0)T_n(x)$ beolvasztható $\ln h(x)$-be. 

    Feltehető, hogy $\theta$ értékkészletének lineáris burka $\real^p$. Ekkor létezik $\mu_k$ $k=1,\dots,p$, 
    hogy $\theta_k=\theta(\mu_k)$, $k=0,\dots,p$ lineárisan függetlenek. Ekkor alkalmas $c_k$ együtthatókkal.
    \begin{displaymath}
      T_n(x)=\sum_{k=1}^p c_k (\ln f_{\mu_k}-\ln f_0)(x).
    \end{displaymath}
    $n>p+1$ esetén $T_n(x)=T_n(y)$ akkor is előfordul, ha $x^*\neq y^*$, ami ellentmond annak, hogy a 
    rendezett minta minimális elégséges statisztika.

  \end{itemize}
  
\end{frame}

\end{document}

\begin{frame}{Momentum generáló függvény, deriválhatóság}
  \begin{proposition}
    Ha $M (\alpha)=\E{e^{\alpha\cdot X}}<\infty$, az origó egy környezetében,
    akkor ott  $M$ deriválható.
  \end{proposition}
  \begin{itemize}
    
  \item  Létezik $\eps>0$, hogy $\E{e^{h\norm{X}}}<\infty$, ha
    $h<\eps$.

    $\real^p$ a normák ekvivalensek, elég
    $\norm{x}_{1}=\frac1p \sum \abs{x_i}$-re számolni. Mivel $\exp$
    konvex $e^{\norm{x}_1}\leq\frac1p \sum e^{\abs*{x_i}}$ és
    %$\norm{h}_1=\sum h_i=1$ és $h_i\geq0$, akkor
    \begin{displaymath}
      e^{h\abs{X_i}}\leq e^{hX_i}+e^{-hX_i},\quad
      \E{e^{h\norm{X}_1}}\leq \frac1p \sum_i(M (h e_i)+M (-h e_i))<\infty
    \end{displaymath}
    ahol $e_1,\dots,e_p$ a természetes bázis $\real^p$-ben.
  \item ha $\norm{\alpha}$ kicsi, akkor
    $\abs*{\alpha\cdot X}<\eps\norm{X}$ és a dominált konvergencia tétel miatt
    \begin{displaymath}
      M (\alpha)=\E{\sum_k \frac{(\alpha\cdot X)^n}{n!}}
      =\sum_n \frac{\E{(\alpha\cdot X)^n}}{n!}\quad\implies\quad 
      \partial_{i_1,\dots,i_k}M (0) =\E{\prod_{j}X_{i_j}}
    \end{displaymath}
  \end{itemize}

\end{frame}

\end{document}
