%\def\option{}
\providecommand{\option}{handout}
\documentclass[aspectratio=169,notheorems,9pt,\option]{beamer}


\usepackage{marvosym}
%\usepackage[magyar]{babel}
\usepackage{polyglossia}
\setmainlanguage{magyar}

% \usepackage{pvdefs}
\usepackage{etoolbox,mathtools}

%\makeatletter
\def\ifbrace{\@ifnextchar{\bgroup}}

\@ifundefined{texexp}{%
  \let\texexp\exp
  \def\exp{%
    \texexp\ifbrace{\cbr*}{}%
  }
}\relax
 
\def\isparam#1{\ifbrace#1\relax}

\DeclarePairedDelimiter{\abs}{\lvert}{\rvert}
\DeclarePairedDelimiter{\zjel}{(}{)}
\DeclarePairedDelimiter\br\lbrack\rbrack
\DeclarePairedDelimiter\event\lbrace\rbrace
\DeclarePairedDelimiter{\norm}{\|}{\|}
\let\smallset\event
\let\cbr\event
\DeclarePairedDelimiterX\set[2]\{\}{%
  #1\nonscript\::\allowbreak\nonscript\:\mathopen{}#2%
}

\let\PEfont\mathbb
\providecommand\given{}

\def\redefgiven#1{%
  \renewcommand\given{%
    \nonscript\:%
    #1\vert%
    \allowbreak%
    \nonscript\:%
    \mathopen{}%
  }%
}
\DeclarePairedDelimiterX\PEzjel[1](){\redefgiven{\delimsize}#1}
\def\aftersuperscript#1^#2{%
  ^{#2}#1%
}
\newcommand{\PE}[1][]{%
  \PEfont{\Pe}%
  \ifblank{#1}{}{_{#1}}%
  \@ifnextchar{^}{%
    \aftersuperscript{\ifbrace{\PEzjel*}{}}%
    }{%
      \ifbrace{\PEzjel*}{}%
    }%
  %\ifbrace{\PEzjel*}{}%
}
\newcommand{\E}{\def\Pe{E}\PE}
\newcommand{\D}{\def\Pe{D}\PE}
\renewcommand{\P}{\def\Pe{P}\PE}

\def\cov{\operatorname{cov}}
\let\eps\varepsilon
\def\ito/{It\^o}
\let\Ito=\ito
\def\eqinlaw{\stackrel{d}{=}}

\def\strip#1{%
    \expandafter\stripi\detokenize{#1}\relax
}
\def\stripi#1#2\relax{%
    \def\stripi##1##2\relax{%
    \ifx##1#1\else\expandafter##1\fi##2%
    }%
}
\expandafter\stripi\detokenize{\ }\relax

\def\strip#1{%
    \expandafter\stripi\string#1\relax
}
\def\stripi#1#2\relax{%
    \def\stripi##1##2\relax{%
    \ifx##1#1\else\expandafter##1\fi##2%
    }%
}
\expandafter\stripi\string\ \relax

\DeclareListParser{\dolist}{}

\def\defprefix#1#2{
    \def\do##1{\csdef{#1\strip##1}{{#2{##1}}}}%
    \ifbrace\dolist\do
}

\defprefix{c}{\mathcal}{XAPIDLCSGTHEN}
\defprefix{b}{\bar}{hX}
\defprefix{t}{\tilde}{STQbCf\lambda \Omega \sigma }
\defprefix{}{\mathcal}{BF}
\defprefix{mf}{\mathfrak}{X}

\def\sign{\operatorname{sign}}
\def\real{\mathbb{R}}
\def\Z{\mathbb{Z}}
\def\Q{\mathbb{Q}}

\def\Isymb{{\mathbf 1}}
\def\substr#1{_{\zjel{#1}}}
\newcommand{\I}[1][]{%
  \Isymb\ifblank{#1}{\substr}{_{#1}}%
}

\def\Isymb{{\mathbb 1}}

\def\I{\@ifstar\I@star\I@}
\def\I@star{%
  %\message{I@star arg:|#1|^^J}%
  \Isymb\ifbrace{_}{}% } %{_{#1}}%
  }
\newcommand{\I@}[2][\zjel]{%
  \Isymb_{#1{#2}}% } %{_{#1}}%
}



\def\interior{\operatorname{int}}
\def\tg{\operatorname{tg}}
\DeclarePairedDelimiterX{\zfrac}[2](){\frac{#1}{#2}}
\newcommand\independent{\protect\mathpalette{\protect\independenT}{\perp}}
\def\independenT#1#2{\mathrel{\rlap{$#1#2$}\mkern2mu{#1#2}}}


\newtheorem{df}{Definíció}
\newtheorem{theorem}{Tétel}
\newtheorem{lemma}{Lemma}
\newtheorem{proposition}{Állítás}
\newtheorem{corollary}{Következmény}



\def\diag{\operatorname{diag}}
\def\Id{\mathbf{I}}
\let\PEfont\mathbb
\def\npto{\buildrel p\over \nrightarrow}
\def\ndto{\buildrel d\over \nrightarrow}
%\def\dto{\buildrel d\over \rightarrow}
\def\dto{\stackrel{d}{\to}}
\def\pto{\stackrel{p}{\to}}
\def\tr{\operatorname{Tr}}
\def\arctg{\operatorname{arctg}}
\let\nto\nrightarrow

\def\Leb{\operatorname{Leb}}


\def\readrest#1\enddate{}
\def\setdate@#1-#2-#3\relax{\def\lecturedate{#1.#2.#3.}\readrest}
\def\setdate{\expandafter\setdate@\jobname\relax-\relax-\relax\enddate}
\setdate

% \usetheme{Bruno}
\usecolortheme{Bruno}
% \usefonttheme{Bruno}
% \useinnertheme{Bruno}
\useoutertheme{Bruno}

\newrobustcmd{\continue}{%
  \onslide<\value{beamerpauses}->\relax
}


\beamerdefaultoverlayspecification{<+->}

\mode<presentation>

\RequirePackage[math-style=TeX, bold-style=upright]{unicode-math}
\setmathfont{Latin Modern Math}[Scale=1]
\setsansfont[Ligatures=TeX, ItalicFont={Fira Sans Italic}, BoldFont={Fira Sans SemiBold}, Scale=0.9]{Fira Sans Book}
\usepackage[mathcal]{euscript}

% Title page
\setbeamerfont{title}{
    size=\LARGE,
    shape=\bfseries
}

\setbeamerfont{subtitle}{
    size=\large,
    shape=\normalfont
}

\setbeamerfont{author}{
    size=\normalsize,
    shape=\normalfont
}

\setbeamerfont{details}{
    size=\footnotesize,
    shape=\normalfont
}

% Slide title
\setbeamerfont{frametitle}{size=\LARGE}
\setbeamerfont{framesubtitle}{
    size=\normalsize,
    %shape=\normalfont\scshape
}

% Blocks
\setbeamerfont{block title}{
    size=\normalfont,
    shape=\strut
}

\setbeamerfont{blockdef}{
    size=\large,
    shape=\bfseries
}

% Description
\setbeamerfont{description item}{shape=\bfseries}


% Footer information line with title, author and slide number
\setbeamertemplate{footline}{%
    \leavevmode%
    \hbox{%
        \usebeamerfont{footline}%
        \begin{beamercolorbox}[
            wd=\textwidth,
            ht=3ex,
            dp=1.25ex
        ]{footline}%
            \hspace{1cm}%
            \insertshorttitle%
            \hspace{2em}%
            \insertdate%
            \hfill%
            \insertframenumber{} / \inserttotalframenumber%
            \hspace{1cm}
        \end{beamercolorbox}%
    }
    \vskip0pt%
}

%% Bruno Beamer theme — Version 1.0
%% A lightweight Beamer theme inspired from the Metropolis theme
%%
%% Written in 2017-2019 by:
%% — Rémi Cérès <remiceres@msn.com>
%% — Mattéo Delabre <bonjour@matteodelabre.me>
%%
%% This work is released under the CC0 1.0 Universal license. See the
%% accompanying LICENSE file for details. To the extent possible under
%% law, Rémi Cérès and Mattéo Delabre have waived all copyright and
%% related or neighboring rights to the Bruno Beamer theme. This work
%% is published from France.

% \mode<presentation>

\setbeamertemplate{sections/subsections in toc}[square]
\setbeamertemplate{itemize item}[square]
\setbeamertemplate{itemize subitem}[circle]

% Allow multi-slide frames
\setbeamertemplate{frametitle continuation}{}

% Horizontal separator
\setbeamertemplate{separator}{%
    \usebeamercolor{separator}%
    \textcolor{fg}{\rule{.7\textwidth}{.5pt}}%
}

% Separate authors with newlines
\renewcommand{\beamer@andtitle}{\\}

% Title page. If you want to add a background image on the title slide,
% use the \background command to set the path to that image. Otherwise,
% no image will be included
\RequirePackage{tikz}
\usetikzlibrary{fadings}

\newcommand{\background}[1]%
    {\newcommand{\bruno@background}{#1}}


%% \newcommand\insertcaption{}
\newcommand{\backgroundcaption}[2][white]{\def\bruno@caption{{#1}{#2}}}

\newcommand{\insertcaption}{%
    \ifdefined
    \bruno@caption{%
        \expandafter\textcolor\bruno@caption
    }%
    \fi
}


\tikzfading[
    name=title page picture fading,
    left color=transparent!0,
    right color=transparent!100,
]

\setbeamertemplate{title page}{%
    \begin{minipage}{.7\textwidth}
        \raggedright
        \usebeamerfont{title}
        \inserttitle

        \usebeamertemplate{separator}

        \usebeamerfont{author}
        \vspace{2em}
        \insertauthor

        \vspace{2em}
        \usebeamerfont{details}
        \insertinstitute

        \insertdate
    \end{minipage}

    % Include faded image on the right if defined
    \ifdefined\bruno@background
    \begin{tikzpicture}[remember picture, overlay]
        % Crop image to a trapezium on the right
        \clip (current page.south east)
           -- (current page.north east)
           -- ++(-5.1, 0)
           -- ++(-3, -\paperheight)
           -- cycle;

        % Include background image
        \node[
            anchor=south east,
            inner sep=0,
            outer sep=0
        ] at (current page.south east) {
            \includegraphics[height=\paperheight]
                {\bruno@background}
        };

        \node[anchor=south east,
            inner sep=0,
            outer sep=15pt,
        ] at (current page.south east){\insertcaption};

        % Add a slight shadow
        \fill[
            black, path fading=title page picture fading,
            fading angle=-15
        ]
            (current page.south east)
            rectangle
            ++(-10, \dimexpr\paperheight+1cm);

        
    \end{tikzpicture}
    \fi
}

\mode<all>
\AtBeginDocument{%
  \let\phi\varphi
  \let\theta\vartheta
  \let\setminus\smallsetminus
}

\title{Statisztika előadás}
\date{\lecturedate}
\background{images/young-Ronald-Fisher.jpg}
\backgroundcaption{Ronald Fisher (1913)}

\def\ip#1#2{\langle#1,#2\rangle}
\begin{document}

\maketitle
\begin{frame}{Folytonos illeszkedés vizsgálat}
  \begin{itemize}
    \item $X_1,X_2,\dots,X_n$ minta  $F$ folytonos egy dimenziós
    eloszlásból.
    \item $F_0$ adott folytonos eloszlás függvény.
    \item $H_0: F=F_0$, lehetséges alternatívák
    
    $H_1:F\geq F_0$, $F \neq F_0$ %(egyoldali ellenhipotézis).
    
    $H^*_1:F\neq F_0$ %(egyoldali ellenhipotézis).
    \item Statisztikák:
    \begin{displaymath}
      D_n^+=\sup_t F_n (t)-F_0 (t),
      \quad 
      D_n=\sup_t\abs*{F_n (t)-F_0 (t)}
      \quad\text{Kolmogorov--Szmirnov}
    \end{displaymath}
    \begin{displaymath}
      \omega^2_n=\int_{\real} (F_n (t)-F_0 (t))^2 dF_0 (t),
      % ,\quad D_n=\sup_t\abs*{F_n (t)-F_0 (t)}
      \quad\text{Cramer--von Mises}
    \end{displaymath}
  \end{itemize}
  \begin{proposition}
    A Kolmogorov--Szmirnov és Cramer--von Mises statisztikák
    eloszlásmentesek, azaz $H_0$ fennállása esetén eloszlásuk nem függ $F_0$-tól.
  \end{proposition}
\end{frame}

\begin{frame}{Kolmogorov--Szmirnov statisztika eloszlásmentessége}
  \begin{itemize}
    \item $X_1,\dots,X_n$ $n$--elemű minta $F_0$ folytonos
    eloszlásfüggvényből.
    \item $X_0^*=-\infty<X_1^*<X_2^*<\dots<X_n^*<X_{n+1}^*=\infty$ egy valószínűséggel, mert $i\neq
    j$-re $\P{X_i=X_j}=\E{\P{X_i=X_j|X_j}}=\E{\P{X_i=y}|_{y=X_j}}=0$.
    \item $(X_i^*,X_{i+1}^*)$-n $F_n$ konstans, $F_0$ monoton,
    \begin{displaymath}
      D_n^+ = \sup_{t}F_n (t)-F_0 (t)=\max_{1\leq i\leq n} (F_n (X_i)-F_0 (X_i))\vee(F_n (X_i+)-F_0 (X_i))
    \end{displaymath}
    
    \item $F_n (X_i^*)=\frac{i-1}n$ és $F_n (X_i^*+)=\frac{i}n$. Azaz
    \begin{displaymath}
      D_n^+=\max_i(\tfrac{i-1}n-F_0 (X_i^*))\vee(\tfrac{i}n-F_0 (X_i^*))
    \end{displaymath}
    \item $U_i=F_0 (X_i)$ $(0,1)$-en egyenletes eloszlású minta, $F_0
    (X_i^*)=U_i^*$
    \item $D_n^+$ eloszlása tetszőleges $F_0$ esetén ugyanaz.
    \item Hasonlóan $D_n=\sup_t\abs*{F_n (t)-F (t)}$-re.
  \end{itemize}
  
\end{frame}

\begin{frame}{Cramer--von Mises statisztika eloszlásmentes}
  \begin{itemize}
    \item $X_1,\dots,X_n$ $n$--elemű minta $F_0$ folytonos
    eloszlásfüggvényből.
    \begin{displaymath}
      \omega_n^2=\int (F_n (t)-F_0 (t))^2dF_0 (t)
    \end{displaymath}
    \item Ha $X_0$ a mintától független $F_0$ eloszlású
    \begin{displaymath}
      \omega_n^2=\E{(F_n (X_0)-F_0 (X_0))^2|X_1,\dots,X_n}
    \end{displaymath}
    \item $U_0=F_0 (X_0)$, $U_i=F_0 (X_i)$ $i=1,\dots,n$.
    \begin{displaymath}
      F^U_n (t)=\frac1n\sum_{i=1}^n \I{U_i<t}
    \end{displaymath}
    \item Mivel $\P{U_0\in\smallset{U_1,\dots,U_n}}=0$, ezért $F_n
    (X_0)=F^{U}_n (U_0)$ és
    \begin{displaymath}
      \omega_n^2=\E{(F^U_n (U_0)-U_0)^2|U_1,\dots,U_n}
    \end{displaymath}
    azaz $\omega_n^2$ eloszlása minden folytonos $F_0$ esetén ugyanaz.
  \end{itemize}
  
\end{frame}

\begin{frame}{Folytonos illeszkedés vizsgálat}
  \begin{itemize}
    \item $X_1,X_2,\dots,X_n$ minta  $F$ folytonos egy dimenziós
    eloszlásból.
    \item $F_0$ adott folytonos eloszlás függvény.
    \item $H_0: F=F_0$, lehetséges alternatívák
    
    $H_1:F\geq F_0$, $F \neq F_0$ %(egyoldali ellenhipotézis).
    
    $H^*_1:F\neq F_0$ %(egyoldali ellenhipotézis).
    \item Statisztikák:
    \begin{displaymath}
      D_n^+=\sup_t F_n (t)-F_0 (t),\quad D_n=\sup_t\abs*{F_n (t)-F_0 (t)}\quad\text{Kolmogorov--Szmirnov}
    \end{displaymath}
    \begin{displaymath}
      \omega^2_n=\int_{\real} (F_n (t)-F_0 (t))^2 dF_0 (t),
      % ,\quad D_n=\sup_t\abs*{F_n (t)-F_0 (t)}
      \quad\text{Cramer--von Mises}
    \end{displaymath}
    
    \item Kolmogorov--Szmirnov próba kritikus tartománya
    
    $H_1$ esetében $D_n^+>c$ esetén elvetjük a nullhipotézist
    
    $H^*_1$ esetében $D_n>c$ esetén elvetjük a nullhipotézist
    
    \item Cramer--von Mises próba:  $\omega^2_n>c$ esetén vetjük el $H_0$-t.
    
    \item Nem túl nagy $n$-re léteznek táblázatok. 
    \item A skálázott próba statisztikának létezik határeloszlása,
    általában a pontos eloszlást a határeloszlással közelítjük.
  \end{itemize}
\end{frame}

\begin{frame}{Skálázott próbastatisztikák határeloszlása}
  \begin{itemize}
    \item
    \begin{displaymath}
      D_n^+=\sup_t F_n (t)-F_0 (t),\quad D_n=\sup_t\abs*{F_n (t)-F_0 (t)}
    \end{displaymath}
    \item
    \begin{displaymath}
      \lim_{n\to\infty}\P{\sqrt{n}D_n^+>x}
      =e^{-2x^2}=\frac{\phi (2x)}{\phi (0)},
    \end{displaymath}
    ahol $\phi$ a standard normális sűrűség függvény.
    
    Más szóval $n(D_n^+)^2\dto\exp{2}$.
    \item
    \begin{displaymath}
      \lim_{n\to\infty}\P{\sqrt{n}D_n<x} =
      \sum_{k\in \Z} (-1)^{k} e^{-2k^2x^2}=\sum_{k\in \Z} (-1)^{k}\frac{\phi (2kx)}{\phi (0)}
    \end{displaymath}
    \item
    \begin{displaymath}
      n\omega_n^2  \dto\sum_{k=1}^\infty
      \frac{Z_k^2}{k^2\pi^2},\quad\text{ahol $Z_1,Z_2,\dots$ független $N (0,1)$-ek.}
    \end{displaymath}
  \end{itemize}
\end{frame}


\begin{frame}{Brown híd, mese}
  \begin{itemize}
    \item Legyen $U_1,\dots,U_n$ $(0,1)$-n egyenletes minta.
    \begin{displaymath}
      X^{(n)}=\zjel*{\frac1{\sqrt{n}} \sum\nolimits_{i=1}^n(\I{U_i\leq t}-t)}_{t\in[0,1]}=\sqrt{n}\zjel*{F_n(t)-t}_{t\in[0,1]}
    \end{displaymath}
    valószínűségi változók paraméterezett családja, azaz folyamat.
    \item Ha $\underbar{t}=(t_1,\dots,t_r)\in[0,1]^r$ akkor 
    \begin{displaymath}
      X^{(n)}_{\underbar{t}}=\tfrac1{\sqrt{n}}\zjel*{\sum\nolimits_{k=1}^n Y_k-\E{Y_k}},
      \quad\text{ahol}\quad
      Y_k=\zjel{\I{U_k<t_1},\dots,\I{U_k<t_r}}.
    \end{displaymath} 
    $X^{(n)}_{\underbar t}\dto N(0,\Sigma(Y))$, ahol $\cov(\I{U<s},\I{U<t})=\P{U<s,\, U<t}-\P{U<s}\P{U<t}=s\wedge t-st$.
    \item $t\mapsto X^{(n)}_t (\omega)$ jobbról folytonos, balról
    határértékkel rendelkező függvény (trajektória).
    \begin{displaymath}
      D[0,1]=\set{w:[0,1]\to\real}{\text{$w$ jobbról folytonos, balról
      határértékkel rendelkezik}}
    \end{displaymath}
    %\item 
    $D[0,1]$ alkalmas %megadható olyan 
    metrikával %, amivel $D[0,1]$ 
    teljes szeparábilis metrikus tér. $X^{(n)}$  véletlen elem ebből a térből.
    \item Megmutatható, hogy $X^{(n)}$ eloszlásban konvergens, az
    eloszlásbeli limesz neve Brown híd.
    
    $(B_t)_{t\in[0,1]}$ \textbf{Brown híd}, ha folytonos trajektóriájú Gauss
    folyamat, $\E{B_t}=0$, $\cov (B_t,B_s)=s\wedge t-st$.
    
    \item $\phi (w)=\max_s w (s)$, $\max_s \abs*{w (s)}$, $\int_0^1w^2
    (s)d s$ folytonos funkcionálok $D[0,1]$-en. 
    % \item
    \begin{displaymath}
      \sqrt{n}D_n^+\dto \max_{s\in[0,1]} B_s,
      \quad\sqrt{n}D_n\dto \max_{s\in[0,1]}\abs*{B_s},
      \quad n\omega_n^2\dto \int_0^1B_s^2d s,\quad\text{ahol $B$ Brown híd}
    \end{displaymath}
  \end{itemize}
\end{frame}

\begin{frame}{Skálazott Cramer--von Mises statisztika határeloszlása}
  \begin{itemize}
    \item $(B_t)_{t\in[0,1]}$ \textbf{Brown híd}, ha folytonos trajektóriájú Gauss
    folyamat, $\E{B_t}=0$, $\cov (B_t,B_s)=s\wedge t-st$.
    
    \begin{displaymath}
      X^{(n)}=\zjel*{\frac1{\sqrt{n}} \sum\nolimits_{i=1}^n(\I{U_i\leq t}-t)}_{t\in[0,1]}
      =\sqrt{n}\zjel*{F_n(t)-t}_{t\in[0,1]},
      \quad X^{(n)}\dto B\quad \text{a $D[0,1]$ metrikus térben.} 
    \end{displaymath}
    \item $e_n(t)=\sqrt{2}\sin (n\pi t)$. $\set{e_n}{n\geq1}$ teljes ortonormált rendszer (bázis) 
    az $L^2([0,1])$ Hilbert térben. 
    \begin{displaymath}
      n\omega_n^2\eqinlaw\int_0^1 (X^{(n)}_s)^2d s
      \dto \int_0^1 B_s^2 d s 
      =\sum_{n=1}^\infty \zjel*{\int_0^1 B_se_n(s)ds}^2
    \end{displaymath}
    Itt a $\xi_n=\int_0^1 B_s e_n(s)ds$  változók együttesen normálisak. $\E{\xi_n}=\int_0^1\E{B_s}e_n(s)ds=0$ 
    és 
    \begin{displaymath}
      \cov(\xi_n,\xi_m)
      =\E{\int_0^1\int_0^1 B_se_n(s)B_te_m(t)dsdt}
      =\int_{[0,1]^2} \cov(B_s,B_t)e_n(s)e_m(t)d s d t
      =\I{n=m} \frac1{n^2\pi^2}
    \end{displaymath}
    Azaz a $\xi_n$ változók függetlenek és $Z_n=n\pi\xi_n$ független $N(0,1)$-ek 
    sorozata.
    % $\xi^2_n\eqinlaw Z_n^2/(n^2\pi^2)$ és
    
    \item $n\omega^2_n\dto \sum_{n=1}^\infty \frac{Z_n^2}{n^2\pi^2}$
  \end{itemize}
\end{frame}

\begin{frame}{Skálázott Kolmogorov--Szmirnov statisztika határeloszlása}
  \begin{itemize}
    \item  $(B_t)_{t\in[0,1]}$ \textbf{Brown híd}, ha folytonos trajektóriájú Gauss
    folyamat, $\E{B_t}=0$, $\cov (B_t,B_s)=s\wedge t-st$.
    
    \begin{displaymath}
      X^{(n)}=\zjel*{\frac1{\sqrt{n}} \sum\nolimits_{i=1}^n(\I{U_i\leq t}-t)}_{t\in[0,1]}
      =\sqrt{n}\zjel*{F_n(t)-t}_{t\in[0,1]},
      \quad X^{(n)}\dto B\quad \text{a $D[0,1]$ metrikus térben.} 
    \end{displaymath}
    \item $Z\sim N(0,1)$ független a $B$ Brown hídtól és  $\beta_t=B_t+tZ$, $t\in[0,1]$.
    \begin{displaymath}
      \cov(\beta_s,\beta_t)=\cov(B_s,B_t)+\cov(sZ,tZ)=\min(s,t)=s\wedge t.
    \end{displaymath}
    $(\beta_t)_{t\in[0,1]}$ \textbf{Brown mozgás}, azaz folytonos trajektóriájú Gauss folyamat nulla várható értékkel 
    és $\cov(\beta_s,\beta_t)=s\wedge t$ kovariancia függvénnyel. 
    \item 
    \begin{displaymath}
      \P{\max_{s\in[0,1]} B_s\geq x}
      =\lim_{\eps\to0}\P{\max_{s\leq 1} B_s+sZ\geq x \given \abs{Z}<\eps}
      =\lim_{\eps\to0}\P{\max_{s\leq 1} \beta_s\geq x \given \abs{\beta_1}<\eps}
    \end{displaymath}
    \item $u<s<t$ esetén $\cov(\beta_t-\beta_s,\beta_u)=0$, azaz $\set{\beta_u}{u\in[0,s]}$ 
    és $\set{\beta_t-\beta_s}{t\in[s,1]}$ függetlenek. 
    \item $\set{\beta_t-\beta_s}{t\in[s,1]}\eqinlaw \set{-(\beta_t-\beta_s)}{t\in[s,1]}$ és így 
    \begin{displaymath}
      \set{\beta_t=\beta_{s\wedge t}+(\beta_t-\beta_{s\wedge t})}{t\in[0,1]}\eqinlaw\set{\tbeta_t=\beta_{t\wedge s}-(\beta_{t}-\beta_{t\wedge s})}{t\in[0,1]}
    \end{displaymath}
    Ez a tükrözési elv determinisztikus $s$ időpontra. Igaz marad akkor is, ha pl. 
    $\tau=\inf\set{s\in[0,1]}{\beta_s\geq x}$. 
  \end{itemize}
\end{frame}

\begin{frame}{Skálázott Kolmogorov--Szmirnov statisztika határeloszlása}
  \begin{itemize}
    \item<*>  $(B_t)_{t\in[0,1]}$ \textbf{Brown híd}, ha folytonos trajektóriájú Gauss
    folyamat, $\E{B_t}=0$, $\cov (B_t,B_s)=s\wedge t-st$.
    
    \begin{displaymath}
      X^{(n)}=\zjel*{\frac1{\sqrt{n}} \sum\nolimits_{i=1}^n(\I{U_i\leq t}-t)}_{t\in[0,1]}
      =\sqrt{n}\zjel*{F_n(t)-t}_{t\in[0,1]},
      \quad X^{(n)}\dto B\quad \text{a $D[0,1]$ metrikus térben.} 
    \end{displaymath}
    \item<*> $Z\sim N(0,1)$ független a $B$ Brown hídtól és  $\beta_t=B_t+tZ$. $\beta$ Brown mozgás.
    \begin{displaymath}
      \P{\max_{s\in[0,1]} B_s\geq x}
      % =\lim_{\eps\to0}\P{\max_{s\leq 1} B_s+sZ\geq x \given \abs{Z}<\eps}
      =\lim_{\eps\to0}\P{\max_{s\leq 1} \beta_s\geq x \given \abs{\beta_1}<\eps}
    \end{displaymath}
    \item $x>0$, $\tau=\inf\set{s}{\beta_s\geq x}$. Ekkor 
    $\tbeta_t=\beta_{t\wedge \tau}-(\beta_t-\beta_{t\wedge\tau})=2\beta_{t\wedge\tau}-\beta_t$ is Brown mozgás és
    \begin{displaymath}
      \P{\max_{s\leq 1}\beta_s\geq x,\,\abs{\beta_1}<\eps}
      =\P{\max_{s\leq 1}\tbeta_s\geq x,\,\abs{\tbeta_1-2x}<\eps}
      =\P{\abs{\tbeta_1-2x}<\eps},\quad\text{ha $\eps<x$}.
    \end{displaymath}
    \item  $\beta_1,\tbeta_1\sim N(0,1)$
    \begin{displaymath}
      \hphantom{\hbox{}\hspace{-2em}}
      \lim_{n\to\infty}\P{\sqrt{n}D_n^{+}\geq x}=\P{\max_{s\leq 1} B_s\geq x}
      =\lim_{\eps\to 0}\P{\max_{s\leq 1}\beta_s\geq x\given \abs{\beta_1}<\eps}
      =\lim_{\eps\to0}\frac{\P{\abs{\tbeta_1-2x}<\eps}}{\P{\abs{\beta_1}<\eps}}
      =\frac{\phi(2x)}{\phi(0)}
      =e^{-2x^2}.
    \end{displaymath}
  \end{itemize}
\end{frame}

\section*{Klasszikus próbák optimalitása}

\begin{frame}{Emlékeztető. Neyman-Pearson lemma}
  \begin{itemize}
    \item \textbf{Egyszerű} a hipotézis vizsgálati feladat, ha
    $\Theta_0,\Theta_1$ egyelemű. Ekkor $H_0:f=f_0$, $H_1:f=f_1$ alakú
    a feladat, ahol $f_0$, $f_1$ a két eloszlás $\P[0]$, $\P[1]$ sűrűségfüggvénye
    alkalmas domináló mértékre  nézve (pl. $\lambda=\frac12 (\P[0]+\P[1])$).
    \item \textbf{Likelihood hányados} statisztika $f_1/f_0$
  \end{itemize}
  \begin{lemma}
    Egyszerű hipotézis vizsgálati feladat, $f_0>0$. 
    Tegyük fel, hogy a  likelihood hányados $f_1/f_0$
    folytonos eloszlású $H_0$ és $H_1$ mellett is.
    
    Ekkor
    \begin{itemize}[<*>]
      \item Minden $\alpha\in(0,1)$-re létezik $c_\alpha$ úgy,
      hogy $\phi=\I{f_1>c_\alpha f_0}$ terjedelme $\alpha$. (Azaz $\E[0]{\phi}=\alpha$)
      %T>c_\alpha}=\alpha$)
      \item $\phi$ a legerősebb $\alpha$ terjedelmű próba.
      \item Ha $\alpha (\bar\phi)\leq \alpha$ és
      $\E[1]{\bar{\phi}}=\E[1]{\phi}$, akkor $\bar{\phi}=\phi$
      ($\P[0]$, $\P[1]$ majdnem mindenütt).
    \end{itemize}
  \end{lemma}
  \continue
  Nekünk annyi kell, hogy  a likelihood hányados próba a vele azonos terjedelmű próbák között a legerősebb.
\end{frame}


\begin{frame}{Egyoldali ellenhipotézis mellett, az $u$-próba 
  egyenletesen legerősebb I.}
  
  A feladat: $X_1,\dots,X_n$ minta $N(\mu,\sigma^2)$ eloszlásból, $\sigma$ ismert.
  $H_0: \mu=\mu_0$, $H_1:\mu>\mu_0$.
  
  A sűrűségfüggvény
  \begin{displaymath}
    f_\mu (x)=\zfrac*1{2\pi\sigma^2}^{n/2}\exp{-\frac1{2\sigma^2}\sum_i
    (x_i-\mu)^2}=h (x)\exp{\mu \tT (x)-b (\mu)}
  \end{displaymath}
  ahol
  \begin{displaymath}
    h (x)=\zfrac*1{2\pi\sigma^2}^{n/2}e^{-\frac1{2\sigma^2}\sum_i x_i^2},\quad
    b(\mu)=n\frac{\mu^2}{2\sigma^2},\quad \tT (x)=\frac{\sum_i x_i}{\sigma^2}
  \end{displaymath}
  \begin{itemize}
    \item Elegendő a $\mu_0=0$ esetet nézni.
    \item Egyszerű hipotézis vizsgálati feladat ($0<\mu_1$), $H_0:\mu=0$,
    $H_1:\mu=\mu_1$.
    \item A likelihood hányados
    statisztika 
    \begin{displaymath}
      \frac{f_1}{f_0} (x)=\exp{(\mu_1-0)\tT (x)-(b (\mu_1)-b(0))}
      \quad\text{$\tT$ szigorúan monoton növő függvénye}
    \end{displaymath}
    \item $\tT$ eloszlása folytonos, azaz a legerősebb próba nem
    randomizált és
    \begin{displaymath}
      \mfX_1=\event{\tT>c}\quad\text{alakú, ahol}\quad \P[0]{\tT>c}=\alpha
    \end{displaymath}
  \end{itemize}
\end{frame}

\begin{frame}[<*>]{Egyoldali ellenhipotézis mellett, az $u$-próba
  egyenletesen legerősebb II.}
  
  \begin{itemize}
    \item<*> $X_1,\dots,X_n$ minta $N(\mu,\sigma^2)$ eloszlásból, $\sigma$ ismert.
    Egyszerű hipotézis vizsgálati feladat ($0<\mu_1$), $H_0:\mu=0$,
    $H_1:\mu=\mu_1$.
    \item<*> $\alpha$ terjedelmű likelihood hányados próba
    \begin{displaymath}
      \mfX_1=\event{\tT>c}
      \quad\text{alakú, ahol}\quad
      \tT=\frac{1}{\sigma^2}\sum_i X_i,\quad \P[0]{\tT>c}=\alpha
    \end{displaymath}
    
    \item A likelihood hányados próba nem függ
    $\mu_1\in(0,\infty)$-től.
    \item A fenti próba egyenletesen legerősebb a $H_1:\mu>0$
    ellenhipotézisre is.
    
    Ugyanis, ha a $\phi$ döntési függvénnyel megadott próba terjedelme
    $\alpha$ és valamely $\mu_1>0$ pontban az ereje nem kisebb, mint a
    fenti próba ereje, akkor a Neyman--Pearson lemma miatt $\phi$ is
    likelihood hányados próba, vagyis azonos a 
    fenti próbával.
    \item Az elutasítási tartomány
    \begin{displaymath}
      \event*{\sum_{i} X_i>\sigma^2 c}=
      \event*{T(X)>u_\alpha},
      \quad\text{ahol}\quad
      T (X)=\sqrt{n}\frac{\bar{X}}{\sigma},\quad \P[0]{T>u_\alpha}=\alpha.
    \end{displaymath}
    $T(X)\sim N(0,1)$ a $H_0: \mu=0$ hipotézis mellett.
  \end{itemize}
\end{frame}

\begin{frame}[<*>]{$H_1:\mu\neq0$-ra nincs egyenletesen legerősebb próba}
  
  \begin{itemize}
    \item $H_0:\mu=0$, $H_1^*:\mu=\mu_1$ feladatra, a likelihood hányados
    próba a legerősebb. 
    \item Mivel a likelihood hányados eloszlása $H_0$ mellett folytonos
    a likelihood hányados próba nem randomizált, így egyértelmű.
    \item Ha $\phi$ egyenletesen legerősebb lenne, akkor minden
    $\mu_1\neq0$-ra egybeesne a $H^*$ feladathoz tartozó likelihood
    hányados próbával.
    \item $\mu_1>0$ esetén $\phi (x)=\I{T (x) >u_\alpha}$
    
    \item $\mu_1<0$ esetén $\phi (x)=\I{T (x) <-u_\alpha}$
    
    \item Ezek egyszerre nem teljesülhetnek, nincs ilyen $\phi$ és nincs
    egyenletesen legerősebb próba.
  \end{itemize}
\end{frame}


\begin{frame}{$H_0:\mu=0$, $H_1:\mu\neq0$-ra, az $u$-próba egyenletesen
  legerősebb torzítatlan próba}
  $\Theta$ konvex, nyílt és (R) teljesül,
    $\Theta_0=\smallset{\theta_0}$, $\phi$ torzítatlan. %, pl. $\P[\theta]=N (\mu,\sigma^2)$, $\mu\in\real=\Theta$.
    Ekkor $\psi(\theta)=\E[\theta]{\phi}$ folytonosan differenciálható, minimumhely $\theta=\theta_0$-ban van és $\E[\theta_0]{\ell' (\theta_0) }=0$.
    \begin{displaymath}
      \cov_{\theta_0} (\phi,\ell' (\theta_0))=\E[\theta_0]{\phi\ell'(\theta_0)}=\psi'(\theta_0)=0.
    \end{displaymath}

  \begin{proposition}[Bizonyítás később]
    $\Theta\subset \real$ nyílt, $\set{\P[\theta]}{\theta\in\Theta}$ exponenciális eloszláscsalád. % $H_0:\theta=0$, $H_1:\theta\neq 0$.
    \begin{displaymath}
      \phi=\I{\ell'(0)\notin[a_1,a_2]},\quad \Phi=\set{\bphi}{\E[0]{\bphi}=\E[0]{\phi},\quad \E[0]{\bphi\ell'(0)}=0}
    \end{displaymath}
    Ha $\phi\in \Phi$, akkor $\phi$ egyenletesen legerősebb próba $\Phi$-ben a
    $H_0:\theta=0$, $H_1:\theta\neq0$ feladatra és $a_1<0<a_2$.
  \end{proposition}
  \begin{itemize}
    \item $\set{N (\mu,\sigma^2)}{\mu\in\real}$, $\sigma^2>0$ rögzített,
    esetén ($\theta=\mu$)
    \begin{displaymath}
      \ell (\mu,x)=\ln h (x)+\mu \tT (x) - b (\mu),\quad
      \text{ahol}\quad
      \tT (x)=\frac{\sum_i x_i}{\sigma^2},\quad b (\theta)=\frac{n\mu^2}{2\sigma^2}
      \quad \ell' (\mu)=\frac{n}{\sigma^2}(\bar{X}-\mu)
    \end{displaymath}
    \item  $u$ próba próba statisztikája: $T=\frac{\sqrt{n}}{\sigma} \bX$, 
    döntési függvénye: $\phi=\I{\abs{T}>u_{\alpha/2}}$. $\phi$ a fenti alakú.
    \item  $\phi$ torzítatlan: $\P[\mu]$ alatt $T\sim N(\sqrt{n}\frac{\mu}{\sigma},1)$, %, ha a paraméter $\theta=\mu$. 
    % \frac{n\mu}{\sigma^2}$  $\phi\in\Phi$, mert 
    a $\psi(\mu)=\E[\mu]{\phi}$ jelöléssel, ha $Z\sim N(0,1)$
    \begin{displaymath}
      \psi(\mu)=\P{-u_{\alpha/2}-\tfrac{\sqrt{n}\mu}{\sigma}<Z<+u_{\alpha/2}-\tfrac{\sqrt{n}\mu}{\sigma}},\quad
      \psi'(\mu)=
      c\zjel*{f_Z\zjel*{u_{\alpha/2}-\tfrac{\sqrt{n}\mu}{\sigma}}-f_Z\zjel*{-u_{\alpha/2}-\tfrac{\sqrt{n}\mu}{\sigma}}}
    \end{displaymath}
  \end{itemize}  
\end{frame}

\begin{frame}
  \begin{proposition}
    $\Theta\subset \real$ nyílt, $\set{\P[\theta]}{\theta\in\Theta}$ exponenciális eloszláscsalád. % $H_0:\theta=0$, $H_1:\theta\neq 0$.

      $\phi=\I{\ell'(0)\notin[a_1,a_2]},\quad \Phi=\set{\bphi}{\E[0]{\bphi}=\E[0]{\phi},\quad \E[0]{\bphi\ell'(0)}=0}$.

    Ha $\phi\in \Phi$, akkor $\phi$ egyenletesen legerősebb próba $\Phi$-ben a
    $H_0:\theta=0$, $H_1:\theta\neq0$ feladatra.
  \end{proposition}
  
  \begin{lemma}[Bizonyítás később]
    $\theta_1<0<\theta_2$, $p\in (0,1)$. $H_0:f=f_{0}$, $H_1^*: f=pf_{\theta_1}+ (1-p)f_{\theta_2}$.
    
    Ekkor %$\forall\,\theta_1<0<\theta_2$-re 
    létezik $p=p (\mu_1,\mu_2)$ úgy, hogy $H^*$-hoz $\phi$ a legerősebb a vele azonos terjedelmű próbák között és
    \begin{displaymath}
      \exists \lim\nolimits_{\theta_2\to0}\tfrac{p (\theta_1,\theta_2)}{\theta_2}\in (0,\infty),\
      \quad 
      \lim\nolimits_{\theta_1\to0}\tfrac{1-p (\theta_1,\theta_2)}{\theta_1}\in (0,\infty).
    \end{displaymath}
  \end{lemma}
  \begin{itemize}
    \item 
    Ezt felhasználva, ha $\bar{\phi}\in\Phi$, akkor
    \begin{displaymath}
      \lim\nolimits_{\theta_2\to0}\tfrac{1}{\theta_2}\zjel{\E[\theta_2]{\bar{\phi}}-\E[0]{\bar{\phi}}}
      =\E[0]{\bar{\phi}\ell' (0)}=0,
      \quad
      \lim\nolimits_{\theta_2\to0}\tfrac{1}{\theta_2}\zjel{\E[\mu_2]{\phi}-\E[0]{\phi}}
      =\E[0]{\phi\ell' (0)}=0
    \end{displaymath}
    és
    \begin{displaymath}
      \tfrac{1}{p(\theta_1,\theta_2)}\E[\theta_2]{\bar{\phi}-\phi}\to0,\quad\text{ha $\theta_2\to0$
      és $\E[0]{\bar{\phi}}= \E[0]{\phi}$}
    \end{displaymath}
    \item Mivel $\phi$ legerősebb próba $H^*$-ra
    \begin{displaymath}
      p\E[\theta_1]{\bar{\phi}}+ (1-p)\E[\theta_2]{\bar{\phi}}\leq
      p\E[\theta_1]{\phi}+ (1-p)\E[\theta_2]{\phi},
      \quad\implies
      \E[\theta_1]{\phi-\bar{\phi}}\geq \tfrac{1-p(\theta_1,\theta_2)}{p(\theta_2)}\E[\theta_2]{\bar{\phi}-\phi}\to0.
      %\quad\theta_2\to0.
    \end{displaymath}
  \end{itemize}
\end{frame}

\begin{frame}{Lemma bizonyítása}
  \begin{itemize}
    \item $\theta_1<0<\theta_2$. $H_0:f=f_0$, $H_1^*:f=pf_1+ (1-p)f_2$, ahol 
    \begin{displaymath}
      f_i  = f_{\theta_i}, \quad f_\theta(x) = h (x)e^{\theta\cdot T(x) - b (\theta)},
    \end{displaymath}
    Feltehető, hogy $T=\ell'(0)$. %és $b (0)=b' (0)=0$, azaz $\partial\ell (0)=T$.
    \item A likelihood hányados 
    \begin{displaymath}
      \frac{pf_1+ (1-p)f_2}{f_0} (x) 
      =p e^{\theta_1 t-b (\theta_1)}+(1-p)e^{\theta_2 t-b (\theta_2)}|_{t=T(x)}=g (T(x))
    \end{displaymath}
    Ez a $t$-nek konvex függvénye, $\lim_{t\to\pm\infty} =\infty$
    \item A kritikus tartomány $\event*{T\notin [a_1,a_2]}$ alakú $a_1<0<a_2$, ha
    \begin{displaymath}
      g (a_1)=g (a_2),\quad\text{azaz}\quad
      p\zjel{e^{\theta_1 a_1}-e^{\theta_1 a_2}}e^{-b (\theta_1)}
      = (1-p)\zjel{e^{\theta_2 a_2}-e^{\theta_2 a_1}}e^{-b (\theta_2)}
    \end{displaymath}
    Ennek megoldása $p$-re $p (\theta_1,\theta_2)$. Ha $\theta_1$ rögzített és $\theta_2\to0$, akkor $p(\theta_1,\theta_2)\to0$
    \item Ha $\theta_2$-vel osztunk és $\theta_2\to0$, akkor
    \begin{displaymath}
      \hphantom{\hbox{}\hspace{-1em}}
      \lim_{\theta_2\searrow0}\frac{p (\theta_1,\theta_2)}{\theta_2} 
      \zjel*{e^{\theta_1 a_1}-e^{\theta_1 a_2}}e^{-b(\theta_1)}
      = \lim_{\theta_2\searrow0} (1-p(\theta_1,\theta_2)) 
      e^{-b (\theta_2)} \zjel*{\frac{e^{\theta_2a_2}-1}{\theta_2}-\frac{e^{\theta_2a_1}-1}{\theta_2}}
      %\frac{\sinh (\theta_2c)}{\theta_2 c}
      = e^{-b(0)}(a_2-a_1)
    \end{displaymath}
  \end{itemize}
\end{frame}



\begin{frame}{Egyenletesen legerősebb próbák zavaró paraméter mellett}
  \begin{itemize}
    \item $t$-próba esetét nézzük részletesebben.
    \item $X_1,\dots,X_n\sim N (\mu,\sigma^2)$ minta, csak $\mu$ érdekel
    minket. 
    \item Elég a $H_0:\mu=0$ esetet nézni.
    \item Az eloszláscsalád dominált
    \begin{displaymath}
      f_{\mu,\sigma^2} (x)=\exp{-\frac1{2\sigma^2}\sum(x_i-\mu)^2 + c
      (\sigma)}=
      \exp{-\frac{1}{2\sigma^2}\sum x_i^2+\frac\mu{\sigma^2}\sum x_i+c (\mu,\sigma)}
    \end{displaymath}
    \item Átparaméterezés: $\theta=\frac{n\mu}{\sigma^2}$, 
    $T(x)=\bar{x}$, $\tau=-\frac{n}{2\sigma^2}$, 
    $S (x)=\frac1n\sum x_i^2=s_n^2+T^2$.
    \item Átparaméterezés után a kérdés $H_0:\theta=0$, $H_1:\theta>0$
    ($H_1:\theta<0$) vagy kétoldali ellenhipotézis esetében $H_1:\theta\neq0$.
    \item Az átparaméterezett család sűrűségfüggvénye:
    \begin{displaymath}
      f_{\theta,\tau} (x) =\exp{\theta T (x)+\tau S (x)+b (\theta,\tau)}
    \end{displaymath}
    alakú. 
    \item A paramétertér eredetileg $\real\times (0,\infty)$,
    átparaméterezés után $\real\times (-\infty,0)$.
    \item $T,S$ elégséges statisztika. $b$ a kummuláns generáló
    függvénnyel van kapcsolatban.
  \end{itemize}
\end{frame}

\begin{frame}{Egyenletesen legerősebb próbák zavaró paraméter mellett}
  
  \begin{itemize}
    \item Átparaméterezés után a kérdés $H_0:\theta=0$, $H_1:\theta>0$
    ($H_1:\theta<0$) vagy kétoldali ellenhipotézis esetében
    $H_1:\theta\neq0$.
    \item Az átparamétezett család sűrűségfüggvénye:
    \begin{displaymath}
      f_{\theta,\tau} (x) =\exp{\theta T (x)+\tau S (x)+b (\theta,\tau)}
    \end{displaymath}
    alakú. 
    \item $T,S$ elégséges statisztika. $b$ a $(T,S)$ kummuláns generáló
    függvényével van kapcsolatban.
    \item Tetszőleges $\phi$ döntési függvényre $\E{\phi|(T,S)}$
    ugyanolyan jó próbát ad, vagyis elég a $(T,S)$ statisztika
    eloszlásainak a családját nézni.
    
    \item Ha  $\cP$ domináló mértéke $\P[0]$ (valószínűségi mérték), akkor
    $(T,S)$ eloszlásainak a családját pl. $Q_0=\P[0]\circ (T,S)^{-1}$ dominálja
    és a sűrűségfüggvények
    \begin{displaymath}
      g_{\theta,\tau} (t,s)=\exp{\theta t+\tau s-b (\theta,\tau)}
    \end{displaymath}
    alakúak.
    \item $Q_0$ mérték a $\real\times\real^p$ téren, de nem feltételenül
    szorzat mérték!
    \item Nézzük a feladatot az $S$-re vett feltétel mellett.
  \end{itemize}
\end{frame}

\begin{frame}{$(T,S)$ feltételes eloszlásai}
  \begin{itemize}
    \item $\P[0]$ domináló mérték $X$ eloszlásaira, $Q_0=\P[0]\circ (T,S)^{-1}$.
    \item $T,S$ eloszlása $\P[\theta,\tau]$ mellett abszolút folytonos
    $Q_0$-ra nézve, a sűrűségfüggvény
    \begin{displaymath}
      g_{\theta,\tau} (t,s)=\exp{\theta t+\tau s+b (\theta,\tau)}
    \end{displaymath}
    
    
    \item Legyen $Q(H,S)$ a $\P[0]{T\in H|S}$ feltételes eloszlás
    reguláris változata.% Ezzel a jelöléssel
    \begin{displaymath}
      Q_0 (A\times B)=\P[0]{T\in A,S\in B}=\E[0]{\P[0]{T\in A|S}\I{S\in B}}
      =\int_{B} Q (A,s) \tQ (ds),
      \quad\text{ahol}\quad\tQ=\P[0]\circ S^{-1}
    \end{displaymath}
    \item  Ezzel a jelöléssel
    \begin{align*}
      \P[\theta,\tau]{T\in B, S\in A}
      &=\int \I{t\in B}\I{s\in B}e^{\theta t+\tau s+b
      (\theta,\tau)}Q_0 (dt,ds)\\
      &= \int_A\int_Be^{\theta t} Q (dt,s)e^{\tau s+b(\theta,\tau)}\tQ(ds)=
      \E[\theta,\tau]{h(B,S)}
    \end{align*}
    ahol
    \begin{displaymath}
      h (B,s)=\int_Be^{\theta t+\tb (\theta,s)} Q(dt,s),
      \quad\text{és}\quad e^{-\tb (\theta,s)}=\int e^{\theta t} Q(dt,s) 
    \end{displaymath}
    
    
    \item Összefoglalva, $\P[\theta,\tau]{T\in H|S}$ feltételes
    eloszlás reguláris változata $S=s$ mellett csak $\theta$-tól függ
    és egy paraméteres exponenciális családot alkot  $Q (dt,s)$
    domináló mértékkel
    
  \end{itemize}
\end{frame}

\begin{frame}{Egyenletesen legerősebb próbák zavaró paraméter mellett,
  folyt.}
  \begin{itemize}
    \item $H_0:\theta=0$, $H_1:\theta>0$. $\phi=\phi (S,T)$ torzítatlan $\alpha$
    terjedelmű próba. Ekkor $\psi (\theta,\tau)=\E[\theta,\tau]{\phi}$
    folytonos és a torzítatlanság miatt $\psi (0,\tau)=\alpha$ minden
    $\tau$-ra.
    \item A $\P[0,\tau]$ egy paraméteres család is exponenciális $S$
    teljes és elégséges statisztikával
    \begin{displaymath}
      f_{0,\tau} (x)=\exp{\tau S (x)+b (0,\tau)}
    \end{displaymath}
    $\set{(0,\tau)}{(0,\tau)\in\Theta}$,  a $\Theta_0,\Theta_1$ közös
    határa.
    
    \item Ha $S$ teljes a $\set{\P[0,\tau]}{\tau}$ családra, akkor
    azt mondjuk, $S$ a \textbf{határon teljes}.
    \item Mivel $S$ teljes a határon, ezért
    \begin{displaymath}
      \E[0,\tau]{\E[0,\tau]{\phi (T,S)|S}-\alpha}=0
      \quad\implies\quad
      \E[0,\tau]{\phi (T,S)|S}=\alpha
    \end{displaymath}
    
    \item Ha $\phi$ torzítatlan $\alpha$ terjedelmű, akkor $\E[(0,\tau)](\phi)=\alpha$ minden $\tau$-ra.
    
    $S$ teljessége miatt $\phi(T,S)$  $\alpha$
    terjedelmű a feltételes változatra is, azaz a
    $\P[\theta,\tau]{T\in H|S}$ eloszlás családra.
    
    \item A feltételes változatra a likelihood hányados
    \begin{displaymath}
      \frac{e^{\theta_1 t+\tb (\theta_1,s)}}{e^{\tb (0,s)}} \quad\text{$t=T$ monoton növő függvénye}
    \end{displaymath}
    Egyoldali ellenhipotézis esetén a feltételes változatra a legerősebb próba alakja
    \begin{displaymath}
      \phi (T,S)=\I{T>c (S)}
    \end{displaymath}
    %alakú egyoldali ellenhipotézisre.   
  \end{itemize}
  
\end{frame}

\begin{frame}{A $t$-próba egyenletesen legerősebb a torzítatlan próbák
  között}
  \begin{itemize}
    \item $X_1,\dots,X_n\sim N (\mu,\sigma^2)$ minta, csak $\mu$ érdekel
    minket. 
    \item A $\theta=\frac{n\mu}{\sigma^2}$, $T
    (x)=\bar{x}$, $\tau=-\frac{n}{2\sigma^2}$, $S (x)=\frac1n\sum
    x_i^2=s_n^2+T^2$ átparaméterezés után
    \begin{displaymath}
      f_{\mu,\sigma^2} (x)=f_{\theta,\tau}(x)= \exp{\theta T (x)+\tau
      S (x)+b (\theta,\tau)}
      \quad
      H_0:\theta=0, \, H_1:\theta>0.
    \end{displaymath}
    \item Áttértünk %a $T,S$ elégséges statisztika eloszlásaira, majd
    a $\P[\theta,\tau]{T\in H|S}$  feltételes eloszlásokra. Ha $\phi=\phi (T,S)$
    torzítatlan $\alpha$ terjedelmű, akkor $\alpha$ terjedelmű a
    feltételes feladatra is $\E[0,\tau]{\phi (T,S)|S}=\alpha$.
    \item A feltételes feladatban egy paraméteres család marad
    $\P[\theta,\tau]{T\in H|S}$ nem függ $\tau$-tól. A likelihood
    hányados $T$ monoton növő függvénye az egyenletesen legerősebb
    próba $\phi (t,s)=\I{t>c (s)}$ alakú.
    
    \item A $t$ próba $\phi (T,S)=\I{T/\sqrt{S-T^2}>c}$ torzítatlan, mert
    $T/s_n$  eloszlása minden $(0,\tau)$ párra ugyanaz.
    
    \item Mivel $c\geq 0$-ra
    \begin{displaymath}
      T>c\sqrt{S-T^2}\quad\iff\quad T^2>c^2 (S-T^2),\,T>0
      \quad\iff\quad
      T>\frac{c}{\sqrt{1+c^2}}\sqrt{S}=\bar{c}\sqrt{S}
    \end{displaymath}
    $\phi=\I{T>\bar{c}\sqrt{S}}$ alakú és egyenletesen legerősebb a feltételes
    feladatra.
    Hasonlóan $c<0$-ra
    
    \item Ha $\bar{\phi}$ tetszőleges $\alpha$ terjedelmű torzítatlan. Feltehető, hogy $\bar{\phi}=\phi(T,S)$ és
    ekkor 
    \begin{displaymath}
      \E[\theta,\tau]{\phi-\bar{\phi}} =
      \E[\theta,\tau] {\E[\theta,\tau]{\phi-\bar{\phi}|S}}\geq0\quad\forall\theta,\tau
    \end{displaymath}
  \end{itemize}
\end{frame}

\end{document}   
