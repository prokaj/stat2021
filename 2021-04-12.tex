%\def\option{}
\providecommand{\option}{handout}
\documentclass[aspectratio=169,notheorems,9pt,\option]{beamer}

\usepackage{marvosym}
%\usepackage[magyar]{babel}
\usepackage{polyglossia}
\setmainlanguage{magyar}

% \usepackage{pvdefs}
\usepackage{etoolbox,mathtools}

%\makeatletter
\def\ifbrace{\@ifnextchar{\bgroup}}

\@ifundefined{texexp}{%
  \let\texexp\exp
  \def\exp{%
    \texexp\ifbrace{\cbr*}{}%
  }
}\relax
 
\def\isparam#1{\ifbrace#1\relax}

\DeclarePairedDelimiter{\abs}{\lvert}{\rvert}
\DeclarePairedDelimiter{\zjel}{(}{)}
\DeclarePairedDelimiter\br\lbrack\rbrack
\DeclarePairedDelimiter\event\lbrace\rbrace
\DeclarePairedDelimiter{\norm}{\|}{\|}
\let\smallset\event
\let\cbr\event
\DeclarePairedDelimiterX\set[2]\{\}{%
  #1\nonscript\::\allowbreak\nonscript\:\mathopen{}#2%
}

\let\PEfont\mathbb
\providecommand\given{}

\def\redefgiven#1{%
  \renewcommand\given{%
    \nonscript\:%
    #1\vert%
    \allowbreak%
    \nonscript\:%
    \mathopen{}%
  }%
}
\DeclarePairedDelimiterX\PEzjel[1](){\redefgiven{\delimsize}#1}
\def\aftersuperscript#1^#2{%
  ^{#2}#1%
}
\newcommand{\PE}[1][]{%
  \PEfont{\Pe}%
  \ifblank{#1}{}{_{#1}}%
  \@ifnextchar{^}{%
    \aftersuperscript{\ifbrace{\PEzjel*}{}}%
    }{%
      \ifbrace{\PEzjel*}{}%
    }%
  %\ifbrace{\PEzjel*}{}%
}
\newcommand{\E}{\def\Pe{E}\PE}
\newcommand{\D}{\def\Pe{D}\PE}
\renewcommand{\P}{\def\Pe{P}\PE}

\def\cov{\operatorname{cov}}
\let\eps\varepsilon
\def\ito/{It\^o}
\let\Ito=\ito
\def\eqinlaw{\stackrel{d}{=}}

\def\strip#1{%
    \expandafter\stripi\detokenize{#1}\relax
}
\def\stripi#1#2\relax{%
    \def\stripi##1##2\relax{%
    \ifx##1#1\else\expandafter##1\fi##2%
    }%
}
\expandafter\stripi\detokenize{\ }\relax

\def\strip#1{%
    \expandafter\stripi\string#1\relax
}
\def\stripi#1#2\relax{%
    \def\stripi##1##2\relax{%
    \ifx##1#1\else\expandafter##1\fi##2%
    }%
}
\expandafter\stripi\string\ \relax

\DeclareListParser{\dolist}{}

\def\defprefix#1#2{
    \def\do##1{\csdef{#1\strip##1}{{#2{##1}}}}%
    \ifbrace\dolist\do
}

\defprefix{c}{\mathcal}{XAPIDLCSGTHEN}
\defprefix{b}{\bar}{hX}
\defprefix{t}{\tilde}{STQbCf\lambda \Omega \sigma }
\defprefix{}{\mathcal}{BF}
\defprefix{mf}{\mathfrak}{X}

\def\sign{\operatorname{sign}}
\def\real{\mathbb{R}}
\def\Z{\mathbb{Z}}
\def\Q{\mathbb{Q}}

\def\Isymb{{\mathbf 1}}
\def\substr#1{_{\zjel{#1}}}
\newcommand{\I}[1][]{%
  \Isymb\ifblank{#1}{\substr}{_{#1}}%
}

\def\Isymb{{\mathbb 1}}

\def\I{\@ifstar\I@star\I@}
\def\I@star{%
  %\message{I@star arg:|#1|^^J}%
  \Isymb\ifbrace{_}{}% } %{_{#1}}%
  }
\newcommand{\I@}[2][\zjel]{%
  \Isymb_{#1{#2}}% } %{_{#1}}%
}



\def\interior{\operatorname{int}}
\def\tg{\operatorname{tg}}
\DeclarePairedDelimiterX{\zfrac}[2](){\frac{#1}{#2}}
\newcommand\independent{\protect\mathpalette{\protect\independenT}{\perp}}
\def\independenT#1#2{\mathrel{\rlap{$#1#2$}\mkern2mu{#1#2}}}


\newtheorem{df}{Definíció}
\newtheorem{theorem}{Tétel}
\newtheorem{lemma}{Lemma}
\newtheorem{proposition}{Állítás}
\newtheorem{corollary}{Következmény}



\def\diag{\operatorname{diag}}
\def\Id{\mathbf{I}}
\let\PEfont\mathbb
\def\npto{\buildrel p\over \nrightarrow}
\def\ndto{\buildrel d\over \nrightarrow}
%\def\dto{\buildrel d\over \rightarrow}
\def\dto{\stackrel{d}{\to}}
\def\pto{\stackrel{p}{\to}}
\def\tr{\operatorname{Tr}}
\def\arctg{\operatorname{arctg}}
\let\nto\nrightarrow

\def\Leb{\operatorname{Leb}}


\def\readrest#1\enddate{}
\def\setdate@#1-#2-#3\relax{\def\lecturedate{#1.#2.#3.}\readrest}
\def\setdate{\expandafter\setdate@\jobname\relax-\relax-\relax\enddate}
\setdate

% \usetheme{Bruno}
\usecolortheme{Bruno}
% \usefonttheme{Bruno}
% \useinnertheme{Bruno}
\useoutertheme{Bruno}

\newrobustcmd{\continue}{%
  \onslide<\value{beamerpauses}->\relax
}


\beamerdefaultoverlayspecification{<+->}

\mode<presentation>

\RequirePackage[math-style=TeX, bold-style=upright]{unicode-math}
\setmathfont{Latin Modern Math}[Scale=1]
\setsansfont[Ligatures=TeX, ItalicFont={Fira Sans Italic}, BoldFont={Fira Sans SemiBold}, Scale=0.9]{Fira Sans Book}
\usepackage[mathcal]{euscript}

% Title page
\setbeamerfont{title}{
    size=\LARGE,
    shape=\bfseries
}

\setbeamerfont{subtitle}{
    size=\large,
    shape=\normalfont
}

\setbeamerfont{author}{
    size=\normalsize,
    shape=\normalfont
}

\setbeamerfont{details}{
    size=\footnotesize,
    shape=\normalfont
}

% Slide title
\setbeamerfont{frametitle}{size=\LARGE}
\setbeamerfont{framesubtitle}{
    size=\normalsize,
    %shape=\normalfont\scshape
}

% Blocks
\setbeamerfont{block title}{
    size=\normalfont,
    shape=\strut
}

\setbeamerfont{blockdef}{
    size=\large,
    shape=\bfseries
}

% Description
\setbeamerfont{description item}{shape=\bfseries}


% Footer information line with title, author and slide number
\setbeamertemplate{footline}{%
    \leavevmode%
    \hbox{%
        \usebeamerfont{footline}%
        \begin{beamercolorbox}[
            wd=\textwidth,
            ht=3ex,
            dp=1.25ex
        ]{footline}%
            \hspace{1cm}%
            \insertshorttitle%
            \hspace{2em}%
            \insertdate%
            \hfill%
            \insertframenumber{} / \inserttotalframenumber%
            \hspace{1cm}
        \end{beamercolorbox}%
    }
    \vskip0pt%
}

%% Bruno Beamer theme — Version 1.0
%% A lightweight Beamer theme inspired from the Metropolis theme
%%
%% Written in 2017-2019 by:
%% — Rémi Cérès <remiceres@msn.com>
%% — Mattéo Delabre <bonjour@matteodelabre.me>
%%
%% This work is released under the CC0 1.0 Universal license. See the
%% accompanying LICENSE file for details. To the extent possible under
%% law, Rémi Cérès and Mattéo Delabre have waived all copyright and
%% related or neighboring rights to the Bruno Beamer theme. This work
%% is published from France.

% \mode<presentation>

\setbeamertemplate{sections/subsections in toc}[square]
\setbeamertemplate{itemize item}[square]
\setbeamertemplate{itemize subitem}[circle]

% Allow multi-slide frames
\setbeamertemplate{frametitle continuation}{}

% Horizontal separator
\setbeamertemplate{separator}{%
    \usebeamercolor{separator}%
    \textcolor{fg}{\rule{.7\textwidth}{.5pt}}%
}

% Separate authors with newlines
\renewcommand{\beamer@andtitle}{\\}

% Title page. If you want to add a background image on the title slide,
% use the \background command to set the path to that image. Otherwise,
% no image will be included
\RequirePackage{tikz}
\usetikzlibrary{fadings}

\newcommand{\background}[1]%
    {\newcommand{\bruno@background}{#1}}


%% \newcommand\insertcaption{}
\newcommand{\backgroundcaption}[2][white]{\def\bruno@caption{{#1}{#2}}}

\newcommand{\insertcaption}{%
    \ifdefined
    \bruno@caption{%
        \expandafter\textcolor\bruno@caption
    }%
    \fi
}


\tikzfading[
    name=title page picture fading,
    left color=transparent!0,
    right color=transparent!100,
]

\setbeamertemplate{title page}{%
    \begin{minipage}{.7\textwidth}
        \raggedright
        \usebeamerfont{title}
        \inserttitle

        \usebeamertemplate{separator}

        \usebeamerfont{author}
        \vspace{2em}
        \insertauthor

        \vspace{2em}
        \usebeamerfont{details}
        \insertinstitute

        \insertdate
    \end{minipage}

    % Include faded image on the right if defined
    \ifdefined\bruno@background
    \begin{tikzpicture}[remember picture, overlay]
        % Crop image to a trapezium on the right
        \clip (current page.south east)
           -- (current page.north east)
           -- ++(-5.1, 0)
           -- ++(-3, -\paperheight)
           -- cycle;

        % Include background image
        \node[
            anchor=south east,
            inner sep=0,
            outer sep=0
        ] at (current page.south east) {
            \includegraphics[height=\paperheight]
                {\bruno@background}
        };

        \node[anchor=south east,
            inner sep=0,
            outer sep=15pt,
        ] at (current page.south east){\insertcaption};

        % Add a slight shadow
        \fill[
            black, path fading=title page picture fading,
            fading angle=-15
        ]
            (current page.south east)
            rectangle
            ++(-10, \dimexpr\paperheight+1cm);

        
    \end{tikzpicture}
    \fi
}

\mode<all>
\AtBeginDocument{%
  \let\phi\varphi
  \let\theta\vartheta
  \let\setminus\smallsetminus
}

\title{Statisztika előadás}
\date{\lecturedate}
\background{images/young-Ronald-Fisher.jpg}
\backgroundcaption{Ronald Fisher (1913)}


\begin{document}

\maketitle

\begin{frame}{Rásonyi Miklós habilitációs előadása}
\vfill
  \begin{center}
    \textbf{Centrális határeloszlás tétel}
    
    \vspace{1cm}

    2021. április 16. (péntek), 10 óra 30 perc
\end{center}
\vfill

\url{https://zoom.us/j/99761822815?pwd=dTdKM3l5M1NpTER6T0JLbW5MNVhiZz09}{}\\
%\begin{verbatim}
Meeting ID: 997 6182 2815\\
Passcode: 286612
%\end{verbatim}
\end{frame} 

\begin{frame}{Hipotézis vizsgálati feladatok, példák}
  \begin{itemize}
    \item A kérdés nem az ismeretlen paraméter értéke hanem az, hogy egy állítás (hipotézis) teljesül-e rá.
    \item Egy gyártósor működése elfogadható, ha a selejt arány $5\%$ alatt van. Ha a gépet beállítják, 
    akkor ez az arány $1\%$, de a használat során a gépsor elállítódik és a selejt arány idővel nő.
    \item Minden nap végén mintát veszünk a termékekből, ez alapján szeretnénk dönteni arról, 
    hogy időszerű-e a gépsor beállítása. 
    \item Itt a hipotézis (nullhipotézis) az, hogy selejt arány az elfogadható szint alatt van.
    \item Egy döntési szabály (statisztikai próba) lehet az, hogy ha a mintában a selejt arány egy \textit{kritikus értéket} 
    meghalad, akkor a gép beállítása mellett döntünk, ellenkező esetben folyhat tovább a termelés.
    \item Mekkora legyen a kritikus érték?
    
    Ha a kritikus érték túl nagy, akkor a selejt arány elfogadhatatlanul magas lesz, ha túl kicsi akkor 
    feleslegesen áll a termelés a gép beállítása alatt.
  \end{itemize}  
\end{frame}


\begin{frame}{Hipotézis vizsgálati feladat, fogalmak}
    \begin{itemize}
    \item \textbf{Statisztikai hipotézis}. A minta eloszlásának valamely
      jellemzőjére vonatkozó kijelentés.
      
      $(\Omega,\cA,\cP)$ statisztikai mező. $\cP=\cP_0\cup\cP_1$,
      ($\cP_0\cap\cP_1=\emptyset$).
      
      $H_0$ \textbf{nullhipotézis}: a valódi eloszlás $\in\cP_0$,
      
      $H_1$ \textbf{ellenhipotézis}: a valódi eloszlás $\in\cP_1$.
    
   
    \item Ha $\cP=\set{\P[\theta]}{\theta\in\Theta}$ paraméteres család,
      akkor $\cP=\cP_0\cup^*\cP_1$ megfelel egy
      $\Theta=\Theta_0\cup^*\Theta_1$ felbontásnak. 
      $H_0:\,\theta\in\Theta_0$, $H_1:\,\theta\in\Theta_1$.
    
    \item A bevezető példában indikátor mintánk volt $\Omega=\mfX=\smallset{0,1}^n$. 
      $\theta\in\Theta=(0,1)$ az ismeretlen selejt arány. 
      
      $H_0: \theta\leq 0.05$, $H_1: \theta>0.05$, $\Theta_0=(0,0.05]$ és $\Theta_1=(0,05,1)$.
    \item A $H_0$, ill. $H_1$ hipotézis egyszerű, ha $\cP_0$, ill.
      $\cP_1$ egy elemű.
    \item Egy hipotézis összetett, ha nem egyszerű, azaz a $\cP_0$
      vagy $\cP_1$ több elemű.
    \item \textbf{Statisztikai próba}. Döntés a minta alapján
      $H_0$-ról. Azaz $\mfX=\mfX_0\cup^*\mfX_1$. Ha
      a megfigyelés $X\in\mfX_0$, akkor elfogadjuk a nullhipotézist, ellenkező
      esetben elutasítjuk.
  
      $\mfX_0$ neve \textbf{elfogadási tartomány}
  
      $\mfX_1$ neve elutasítási, vagy \textbf{kritikus tartomány}
    \item A bevezető példában ha $c$ a kritikus érték, akkor $\mfX_1=\set{x\in\mfX}{\bar{x}>c}$.
    \end{itemize}
    
  \end{frame}
  
  
  \begin{frame}{Hiba lehetőségek}
   
    \begin{itemize}
    \item 
    
    \textbf{Elsőfajú hiba}. A nullhipotézis igaz, de elvetjük
      (téves elutasítás)
    
      \textbf{Másodfajú hiba}. A nullhipotézis nem igaz, de mégis elfogadjuk
      (téves elfogadás)
    
    \item A bevezető példában: 
    
    $H_0: \theta\leq0.05$, azaz elsőfajú hibát követünk el, ha a gép beállítása még nem időszerű
    de a mintában a selejt arány a kritikus érték felett van.

    $H_1: \theta>0.05$, azaz másodfajú hibát követünk el, ha a selejt arány a mintában kicsi, pedig a gép beállítása időszerű.

    \item \textbf{Fix mintaelemszám esetén az egyik fajta hiba valószínűségét
      csak a másik növelése árán lehet csökkenteni.}
      
    \item Feladat. Az elsőfajú hiba valószínűsége legyen kicsi (tipikusan
      nem nagyobb, mint 5\%) és ezen feltétel mellett a másodfajú hiba valószínűsége
      legyen minél kisebb.

    \item Jelentés: A nullhipotézist, ha lehet elfogadjuk. Csak akkor
      utasítjuk el, ha az adatok ennek ,,ellentmondanak''.
  
      $H_0$ elfogadása csak annyit jelent, hogy az adatok alapján lehetséges
      hogy  $H_0$ igaz.
  
      $H_0$ elutasítása az \textbf{informatív döntés}, ez ugyanis annyit jelent,
      hogy a megfigyelések $H_0$ ellen szólnak. %, az valószínűleg nem igaz.
    \end{itemize}
  
    \continue
    Példák.
    \begin{itemize}
    \item Gyógyszerkutatásban: $H_0:$ az új hatóanyag hatástalan, $H_1:$
      az új hatóanyag hatásos. 
    \item Növénynemesítés. $H_0:$ az új változat termésátlaga nem nagyobb
      mint a most használt fajtáké. $H_1:$ az új változat termésátlaga nagyobb.
    \end{itemize}
    
  \end{frame}
  
  \begin{frame}{Véletlenített próba, próbafüggvény, terjedelem, szint, erő}
    \begin{itemize}
    \item Egy statisztikai próba leírható az elutasítási tartomány
      $\mfX_1$ indikátorával. $\phi (x)=\I{x\in \mfX_1}$.
  
      Ekkor $\phi (x)$ az elutasítás valószínűsége, ha a megfigyelés
      $x$. $\phi$ neve \textbf{próbafüggvény} vagy \textbf{döntési függvény}.
      
    \item Véletlenített próba. $\phi:\mfX\to[0,1]$.
  
      Ha $\phi:\mfX\to\smallset{0,1}$, akkor a próba nem véletlenített.
      
    \item A $\phi$ próba \textbf{terjedelme} az elsőfajú hiba elkövetésének
      maximális valószínűsége
      \begin{displaymath}
        \alpha (\phi) =\sup_{\theta\in\Theta_0}\E[\theta]{\phi}
      \end{displaymath}
    \item A $\phi$ próba \textbf{ereje} annak a valószínűsége, hogy $H_0$ nem
      teljesül és ezt felismerjük. Ez egy $\Theta_1$-n értelmezett
      $\psi$ függvény, az \textbf{erőfüggvény}
      \begin{displaymath}
        \psi (\theta)=\psi (\theta,\phi)=\E[\theta]{\phi}, \quad \theta\in\Theta_1
      \end{displaymath}
    \item Az azonos terjedelmű próbákat az erőfüggvény alapján hasonlíthatjuk össze. 
    \end{itemize}
    \continue
    Megjegyzés. Gyakran a legfeljebb $\alpha$ terjedelmű próbát is
    $\alpha$ terjedelműnek mondják.
  
    Ha a próba terjedelme $\alpha$, akkor a szintje $1-\alpha$.
  \end{frame}
  
  \begin{frame}{Statisztikai próbák jellemzői}
  
    $\phi:\mfX\to[0,1]$ próba függvény. $\phi (x)$ a
    $H_0$ %nullhipotézis
    elutasításának valószínűsége, ha a megfigyelés $x$. 
    \begin{itemize}
    \item $\phi$ \textbf{torzítatlan}, ha ereje nem kisebb a
      terjedelménél:
      \begin{displaymath}
        \sup_{\theta\in\Theta_0} \E[\theta] {\phi (X)}\leq \inf_{\theta\in\Theta_1}\E[\theta]{\phi (X)}
      \end{displaymath}
    \item $\phi$ \textbf{egyenletesen legerősebb} próba (e.l.p.), ha
      ereje legalább akkora, mint tetszőleges vele azonos terjedelmű
      próba ereje.
      \begin{displaymath}
        \psi (\theta,\phi)=\E[\theta]{\phi}=\sup\set{\psi
          (\theta,\phi')}{\alpha (\phi')=\alpha (\phi)} \quad\forall\theta\in\Theta_1
      \end{displaymath}
      $\alpha (\phi)=\sup_{\theta\in\Theta_0}\E[\theta]{\phi}$.
  
      Egyenletesen legerősebb próba nem feltétlenül létezik.
  
      Ha $\phi$ egyenletesen legerősebb, akkor torzítatlan is, mert ereje
      legalább akkora, mint a %az $\alpha (\phi)$ terjedelmű
      $\phi'(x)=\alpha (\phi)$ próba ereje.
    \item $\phi$ \textbf{legszigorúbb}, ''ha legjobban közelíti az
      erőfüggvény burkolóját'', azaz $\phi$ minimalizálja
      \begin{displaymath}
        \sup_{\theta\in\Theta_1}(\bpsi (\theta,\alpha)-\psi(\theta,\phi))
        \quad\text{értékét, ahol}\quad
        \bpsi (\theta,\alpha)=\sup\set{\psi (\theta,\phi)}{\alpha
          (\phi)=\alpha},\quad\theta\in\Theta_1
        % \quad
        %\text{\phi minimalizálja}&\quad
        %=\inf\set{\psi (\theta,\alpha)-\psi
        %(\theta,\phi_1)}{\alpha (\phi_1)=\alpha (\phi)}
      \end{displaymath}
      
    \item  $\phi_n$ $n$-elemű mintára épülő $\alpha$ terjedelmű próba.
      $(\phi_n)$ \textbf{konzisztens}, ha $\psi (\theta,\phi_n)\to1$
      %minden
      $\forall\theta\in\Theta_1$-re.
    \end{itemize}
  \end{frame}
  
  
  \begin{frame}{Próbák konstruálása}
    \begin{itemize}
    \item Látni fogjuk, hogy a legtöbb próba $\phi (x)=\I{T (x)>c}$
      alakú. 
    \item $T$ olyan statisztika, ami $\theta\in\Theta_1$ mellett
      tipikusan nagy értékekre vezet, míg $\theta\in\Theta_0$ esetén
      nem.
    \item $T$ neve \textbf{próba statisztika}, $c$ neve \textbf{kritikus
        érték}
    \item Ha $S$ elégséges statisztika, akkor $\phi'
      =\E[\theta]{\phi|S}$ nem függ $\theta$-tól, terjedelme, ereje
      azonos $\phi$-vel. Azaz $\phi$-t elég az elégséges statisztika
      függvényei között keresni.
    \item Ha determinisztikus próbával a megengedett $\alpha$
      terjedelem nem érhető el, akkor véletlenítéssel lehet javítani a
      próbán.
  
      Azaz
      \begin{displaymath}
        \alpha'=\sup_{\theta\in\Theta0}\P[\theta]{T>c}<\alpha\quad
        \sup_{\theta\in\Theta0}\P[\theta]{T\geq c}\geq\alpha
      \end{displaymath}
      akkor
      \begin{displaymath}
        \phi (x)=
        \begin{cases}
          1 &T (x)>c\\
          p &T (x)=c\\
          0 &T (x)<c
        \end{cases}
      \end{displaymath}
      $p\sup_{\theta\in\Theta_0}\P[\theta]{T=c}=\alpha-\alpha'$.
    \end{itemize}    
  \end{frame}
  
  \begin{frame}{Neyman--Pearson lemma I.}
    
    \begin{lemma}
      Egyszerű hipotézis vizsgálati feladatban a likelihood hányados próba
      a legerősebb a vele azonos terjedelmű próbák között.
      
      Azaz, ha $c>0$, $g:\mfX\to [0,1]$, és
      \begin{displaymath}
        \phi (x)=
        \begin{cases}
          1 & f_1 (x)> cf_0 (x)\\
          g(x) & f_1 (x)=cf_0 (x)\\
          0 & f_1 (x)<cf_0 (x)
        \end{cases}
      \end{displaymath}
      akkor tetszőleges $\phi'$ próbafüggvényre
      \begin{displaymath}
        \E[0]{\phi'}\leq \E[0]{\phi}\quad\implies\quad
        \E[1]{\phi'}\leq \E[1]{\phi}
      \end{displaymath}
    \end{lemma}
    \continue
    \begin{displaymath}
      \E[1]{\phi-\phi'}=\int_{\mfX}
      (\phi-\phi')f_1d\lambda\geq
      \int_{\mfX}
      (\phi-\phi')cf_0d\lambda=c\E[0]{\phi-\phi'}\geq0
    \end{displaymath}
    Ugyanis $(\phi-\phi')(f_1-cf_0)\geq 0$. Ez eset szétválasztással látható.
  \end{frame}
  
  \begin{frame}{Neyman--Pearson lemma II.}
    
    \begin{lemma}[Egzisztencia]
      Egyszerű hipotézis vizsgálati feladatban tetszőleges $\alpha\in
      (0,1)$-re létezik $\alpha$ terjedelmű  likelihood hányados próba.
    \end{lemma}
    Megjegyzés. Ez a legerősebb a legfeljebb $\alpha$ terjedelmű próbák között.
    \continue
      
    $c>0$, $p\in [0,1]$, megválasztásával elérhető, hogy a
    \begin{displaymath}
        \phi (x)=
        \begin{cases} 
          1 & f_1 (x)> cf_0 (x)\\
          p & f_1 (x)=cf_0 (x)\\
          0 & f_1 (x)<cf_0 (x)
        \end{cases}
      \end{displaymath}
      próba terjedelme pontosan $\alpha$ legyen. 

      Ha $c=\inf\set{c>0}{\P[0]{f_1(X)>cf_0(X)}<\alpha}$, akkor 
      \begin{displaymath}
        \P[0]{f_1(X)\alert{>}cf_0(X)}\leq \alpha \leq \P[0]{f_1(X)\alert{\geq} cf_0(X)}. 
      \end{displaymath}
      és így létezik $p\in[0,1]$, hogy
      \begin{displaymath}
        \alpha = (1-p)\P[0]{f_1(X)\alert{>}cf_0(X)}+p\P[0]{f_1(X)\alert{\geq}cf_0(X)}=\E[0]{\phi}.
      \end{displaymath} 
      % $\P{\P[0]{f_1(X)\geq cf_0(X)}\geq\alpha$
      % \begin{displaymath}
      %   \E[0]{\phi}=\P[0]{f_1(X)>cf_0(X)}+p\P{f_1(X)=cf_0(X)}
      % \end{displaymath}
      % Itt $c=\inf\set{c}{\P[0]{f_1(X)>cf_0(X)}<\alpha}$. Ekkor $\P{\P[0]{f_1(X)\geq cf_0(X)}\geq\alpha$.

  \end{frame}
  
  
  \begin{frame}{Neyman--Pearson lemma III.}
    
    \begin{lemma}[Unicitás]
      Egyszerű hipotézis vizsgálati feladatban, ha  $\phi_1$ legerősebb
      próba a legfeljebb $\alpha$ terjedelmű próbák között, akkor $\phi_1$
      likelihood hányados próba és vagy $\alpha (\phi_1)=\alpha$ vagy $\psi (\phi_1)=1$.
    \end{lemma}
    \begin{itemize}
    \item Legyen $\phi$ a pontosan $\alpha$ terjedelmű (esetleg
      véletlenített) likelihood hányados próba ($c$ a kritikus
      értéke). Ekkor
      \begin{displaymath}
        0\geq \E[1]{\phi-\phi_1}-c\E[0]{\phi-\phi_1}=\int (f_1-cf_0)
        (\phi-\phi_1)d\lambda\geq 0
      \end{displaymath}
  
     
    \item Azaz $(f_1-cf_0) (\phi-\phi_1)=0$ $\lambda$ m.m., Vagyis $\phi_1$
      is likelihood hányados próba.
      
    \item Ha $\alpha (\phi_1)<\alpha$, és $\psi (\phi)<1$, akkor véletlenítéssel növelhető
      lenne az ereje. Mivel $\phi_1$ a legerősebb próba, ez nem
      lehetséges, tehát vagy $\alpha (\phi_1)=\alpha$ vagy $\psi (\phi_1)=1$. 
    \end{itemize}  
  \end{frame}

  \begin{frame}{Példa}
    \begin{itemize}
      \item $X_1,\dots,X_n$ indikátor minta. $H_0: \theta=\theta_0$, $H_1:\theta=\theta_1$, ahol $\theta_0<\theta_1$.
      \item Ekkor $f_i(x)=\theta_i^{\sum x_k}(1-\theta_i)^{\sum(1-x_k)}$ és
      \begin{displaymath}
        \frac{f_1(x)}{f_0(x)}=\zfrac*{1-\theta_0}{1-\theta_1}^n\cdot\zjel*{\frac{\theta_1}{1-\theta_1}:\frac{\theta_0}{1-\theta_0}}^{\sum x_i}.
      \end{displaymath}
      \item A likelihood hányados a $\sum X_i$ statisztika monoton növő függvénye, 
      azaz a likelihood hányados próba döntési függvénye:
      \begin{displaymath}
        \phi(x)=\I{\sum x_k>c}+p\I{\sum x_k=c},
      \end{displaymath}
      ahol $c$ és $p$ értékét a $\E[0]{\phi(X)}=\alpha$ összefüggés határozza meg.

      \item Vegyük észre, hogy  a döntési függvény nem függ $\theta_1$ konkrét értékétől, 
      csak a $\theta_1>\theta_0$ összefüggést használtuk. A kapott próba egyenletesen 
      legerősebb a $H_1: \theta>\theta_0$ ellenhipotézissel szemben is.
    \end{itemize}    
  \end{frame}

  \begin{frame}{Példa folytatása, aszimptotikus viselkedés}
    \begin{itemize}
      \item $X_1,\dots,X_n$ indikátor minta. $H_0: \theta=\theta_0$, $H_1:\theta=\theta_1$, ahol $\theta_0<\theta_1$.
      \item A likelihood hányados a $\sum X_i$ statisztika monoton növő függvénye, 
      azaz a likelihood hányados próba döntési függvénye:
      \begin{displaymath}
        \phi(x)=\I{\sum x_k>c}+p\I{\sum x_k=c},\quad
        \text{ahol $c$ és $p$ értékét a $\E[0]{\phi(X)}=\alpha$ összefüggés adja.}
      \end{displaymath}
      %ahol $c$ és $p$ értékét a $\E[0]{\phi(X)}=\alpha$ összefüggés határozza meg.
      \item Ha $n$ nagy, akkor $n^{-1/2}\sum_i (X_i-\theta_0)$ közelítőleg $N(0,\theta_0(1-\theta_0))$ eloszlású 
      és a kritikus érték ez alapján választható. $c_n=n\theta_0+\sqrt{n\theta_0(1-\theta_0)}\Phi^{-1}(1-\alpha)$,
      nagy $n$-re $p=0$ értékét vehetjük nullának.
      % \item A próba ereje:
      % \begin{displaymath}
      %   \psi(\theta_1)=
      %   \P[1]{\sum X_i>c_n}
      %   %=\P[1]{\frac{\sum(X_i-\theta_1)}{\sqrt{n(\theta_1(1-\theta_1))}}>\frac{c_n-n\theta_1}{\sqrt{n\theta_1(1-\theta_1)}}}
      %   %=\P{Z> -a\sqrt{n}+b}=\P{-Z<a\sqrt{n}-b},
      %   %\quad\text{ahol $Z\sim N(0,1)$}
      % \end{displaymath}
      %$a=-(\theta_0-\theta_1)>0$, $b=\sqrt{\frac{\theta_0(1-\theta_0)}{\theta_1(1-\theta_1)}}\Phi^{-1}(1-\alpha)>0$.
      \item Emlékeztető
      \begin{proposition}<.->[Hoeffding egyenlőtlenség]
        $Z_1,\dots,Z_n$ független változók, $0\leq Z_i\leq 1$, $\eps>0$. 
        Ekkor $\P{\sum Z_i> \sum \E{Z_i}+n\eps}\leq e^{-2n\eps^2}$
      \end{proposition}
      $(1-Z)$-re áttérve: $\P{\sum Z_i<\sum \E{Z_i}-n\eps}\leq e^{-2n\eps^2}$.
      \begin{displaymath}
        1-\psi(\theta_1)
        =\P[1]{\sum_{i=1}^n X_i<c_n}
        %=\P[1]{\sum_{i=1}^n (1-X_i)>n-c_n}
        \leq \exp{-2n\zfrac*{n\theta_1-c_n}{n}^2}
        \leq e^{-2n(\theta_1-\theta_0)^2+a\sqrt{n}+b}
      \end{displaymath}
      Azaz az erő geometriai sebességgel tart egyhez, 
      ha minta elemszám végtelenhez tart.
    \end{itemize} 
  \end{frame}
  
  \begin{frame}{Általánosítás: kritikus érték aszimptotikája}
    \begin{itemize}
    \item Egyszerű hipotézis vizsgálati feladat. $f_1,f_0$ egy mintaelem
      sűrűségfüggvénye. Tegyük fel, hogy $f_1,f_0>0$.
    \item %Jelölés
      \begin{displaymath}
        \xi_i=-\log  \frac{f_1(X_i)}{f_0 (X_i)},\quad\text{ekkor}\quad
        \E[0]{\xi_i}=\int -\log\frac{f_1}{f_0}f_0d\lambda= D(\P[0]\|\P[1]).
      \end{displaymath}
      Belátható, hogy ha $f_1\neq f_0$, akkor $0<\E[0]{\xi_i}$.
      Tegyük fel, hogy $\E[0]{\xi_i}<\infty$ és $\sigma^2=\D[0]^2{\xi}<\infty$.
    \item A likelihood hányados próba $\sum_i\xi_i$-vel is felírható.
      \begin{displaymath}
        T_n=\prod_{i=1}^n\frac{f_1 (X_i)}{f_0 (X_i)}>c_n\quad\iff\quad
        \sum\frac{\xi_i- D(\P[0]\|\P[1])}{\sqrt{n}\sigma}<t_n,\quad
        c_n=\exp{-n D(\P[0]\|\P[1])-\sqrt{n}\sigma t_n}.
      \end{displaymath}
    \item Terjedelem $\alpha$. A CHT alapján
      \begin{displaymath}
        \sum_{i=1}^n\frac{\xi_i-D(\P[0]\|\P[1])}{\sqrt{n}\sigma}\dto N (0,1)
        \quad\text{ha $H_0$ igaz, és}\quad 
        \P[0]{\sum_{i=1}^n\frac{\xi_i-D(\P[0]\|\P[1])}{\sqrt{n}\sigma}<t_n}\to\alpha.
        %\implies\quad
        %t_n\to\Phi^{-1} (\alpha)
      \end{displaymath}
      Belátjuk, hogy ebből $t_n\to \Phi^{-1}(\alpha)$ következik.
    \item Ha
      \begin{displaymath}
        \hat c_n=\exp{-n D(\P[0]\|\P[1])-\sqrt{n}\sigma \Phi^{-1}
          (\alpha)},\quad\hat\phi_n (x)=\I{T_n>\hat c_n},\quad
        \text{akkor}\quad\alpha (\hat\phi_n)\to\alpha
      \end{displaymath}      
    \end{itemize}
\end{frame}

\begin{frame}{Kullback-Leibler divergencia}
  \begin{df}
    $\P[0]$, $\P[1]$ valószínűségi mértékek az $(\Omega,\B)$ mérhető téren. 
    \begin{displaymath}
      D(\P[0]\|\P[1]) = \begin{cases}
        \E[0]{-\log\frac{d\P[1]}{d\P[0]}}& \text{ha $\P[1]\ll\P[0]$}\\
        \infty&\text{különben}
      \end{cases}
    \end{displaymath}
  \end{df}
  \begin{itemize}
    \item $D(\P[0]\|\P[1])=\int -\log\frac{f_1}{f_0} f_0$, ha $f_1,f_0$ a két sűrűségfüggvény.
    Ez a formula nem függ a domináló mérték választásától.
    \item $-\log$ konvex, ezért $\int-\log\zfrac*{f_1}{f_0} f_0\geq -\log\zjel*{\int f_1}=0$.
    \item $-\log$ szigorúan konvex, így egyenlőség csak akkor lehet, ha $-\log \zfrac*{f_1}{f_0}=0$ $\P[0]$ mm., vagyis 
    $f_1=f_0$.
  \end{itemize}    
  \end{frame}

  \begin{frame}{Két apróság}
    \begin{proposition}
      $\Phi$ $N (0,1)$ eloszlásfüggvénye, $(F_n)$ eloszlásfüggvények
      sorozata és  $F_n (t)\to \Phi (t)$ minden $t$-re.  Ekkor
      \begin{enumerate}[<*>]
      \item $\sup\abs*{F_n-\Phi}\to0$ 
      \item $F_n (t_n)\to \Phi (t)$ pontosan akkor ha $t_n\to t$.
      \end{enumerate}
    \end{proposition}
    \begin{itemize}
    \item (1)-et láttuk a  Glivenko--Cantelli
      tétel kapcsán. $F_n,\Phi$ monoton, ha $t_i=\Phi^{-1}(\frac ir)$, $i=1,\dots,r-1$, akkor
      $F_n(t_i)\to\Phi(t_i)=\frac ir$ és %tetszőleges $t$-re 
      $\limsup_{n\to\infty}\sup \abs{F_n-\Phi}
      \leq\lim_{n\to\infty}\max_i \abs{F_n(t_i)-\Phi(t_i)}+\frac1r
      \leq \frac{1}{r}$.
    \item (2) Ha $t_n\to t$ ($t=\pm\infty$ is lehet)
      \begin{displaymath}
        \abs*{F_n (t_n)-\Phi (t)}\leq \abs*{F_n (t_n)-\Phi (t_n)}+\abs*{\Phi(t_n)-\Phi (t)}
        \leq \sup\abs*{F_n-\Phi}+\abs*{\Phi(t_n)-\Phi (t)}\to0
      \end{displaymath}
      
    \item (2) Ha $t_n\not\to t$, akkor létezik $(t'_n)\subset (t_n)$
      részsorozat, aminek létezik limesze $t'$ és az nem $t$ ($t'_n\to\pm\infty$ lehetséges)
      \begin{displaymath}
        F_n (t'_n)\to\Phi (t')\neq\Phi (t)
      \end{displaymath}
      azaz $F_n (t_n)\not\to\Phi (t)$.
    \end{itemize}
    \continue
    Következmény. Ha $t_n=\sup\set{t}{F_n (t)<\alpha}$, akkor $t_n\to\Phi^{-1} (\alpha)$.
  \end{frame}

  \begin{frame}{Erő aszimptotikája}
    \begin{itemize}
    \item Egyszerű hipotézis vizsgálati feladat. $f_1,f_0$ egy mintaelem
      sűrűségfüggvénye. Tegyük fel, hogy $f_1,f_0>0$.
    \item Az $\alpha$ terjedelmű likelihood hányados próba
    \begin{displaymath}
      \phi_n(x)=\I{T_n(x)>c_n}+p_n\I{T_n(x)=c_n},\quad\text{ahol}\quad
      T_n(x)=\prod_{i=1}^n \frac{f_1(x_i)}{f_0(x_i)},\quad
      c_n=e^{-n D(\P[0]\|\P[1])-\sqrt{n}\sigma t_n},\quad t_n\to\Phi^{-1}(\alpha)
    \end{displaymath}
    % \item %Jelölés
    %   \begin{displaymath}
    %     \xi_i=-\log  \frac{f_1(X_i)}{f_0 (X_i)},\quad\text{ekkor}\quad
    %     \E[0]{\xi_i}=\int \log\frac{f_0}{f_1}f_0d\lambda= D(\P[0]\|\P[1])>0            
    %   \end{displaymath}
    %   Tegyük fel, hogy $\E[0]{\xi_i}<\infty$ és $\sigma^2=\D[0]^2{\xi}<\infty$.
    \item $\psi_n$ az erő. A Markov egyenlőtlenségből
      \begin{displaymath}
        1-\psi_n\leq
        \P[1]{\prod_{i=1}^n\frac{f_1 (X_i)}{f_0 (X_i)}\leq c_n}
        =\P[1]{\prod_{i=1}^n\frac{f_0 (X_i)}{f_1 (X_i)}\geq \frac1{c_n}}
        \leq c_n\E[1]{ \prod_{i=1}^n\frac{f_0 (X_i)}{f_1 (X_i)}}=c_n
      \end{displaymath}
      
    \item
      \begin{displaymath}
        1-\psi_n\leq \exp{-n D (\P[0]\|\P[1])+O(\sqrt{n})}
      \end{displaymath}
      Az erő geometria sebességgel tart $1$-hez, ha $n\to\infty$.
    \end{itemize}
  \end{frame}
\end{document}

  \begin{frame}{Klasszikus próbák. $u$-próba}
    \begin{itemize}
    \item $X_1,\dots,X_n$ minta $N (\mu,\sigma^2)$ eloszlásból, $\sigma$
      ismert, $\mu$ ismeretlen.
    \item Esetek:
  
      $H_0:\mu=\mu_0$, $H_1:\mu<\mu_0$ egyoldali ellenhipotézis
  
      $H_0:\mu=\mu_0$, $H_1:\mu>\mu_0$ egyoldali ellenhipotézis
      
      $H_0:\mu=\mu_0$, $H_1:\mu\neq\mu_0$ kétoldali ellenhipotézis
   
    \item Próba statisztika:
      \begin{displaymath}
        T (X)=\sqrt{n}\frac{\bar{X}-\mu_0}{\sigma}
      \end{displaymath}
    \item $T$ eloszlása $H_0$ mellett $N (0,1)$
    \item $u_\alpha=\Phi^{-1} (1-\alpha)$, azaz
      $\P{Z>u_\alpha}=\alpha$, ha $Z\sim N (0,1)$
    \item $\alpha$ terjedelmű próba. Kritikus tartomány:
  
      $H_1:\mu<\mu_0$ esetén $T<-u_\alpha$
  
      $H_1:\mu>\mu_0$ esetén $T>u_\alpha$
  
      $H_1:\mu\neq \mu_0$ esetén $\abs*{T}>u_{\alpha/2}$
  
    \end{itemize}
  \end{frame}
  
  
  \begin{frame}{Emlékeztető. Neyman-Pearson lemma}
    \begin{itemize}
    \item \textbf{Egyszerű} a hipotézis vizsgálati feladat, ha
      $\Theta_0,\Theta_1$ egyelemű. Ekkor $H_0:f=f_0$, $H_1:f=f_1$ alakú
      a feladat, ahol $f_0$, $f_1$ a két eloszlás $\P[0]$, $\P[1]$ sűrűségfüggvénye
      alkalmas domináló mértékre  nézve (pl. $\lambda=\frac12 (\P[0]+\P[1])$)).
    \item \textbf{Likelihood hányados} statisztika
      \begin{displaymath}
        T (x)=\frac{f_1 (x)}{f_0 (x)}
      \end{displaymath}
    \end{itemize}
    \begin{lemma}
      Egyszerű hipotézis vizsgálati feladat, $f_0>0$. A likelihood hányados $T=f_1/f_0$
      folytonos eloszlású $H_0$ és $H_1$ mellett.
  
      Ekkor
      \begin{itemize}[<*>]
      \item Minden $\alpha\in(0,1)$-re létezik $c_\alpha$ úgy,
        hogy $\phi=\I{T>c_\alpha}$ terjedelme $\alpha$. (Azaz $\P[0]{T>c_\alpha}=\alpha$)
      \item $\phi=\I{T>c_\alpha}$ a legerősebb $\alpha$ terjedelmű próba.
      \item Ha $\alpha (\bar\phi)\leq \alpha$ és
        $\E[1]{\bar{\phi}}=\E[1]{\phi}$, akkor $\bar{\phi}=\phi$
        ($\P[0]$, $\P[1]$ majdnem mindenütt).
      \end{itemize}
    \end{lemma}
  
  \end{frame}
  
  
  \begin{frame}{Egyoldali ellenhipotézis mellett, az $u$-próba
      egyenletesen legerősebb I.}
    A sűrűségfüggvény
    \begin{displaymath}
      f_\mu (x)=\zfrac*1{2\pi\sigma^2}^{n/2}\exp{-\frac1{2\sigma^2}\sum_i
        (x_i-\mu)^2}=h (x)\exp{\mu T (x)-b (\mu)}
    \end{displaymath}
    ahol
    \begin{displaymath}
      h (x)=\zfrac*1{2\pi\sigma^2}^{n/2}e^{-\frac1{2\sigma^2}\sum_i x_i^2},\quad
      b(\mu)=n\frac{\mu^2}{2\sigma^2},\quad T (x)=\frac{\sum_i x_i}{\sigma^2}
    \end{displaymath}
    \begin{itemize}
    \item Elegendő a $\mu_0=0$ esetet nézni.
    \item Egyszerű hipotézis vizsgálati feladat ($0<\mu_1$), $H_0:\mu=0$,
      $H_1:\mu=\mu_1$.
    \item A likelihood hányados %próba próba 
    statisztika %ája
      \begin{displaymath}
        \frac{f_1}{f_0} (x)=\exp{(\mu_1-0)T (x)-(b (\mu_1)-b
          (\mu_0))}\quad\text{$T$ szigorúan monoton növő függvénye}
      \end{displaymath}
    \item $T$ eloszlása folytonos, azaz a legerősebb próba nem
      randomizált és
      \begin{displaymath}
        \mfX_1=\event{T>c}\quad\text{alakú, ahol}\quad \P[0]{T>c}=\alpha
      \end{displaymath}
    \end{itemize}
  \end{frame}
  
  \begin{frame}{Egyoldali ellenhipotézis mellett, az $u$-próba
      egyenletesen legerősebb II.}
    \begin{itemize}
    \item<*> Egyszerű hipotézis vizsgálati feladat ($0<\mu_1$), $H_0:\mu=0$,
      $H_1:\mu=\mu_1$.
    \item<*> $\alpha$ terjedelmű likelihood hányados próba
      \begin{displaymath}
        \mfX_1=\event{T>c}
        \quad\text{alakú, ahol}\quad
        T=\frac{1}{\sigma^2}\sum_i X_i,\quad \P[0]{T>c}=\alpha
      \end{displaymath}
      
    \item A likelihood hányados próba nem függ
      $\mu_1\in(0,\infty)$-től.
    \item A fenti próba egyenletesen legerősebb a $H_1:\mu>0$
      ellenhipotézisre is.
  
      Ugyanis, ha a $\phi$ próbafüggvénnyel megadott próba terjedelme
      $\alpha$ és valamely $\mu_1>0$ pontban az ereje nem kisebb, mint a
      fenti próba ereje, akkor a Neyman--Pearson lemma miatt $\phi$ is
      likelihood hányados próba, vagyis azonos a 
      fenti próbával.
    \item Az elutasítási tartomány
      \begin{displaymath}
        \event{\sum_{i} X_i>\sigma^2 c}=
        \event{T(X)>u_\alpha},
        \quad\text{ahol}\quad
        T (X)=\sqrt{n}\frac{\bar{X}}{\sigma},\quad \P[0]{T>u_\alpha}=\alpha.
      \end{displaymath}
    \end{itemize}
  \end{frame}
  
  \begin{frame}[<*>]{$H_1:\mu\neq0$-ra nincs egyenletesen legerősebb próba}
  
    \begin{itemize}
    \item $H_0:\mu=0$, $H_1^*:\mu=\mu_1$ feladatra, a likelihood hányados
      próba a legerősebb. 
    \item Mivel a likelihood hányados eloszlása $H_0$ mellett folytonos
      a likelihood hányados próba nem randomizált, így egyértelmű.
    \item Ha $\phi$ egyenletesen legerősebb lenne, akkor minden
      $\mu_1\neq0$-ra egybeesne a $H^*$ feladathoz tartozó likelihood
      hányados próbával.
    \item $\mu_1>0$ esetén $\phi (x)=\I{T (x) >u_\alpha}$
      
    \item $\mu_1<0$ esetén $\phi (x)=\I{T (x) <-u_\alpha}$
      
    \item Ezek egyszerre nem teljesülhetnek, nincs ilyen $\phi$ és nincs
      egyenletesen legerősebb próba.
    \end{itemize}
  \end{frame}
  
  \begin{frame}{$H_1:\mu\neq0$-ra, az $u$-próba egyenletesen
      legerősebb \textbf{torzítatlan} próba I.}
    \begin{itemize}
    \item Emlékeztető. A $\phi$ próba  \textbf{torzítatlan}, ha
      \begin{displaymath}
        \psi (\theta)=\E[\theta]{\phi}\quad \text{$\Theta_0$-n nem
          nagyobb mint $\Theta_1$-n}
      \end{displaymath}
      
    \item Egy elemű $\Theta_0=\smallset{\theta_0}$ esetén $\phi$
      torzítatlan, ha
      \begin{displaymath}
        \psi (\theta_0)=\min_{\theta}\psi (\theta)
      \end{displaymath}
    \item Tegyük fel, hogy $\Theta$ konvex, nyílt és (R) teljesül.
      % , pl. $\P[\theta]=N (\mu,\sigma^2)$, $\mu\in\real=\Theta$.
      Ekkor
      \begin{displaymath}
        \partial_\theta\psi (\theta)=\E[\theta]{\phi\partial\ell(\theta)},
        \quad 0=\E[\theta]{\partial\ell(\theta)}
        \quad\text{és}\quad
        \E[\theta_0]{\phi\partial\ell (\theta_0)}=0,
        \quad
        \text{azaz}
        \quad
        \cov_{\theta_0} (\phi,\partial\ell (\theta_0))=0.
      \end{displaymath}
    \end{itemize}  
  \end{frame}
  
  \begin{frame}{$H_1:\mu\neq0$-ra, az $u$-próba egyenletesen
      legerősebb torzítatlan próba II.}
    \begin{itemize}[<*>]
    \item $\Theta$ konvex, nyílt és (R) teljesül,
      $\Theta_0=\smallset{\theta_0}$, $\phi$ torzítatlan.
      % , pl. $\P[\theta]=N (\mu,\sigma^2)$, $\mu\in\real=\Theta$.
      Ekkor
      \begin{displaymath}
        \cov_{\theta_0} (\phi,\partial\ell (\theta_0))=0.
      \end{displaymath}
    \item $\set{N (\mu,\sigma^2)}{\mu\in\real}$, $\sigma^2>0$ rögzített,
      esetén
      \begin{displaymath}
        \ell (\mu,x)=\ln h (x)+\mu T (x) - b (\mu),\quad
        \text{ahol}\quad
        T (x)=\frac{\sum_i x_i}{\sigma^2},\quad b (\theta)=\frac{n\mu^2}{2\sigma^2}
      \end{displaymath}
    \item $\partial\ell (\mu,x)=\frac{n}{\sigma^2}(\bar{X}-\mu)$
    \end{itemize}  
    \begin{proposition}
      \begin{displaymath}
        \phi=\I{\abs*{\partial\ell (0)}>c}, \quad
        \alpha=\alpha (\phi),
        \quad 
        \Phi=\set{\bar{\phi}}{\E[0]{\bar{\phi}}=\alpha,\,
        \cov_{0} (\bar{\phi},\partial\ell (0))=0}
      \end{displaymath}
      Ekkor $\phi$ az egyenletesen legerősebb próba $\Phi$-ben a
      $H_0:\mu=0$, $H_1:\mu\neq0$ feladatra.
    \end{proposition}
    \continue
    \begin{displaymath}
      \partial\ell (0)=\frac{n}{\sigma^2}\bar{X},
      \quad
      \event*{\abs{\partial\ell (0)}>c}
      =\event*{\abs{\sqrt{n}\tfrac{\bar{X}}{\sigma}}>u_{\alpha/2}}
    \end{displaymath}
  \end{frame}
  
  \begin{frame}{$H_1:\mu\neq0$-ra, az $u$-próba egyenletesen
      legerősebb torzítatlan próba III.}
    \begin{lemma}
      %\begin{itemize}
      %\item
      $\mu_1<\mu_0=0<\mu_2$,
      %$f_i=f_{\mu_i}$ a
      %sűrűségfüggvények,
      $p\in (0,1)$.
        % \item
        $H_0:f=f_{\mu_0}$, $H_1^*: f=pf_{\mu_1}+ (1-p)f_{\mu_2}$.
        % \item Belátjuk, hogy
  
        Ekkor $\forall\,\mu_1<0<\mu_2$-re létezik
        $p=p (\mu_1,\mu_2)$ úgy, hogy $H^*$-hoz tartozó $\alpha$ terjedelmű
        legerősebb próba $\phi=\I{\abs*{\partial\ell (0)}>c}$
        alakú és
        \begin{displaymath}
          \exists \lim_{\mu_2\to0}\frac{p (\mu_1,\mu_2)}{\mu_2}\in (0,\infty),\
          \quad 
          \lim_{\mu_1\to0}\frac{1-p (\mu_1,\mu_2)}{\mu_1}\in (0,\infty).
        \end{displaymath}
      %\end{itemize}
    \end{lemma}
    \begin{itemize}
    \item Ezt felhasználva, ha $\bar{\phi}\in\Phi$, akkor
      \begin{displaymath}
        \lim_{\mu_2\to0}\frac{1}{\mu_2}\zjel{\E[\mu_2]{\bar{\phi}}-\E[0]{\bar{\phi}}}
        =\E[0]{\bar{\phi}\partial\ell (0)}=0,
        \quad
        \lim_{\mu_2\to0}\frac{1}{\mu_2}\zjel{\E[\mu_2]{\phi}-\E[0]{\phi}}
        =\E[0]{\phi\partial\ell (0)}=0
      \end{displaymath}
      és
      \begin{displaymath}
        \frac{1}{p}\E[\mu_2]{\bar{\phi}-\phi}\to0,\quad\text{ha $\mu_2\to0$
        és $\E[0]{\bar{\phi}}= \E[0]{\phi}$}
      \end{displaymath}
      mivel $\phi$ legerősebb próba $H^*$-ra
      \begin{displaymath}
        p\E[\mu_1]{\bar{\phi}}+ (1-p)\E[\mu_2]{\bar{\phi}}\leq
        p\E[\mu_1]{\phi}+ (1-p)\E[\mu_2]{\phi},
        \quad\implies
        \E[\mu_1]{\phi-\bar{\phi}}\geq \frac{1-p}{p}\E[\mu_2]{\bar{\phi}-\phi}\to0
      \end{displaymath}
    \end{itemize}
  \end{frame}
  
  \begin{frame}{Lemma bizonyítása}
    \begin{itemize}
    \item $H_0:f=f_0$, $H_1^*:f=pf_1+ (1-p)f_2$. Itt
      \begin{displaymath}
        f_i (x) = f_{\mu_i} (x)= h (x)e^{\mu_i\cdot T(x) - b (\mu_i)},
        \quad 
        T(x)=\frac{1}{\sigma^2}\sum x_i,
        \quad
        b(\mu)=\frac{n\mu^2}{2\sigma^2},
        %,\quad pf_1 (x)+ (1-p)f_2 (x)=p h (x)e^{\mu_1\tT (x)+b (\mu_1)}+ (1-p)h (x)e^{\mu_2\tT (x)+b (\mu_2)}
      \end{displaymath}
      ahol $\mu_1<\mu_0=0<\mu_2$ és $b (0)=b' (0)=0$, azaz $\partial\ell (0)=T$.
    \item A likelihood hányados 
      \begin{displaymath}
        \frac{pf_1+ (1-p)f_2}{f_0} (x) 
        =p e^{\mu_1 t-b (\mu_1)}+(1-p)e^{\mu_2 t-b (\mu_2)}|_{t=T (x)}=g (T (x))
      \end{displaymath}
      Ez a $t$-nek konvex függvénye, $\lim_{t\to\pm\infty} =\infty$
    \item A kritikus tartomány $\event*{\abs*{T}>c}$ alakú, ha
      \begin{displaymath}
        g (c)=g (-c),\quad\text{azaz}\quad
        p\zjel{e^{\mu_1 c}-e^{-\mu_1c}}e^{-b (\mu_1)}
        = (1-p)\zjel{e^{-\mu_2 c}-e^{\mu_2c}}e^{-b (\mu_2)}
      \end{displaymath}
      Ennek megoldása $p$-re $p (\mu_1,\mu_2)$.
    \item Ha $\mu_2$-vel osztunk és $\mu_2\to0$, akkor
      \begin{displaymath}
        \lim_{\mu_2\searrow0}\frac{p (\mu_1,\mu_2)}{\mu_2} \sinh(\mu_1c)e^{-b(\mu_1)}
        = \lim_{\mu_2\searrow0} (1-p(\mu_1,\mu_2)) e^{-b (\mu_2)} c\frac{\sinh (\mu_2c)}{\mu_2 c}
        = c
      \end{displaymath}
    \end{itemize}
  \end{frame}
\end{document}

