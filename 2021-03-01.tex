%\def\option{}
\providecommand{\option}{handout}
\documentclass[aspectratio=169,notheorems,9pt,\option]{beamer}



\makeatletter
\def\readrest#1\enddate{}
\def\setdate@#1-#2-#3\relax{\def\lecturedate{#1.#2.#3.}\readrest}
\def\setdate{\expandafter\setdate@\jobname\relax-\relax-\relax\enddate}

\makeatother

\setdate


\usepackage{marvosym}
%\usepackage[magyar]{babel}
\usepackage{polyglossia}
\setmainlanguage{magyar}

% \usetheme{Bruno}

\usecolortheme{Bruno}
% \usefonttheme{Bruno}
% \useinnertheme{Bruno}
\useoutertheme{Bruno}

\usepackage{pvdefs}
\newtheorem{df}{Definíció}
\newtheorem{theorem}{Tétel}
\newtheorem{lemma}{Lemma}
\newtheorem{proposition}{Állítás}
\newtheorem{corollary}{Következmény}

\defcx{XAPIDLCSGTHEN}
\deftx{STQbCf\lambda\Omega\sigma}
\defmf{X}

\def\diag{\operatorname{diag}}
\def\Id{\mathbf{I}}
\let\PEfont\mathbb
\def\npto{\buildrel p\over \nrightarrow}
\def\ndto{\buildrel d\over \nrightarrow}
\def\dto{\buildrel d\over \rightarrow}
\def\dto{\stackrel{d}{\to}}
\let\nto\nrightarrow

\def\Leb{\operatorname{Leb}}

\newrobustcmd{\continue}{%
  \onslide<\value{beamerpauses}->\relax
}


\beamerdefaultoverlayspecification{<+->}

\mode<presentation>

\RequirePackage[math-style=TeX, bold-style=upright]{unicode-math}
\setmathfont{Latin Modern Math}[Scale=1]
% \setsansfont{Kurier Cond}
\setsansfont[Ligatures=TeX, ItalicFont={Fira Sans Italic}, BoldFont={Fira Sans SemiBold}, Scale=0.9]{Fira Sans Book}
%\setsansfont{Ubuntu}

%\usepackage{mathspec}
%\setmathsfont{lm}
%(Digits,Latin,Greek)
%[Numbers={Lining,Proportional}]{lm}
%\setmathsfont(Digits,Greek)
%[Uppercase=Plain,Lowercase=Regular,Scale=MatchLowercase]
%{GFS Porson}


% Title page
\setbeamerfont{title}{
    size=\LARGE,
    shape=\bfseries
}

\setbeamerfont{subtitle}{
    size=\large,
    shape=\normalfont
}

\setbeamerfont{author}{
    size=\normalsize,
    shape=\normalfont
}

\setbeamerfont{details}{
    size=\footnotesize,
    shape=\normalfont
}

% Slide title
\setbeamerfont{frametitle}{size=\LARGE}
\setbeamerfont{framesubtitle}{
    size=\normalsize,
    %shape=\normalfont\scshape
}

% Blocks
\setbeamerfont{block title}{
    size=\normalfont,
    shape=\strut
}

\setbeamerfont{blockdef}{
    size=\large,
    shape=\bfseries
}

% Description
\setbeamerfont{description item}{shape=\bfseries}


% Footer information line with title, author and slide number
\setbeamertemplate{footline}{%
    \leavevmode%
    \hbox{%
        \usebeamerfont{footline}%
        \begin{beamercolorbox}[
            wd=\textwidth,
            ht=3ex,
            dp=1.25ex
        ]{footline}%
            \hspace{1cm}%
            \insertshorttitle%
            \hspace{2em}%
            \insertdate%
            \hfill%
            \insertframenumber{} / \inserttotalframenumber%
            \hspace{1cm}
        \end{beamercolorbox}%
    }
    \vskip0pt%
}

%% Bruno Beamer theme — Version 1.0
%% A lightweight Beamer theme inspired from the Metropolis theme
%%
%% Written in 2017-2019 by:
%% — Rémi Cérès <remiceres@msn.com>
%% — Mattéo Delabre <bonjour@matteodelabre.me>
%%
%% This work is released under the CC0 1.0 Universal license. See the
%% accompanying LICENSE file for details. To the extent possible under
%% law, Rémi Cérès and Mattéo Delabre have waived all copyright and
%% related or neighboring rights to the Bruno Beamer theme. This work
%% is published from France.

% \mode<presentation>

\setbeamertemplate{sections/subsections in toc}[square]
\setbeamertemplate{itemize item}[square]
\setbeamertemplate{itemize subitem}[circle]

% Allow multi-slide frames
\setbeamertemplate{frametitle continuation}{}

% Horizontal separator
\setbeamertemplate{separator}{%
    \usebeamercolor{separator}%
    \textcolor{fg}{\rule{.7\textwidth}{.5pt}}%
}

% Separate authors with newlines
\renewcommand{\beamer@andtitle}{\\}

% Title page. If you want to add a background image on the title slide,
% use the \background command to set the path to that image. Otherwise,
% no image will be included
\RequirePackage{tikz}
\usetikzlibrary{fadings}
\newcommand{\background}[1]%
    {\newcommand{\bruno@background}{#1}}


%% \newcommand\insertcaption{}
\newcommand{\backgroundcaption}[2][white]{\def\bruno@caption{{#1}{#2}}}

\newcommand{\insertcaption}{\ifdefined\bruno@caption{\expandafter\textcolor\bruno@caption}\fi}

\tikzfading[
    name=title page picture fading,
    left color=transparent!0,
    right color=transparent!100,
]

\setbeamertemplate{title page}{%
    \begin{minipage}{.7\textwidth}
        \raggedright
        \usebeamerfont{title}
        \inserttitle

        \usebeamertemplate{separator}

        \usebeamerfont{author}
        \vspace{2em}
        \insertauthor

        \vspace{2em}
        \usebeamerfont{details}
        \insertinstitute

        \insertdate
    \end{minipage}

    % Include faded image on the right if defined
    \ifdefined\bruno@background
    \begin{tikzpicture}[remember picture, overlay]
        % Crop image to a trapezium on the right
        \clip (current page.south east)
           -- (current page.north east)
           -- ++(-5.1, 0)
           -- ++(-3, -\paperheight)
           -- cycle;

        % Include background image
        \node[
            anchor=south east,
            inner sep=0,
            outer sep=0
        ] at (current page.south east) {
            \includegraphics[height=\paperheight]
                {\bruno@background}
        };

        \node[anchor=south east,
            inner sep=0,
            outer sep=15pt,
        ] at (current page.south east){\insertcaption};

        % Add a slight shadow
        \fill[
            black, path fading=title page picture fading,
            fading angle=-15
        ]
            (current page.south east)
            rectangle
            ++(-10, \dimexpr\paperheight+1cm);

        
    \end{tikzpicture}
    \fi
}

\mode<all>
\AtBeginDocument{%
  \let\phi\varphi
  \let\theta\vartheta
  \let\setminus\smallsetminus
}


\title{Statisztika előadás}
\date{\lecturedate}
\background{images/young_Ronald_Fisher.jpg}
\backgroundcaption{Ronald Fisher (1913)}


\begin{document}

\maketitle
\begin{frame}{Becslés, torzítatlanság, hatásosság}
  \begin{df}
    $\cP=\set{\P[\theta]}{\theta\in \Theta}$. $g:\Theta\to\real$. 

    $T$ statisztika a $g(\theta)$ torzítatlan becslése, ha $\E[\theta]{T}=g(\theta)$.

    $T$ a $g(\theta)$ hatásos becslése, ha torzítatlan $\D[\theta]^2{T}$ véges minden $\theta\in\Theta$-ra 
    és tetszőleges $T'$ torzítatlan becslésre $\D[\theta]^2{T}\leq\D[\theta]^2{T'}$.
  \end{df}

  \begin{theorem}[Rao-Blackwell-Kolmogorov]
    $S$ elégséges a $\cP$ eloszláscsaládra és $T$ torzítatlan $g:\Theta\to\real$-re. 
    Ekkor $\E[\theta]{T\given  S}=h(S)$ 
    közös változata torzítatlan $g(\theta)$-ra és $\D[\theta]^2{h(S)}\leq \D[\theta]^2{T}$.

    Ha az $S$ statisztika teljes is, akkor $\E[\theta]{T\given  S}$ hatásos becslés.
  \end{theorem}
\end{frame}

\begin{frame}{Alsó korlát torzítatlan becslés szórásnégyzetére}
  \begin{itemize}
    \item $\P[0]$ $\P[1]$ eloszlások, $f_0$, $f_1$ sűrűséggel az $\mfX$ mintatéren, $\P[1]\ll\P[0]$.
    \item $T$ statisztika, $\E[i]{T}=g(i)$ $i=0,1$. 
    \begin{displaymath}
      g(1)-g(0)=\E[1]{T}-\E[0]{T}=\E[0]{T\zjel*{\frac{f_1}{f_0}-1}}
    \end{displaymath}
    \item $\E[0]{f_1/f_0-1}=0$ és $U=f_1/f_0-1$ jelöléssel
    \begin{displaymath}
      \abs{g(1)-g(0)}=\abs{\cov_0(T,U)}\leq\D[0]{T}\D[0]{U}
      \quad\implies\quad
      \D[0]^2{T}\geq \frac{(g(1)-g(0))^2}{\D[0]^2(U)}.
    \end{displaymath}
  \end{itemize}
  \continue
  Példa
  \begin{itemize}
    \item $X$ $n$ elemű minta $U(0,\theta)$ eloszlásból, $\theta>0$.
    \item $T$ torzítatlan becslés $\theta$-ra. 
    \item $0< \theta_1=\alpha\theta_0<\theta_0$, mellett $\P[1]\ll\P[0]$ és 
    $U(x)=\frac{f_1(x)}{f_0(x)}-1$, 
    $\frac{f_1(x)}{f_0(x)}%=\zfrac*{\theta_0}{\theta_1}^n\I{\max x_i<\theta_1}
    =\alpha^{-n}\I{\max x_i<\theta_1}$.
    \begin{displaymath}
      \D[0]^2{U}=\E[0]{U^2}=\E{\zfrac*{f_1}{f_0}^2}-1=\alpha^{-n}-1
      %\zjel*{\alpha^{-n}-1}^2\cdot\zfrac*{\theta_1}{\theta_0}^n
      %+1^2\cdot\zjel*{1-\zfrac*{\theta_1}{\theta_0}^n}=\alpha^{-n}-1
      \implies \D[0]^2{T}\geq 
      \theta_0^2 \sup_{\alpha\in(0,1)}\frac{(1-\alpha)^2}{\alpha^{-n}-1}
      \geq C\frac{\theta_0^2}{n^2},\quad\zjel{\alpha=1-\tfrac1n}.
      %\frac{(1-\alpha^n)^2}{\alpha^n}+1-\alpha^n(1-\alpha^n)\alpha^{-n}=\alpha^{-n}+1
    \end{displaymath}
  \end{itemize}
\end{frame}

\begin{frame}{Példa. Indikátor minta}
  % \begin{itemize}
  %   \item $\P[0]$ $\P[1]$ eloszlások, $f_0$, $f_1$ sűrűséggel az $\mfX$ mintatéren, $\P[1]\ll\P[0]$.
  %   \item $T$ statisztika, $\E[i]{T}=g(i)$ $i=0,1$. 
  %   \item $\E[0]{f_1/f_0-1}=0$ és $U=f_1/f_0-1$ jelöléssel
  %   \begin{displaymath}
  %     \abs{g(1)-g(0)}=\abs{\cov_0(T,U)}\leq\D[0]{T}\D[0]{U}
  %     \quad\implies\quad
  %     \D[0]^2{T}\geq \frac{(g(1)-g(0))^2}{\D[0]^2(U)}.
  %   \end{displaymath}
  % \end{itemize}
  % \continue
  % Példa
  \begin{itemize}
    \item $X$ $n$ elemű minta $\theta$ paraméterű indikátor mintából, $\theta\in(0,1)$. 
    \item $T$ torzítatlan becslés $p(\theta)$-ra, ahol $p$ legfeljebb $n$-edfokú polinom. 
    \item $\P[\theta]$ ekvivalens a számláló mértékkel minden $\theta$ mellett és 
    $\theta_1,\theta_0\in(0,1)$ esetén 
    \begin{displaymath}
      \lim_{\theta_1\to\theta_0}\frac1{\theta_1-\theta_0}\zjel*{\frac{f_{\theta_1}(x)}{f_{\theta_0}(x)}-1}
      =\partial_\theta \ln f_{\theta_0}(x)=\frac{s}{\theta_0}-\frac{n-s}{1-\theta_0},\quad\text{ahol $s=\sum x_i$}.
      % \frac{f_{\theta_1(x)}}{f_\theta_0(x)}=\frac{\theta_1^s(1-\theta_1)^{n-s}}{\theta_0^s(1-\theta_0)^{n-s}},\quad\text{ahol $s=\sum x_i$},
      % \quad
      % \E[\theta_0]{\zfrac{f_{\theta_1}}{f_\theta_0}^2}
      % =\sum_{s=0}^n \binom{n}{s}\frac{\theta_1^{2s}(1-\theta_1)^{2(n-s)}}{\theta_0^s(1-\theta_0)^{n-s}}
      % =\zjel*{\frac{\theta^2_1}{\theta_0}+\frac{(1-\theta_1)^2}{1-\theta_0}}^n=
    \end{displaymath}
    \item 
    \begin{displaymath}
      p'(\theta_0)=\lim_{\theta_1\to\theta_0}\frac{\E[\theta_0]{T\zjel*{\frac{f_{\theta_1}}{f_{\theta_0}}-1}}}{\theta_1-\theta_0}
      =\cov_{\theta_0}(T,\partial_\theta\ln f_{\theta_0})\leq \D[\theta_0]{T}\D[\theta_0]{\partial_\theta\ln f_{\theta_0}}
    \end{displaymath}
    \item
    \begin{displaymath}
      \D[\theta]^2{T}\geq 
      \frac{\zjel{p'(\theta)}^2}{\D[\theta]^2{\partial_\theta\ln f_\theta}}=(p'(\theta))^2 \frac 1n\theta(1-\theta),
      \quad\text{hiszen}\quad
      \D[\theta]^2{\partial_\theta\ln f_\theta}
      =\D[\theta]^2{\frac{\sum X_i}{\theta(1-\theta)}}=\frac{n}{\theta(1-\theta)}.
    \end{displaymath}
    Ez a Cramer-Rao egyenlőtlenség speciális esete.
    \item $p(\theta)=\theta$ esetén az alsó becslés a mintaátlag szórásnégyzete, 
    azaz a mintaátlag hatásos  becslése a valószínűségnek, vagyis $\theta$-nak.
  \end{itemize}
\end{frame}

\begin{frame}{Cramer-Rao egyenlőtlenség, információs határ}
  \begin{theorem}
  \begin{itemize}[<*>]
    \item $\cP=\set{\P[\theta]}{\theta\in\Theta}$ dominált család $f_\theta$ sűrűségekkel, $\Theta\subset\real^p$. 
    \item $\ell(\theta)=\ell(\theta,x)=\ln f_\theta(x)$ a \textbf{loglikelihood} függvény. 
    \item $\theta\in\interior\Theta$, $\ell'=\ell'(\theta)=\partial_\theta\ell(\theta)$ ($p$-dimenziós sorvektor) 
    létezik és $\E[\theta]{\ell'(\theta)}=0$.
    \item $I(\theta)=\Sigma_\theta((\ell'(\theta))^T)
    =\E[\theta]{\ell'(\theta))^T\ell'(\theta)}$ invertálható. $I(\theta)$ neve \textbf{Fisher információ}.
    \item $T:\mfX\to\real^q$ statisztikára $g(\theta)=\E[\theta]{T}$ deriválható és 
    $g'(\theta)=\E[\theta]{T\ell'(\theta)}$
  \end{itemize}
  Ekkor 
  \begin{displaymath}
    \Sigma_{\theta}(T)\geq G I^{-1}(\theta)G^T,\quad\text{ahol $G=g'(\theta)\in\real^{q\times p}$ és }
  \end{displaymath}
  \end{theorem}
  \begin{itemize}
    \item $A\in\real^{q\times p}$ mátrixszal $T-A\partial_\theta\ell(\theta)$ $q$-dimenziós vektor változó.
    % \item $\ell'=\partial_\theta\ell(\theta)$ jelöléssel 
    \begin{align*}
      0\leq \Sigma_\theta(T-A(\ell')^T)
      &=\Sigma_\theta(T)-A\E[\theta]{(\ell')^T T^T}-\E[\theta]{T\ell'}A^T+A\E[\theta]{(\ell')^T\ell'}A^T \\
      &=\Sigma_\theta(T)-AG^T-GA^T+A I(\theta) A^T 
    \end{align*}
    \item Ha itt $A=G^TI^{-1}(\theta)$, akkor
     \begin{displaymath}
       0\leq \Sigma_\theta(T)-G^TI^{-1}(\theta)G\quad\implies\quad \Sigma_\theta(T)\geq G^TI^{-1}(\theta) G.
     \end{displaymath}
  \end{itemize}
  Ha $\E[\theta]{\ell'(\theta)}=0$, akkor az $n$ elemű minta Fisher információja $I_n(\theta)=I_1(\theta)$ 
  az alsó becslés $1/n G^TI_1^{-1}G$.
  %\continue
  %$I(\theta)$ neve \textbf{Fisher} információ. Az ún. \textbf{gyenge regularitási feltétel} garantálja, hogy
  %a $T$-re kirótt  minimális feltételek mellett a tétel feltételei teljesüljenek.
\end{frame}


\begin{frame}[<*>]{Cramer-Rao egyenlőtlenség, speciális esetek}
  \begin{theorem}
    \begin{itemize}[<*>]
      \item $\cP=\set{\P[\theta]}{\theta\in\Theta}$ dominált család $f_\theta$ sűrűségekkel, $\Theta\subset\real^p$. 
      $\ell(\theta)$ loglikelihooddal. 
      \item $\theta\in\interior\Theta$, $\ell'$
      létezik, $\E[\theta]{\ell'(\theta)}=0$, 
      $I(\theta)=\E[\theta]{\ell'(\theta))^T\ell'(\theta)}$ invertálható. % $I(\theta)$ neve \textbf{Fisher információ}.
      \item $T:\mfX\to\real^q$ statisztikára $g(\theta)=\E[\theta]{T}$ deriválható és 
      $g'(\theta)=\E[\theta]{T\ell'(\theta)}$
    \end{itemize}
    %Ekkor 
    \begin{displaymath}
      \Sigma_{\theta}(T)\geq G I^{-1}(\theta)G^T,\quad\text{ahol $G=g'(\theta)\in\real^{q\times p}$ és }
    \end{displaymath}
  \end{theorem}

  \begin{corollary}
    \begin{itemize}[<*>]
      \item $\Theta\subset\real$, $g:\Theta\to\real$. Ha a fenti feltételek teljesülnek, akkor
      \begin{displaymath}
        \D[\theta]^2(T)\geq \frac{(g'(\theta))^2}{I(\theta)}
      \end{displaymath}  
      \item $g(\theta)=\theta$. $T=\hat\theta$ a paraméter torzítatlan becslése. Ha a fenti feltételek teljesülnek, akkor
      \begin{displaymath}
        \Sigma_{\theta}(\hat\theta)\geq I^{-1}(\theta).
      \end{displaymath}  
    \end{itemize}
  \end{corollary}
\end{frame}

  % mi ez, mire jó, példák, összefüggések.
  % 
\begin{frame}{Fisher információ}
  \begin{df}
    \begin{itemize}[<*>]
      \item $\cP=\set{\P[\theta]}{\theta\in\Theta}$ dominált eloszláscsalád az $\mfX$ mintatéren, ahol $\Theta\subset\real^p$ nyílt.
      \item $f_\theta=\frac{d\P[\theta]}{d\lambda}$ ahol $\lambda$ a domináló mérték. 
      $\ell(\theta)=\ell(\theta,x)=\ln f_\theta(x)$ a \textbf{loglikelihood} függvény.
      \item 
      $\ell'(\theta)=\partial_\theta\ell(\theta,x)$ p-dimenziós sorvektor
      létezik tetszőleges $\theta$ esetén 
      $\P[\theta]$ majdnem minden $x$ mellett.
    \end{itemize}
    \begin{displaymath}
      I(\theta)=\E[\theta]{\ell'(\theta)^T\ell'(\theta)}
    \end{displaymath}
    $I:\Theta\to\real^{p\times p}$ a minta \textbf{Fisher információja}, ha a várható érték létezik. 
  \end{df}
  \begin{itemize}
    \item $I(\theta)$ pozitív szemidefinit mátrix. Ha $a\in\real^p$, akkor 
    $a^T I(\theta) a=\E[\theta]{\zjel{\ell'(\theta)a}^2}\geq0$.
    \item A definícióban a sűrűségek ,,szép'' változatát használjuk, 
    pl. $(\theta,x)\mapsto f_\theta(x)$ mérhető, 
    sőt általában $\theta$-ban folytonosan deriválható.
    \item A Fisher információ nem függ a domináló mérték megválasztásától.
    Ha $\P[0]\in\cP'$, akkor tetszőleges $\lambda$ domináló mértékre
    \begin{displaymath}
      f_\theta=\frac{d\P[\theta]}{d\P[0]}\frac{d\P[0]}{d\lambda}
      \implies 
      \ell(\theta)
      =\ln \frac{d\P[\theta]}{d\P[0]}+\ln \frac{d\P[0]}{d\lambda}
      \implies
      \partial_\theta \ell(\theta)
      =\partial_\theta \ln \frac{d\P[\theta]}{d\P[0]}.
    \end{displaymath} 
    \item Erősebb feltételek mellett érdemes nézni, 
    pl. gyenge regularitási feltétel. 
  \end{itemize}
\end{frame}

\begin{frame}{Formulák Fisher információ}      
  \begin{itemize}
    \item $\ell(\theta)=\ell(\theta,x)=\ln f_\theta(x)$ a \textbf{loglikelihood} függvény.
    $f_\theta(x)\geq0$, azaz $\ell(\theta,x)=-\infty$ előfordulhat. Megállapodás $\partial_\theta f_\theta(x)=0$, ha nem létezik, 
    vagy $f_\theta(x)=0$.
    \begin{displaymath}
      \partial_\theta \ell(\theta,x)
      =\frac{\partial_\theta f_\theta(x)}{f_\theta(x)}
      =\begin{cases}
        0 & f_\theta(x) = 0\\
        \frac{\partial_\theta f_\theta(x)}{f_\theta(x)} & f_\theta(x)>0
      \end{cases}
    \end{displaymath}
    \item Ha $\Theta\subset\real^p$ nyílt, akkor $\ell'(\theta)$ $p$-dimenziós sorvektor értékű statisztika
    \begin{align*}
      I(\theta)&=\E{(\ell'(\theta))^T\ell'(\theta)} \\
      &=\int_{\mfX} \frac{\zjel{\partial_\theta f_\theta(x)}^T\partial_\theta f_\theta(x)} {f^2_\theta(x)}f_\theta(x)\lambda(d x)\\
      &=\int_{\mfX} \frac{\zjel{\partial_\theta f_\theta(x)}^T\partial_\theta f_\theta(x)} {f_\theta(x)}\lambda(d x)\\
      &= 4\int_{\mfX}\zjel{\partial_\theta\sqrt{f_\theta(x)}}^T \partial_\theta\sqrt{f_\theta(x)}\lambda(d x)
    \end{align*}
  \end{itemize}

\end{frame}

\begin{frame}{Fisher információ exponenciális családban}
  Emlékeztető. $\cP$ exponenciális családot alkot, ha alkalmas domináló mérték és paraméterezés 
  mellett $\ln f_\theta(x)=\theta T(x)-b(\theta)$ alakú, $\theta\in\Theta$. 
  \begin{itemize}
    \item Ha $\Theta$ nyílt, de legalább $\theta\in\interior\Theta$, akkor $\theta+\alpha\in\Theta$-ra 
    \begin{displaymath}
      e^{b(\theta+\alpha)}
      =\int_{\mfX}e^{(\theta+\alpha) T(x)} \lambda(d x)
      =e^{b(\theta)}\int_{\mfX}e^{\alpha T(x)} e^{\theta T(x)-b(\theta)} \lambda(d x)
      =e^{b(\theta)}\E[\theta]{e^{\alpha T}} 
    \end{displaymath}
    \item $M(\alpha) = e^{b(\theta+\alpha)-b(\theta)}$ a $T$ statisztika momentum generáló függvénye $\P[\theta]$ alatt.
    \item Kiszámoljuk, hogy ha a momentum generáló függvény véges az origó egy környezetében, akkor ott sima ($C^\infty$) 
    és a nullabeli deriváltak $T$ momentumait adják. 
    $\ln M(\alpha) = b(\theta+\alpha)-b(\theta)$ a kummuláns generáló függvény, ez is sima és a nullabeli deriváltakra
    \begin{displaymath}
      \E[\theta]{T} = \frac{M'(0)}{M(0)}=(\ln M)'(0) = b'(\theta),\quad
      \Sigma_\theta{T} = \frac{M''(0)M(0)-M'(0)^TM'(0)}{M^2(0)} =(\ln M)''(0) = b''(\theta).
    \end{displaymath}
    \item $\ell'(\theta)= T-b'(\theta)=T-\E[\theta]{T}$. 
    \begin{displaymath}
      \E[\theta]{\ell'(\theta)}=0, 
      \quad  
      I(\theta)
      =\E[\theta]{\ell'(\theta)^T\ell'(\theta)}
      =\Sigma_\theta(T)=b''(\theta)=-\E[\theta]{\ell''(\theta)}   
    \end{displaymath}
  \end{itemize}
\end{frame}

\begin{frame}{Momentum generáló függvény, deriválhatóság}
  \begin{proposition}
    Ha $M (\alpha)=\E{e^{\alpha\cdot X}}<\infty$, az origó egy környezetében,
    akkor ott  $M$ sima és $M'(0)=\E{X}^{T}$, $M''(0)=\E{XX^{T}}$.
  \end{proposition}
  \begin{itemize}
    
  \item  Létezik $\eps>0$, hogy $\E{e^{h\norm{X}}}<\infty$, ha
    $h<\eps$.

    $\real^p$ a normák ekvivalensek, elég
    $\norm{x}_{1}=\frac1p \sum \abs*{x_i}$-re számolni. Mivel $\exp$
    konvex $e^{\norm{x}_1}\leq\frac1p \sum e^{\abs*{x_i}}$ és
    %$\norm{h}_1=\sum h_i=1$ és $h_i\geq0$, akkor
    \begin{displaymath}
      e^{h\abs*{X_i}}\leq e^{hX_i}+e^{-hX_i},\quad
      \E{e^{h\norm{X}_1}}\leq \frac1p \sum_i(M (h e_i)+M (-h e_i))<\infty
    \end{displaymath}
    ahol $e_1,\dots,e_p$ a természetes bázis $\real^p$-ben.
  \item ha $\norm{\alpha}$ kicsi, akkor
    $\abs{\alpha\cdot X}<\eps\norm{X}$ és a dominált konvergencia tétel miatt
    \begin{displaymath}
      M (\alpha)=\E{\sum_k \frac{(\alpha\cdot X)^n}{n!}}
      =\sum_n \frac{\E{(\alpha\cdot X)^n}}{n!}\quad\implies\quad 
      \partial_{i_1,\dots,i_k}M (0) =\E{\prod_{j}X_{i_j}}
    \end{displaymath}
  \end{itemize}
\end{frame}

\begin{frame}{Példa}
  \begin{itemize}
    \item $X_1,\dots,X_n$ exponenciális eloszlású minta, ismeretlen $\lambda>0$ paraméterrel.
    \item A minta eloszlásainak a családja exponenciális családot alkot, 
    a domináló mérték a Lebesgue mérték, $\mfX=(0,\infty)^n$
    \begin{displaymath}
      f_\lambda(x)=\lambda^ne^{-\lambda\sum x_i}=\exp{\lambda T(x)-b(\lambda)},
      \quad T(x)=-\sum x_i,
      \quad b(\lambda)=-n\ln(\lambda)
    \end{displaymath}
    \item $I(\lambda)=b''(\lambda)=n\lambda^{-2}$
    \item Definíció alapján számolva:
      \begin{displaymath}
        \ell(\lambda)= n\ln \lambda-\lambda \sum X_i,\quad \ell'(\lambda)=\frac{n}{\lambda}-\sum X_i   
      \end{displaymath}    
      Itt $-\ell'(\lambda)=\sum X_i-\E{\sum X_i}$ és 
      $I(\lambda)=\D[\lambda]^2(\sum X_i)=n \D[\lambda]^2(X_1)=n\lambda^{-2}$
  \end{itemize}
\end{frame}

\begin{frame}{Átparaméterezés és Fisher információ}
  \begin{itemize}
  \item $\Xi\subset\real^q$, nyílt. $h:\Xi\to\Theta$ injektív,
    differenciálható. ($q\leq p$)
  \item $H=h'$
  \item $\cP'=\set{\P[h (\xi)]}{\xi\in \Xi}$. 
  \item $\cI (\xi)$ a Fisher információ $\cP'$-ben.
    
  \item Láncszabály
    \begin{displaymath}
      \partial_\xi\log f_{h (\xi)} (x)=\partial_\theta\log
      f_{\theta} (x)|_{\theta =h (\xi)} H (\xi)
    \end{displaymath}
  \item
    \begin{displaymath}
      \cI (\xi)=H (\xi)^TI (h (\xi))H (\xi)
    \end{displaymath}
  \item Ha $q=1$, akkor $\cI(\xi) = (\theta'(\xi))^2 I(\theta(\xi))$.
  \end{itemize}
\end{frame}

\begin{frame}{Átparaméterezés és információs határ}
  \begin{itemize}
  \item $\Xi,\Theta\subset\real^p$, nyílt, $h:\Xi\to\Theta$ diffeomorfizmus.
  \item $H=h'$ invertálható
  \item $\cP'=\set{\P[h(\xi)]}{\xi\in \Xi}$. 
  \item $\cI (\xi)$ a Fisher információ $\cP'$-ben, $\cI (\xi)=H (\xi)^TI (h (\xi))H (\xi)$.
  \begin{displaymath}
   I(h(\xi))^{-1}= H(\xi)\,\cI^{-1} (\xi)\, H^{T}(\xi)
  \end{displaymath}
    
  \item $g (\theta) =\E[\theta] (T)$, $G=g'$
    
  \item $\E[h (\xi)]{T}=g (h(\xi))$, $(g\circ h)'=(G\circ h) H$
    
  \item Információs határ. Ha a Cramer-Rao egyenlőtlenség feltételei teljesülnek
    \begin{align*}
      \Sigma_{h (\xi)} (T)
      &\geq G (h(\xi)) H \,\cI (\xi)^{-1}\, H^T G (h (\xi))^{T}\\
      &=G (h(\xi)) I (h (\xi))^{-1} G (h (\xi))^{T}
    \end{align*}
  \end{itemize}
  \alert{Az információs határ nem változik.}
\end{frame}

% \begin{frame}{Átparaméterezés hatása}
  
% \end{frame}

\begin{frame}{Példa, indikátor minta Fisher információja}
  \begin{itemize}
    \item $X_1,\dots, X_n$ indikátor minta $p\in(0,1)$ ismeretlen paraméterrel. 
    \item $X$ eloszlásainak a családja exponenciális családot alkot a $\theta=\ln\frac{p}{1-p}$ paraméterezéssel:
    \begin{displaymath}
      f_p(x)=p^{\sum x_i}(1-p)^{n-\sum x_i}=\exp{\ln\frac{p}{1-p}\sum x_i+n\ln(1-p)}=\exp{\theta S(x)+n\ln{1+e^{\theta}}}
    \end{displaymath}
    ahol $S$ a mintaösszeg, $p=1/(1+e^{-\theta})$ $b(\theta)=-n\ln(1-p)=-n\ln(1-\frac1{1+e^{-\theta}})=n\ln(1+e^\theta)$.
    
    Ellenőrzés: $e^{b(0)}=\abs{\mfX}=2^{n}$
    \begin{displaymath}
      e^{b(\theta)}
      =e^{b(0)}\E[\theta=0]{e^{\theta S}}
      =2^{n}\E[\theta=0]^n{e^{\theta X_1}}
      =2^n\zjel*{\frac12+\frac12e^{\theta}}^n
      =(1+e^{\theta})^n
    \end{displaymath}

    \item \begin{displaymath}
      I(\theta)=b''(\theta)
    =n\zfrac*{e^\theta}{1+e^{\theta}}'
    =n\zfrac*{1}{1+e^{-\theta}}'
    =n\frac{e^{-\theta}}{(1+e^{-\theta})^2}=n p(1-p)=\D^2{S}
    \end{displaymath}

    \item Eredeti paraméterezés mellett $\cI(p)=\zjel{\theta'(p)}^2I(\theta(p))$
    \begin{displaymath}
      \theta(p)=\ln p -\ln (1-p),\quad \theta'(p)=\frac1p+\frac1{1-p}=\frac{1}{p(1-p)},\quad \cI(p)=\frac{n}{p(1-p)}
    \end{displaymath}
  \end{itemize}  
\end{frame}

\begin{frame}{Példa, normális minta Fisher információja}
  \begin{itemize}
    \item $X=(X_1,\dots,X_n)\sim N(\mu,\sigma^2)$ eloszlású minta ismeretlen $\mu\in\real$, $\sigma>0$ paraméterrel. 
    \begin{displaymath}
      f_{\mu,\sigma^2}(x)
      = \zfrac{1}{\sigma\sqrt{2\pi}}^n\exp{-\frac1{2\sigma^2}\sum (x_i-\mu)^2}
      = \exp{-\frac12\tau\sum x_i^2+\rho\sum x_i-b(\tau,\rho)}
      %\exp{\ln\frac{p}{1-p}\sum x_i+n\ln(1-p)}=\exp{\theta S(x)+n\ln{1+e^{\theta}}}
    \end{displaymath}
    \item % Elég egy minta elem
    \begin{align*}
      \ell(\mu,\sigma)&=C-n\ln\sigma-\frac12\cdot\sum_i \frac{(X_i-\mu)^2}{\sigma^2}
      %=C-\ln\sigma-\frac12 Z^2,\quad \text{ahol $Z=(X_1-\mu)/sigma\sim N(0,1)}
      \\
      \ell'(\mu,\sigma)
      &=(\sum \frac{X_i-\mu}{\sigma^2},\sum \frac{(X_i-\mu)^2}{\sigma^3}-\frac n\sigma)
      =\frac1\sigma(\sum Z_i,\sum (Z^2_i-1)),\quad\text{ahol $Z_i=\frac1{\sigma}(X_i-\mu)\sim N(0,1)$}
    \end{align*}
    \item $\E[\mu,\sigma]{\ell'(\mu,\sigma)}=0$, 
    \begin{displaymath}
      I(\mu,\sigma)
      =\Sigma_{\mu,\sigma}(\ell'(\mu,\sigma))
      =\frac{n}{\sigma^2}\Sigma((Z_1,Z_1^2-1))
      =\frac1{\sigma^2}  
      \begin{pmatrix}
        \D^2{Z}&\E{Z^3}\\\E{Z^3}&\D^2{Z^2}
      \end{pmatrix}
      =\frac1{\sigma^2}
      \begin{pmatrix}
        1&0\\0&2
      \end{pmatrix}  
    \end{displaymath}
    \item A minta eloszlásai exponenciális családot alkotnak
    % \begin{displaymath}
    %     f_{\mu,\sigma}(x)
    %     = \zfrac{1}{\sigma\sqrt{2\pi}}^n \exp{-\frac1{2\sigma^2}\sum (x_i-\mu)^2}
    %     = \exp{-\frac12\tau\sum x_i^2+\rho\sum x_i-b(\tau,\rho)}
    % \end{displaymath}
    
    HF. $b=?$ Fisher információ a természetes paraméterezés mellett.
    \item A minta eloszlása eltolás és skála paraméteres család. 
    HF. Hogyan számolható a Fisher információ eltolás és skála paraméteres családban.
    
    Egy mintaelem sűrűsége $f_{\mu,\sigma}(x)=\frac1\sigma f\zfrac{x-\mu}\sigma$, $\mu\in\real$, $\sigma>0$. 
    $f$ ,,szép'' sűrűségfüggvény,
    $I(\mu,\sigma)=?$, pl. $f(x)=\frac1{\pi}(1+x^2)^{-1}$ Cauchy eloszlás.
    % \begin{displayma&th}
    %   \theta(\mu,\sigma)=\zjel*{\tfrac1\sigma^2,\tfrac{\mu}{\sigma^2}}
    %   \quad
    %   \theta'(\mu,\sigma)=\begin{pmatrix}
    %     0&-2\sigma^{-3}\\\sigma^{-2}&-2\mu\sigma{-3}
    %   \end{pmatrix}
    %   \quad
    %   \cI(\mu,\sigma)=\begin{pmatrix}
    %     dd
    %   \end{pmatrix}
    %   % \theta(p)=\ln p -\ln (1-p),\quad \theta'(p)=\frac1p+\frac1{1-p}=\frac{1}{p(1-p)},\quad \cI(p)=\frac{n}{p(1-p)}
    % \end{displaymath}
    % \item $X$ eloszlásainak a családja exponenciális családot alkot a $\theta=(\tau,\rho)
    % =(\frac1{\sigma^2},\frac{\mu}{\sigma^2})$ paraméterezéssel:
    % \begin{displaymath}
    %   f_{\mu,\sigma^2}(x)
    %   = \zfrac{1}{\sigma\sqrt{2\pi}}^n\exp{-\frac1{2\sigma^2}\sum (x_i-\mu)^2}
    %   = \exp{-\frac12\tau\sum x_i^2+\rho\sum x_i-b(\tau,\rho)}
    %   %\exp{\ln\frac{p}{1-p}\sum x_i+n\ln(1-p)}=\exp{\theta S(x)+n\ln{1+e^{\theta}}}
    % \end{displaymath}
    % ahol $S=(-\frac12\sum x_i^2,\sum x_i)$. 
    % \begin{displaymath}
    %   b(\tau,\rho)
    %   =\frac{n}2\zjel{\ln\sigma^2+\ln(2\pi)}+\frac{n}2\cdot\frac{\mu^2}{\sigma^2}
    %   =\frac{n}2\zjel*{-\ln(\tau)+\frac{\rho^2}{\tau}+\ln(2\pi)}. 
    % \end{displaymath}
    % \item \begin{displaymath}
    %   I(\theta)=b''(\theta)=\frac n2
    %   \begin{pmatrix}
    %     \frac1{\tau^2}+2\frac{\rho^2}{\tau^3}& -\frac{\rho}{\tau^2}\\
    %     -\frac{\rho}{\tau^2}&\frac{2}\tau
    %   \end{pmatrix}
    %   =\frac n2\begin{pmatrix}
    %     \sigma^4+2\mu^2\sigma^2 & -\mu\sigma^2\\ -\mu\sigma^2 & 2\sigma^2 
    %   \end{pmatrix}
    %   %\Sigma_\theta(S)=n\Sigma_\theta((-\frac12 X_1^2,X_1))= n
    %   %\begin{pmatrix}
    %   %  \tfrac14\D[\theta]^2(X_1^2)&-\tfrac12\cov(X_1^2,X_1)\\
    %   %  -\tfrac12\cov(X_1^2,X_1)&\D[\theta]^2(X_1)
    %   %\end{pmatrix}
    % % =n\zfrac*{e^\theta}{1+e^{\theta}}'
    % % =n\zfrac*{1}{1+e^{-\theta}}'
    % % =n\frac{e^{-\theta}}{(1+e^{-\theta})^2}=n p(1-p)=\D^2{S}
    % \end{displaymath}

    % \item Eredeti paraméterezés mellett $\cI(\mu,\sigma)=(\theta'(\mu,\sigma))^TI(\theta(\mu,\sigma))\theta'(\mu,\sigma)$ HF.
  \end{itemize}  
\end{frame}

% \begin{frame}{Negatív példák}
%   \begin{itemize}
%     \item $X_1$ $U(0,\theta)$ eloszlású egy elemű minta, $\theta>0$ ismeretlen. Formálisan a Fisher információ definiálható
%     \begin{displaymath}
%       \ell(\theta,x)=-\ln\theta \I{x<\theta},\quad \ell'(\theta,x)=-\theta^{-1}\I{x<\theta},\quad 
%       I(\theta)=\E[\theta](\ell'(\theta)^2)=\theta^{-2}
%     \end{displaymath}
%     de $\E[\theta]{\ell'(\theta)}=-\theta^{-1}\neq0$.
%     \item $X_1=\theta (U_1+U_2)$, ahol $U_1,U_2$ független $(0,1)$-en egyenletesek, $\theta>0$ paraméter.
%     $f_\theta(x)=\frac1\theta f(x/\theta)$, ahol $f(t)=f_{U_1+U_2}(t)=(1-\abs{1-t})\I{t<2}$.
%     \begin{displaymath}
%       \ell'(\theta,x)=\frac{-1}{\theta^2}\zjel{f(\tfrac x\theta)+xf'(\tfrac x\theta)}
%       =\frac{-1}{\theta^2}\zjel{1-\abs{x-\theta}+x\sign(\theta-x)}\I{x<2\theta}
%       =\frac{-1}{\theta^2}
%     \end{displaymath}
%   \end{itemize}
  
% \end{frame}

\begin{frame}{Gyenge regularitási feltétel}
  \begin{df}[$(R)$, azaz gyenge regularitási feltétel]
    $\cP=\set{f_\theta d\lambda}{\theta\in\Theta}$ dominált mértékcsalád, $\Theta\subset\real^p$ nyílt.
    \begin{enumerate}[<*>]
      \item $\lambda$ majdnem minden $x$-re $\theta\mapsto \sqrt{f_\theta(x)}$ folytonosan differenciálható.
      \item $I(\theta)$ véges $\theta$-ban folytonos, nem szinguláris ($\det I(\theta)>0$).
    \end{enumerate}
  \end{df}
  \begin{itemize}
    \item $(R)$ nem függ a domináló mértéktől.
    \item Ha $\cP$ exponenciális családot alkot, akkor $(R)$ lényegében mindig teljesül.
    \item Ha $\cP$ $U(0,\theta)$-ból származó $n$-elemű minta eloszlásainak a családja, akkor 
    $\theta\mapsto \sqrt{f_\theta(x)}=\theta^{-n/2}\I{\max x_i<\theta}$ nem deriválható folytonosan majdnem minden 
    $x\in(0,\infty)^n$-re. Nem teljesíti $(R)$-et.
    \item Ha $\Theta_0\subset\Theta$ nyílt, akkor $\cP_0=\set{\P[\theta]}{\theta\in\Theta_0}$ is teljesíti $(R)$-et.
    %\item Ha 1. teljesül, akkor majdnem minden $x$-re $\theta\mapsto \sqrt{f_\theta(x)}$ és $\theta\mapsto \partial_\theta\sqrt{f_\theta(x)}$
    %folytonos, vagyis $\ln
  \end{itemize}
\end{frame}

\begin{frame}{Bederiválhatóság gyenge regularitási feltétel mellett}
  \begin{theorem}
    $\cP$ teljesíti $(R)$-et. $T$ $r$--dimenziós statisztika, $\E[\theta]{\abs{T}^2}<\infty$ 
    és lokálisan korlátos $\theta$-ban. Ekkor $\theta\mapsto\E[\theta]{T}$ folytonosan differenciálható 
    és $\partial_\theta\E[\theta]{T}=\E[\theta]{T\ell'(\theta)}$.
  \end{theorem}
  \begin{itemize}
    \item Elég $r=1$-re igazolni. 
    %\item Feltehető, hogy $\sup_{\theta\in\Theta} \E[\theta]{\abs{T}^2}<\infty$ 
    %és $\sup_{\theta\in\Theta} \norm{I(\theta)}<\infty$.
    %\item Elég a parciális deriváltak létezését és folytonosságát igazolni, ekkor a függvény folytonosan deriválható.
    %\item 
    Legyen $g(\theta)=\E[\theta]{T\ell'(\theta)}=\int_{\mfX} T(x) \partial_\theta f_\theta(x)\lambda(dx)$. 
    \begin{align*}
      \E[\theta_1]{T}-\E[\theta_0]{T}
      &=\int_{\mfX} T(x)\zjel*{f_{\theta_1}(x)-f_{\theta_0}(x)}\lambda(d x)\\
      &\hphantom{{}\hspace{-2em}}=\int_{\mfX} T(x)\int_0^1 \partial_\theta f_{\theta_0+u(\theta_1-\theta_0)}(x)(\theta_1-\theta_0)d u\lambda(d x)
      =\int_0^1 g(\theta_0+u(\theta_1-\theta_0))d u(\theta_1-\theta_0)
    \end{align*}
    \item Elég $g$ folytonossága, ugyanis
    \begin{align*}
      \E[\theta_1]{T}-\E[\theta_0]{T}&=\int_0^1 g(\theta_0+u(\theta_1-\theta_0))d u(\theta_1-\theta_0)\\
      &=g(\theta_0)(\theta_1-\theta_0)+ \int_0^1 g(\theta_0+u(\theta_1-\theta_0))-g(\theta_0)d u(\theta_1-\theta_0)\\
      &=g(\theta_0)(\theta_1-\theta_0)+o(\abs{\theta_1-\theta_0})
    \end{align*}
    % \item Feltehető, hogy a domináló mérték valószínűségi mérték.
  \end{itemize}
\end{frame}

\begin{frame}{$\theta\mapsto \E[\theta]{T\ell'(\theta)}$ folytonos}
  \begin{lemma}
    Ha $\theta\mapsto\sqrt{f_\theta(x)}$ folytonosan differenciálható $\lambda$ majdnem minden $x$-re,
    $\theta\mapsto (I(\theta))_{i,i}$ $i=1,\dots,p$ véges értékű, folytonos és 
    $\sup_{\theta\in\Theta}\E[\theta]{T^2}<\infty$. 
    
    Ekkor $\theta\mapsto \E[\theta]{T\ell'(\theta)}$ folytonos. 
  \end{lemma}
  \begin{itemize}
    \item 
    \begin{displaymath}
      g(\theta)=\E[\theta]{T\ell'(\theta)}
      =\int_{\mfX} T(x) \frac{\partial_\theta f_\theta(x)}{f_\theta(x)}f_\theta(x)\lambda(d x)
      =\int_{\mfX} T(x)\sqrt{f_\theta(x)} 2\partial_\theta \sqrt{f_\theta(x)}\lambda(d x)
    \end{displaymath}
    \item Kell: $g(\theta_n)\to g(\theta_0)$, ha $\theta_n\to\theta_0$. 
    
    Legyen $h_n(x)=T(x)\sqrt{f_{\theta_n}(x)}$, $k_n(x)= 2 \partial_\theta \sqrt{f_{\theta_n}(x)}$. Ezzel a jelöléssel:
    \begin{displaymath}
      g(\theta_n)=\int h_n k_n d\lambda,
      \quad 
      g(\theta_0)=\int h_0k_0d\lambda,
      \quad\text{Cél: \alert{$\int h_n k_{n,i}d\lambda\to\int h_0k_{0,i}d\lambda$}, $i=1,\dots,p$}
    \end{displaymath}
    \item Itt
    \begin{align*}
      \sup_n \int h_n^2d\lambda=\sup_{n}\E[\theta_n]{T^2}<\infty,
      \quad
      \int k_{n,i}^2d\lambda = (I(\theta_n))_{i,i}\to (I(\theta_0))_{i,i}=\int k_{0,i}^2 d\lambda, \quad i=1,\dots,p 
    \end{align*}
    és $\theta\mapsto\sqrt{f_\theta(x)}\in C^1$ $\lambda$ m.m. $x$ miatt $h_n\to h_0$ és $k_n\to k_0$ $\lambda$-m.m. 
    és feltehető, hogy $\lambda(\mfX)=1$. % valószínűségi mérték.
  \end{itemize}
\end{frame}

\begin{frame}{Technikai lemma}
  \begin{lemma}
    $h_n,h_0,k_n,k_0:\mfX\to\real$, $\lambda$ valószínűségi mérték $\B(\mfX)$-en.
    %Jelölés: $\int f =\int_{\mfX} f d\lambda$.

    $h_n\to h_0$, $k_n\to k_0$ $\lambda$ m.m., 
    $\sup_n \int h_n^2d\lambda<\infty$, 
    $\int k_n^2d\lambda\to\int k_0^2d\lambda<\infty$.
    
    Ekkor $\int h_nk_nd\lambda\to\int h_0k_0d\lambda$.
  \end{lemma}
  Jelölés $\int h=\int_{\mfX} h d\lambda$. Vázlat:
  \begin{itemize}
    % \item %, ugyanis 
    % $\sup_n\int h_n^2<\infty$ $\implies$ $\set{h_n}{n\geq 1}$ egyenletesen integrálható $+$ $h_n\to h_0$ 
    % $\implies$ $\int \abs{h_n-h_0} \to0$. %\int h_0d\lambda$
    \item Csonkolás. ($\eps>0$, $\approx_\eps$ jelentése a két oldal eltérése legfeljebb $\eps$) 
    
    Ha elég nagy $K$-ra és elég nagy $n_0$-tól kezdve
    \begin{displaymath}
      \int h_n k_n
      \approx_{2\eps}
      \int h_n k_n\I{\abs{k_n}<2\abs{k_0}<K}
      \approx_{\eps} 
      \int h_0 k_0\I{2\abs{k_0}<K}
      \approx_\eps 
      \int h_0k_0
      \quad\implies\quad \int h_n k_n\to\int h_0k_0
    \end{displaymath}
    %akkor nagy $n$-ekre $\int h_n k_n\approx_{5\eps}\int h_0 k_0$ és $\int h_n k_n\to\int h_0k_0$
    \item Cél: ha $K$ olyan nagy, hogy $R\int k_0^2\I{2\abs{k_0}\geq K}<\eps^2$, ahol $R=\sup_n\int h_n^2$, akkor 
    \begin{align*}
      \limsup_{n\to\infty} \int h_n^2\int k^2_n(1-\I{\abs{k_n}<2\abs{k_0}<K}) &\leq 4R\int k_0^2\I{2\abs{k_0}\geq K}<4\eps^2 \tag{$*$}\\
      \lim_{n\to\infty} \abs*{\int h_n k_n\I{\abs{k_n}<2\abs{k_0}<K}-h_0 k_0\I{2\abs{k_0}<K}}&=0
    \end{align*}
    \item Cauchy-Schwarz: $K$ választása $\implies$ utolsó lépés hibája $\leq\eps$,

    \hphantom{Cauchy-Schwarz: }$(*)$ $\implies$ első lépés hibája $\leq 2\eps$,  elég nagy $n_0$-tól kezdve.


    % \begin{displaymath}  
    %   \int h_n k_n\I{\abs{k_n}<2\abs{k_0}<K}
    %   %=\int (h_n-h_0)k_n\I{\abs{k_n}<2\abs{k_0}<K}d\lambda + \int h_0k_n\I{\abs{k_n}<2\abs{k_0}<K}d\lambda
    %   \to \int h_0k_0\I{2\abs{k_0}<K},
    %   \quad\text{mert $k_n\I{\abs{k_n}<2\abs{k_0}<K}\to k_0\I{2\abs{k_0}<K}$}.
    % \end{displaymath}
    % \item A csonkolással elkövetett hiba ,,egyenletesen kicsi'' ($\limsup$ kicsivé tehető, ha $K$ nagy):
    
    % $\eps>0$-hoz létezik $K$, hogy $R\int k^2_0\I{2\abs{k_0}\geq K}\leq \eps^2$ ahol $R=\zjel*{\sup_n\int h_n^2}$.
    % \begin{displaymath}
    %   \zjel*{\smallint h_n k_n(1-\I{\abs{k_n}<2\abs{k_0}<K})}^2\leq
    %   R %\sup_n\int h_n^2 
    %   \zjel*{\smallint k_n^2\I{\abs{k_n}>2\abs{k_0}}+4k_0^2\I{2\abs{k_0}>K}} 
    %   < 2\eps^2,
    %   \quad\text{ha $n$ elég nagy.}
    % \end{displaymath}
    % Ugyanis $\int k_n^2\I{\abs{k_n}>2\abs{k_0}}\to0$. Ez indoklásra szorul, később.
    % \item Nagy $n$-re ($\approx_\eps$ jelentése a két oldal eltérése legfeljebb $\eps$)
    % \begin{displaymath}
    %   \smallint h_n k_n 
    %   \approx_{2\eps} 
    %   \smallint h_n k_n\I{\abs{k_n}<2\abs{k_0}<K} 
    %   \approx_\eps 
    %   \smallint h_0k_0\I{2\abs{k_0}<K}
    %   \approx_\eps
    %   \smallint h_0k_0
    %   % \abs*{\int h_n k_n-h_0k_0}\leq
    %   % \abs*{\int h_n k_n\I{\abs{k_n}<2\abs{k_0}<K}-h_0k_0\I{2\abs{k_0}\geq K}}
    %   % +\abs*{\int h_n k_n(1-\I{\abs{k_n}<2\abs{k_0}<K})}
    %   % +\abs*{\int h_0k_0\I{2\abs{k_0}\geq K}}\leq 4\eps\quad\text{ha $n$ elég nagy.}
    % \end{displaymath}
    % % \begin{displaymath}
    % %   k_n^2\I{\abs{k_n}<2\abs{k_0}}\to k_0^2,\quad k_n^2\I{\abs{k_n}<2\abs{k_0}}\leq 4k_0^2,\implies
    % %   \int k_n^2\I{\abs{k_n}<2\abs{k_0}}\to \int k_0^2=\lim \int k_n^2 
    % %   \implies \int k_n^2\I{\abs{k_n}\geq2\abs{k_0}}\to0  
    % % \end{displaymath}
    % % \item 
    % % %   \zjel*{\int h_nk_n\I*{A_n^c}d\lambda}\leq \int h_n^2d\lambda \int k_n^2\I*{A_n^c}d\lambda
    % % % \end{displaymath}
    % % % \item 
    % % % \begin{displaymath}
    % % %   k_n^2\I*{A_n}\to k_0^2,\quad k_n^2\I*{A_n}\leq 4k_0^2,\implies
    % % %   \int k_n^2 \I*{A_n}d\lambda\to \int k_0^2 d\lambda=\lim \int k_n^2d\lambda 
    % % %   \implies \int k_n^2\I*{A_n^{c}}d\lambda\to0  
    % % % \end{displaymath}
    % % % \item 
  \end{itemize}
\end{frame}

\begin{frame}{Technikai lemma részletei} 
  % Jelölés $\int h=\int_{\mfX} h d\lambda$.
  \begin{proposition}
    $k_n,k_0:\mfX\to\real$,
    $k_n\to k_0$ $\lambda$ m.m., 
    $\int k_n^2\to\int k_0^2<\infty$.
    
    Ekkor $\int k^2_n\I{\abs{k_n}\geq 2\abs{k_0}}\to0$.
  \end{proposition}
  \continue
  %\begin{itemize}
  %  \item 
  $k^2_n\I{\abs{k_n}< 2\abs{k_0}}\to k_0^2$. Dominált konvergencia tétel ($4k_0^2$ integrálható majoráns)
    \begin{displaymath}
      \smallint k^2_n\I{\abs{k_n}\leq 2\abs{k_0}} \to\smallint k_0^2=\lim\smallint k_n^2 
      \quad\implies\quad 
      \smallint k_n^2(1-\I{\abs{k_n}<2\abs{k_0}})\to0 
    \end{displaymath}
  %\end{itemize}
  \pause
  \begin{corollary}
    \begin{displaymath}
      \limsup_{n\to\infty} \smallint k^2_n(1-\I{\abs{k_n}<2\abs{k_0}<K}) \leq 4\smallint k_0^2\I{2\abs{k_0}\geq K}
    \end{displaymath}
  \end{corollary}
  \begin{align*}
    \smallint k^2_n(1-\I{\abs{k_n}<2\abs{k_0}<K})
    &= \smallint k_n^2\I{\abs{k_n}>2\abs{k_0}}+\smallint k_n^2\I{\abs{k_n}<2\abs{k_0}\geq K}\\
    &\leq \smallint k_n^2\I{\abs{k_n}>2\abs{k_0}}+\smallint 4k_0^2\I{2\abs{k_0}\geq K}
    \to \smallint 4 k_0^2\I{2\abs{k_0}\geq K}
  \end{align*}
\end{frame}

\begin{frame}{Technikai lemma részletei}
  \begin{proposition}
    $h_n,h_0,k_n,k_0:\mfX\to\real$, \alert{$\lambda(\mfX)=1$},
    %$\lambda$ valószínűségi mérték $\B(\mfX)$-en.
    %
    $h_n\to h_0$, $k_n\to k_0$ $\lambda$ m.m., 
    $R=\sup_n \int h_n^2d\lambda<\infty$. Ekkor
    %
    %$
    \begin{displaymath}
      \int h_n k_n\I{\abs{k_n}<2\abs{k_0}<K}\to \int h_0 k_0\I{2\abs{k_0}<K}  
    \end{displaymath}
    %$
  \end{proposition}
  \begin{itemize}
    \item $h_0 k_n\I{\abs{k_n}<2\abs{k_0}<K}\to h_0k_0\I{2\abs{k_0}<K}$ és $\abs{h_0}K$ integrálható majoráns.
    \begin{displaymath}
      \lim \smallint h_0 k_n\I{\abs{k_n}<2\abs{k_0}<K} =\smallint h_0k_0\I{2\abs{k_0}<K}
    \end{displaymath}
    \item $h_n$ $L^2$-ben korlátos, ezért egyenletesen integrálható. Emlékeztető:
    \begin{displaymath}
      \sup_n\int \abs{h_n} \I{\abs{h_n}\geq L}\leq \sup_n\frac1L\int h_n^2=\frac{R}{L}\to0\quad \text{ha $L\to\infty$}.
    \end{displaymath}
    \item $\set{h_n}{n\geq1}$ egyenletesen integrálható, $h_n\to h_0$ m.m. $\implies$ $\int \abs{h_n-h_0}\to0$ $\implies$
    \begin{displaymath}
      \smallint (h_n-h_0) k_n\I{\abs{k_n}<2\abs{k_0}<K}\to0
    \end{displaymath}
    \item 
    \begin{displaymath}
      \smallint h_0k_0\I{2\abs{k_0}<K}
      =\lim \smallint h_0 k_n\I{\abs{k_n}<2\abs{k_0}<K}
      =\lim \smallint h_n k_n\I{\abs{k_n}<2\abs{k_0}<K}.
    \end{displaymath}
  \end{itemize}
\end{frame}

\begin{frame}{$(R)$ következményei}
  \begin{df}[$(R)$, azaz gyenge regularitási feltétel]
    $\cP=\set{f_\theta d\lambda}{\theta\in\Theta}$ dominált mértékcsalád, $\Theta\subset\real^p$ nyílt.
    \begin{enumerate}[<*>]
      \item $\lambda$ majdnem minden $x$-re $\theta\mapsto \sqrt{f_\theta(x)}$ folytonosan differenciálható.
      \item $I(\theta)$ véges $\theta$-ban folytonos, nem szinguláris ($\det I(\theta)>0$).
    \end{enumerate}
  \end{df}
  \begin{lemma}[Már igazoltuk]
    $(R)$ + $\sup_{\theta\in\Theta}\E[\theta]{\abs{T}^2}<\infty$ $\implies$ 
    $g(\theta)=\E[\theta]{T}$ deriválható, $g'(\theta)=\E[\theta]{T\ell'(\theta)}$ folytonos, ahol 
    $\ell(\theta,x)=\log f_{\theta}(x)$.
  \end{lemma}
  \begin{itemize}
    \item $(R)$ teljesül, $T\equiv1$. 
    Ekkor $0=\E[\theta]{\ell'(\theta)}=\int_{\mfX} \partial_\theta f_\theta(x)\lambda(dx)$ 
    $\implies$ $n$ elemű minta Fisher információja az egy elemű minta $I_1(\theta)$ Fisher információjának $n$-szerese.
    \item $(R)$ teljesül, $T$ torzítatlan $g(\theta)$-ra, %ahol $g$ folytonosan deriválható. 
    $\E[\theta]^2{\abs{T}^2}$ 
    lokálisan korlátos. Cramer-Rao egyenlőtlenség feltételei teljesülnek:
    
    ($\ell'$ létezik, $\E[\theta]{\ell'(\theta)}=0$,  $I(\theta)=\E[\theta]{\ell'(\theta))^T\ell'(\theta)}$ invertálható. 
    $g(\theta)=\E[\theta]{T}$ deriválható és $g'(\theta)=\E[\theta]{T\ell'(\theta)}$)
    \begin{displaymath}
      \Sigma_\theta(T)\geq g'(\theta) I(\theta)^{-1}g'(\theta)^T =\frac1n g'(\theta) I_1(\theta)^{-1}g'(\theta)^T%\tag{$*$}
    \end{displaymath}
    
    %\item Összefoglalva: $(R)$ $+$ $g(\theta)=\E[\theta]{T}$ és $\E[\theta]{\abs{T}^2}$ lokálisan korlátos
    %$\theta$-ban $\implies$ $g\in C^{1}$ és  $(*)$.
  \end{itemize}
\end{frame}

\begin{frame}{Példák}
  \begin{itemize}
    \item Ha $\cP$ exponenciális családot alkot $\Theta\subset\real^p$ nyílt. 
    \begin{displaymath}
      f_\theta(x)=\exp{\theta T(x)-b(\theta)},\quad \sqrt{f_\theta(x)}=\exp{\tfrac12\theta T(x)-\tfrac12b(\theta)}
    \end{displaymath}
    ahol $b(\theta)$ sima $\Theta$-n, azaz $\theta\mapsto \sqrt{f_\theta}(x)$ folytonosan differenciálható.
    
    $I(\theta)=b''(\theta)=\Sigma_\theta(T)$, azaz $I(\theta)$ véges értékű és folytonos, 
    $\det I(\theta)>0$ teljesül, ha $T$ a $\real^p$ nem egy valódi affin altérből veszi fel az értékeit.
    \item $X$ Cauchy eloszlás eltolás paraméteres családjából származó $n$ elemű minta. 
    $X$ eloszlásai nem alkotnak exponenciális családot, de
    \begin{displaymath}
      \sqrt{f_\theta(x)}=\prod(1+(x_i-\theta)^2)^{-1/2}
    \end{displaymath}
    folytonosan deriválható, és 
    \begin{displaymath}
      \hphantom{\hspace{-2em}}
      \ell(\theta,x)=\sum_{i} \frac{-1}{2}\ln(1+(x_i-\theta)^2),
      \quad
      \ell'(\theta,x)=\sum_i \frac{x_i-\theta}{1+(x_i-\theta)^2}
      %\quad 
      %\E[\theta]{\ell'(\theta)}=\E[0]{\ell'(0)}=\sum_i \int_{\real} \frac{x_i}{\pi (1+x_i^2)^2} d x_i=0
      %\quad
      % I(\theta) = \E[\theta]{(\ell'(\theta))^2}
    %   =n\int_{\real} \frac{x^2}{(1+x^2)^2}\frac{1}{\pi(1+x^2)}d x=\frac{n}{4}<\infty 
    \end{displaymath}
    \begin{displaymath}
      \E[\theta]{\ell'(\theta)} = n\int_{\real} \frac{x}{\pi (1+x^2)^2} d x=0
      \quad
      I(\theta) = \D[\theta]^2{\ell'(\theta)}
      =n\int_{\real} \frac{x^2}{(1+x^2)^2}\frac{1}{\pi(1+x^2)}d x<\infty 
    \end{displaymath}
    $I(\theta)$ véges, folytonos és nem nulla. 
    $X$ eloszlásainak családja teljesíti a gyenge regularitási feltételt.
    %$\ell'(\theta)$ eloszlása 
  \end{itemize}  
\end{frame}

\begin{frame}{Független minták Fisher információja összeadódik}
  \begin{df}[$(R)$, azaz gyenge regularitási feltétel]
    $\cP=\set{f_\theta d\lambda}{\theta\in\Theta}$ dominált mértékcsalád, $\Theta\subset\real^p$ nyílt.
    \begin{enumerate}[<*>]
      \item $\lambda$ majdnem minden $x$-re $\theta\mapsto \sqrt{f_\theta(x)}$ folytonosan differenciálható.
      \item $I(\theta)$ véges $\theta$-ban folytonos, nem szinguláris ($\det I(\theta)>0$).
    \end{enumerate}
  \end{df}
  \begin{proposition}
    $X$, $Y$ minden $\theta\in\Theta$ mellett függetlenek és eloszlásaikra teljesül $(R)$.

    Ekkor $(X,Y)$ együttes eloszlásaira is teljesül $(R)$ és $I_{X,Y}(\theta)=I_{X}(\theta)+I_{Y}(\theta)$.
  \end{proposition}
  \begin{itemize}
    \item $X$ eloszlásainak domináló mértéke $\lambda_X$, $Y$-ra $\lambda_Y$. Ekkor $(X,Y)$ eloszlásait 
    $\lambda_X\otimes\lambda_Y$ dominálja. 
    \item $f_{(X,Y),\theta}(x,y)=f_{X,\theta}(x)f_{Y,\theta}(y)$. Ha $\sqrt{f_{X,\theta}(x)}$ $\sqrt{f_{Y,\theta}(y)}$ 
    folytonosan deriválható $\theta$-ban $\lambda_X$ majdnem minden $x$-re ill. $\lambda_Y$ majdnem minden $y$-ra, akkor 
    $\sqrt{f_{X,Y,\theta}(x,y)}$ is folytonosan differenciálható $\lambda_X\otimes\lambda_Y$ m.m $(x,y)$-ra. 
    \item $\ln f_{X,Y,\theta}(x,y)=\ln f_{X,\theta}(x)+\ln f_{Y,\theta}(y)$. 
    Ebből $I_{X+Y}(\theta)=I_X(\theta)+I_{Y}(\theta)$ következik. 
    \item $I_{X+Y}(\theta)$ folytonos, véges értékű.
    \item $I_{X+Y}\geq I_{X}$, amiből $\ker I_{X+Y}\subset \ker I_X=\smallset{0}$, 
    $I_{X+Y}$ is invertálható, minden $\theta$-ra.
  \end{itemize}
\end{frame}

\begin{frame}{$S(X)$ Fisher információja nem lehet $X$ információ tartalmánál nagyobb}
  \begin{proposition}
    $X$ eloszlásainak a családjára teljesül $(R)$. $S$ statisztika. Ekkor $S$ eloszlásainak a családja is dominált.

    Ha $S$ eloszlásainak családjára is teljesül $(R)$, akkor $I_{S(X)}(\theta)\leq I_X(\theta)$.
  \end{proposition}
  \begin{itemize}
    %\item $(R)$ nem függ a domináló mértéktől, $\P[0]\in \cP'$ választható. 
    %($\cP'=\set{\sum c_i\cP[\theta_i]}{c_i\geq0,\,\sum c_i=1}$). 
    \item $\lambda\circ S^{-1}$ domináló mérték $S$ eloszlásaihoz: 
    \begin{displaymath}
      \P[\theta]{S\in H}
      =\int_{\mfX} f_\theta(x) \I{S(x)\in H} \lambda (d x)
      % =\E[0]{\frac{d\P[\theta]|_{\sigma(S)}}{d\P[0]|_{\sigma(S)}}\I{S\in H}}
      =\int_H g_\theta(s) \lambda\circ S^{-1}(d s), 
      \quad g_\theta(S)= \frac{d\P[\theta]|_{\sigma(S)}} {d\lambda|_{\sigma(S)}}. 
    \end{displaymath}
    \item Ha $S$ eloszlásai is teljesítik $(R)$-et, $\P[\theta]{S\in H}$-re adott mindkét formulában az 
    integrálon belül lehet deriválni $\theta$ szerint.
    $\ell_X(\theta,x)=\ln f_{\theta}(x)$ 
    és $\ell_S(\theta,s)=\ln g_\theta(s)$ jelöléssel
    \begin{displaymath}
      %\partial_\theta \P[\theta]{S\in H}
      %=
      \int_{\mfX} \partial_\theta  f_{\theta}(x)\I{S(x)\in H} \lambda(d x)
      =\E[\theta]{\I{S\in H} \ell_X'(\theta,X)}
      =\int_{H} g_\theta'(s) \lambda\circ S^{-1}(d s)
      =\E[\theta]{\I{S\in H}\ell'_S(\theta,S)}
    \end{displaymath}
    \item $\E[\theta]{\ell'_X(\theta,X)|S}=\ell'_S(\theta,S)$ és a teljes szórásnégyzet tétel szerint 
    $I_S(\theta)\leq I_X(\theta)$.
  \end{itemize}
\end{frame}

\begin{frame}{Elégséges statisztika Fisher információja}
  \begin{proposition}
    $X$ minta eloszlásainak családjára teljesül $(R)$, $S$ elégséges statisztika. 
    Ekkor $S$ eloszlásainak a családjára is teljesül $(R)$ és $I_{S(X)}(\theta)=I_{X}(\theta)$.
  \end{proposition}
  \begin{itemize}
    \item  $(R)$ nem függ a domináló mértéktől, $\P[0]\in \cP'$ választható. 
    ($\cP'=\set{\sum c_i\cP[\theta_i]}{c_i\geq0,\,\sum c_i=1}$). 
    \item $S$ elégséges, ezért $\P[0]$-ra vonatkozó sűrűségek $f_{\theta}(x)=g_{\theta}(S(x))$ alakúak.
    \item $g_\theta$ $S$ eloszlásának a sűrűsége a $\P[0]\circ S^{-1}$ domináló mértékre nézve.
    \item $\sqrt{f_{\theta}(x)}=\sqrt{g_\theta(S(x))}$, azaz ha $(R)$ teljesül, 
    akkor $\sqrt{g_\theta(s)}$ folytonosan deriválható $\P[0]\circ S^{-1}$ m.m. $s$-re. 
    \item $\ell_X'(\theta,X)=\ell_{S}'(\theta,S)$, amiből $I_X(\theta)=I_{S}(\theta)$ és $(R)$ $I$-re vonatkozó 
    előírásai $I_S$-re is igazak.
  \end{itemize}
\end{frame}

\begin{frame}{Fisher információ és elégséges statisztika}
  \begin{proposition}
    $(R)$ teljesül $X$ és az $S$ statisztika  eloszlásainak családjára is.
     $f_\theta>0$ a mintatéren minden $\theta\in\Theta$-ra és $\Theta$ összefüggő nyílt.
    
     Ha $I_X(\theta)=I_{S(X)}(\theta)$, akkor $S$ elégséges.
  \end{proposition}
  \begin{itemize}
    \item $\P[0]=\P[\theta_0]$ $\theta_0\in \Theta$ választható domináló mértéknek, $f_{\theta_0}\equiv1$.
    \item Láttuk, hogy $\E[\theta]{(\partial_\theta \ln f_\theta)(X)|S }=\E{(\partial_\theta \ln g_\theta)(S)}$, 
    ahol $g_\theta$ $S$ eloszlásainak sűrűsége $\P[\theta_0]\circ S^{-1}$-re.
    \item A feltétel szerint $I_X=I_S$, ami csak akkor lehet, ha 
    $\partial_\theta\ln f_\theta(x)=\partial_\theta g_\theta(S(x))$ minden $\theta$-ra $\P[0]$ m.m..
    \item $\partial_\theta\ln f_\theta(x)$, $\partial_\theta\ln g_\theta(x)$ $\P[0]$ m.m. 
    $x$-re folytonos $\theta$-ban $(R)$ miatt. 
    \item $\P[0]{ \forall\theta\in\Theta,\,\partial_\theta\ln f_\theta(x)=\partial_\theta g_\theta(S(x))}=1$.
    \item 
    \begin{displaymath}
      \ln f_{\theta_1}(x)-\ln f_{\theta_0} (x) = \int_{\gamma} \partial_\theta \ln f_\theta(x) d\theta=
      \int_\gamma \partial_\theta \ln g_\theta (S(x))d\theta= \ln g_{\theta_1}(S(x))-\ln g_{\theta_0}(S(x))
    \end{displaymath}
    \begin{displaymath}
      f_\theta(x)=\frac{g_\theta(S(x))}{g_{\theta_0}(S(x))}
    \end{displaymath}
    azaz $S$ elégséges a faktorizációs tétel miatt.
  \end{itemize}
\end{frame}
\end{document}
\begin{frame}{Összefoglalás}
  
\end{frame}

\begin{frame}{Becslési Módszerek}
  \begin{itemize}
    
  \item Tapasztalati becslések
  \item Momentum módszer
  \item Maximum likelihood becslés
  \item Bayes becslés
  \end{itemize}
\end{frame}

\begin{frame}{Néhány fogalom}
    \begin{df}[Tapasztalati eloszlás]
      $X_1,\dots, X_n$ minta, $P_n^*=\frac1n \sum_{i}\delta_{X_i}$ a
      tapasztalati eloszlás.
    \end{df}

    Ki fogjuk számolni, hogy ha $X_1,X_2,\dots$ a $P$ eloszlásból
    származó független megfigyelések, akkor
    $P_n^*\dto P$ egy valószínűséggel.
    (Glivenko-Cantelli tétel)

    \begin{df}
      $\cP=\set{\P[\theta]}{\theta\in
        \Theta}$,
      $X_1,X_2,\dots$ iid. minta,
      $T_n=T_n (X_1,\dots,X_n)$ 
      $g (\theta)$ becslése $n$ elemű mintából.

      A $T_n$
      becsléssorozat konzisztens, ha $T_n\pto g (\theta)$, és erősen
      konzisztens ha $T_n\to g (\theta)$ egy valószínűséggel.
    \end{df}

\end{frame}

\begin{frame}{Tapasztalati becslések}
  \begin{df}
      $\cP$ eloszlások egy családja, amely tartalmazza a véges sok
      pontra koncentrált eloszlásokat. $\phi$ a $\cP$-n értelmezett
      funkcionál.

      $\phi (P)$ tapasztalati becslése $\phi (P_n^*)$, ahol $P_n^*$ a
      tapasztalati eloszlás. 
    \end{df}

  \begin{itemize}
  \item Példa. $N (\mu,1)$ eloszláscsaládnál kézenfekvő a paramétert a
    mintaátlaggal becsülni.
  \item A paraméter a várható érték, ami véges tartójú eloszlásokra is
    létezik. A becslés a tapasztalati eloszlásból számolt várható érték.
  
  \item Ha a minta a $P$ eloszlásból származik, akkor $P_n^*\dto P$
    egy valószínűséggel, azaz ha $\phi$ a gyenge konvergenciára 
    nézve folytonos, akkor $\phi$ tapasztalati becslése erősen
    konzisztens
    
  \item pl. tapasztalati medián, ha egyértelmű, erősen konzisztens a mediánra.
  \end{itemize}
  
\end{frame}

\begin{frame}{Momentum módszer}
  \begin{itemize}
  \item $\cP=\set{\P[\theta]}{\theta\in\Theta}$, $\theta$-t akarjuk
    becsülni.
  \item Legyen $g:\mathfrak X_1\to\real^p$, és $\phi
    (\theta)=\E[\theta] (g (X_1))$.
    Tegyük fel, hogy $\phi:\Theta\to T$ kölcsönösen egyértelmű és oda-vissza
    folytonos, $T$ nyílt. 
  \item $\psi=\phi (\theta)$  tapasztalati becslése
    \begin{displaymath}
      \hat\psi_n=\frac1n \sum_i g (X_i)
    \end{displaymath}
  \item Ha $n$ elég nagy, akkor $\hat\psi\in T$ és
    $\hat\theta=\phi^{-1} (\hat\psi)$ értelmes. 
  \item $\hat\psi_n\to\psi$ egy valószínűséggel és $\phi^{-1}$ folytonossága
    miatt $\hat\theta_n$ erősen konzisztens, de általában nem torzítatlan.

    $\hat\psi_n$ torzítatlan $\psi$, de $\phi^{-1}$ általában nem
    lineáris.
    
  \item Klasszikus momentum módszernél $g (x)=(x_1^1,\dots,x_1^p)$.
  \end{itemize}

\end{frame}
 
\begin{frame}{Példa}
  
\end{frame}

\begin{frame}{Maximum likelihood becslés}
  \begin{itemize}
  \item $\cP$ dominált mértékcsalád, $f_\theta$ a sűrűségfüggvény,
    $\ell (\theta)$ a loglikelihood.
    \begin{df}
      A $\theta$ paraméter maximum likelihood becslése $\hat\theta (X)$,
      ha $f_{\hat{\theta}} (X)=\sup_{\theta}f_{\theta} (X)$.
    \end{df}

  \item Nem biztos, hogy létezik, nem biztos, hogy egyértelmű.

    pl. $U (\theta,\theta+1)$ eloszláscsalád. $f_\theta
    (x)=\I{X_n^*-1\leq \theta\leq X_1^*}$, azaz ha  $T (X)\in
    [X_n^*-1,X_1^*]$, akkor $T$ maximum likelihood becslése
    $\theta$-nak.

    pl. $f (x)$ sűrűségfüggvény $\real$-en.
    $$f_{\mu,\sigma}
    (x)=\frac{1}\sigma f \zfrac*{x-\mu}{\sigma}
    $$
    $\Theta=\real\times (0,\infty)$. $c\in\real$, $f (c)>0$, $x$ a
    megfigyelt érték
    \begin{displaymath}
      \sup_{\mu,\sigma}\frac{1}{\sigma}f \zfrac*{x-\mu}{\sigma}\geq
      \lim_{\sigma\to0}\frac{1}{\sigma} f\zfrac*{x- (x+c\sigma)}{\sigma}=\infty
    \end{displaymath}
  \end{itemize}  
\end{frame}

\begin{frame}{Maximum likelihood becslés}
  $\psi=g(\theta)$ becslése maximum likelihood elvvel. Ha $g$ kölcsönösen
  egyértelmű, akkor csak átparamétereztük a családot,
  $\hat\psi=g(\hat\theta)$.

  Ha $g$ nem injektív, akkor az indukált likelihoodot szokták
  maximalizálni
  \begin{displaymath}
    f^*_\psi (x)=\sup\set{f_\theta (x)}{g (\theta)=\psi}
  \end{displaymath}

  \begin{proposition}
    \begin{itemize}
    \item Ha $T$ elégséges statisztika és létezik ML becslés, akkor
      van olyan is, amelyik $T$ függvénye.
    \item Ha $\hat\theta$ a $\theta$ ML becslése, akkor $g
      (\hat\theta)$ a $g (\theta)$ ML becslése.
    \end{itemize}
  \end{proposition}
  \begin{itemize}
  \item Ha a maximum hely egyértelmű, akkor a faktorizációs tétel
    miatt $f_\theta (x)=g_\theta (T (x)) h (x)$ csak $T (x)$-en
    keresztül függ a megfigyeléstől. Ha nem egyértelmű, akkor arra
    kell figyelni, hogy $T (x)$ függvényében válasszunk.
    
    pl. $U (\theta,\theta+1)$ esetén elégséges statisztika
    $T (X) =(X_1^*,X_n^*)$. $\hat\theta(=X_n^*-1+X_1^*)/2$ ML becslés $T$
    függvényei között,
    \begin{displaymath}
      \hat\theta+\sin (X_1)\frac{X_1^*-(X_n^*-1)}2
    \end{displaymath}
    pedig nem (feltéve, hogy a minta elemszáma legalább 2).
  \end{itemize}  
\end{frame}

\begin{frame}{Bayes becslés, bevezetés}
  $\cP=\set{\P[\theta]}{\theta\in\Theta}$ eloszláscsalád. $g
  (\theta)$-t szeretnénk becsülni.
  \begin{displaymath}
    L (\theta,T)=(g(\theta)-T)^2,\quad R_T(\theta)=\E[\theta] (L (\theta,T))
  \end{displaymath}
  $L (\theta,T)$ a veszteség, amit a $T$ becslés okoz, $R_T (\theta)$
  az átlagos veszteség, más néven rizikó.

  Bayes becslésnél a cél a
  \begin{displaymath}
    R_T (Q)=\int_\Theta R_T (\theta) Q (d\theta)\quad \text{apriori rizikó}
  \end{displaymath}
  minimalizálása.

  $Q$ gyakran valószínűség eloszlás. Ilyenkor úgy gondolhatunk a feladatra,
  hogy $\theta$ is valószínűségi változó, melynek eloszlása (a priori eloszlás)
  $Q$.
  $\P[t]$ a minta feltételes eloszlása a $\theta=t$ feltétel mellett.

  \begin{displaymath}
    R_T (Q)=\E{L (\theta,T)}=\E{(g (\theta)-T (X))^2}\geq \E{(g (\theta)-\E{g (\theta)|X})^2}
  \end{displaymath}

  A Bayes becslés $g (\theta)$ feltételes várható értéke a mintára nézve.
\end{frame}

\end{document}

  